\pdfoutput=1
%% Author: PGL  Porta Mana
%% Created: 2019-12-15T00:46:38+0100
%% Last-Updated: 2019-12-25T11:22:53+0100
%%%%%%%%%%%%%%%%%%%%%%%%%%%%%%%%%%%%%%%%%%%%%%%%%%%%%%%%%%%%%%%%%%%%%%%%%%%%
\newif\ifarxiv
\arxivfalse
\ifarxiv\pdfmapfile{+classico.map}\fi
\newif\ifafour
\afourfalse% true = A4, false = A5
\newif\iftypodisclaim % typographical disclaim on the side
\typodisclaimtrue
\newcommand*{\memfontfamily}{zplx}
\newcommand*{\memfontpack}{newpxtext}
\documentclass[\ifafour a4paper,12pt,\else a5paper,10pt,\fi%extrafontsizes,%
onecolumn,oneside,article,%french,italian,german,swedish,latin,
british%
]{memoir}
\newcommand*{\firstdraft}{13 December 2019}
\newcommand*{\firstpublished}{\today}
\newcommand*{\updated}{\ifarxiv***\else\today\fi}
\newcommand*{\propertitle}{Dimensional analysis on differential manifolds [draft]%\\{\large ***}%
}% title uses LARGE; set Large for smaller
\newcommand*{\pdftitle}{\propertitle}
\newcommand*{\headtitle}{Dimensional analysis on differential manifolds}
\newcommand*{\pdfauthor}{P.G.L.  Porta Mana}
\newcommand*{\headauthor}{Porta Mana}
\newcommand*{\reporthead}{\ifarxiv\else Open Science Framework \href{https://doi.org/10.31219/osf.io/***}{\textsc{doi}:10.31219/osf.io/***}\fi}% Report number

%%%%%%%%%%%%%%%%%%%%%%%%%%%%%%%%%%%%%%%%%%%%%%%%%%%%%%%%%%%%%%%%%%%%%%%%%%%%
%%% Calls to packages (uncomment as needed)
%%%%%%%%%%%%%%%%%%%%%%%%%%%%%%%%%%%%%%%%%%%%%%%%%%%%%%%%%%%%%%%%%%%%%%%%%%%%

%\usepackage{pifont}

%\usepackage{fontawesome}

\usepackage[T1]{fontenc} 
\input{glyphtounicode} \pdfgentounicode=1

\usepackage[utf8]{inputenx}

%\usepackage{newunicodechar}
% \newunicodechar{Ĕ}{\u{E}}
% \newunicodechar{ĕ}{\u{e}}
% \newunicodechar{Ĭ}{\u{I}}
% \newunicodechar{ĭ}{\u{\i}}
% \newunicodechar{Ŏ}{\u{O}}
% \newunicodechar{ŏ}{\u{o}}
% \newunicodechar{Ŭ}{\u{U}}
% \newunicodechar{ŭ}{\u{u}}
% \newunicodechar{Ā}{\=A}
% \newunicodechar{ā}{\=a}
% \newunicodechar{Ē}{\=E}
% \newunicodechar{ē}{\=e}
% \newunicodechar{Ī}{\=I}
% \newunicodechar{ī}{\={\i}}
% \newunicodechar{Ō}{\=O}
% \newunicodechar{ō}{\=o}
% \newunicodechar{Ū}{\=U}
% \newunicodechar{ū}{\=u}
% \newunicodechar{Ȳ}{\=Y}
% \newunicodechar{ȳ}{\=y}

\newcommand*{\bmmax}{0} % reduce number of bold fonts, before font packages
\newcommand*{\hmmax}{0} % reduce number of heavy fonts, before font packages

\usepackage{textcomp}

%\usepackage[normalem]{ulem}% package for underlining
% \makeatletter
% \def\ssout{\bgroup \ULdepth=-.35ex%\UL@setULdepth
%  \markoverwith{\lower\ULdepth\hbox
%    {\kern-.03em\vbox{\hrule width.2em\kern1.2\p@\hrule}\kern-.03em}}%
%  \ULon}
% \makeatother

\usepackage{amsmath}

\usepackage{mathtools}
\addtolength{\jot}{\jot} % increase spacing in multiline formulae
\setlength{\multlinegap}{0pt}

%\usepackage{empheq}% automatically calls amsmath and mathtools
%\newcommand*{\widefbox}[1]{\fbox{\hspace{1em}#1\hspace{1em}}}

%%%% empheq above seems more versatile than these:
%\usepackage{fancybox}
%\usepackage{framed}

% \usepackage[misc]{ifsym} % for dice
% \newcommand*{\diceone}{{\scriptsize\Cube{1}}}

\usepackage{amssymb}

\usepackage{amsxtra}

\usepackage{tensor}

\usepackage[main=british,french,italian,german,swedish,latin,esperanto]{babel}\selectlanguage{british}
\newcommand*{\langfrench}{\foreignlanguage{french}}
\newcommand*{\langgerman}{\foreignlanguage{german}}
\newcommand*{\langitalian}{\foreignlanguage{italian}}
\newcommand*{\langswedish}{\foreignlanguage{swedish}}
\newcommand*{\langlatin}{\foreignlanguage{latin}}
\newcommand*{\langnohyph}{\foreignlanguage{nohyphenation}}

\usepackage[autostyle=false,autopunct=false,english=british]{csquotes}
\setquotestyle{british}

\usepackage{amsthm}
\newcommand*{\QED}{\textsc{q.e.d.}}
\renewcommand*{\qedsymbol}{\QED}
\theoremstyle{remark}
\newtheorem{note}{Note}
\newtheorem*{remark}{Note}
\newtheoremstyle{innote}{\parsep}{\parsep}{\footnotesize}{}{}{}{0pt}{}
\theoremstyle{innote}
\newtheorem*{innote}{}

\usepackage[shortlabels,inline]{enumitem}
\SetEnumitemKey{para}{itemindent=\parindent,leftmargin=0pt,listparindent=\parindent,parsep=0pt,itemsep=\topsep}
% \begin{asparaenum} = \begin{enumerate}[para]
% \begin{inparaenum} = \begin{enumerate*}
\setlist{itemsep=0pt,topsep=\parsep}
\setlist[enumerate,2]{label=\alph*.}
\setlist[enumerate]{label=\arabic*.,leftmargin=1.5\parindent}
\setlist[itemize]{leftmargin=1.5\parindent}
\setlist[description]{leftmargin=1.5\parindent}
% old alternative:
% \setlist[enumerate,2]{label=\alph*.}
% \setlist[enumerate]{leftmargin=\parindent}
% \setlist[itemize]{leftmargin=\parindent}
% \setlist[description]{leftmargin=\parindent}

\usepackage[babel,theoremfont,largesc]{newpxtext}

\usepackage[bigdelims,nosymbolsc%,smallerops % probably arXiv doesn't have it
]{newpxmath}
\linespread{1.083}%\useosf
%% smaller operators for old version of newpxmath
\makeatletter
\def\re@DeclareMathSymbol#1#2#3#4{%
    \let#1=\undefined
    \DeclareMathSymbol{#1}{#2}{#3}{#4}}
%\re@DeclareMathSymbol{\bigsqcupop}{\mathop}{largesymbols}{"46}
%\re@DeclareMathSymbol{\bigodotop}{\mathop}{largesymbols}{"4A}
\re@DeclareMathSymbol{\bigoplusop}{\mathop}{largesymbols}{"4C}
\re@DeclareMathSymbol{\bigotimesop}{\mathop}{largesymbols}{"4E}
\re@DeclareMathSymbol{\sumop}{\mathop}{largesymbols}{"50}
\re@DeclareMathSymbol{\prodop}{\mathop}{largesymbols}{"51}
\re@DeclareMathSymbol{\bigcupop}{\mathop}{largesymbols}{"53}
\re@DeclareMathSymbol{\bigcapop}{\mathop}{largesymbols}{"54}
%\re@DeclareMathSymbol{\biguplusop}{\mathop}{largesymbols}{"55}
\re@DeclareMathSymbol{\bigwedgeop}{\mathop}{largesymbols}{"56}
\re@DeclareMathSymbol{\bigveeop}{\mathop}{largesymbols}{"57}
%\re@DeclareMathSymbol{\bigcupdotop}{\mathop}{largesymbols}{"DF}
%\re@DeclareMathSymbol{\bigcapplusop}{\mathop}{largesymbolsPXA}{"00}
%\re@DeclareMathSymbol{\bigsqcupplusop}{\mathop}{largesymbolsPXA}{"02}
%\re@DeclareMathSymbol{\bigsqcapplusop}{\mathop}{largesymbolsPXA}{"04}
%\re@DeclareMathSymbol{\bigsqcapop}{\mathop}{largesymbolsPXA}{"06}
\re@DeclareMathSymbol{\bigtimesop}{\mathop}{largesymbolsPXA}{"10}
%\re@DeclareMathSymbol{\coprodop}{\mathop}{largesymbols}{"60}
%\re@DeclareMathSymbol{\varprod}{\mathop}{largesymbolsPXA}{16}
\makeatother
%%
%% With euler font cursive for Greek letters - the [1] means 100% scaling
\DeclareFontFamily{U}{egreek}{\skewchar\font'177}%
\DeclareFontShape{U}{egreek}{m}{n}{<-6>s*[1]eurm5 <6-8>s*[1]eurm7 <8->s*[1]eurm10}{}%
\DeclareFontShape{U}{egreek}{m}{it}{<->s*[1]eurmo10}{}%
\DeclareFontShape{U}{egreek}{b}{n}{<-6>s*[1]eurb5 <6-8>s*[1]eurb7 <8->s*[1]eurb10}{}%
\DeclareFontShape{U}{egreek}{b}{it}{<->s*[1]eurbo10}{}%
\DeclareSymbolFont{egreeki}{U}{egreek}{m}{it}%
\SetSymbolFont{egreeki}{bold}{U}{egreek}{b}{it}% from the amsfonts package
\DeclareSymbolFont{egreekr}{U}{egreek}{m}{n}%
\SetSymbolFont{egreekr}{bold}{U}{egreek}{b}{n}% from the amsfonts package
% Take also \sum, \prod, \coprod symbols from Euler fonts
\DeclareFontFamily{U}{egreekx}{\skewchar\font'177}
\DeclareFontShape{U}{egreekx}{m}{n}{%
       <-7.5>s*[0.9]euex7%
    <7.5-8.5>s*[0.9]euex8%
    <8.5-9.5>s*[0.9]euex9%
    <9.5->s*[0.9]euex10%
}{}
\DeclareSymbolFont{egreekx}{U}{egreekx}{m}{n}
\DeclareMathSymbol{\sumop}{\mathop}{egreekx}{"50}
\DeclareMathSymbol{\prodop}{\mathop}{egreekx}{"51}
\DeclareMathSymbol{\coprodop}{\mathop}{egreekx}{"60}
\makeatletter
\def\sum{\DOTSI\sumop\slimits@}
\def\prod{\DOTSI\prodop\slimits@}
\def\coprod{\DOTSI\coprodop\slimits@}
\makeatother
\input{definegreek.tex}% Greek letters not usually given in LaTeX.

%\usepackage%[scaled=0.9]%
%{classico}%  Optima as sans-serif font
\renewcommand\sfdefault{uop}
\DeclareMathAlphabet{\mathsf}  {T1}{\sfdefault}{m}{sl}
\SetMathAlphabet{\mathsf}{bold}{T1}{\sfdefault}{b}{sl}
\newcommand*{\mathte}[1]{\textbf{\textit{\textsf{#1}}}}
% Upright sans-serif math alphabet
% \DeclareMathAlphabet{\mathsu}  {T1}{\sfdefault}{m}{n}
% \SetMathAlphabet{\mathsu}{bold}{T1}{\sfdefault}{b}{n}

% DejaVu Mono as typewriter text
\usepackage[scaled=0.84]{DejaVuSansMono}

\usepackage{mathdots}

\usepackage[usenames]{xcolor}
% Tol (2012) colour-blind-, print-, screen-friendly colours, alternative scheme; Munsell terminology
\definecolor{mypurpleblue}{RGB}{68,119,170}
\definecolor{myblue}{RGB}{102,204,238}
\definecolor{mygreen}{RGB}{34,136,51}
\definecolor{myyellow}{RGB}{204,187,68}
\definecolor{myred}{RGB}{238,102,119}
\definecolor{myredpurple}{RGB}{170,51,119}
\definecolor{mygrey}{RGB}{187,187,187}
% Tol (2012) colour-blind-, print-, screen-friendly colours; Munsell terminology
% \definecolor{lbpurple}{RGB}{51,34,136}
% \definecolor{lblue}{RGB}{136,204,238}
% \definecolor{lbgreen}{RGB}{68,170,153}
% \definecolor{lgreen}{RGB}{17,119,51}
% \definecolor{lgyellow}{RGB}{153,153,51}
% \definecolor{lyellow}{RGB}{221,204,119}
% \definecolor{lred}{RGB}{204,102,119}
% \definecolor{lpred}{RGB}{136,34,85}
% \definecolor{lrpurple}{RGB}{170,68,153}
\definecolor{lgrey}{RGB}{221,221,221}
%\newcommand*\mycolourbox[1]{%
%\colorbox{mygrey}{\hspace{1em}#1\hspace{1em}}}
\colorlet{shadecolor}{lgrey}

\usepackage{bm}

\usepackage{microtype}

\usepackage[backend=biber,mcite,%subentry,
citestyle=authoryear-comp,bibstyle=pglpm-authoryear,autopunct=false,sorting=ny,sortcites=false,natbib=false,maxcitenames=1,maxbibnames=8,minbibnames=8,giveninits=true,uniquename=false,uniquelist=false,maxalphanames=1,block=space,hyperref=true,defernumbers=false,useprefix=true,sortupper=false,language=british,parentracker=false]{biblatex}
\DeclareSortingScheme{ny}{\sort{\field{sortname}\field{author}\field{editor}}\sort{\field{year}}}
\iffalse\makeatletter%%% replace parenthesis with brackets
\newrobustcmd*{\parentexttrack}[1]{%
  \begingroup
  \blx@blxinit
  \blx@setsfcodes
  \blx@bibopenparen#1\blx@bibcloseparen
  \endgroup}
\AtEveryCite{%
  \let\parentext=\parentexttrack%
  \let\bibopenparen=\bibopenbracket%
  \let\bibcloseparen=\bibclosebracket}
\makeatother\fi
\DefineBibliographyExtras{british}{\def\finalandcomma{\addcomma}}
\renewcommand*{\finalnamedelim}{\addcomma\space}
\setcounter{biburlnumpenalty}{1}
\setcounter{biburlucpenalty}{0}
\setcounter{biburllcpenalty}{1}
\DeclareDelimFormat{multicitedelim}{\addsemicolon\space}
\DeclareDelimFormat{compcitedelim}{\addsemicolon\space}
\DeclareDelimFormat{postnotedelim}{\space}
\ifarxiv\else\addbibresource{portamanabib.bib}\fi
\renewcommand{\bibfont}{\footnotesize}
%\appto{\citesetup}{\footnotesize}% smaller font for citations
\defbibheading{bibliography}[\bibname]{\section*{#1}\addcontentsline{toc}{section}{#1}%\markboth{#1}{#1}
}
\newcommand*{\citep}{\footcites}
\newcommand*{\citey}{\footcites}%{\parencites*}
%\renewcommand*{\cite}{\parencite}
%\renewcommand*{\cites}{\parencites}
\providecommand{\href}[2]{#2}
\providecommand{\eprint}[2]{\texttt{\href{#1}{#2}}}
\newcommand*{\amp}{\&}
% \newcommand*{\citein}[2][]{\textnormal{\textcite[#1]{#2}}%\addtocategory{extras}{#2}
% }
\newcommand*{\citein}[2][]{\textnormal{\textcite[#1]{#2}}%\addtocategory{extras}{#2}
}
\newcommand*{\citebi}[2][]{\textcite[#1]{#2}%\addtocategory{extras}{#2}
}
\newcommand*{\subtitleproc}[1]{}
\newcommand*{\chapb}{ch.}
%
% \def\arxivp{}
% \def\mparcp{}
% \def\philscip{}
% \def\biorxivp{}
% \newcommand*{\arxivsi}{\texttt{arXiv} eprints available at \url{http://arxiv.org/}.\\}
% \newcommand*{\mparcsi}{\texttt{mp\_arc} eprints available at \url{http://www.ma.utexas.edu/mp_arc/}.\\}
% \newcommand*{\philscisi}{\texttt{philsci} eprints available at \url{http://philsci-archive.pitt.edu/}.\\}
% \newcommand*{\biorxivsi}{\texttt{bioRxiv} eprints available at \url{http://biorxiv.org/}.\\}
\newcommand*{\arxiveprint}[1]{%\global\def\arxivp{\arxivsi}%\citeauthor{0arxivcite}\addtocategory{ifarchcit}{0arxivcite}%eprint
\texttt{\urlalt{https://arxiv.org/abs/#1}{arXiv:\hspace{0pt}#1}}%
%\texttt{\href{http://arxiv.org/abs/#1}{\protect\url{arXiv:#1}}}%
%\renewcommand{\arxivnote}{\texttt{arXiv} eprints available at \url{http://arxiv.org/}.}
}
\newcommand*{\haleprint}[1]{%\global\def\arxivp{\arxivsi}%\citeauthor{0arxivcite}\addtocategory{ifarchcit}{0arxivcite}%eprint
\texttt{\urlalt{https://hal.archives-ouvertes.fr/#1}{HAL:\hspace{0pt}#1}}%
%\texttt{\href{http://arxiv.org/abs/#1}{\protect\url{arXiv:#1}}}%
%\renewcommand{\arxivnote}{\texttt{arXiv} eprints available at \url{http://arxiv.org/}.}
}
\newcommand*{\mparceprint}[1]{%\global\def\mparcp{\mparcsi}%\citeauthor{0mparccite}\addtocategory{ifarchcit}{0mparccite}%eprint
\texttt{\urlalt{http://www.ma.utexas.edu/mp_arc-bin/mpa?yn=#1}{mp\_arc:\hspace{0pt}#1}}%
%\texttt{\href{http://www.ma.utexas.edu/mp_arc-bin/mpa?yn=#1}{\protect\url{mp_arc:#1}}}%
%\providecommand{\mparcnote}{\texttt{mp_arc} eprints available at \url{http://www.ma.utexas.edu/mp_arc/}.}
}
\newcommand*{\philscieprint}[1]{%\global\def\philscip{\philscisi}%\citeauthor{0philscicite}\addtocategory{ifarchcit}{0philscicite}%eprint
\texttt{\urlalt{http://philsci-archive.pitt.edu/archive/#1}{PhilSci:\hspace{0pt}#1}}%
%\texttt{\href{http://philsci-archive.pitt.edu/archive/#1}{\protect\url{PhilSci:#1}}}%
%\providecommand{\mparcnote}{\texttt{philsci} eprints available at \url{http://philsci-archive.pitt.edu/}.}
}
\newcommand*{\biorxiveprint}[1]{%\global\def\biorxivp{\biorxivsi}%\citeauthor{0arxivcite}\addtocategory{ifarchcit}{0arxivcite}%eprint
\texttt{\urlalt{https://doi.org/10.1101/#1}{bioRxiv doi:\hspace{0pt}10.1101/#1}}%
%\texttt{\href{http://arxiv.org/abs/#1}{\protect\url{arXiv:#1}}}%
%\renewcommand{\arxivnote}{\texttt{arXiv} eprints available at \url{http://arxiv.org/}.}
}
\newcommand*{\osfeprint}[1]{%
\texttt{\urlalt{https://doi.org/10.17605/osf.io/#1}{Open Science Framework doi:10.17605/osf.io/#1}}%
}

\usepackage{graphicx}

%\usepackage{wrapfig}

%\usepackage{tikz-cd}

\PassOptionsToPackage{hyphens}{url}\usepackage[hypertexnames=false]{hyperref}

\usepackage[depth=4]{bookmark}
\hypersetup{colorlinks=true,bookmarksnumbered,pdfborder={0 0 0.25},citebordercolor={0.2667 0.4667 0.6667},citecolor=mypurpleblue,linkbordercolor={0.6667 0.2 0.4667},linkcolor=myredpurple,urlbordercolor={0.1333 0.5333 0.2},urlcolor=mygreen,breaklinks=true,pdftitle={\pdftitle},pdfauthor={\pdfauthor}}
% \usepackage[vertfit=local]{breakurl}% only for arXiv
\providecommand*{\urlalt}{\href}

\usepackage[british]{datetime2}
\DTMnewdatestyle{mydate}%
{% definitions
\renewcommand*{\DTMdisplaydate}[4]{%
\number##3\ \DTMenglishmonthname{##2} ##1}%
\renewcommand*{\DTMDisplaydate}{\DTMdisplaydate}%
}
\DTMsetdatestyle{mydate}

%%%%%%%%%%%%%%%%%%%%%%%%%%%%%%%%%%%%%%%%%%%%%%%%%%%%%%%%%%%%%%%%%%%%%%%%%%%%
%%% Layout. I do not know on which kind of paper the reader will print the
%%% paper on (A4? letter? one-sided? double-sided?). So I choose A5, which
%%% provides a good layout for reading on screen and save paper if printed
%%% two pages per sheet. Average length line is 66 characters and page
%%% numbers are centred.
%%%%%%%%%%%%%%%%%%%%%%%%%%%%%%%%%%%%%%%%%%%%%%%%%%%%%%%%%%%%%%%%%%%%%%%%%%%%
\ifafour\setstocksize{297mm}{210mm}%{*}% A4
\else\setstocksize{210mm}{5.5in}%{*}% 210x139.7
\fi
\settrimmedsize{\stockheight}{\stockwidth}{*}
\setlxvchars[\normalfont] %313.3632pt for a 66-characters line
\setxlvchars[\normalfont]
\setlength{\trimtop}{0pt}
\setlength{\trimedge}{\stockwidth}
\addtolength{\trimedge}{-\paperwidth}
% The length of the normalsize alphabet is 133.05988pt - 10 pt = 26.1408pc
% The length of the normalsize alphabet is 159.6719pt - 12pt = 30.3586pc
% Bringhurst gives 32pc as boundary optimal with 69 ch per line
% The length of the normalsize alphabet is 191.60612pt - 14pt = 35.8634pc
\ifafour\settypeblocksize{*}{32pc}{1.618} % A4
%\setulmargins{*}{*}{1.667}%gives 5/3 margins % 2 or 1.667
\else\settypeblocksize{*}{26pc}{1.618}% nearer to a 66-line newpx and preserves GR
\fi
\setulmargins{*}{*}{1}%gives equal margins
\setlrmargins{*}{*}{*}
\setheadfoot{\onelineskip}{2.5\onelineskip}
\setheaderspaces{*}{2\onelineskip}{*}
\setmarginnotes{2ex}{10mm}{0pt}
\checkandfixthelayout[nearest]
\fixpdflayout
%%% End layout
%% this fixes missing white spaces
\pdfmapline{+dummy-space <dummy-space.pfb}\pdfinterwordspaceon%

%%% Sectioning
\newcommand*{\asudedication}[1]{%
{\par\centering\textit{#1}\par}}
\newenvironment{acknowledgements}{\section*{Thanks}\addcontentsline{toc}{section}{Thanks}}{\par}
\makeatletter\renewcommand{\appendix}{\par
  \bigskip{\centering
   \interlinepenalty \@M
   \normalfont
   \printchaptertitle{\sffamily\appendixpagename}\par}
  \setcounter{section}{0}%
  \gdef\@chapapp{\appendixname}%
  \gdef\thesection{\@Alph\c@section}%
  \anappendixtrue}\makeatother
\counterwithout{section}{chapter}
\setsecnumformat{\upshape\csname the#1\endcsname\quad}
\setsecheadstyle{\large\bfseries\sffamily%
\centering}
\setsubsecheadstyle{\bfseries\sffamily%
\raggedright}
%\setbeforesecskip{-1.5ex plus 1ex minus .2ex}% plus 1ex minus .2ex}
%\setaftersecskip{1.3ex plus .2ex }% plus 1ex minus .2ex}
%\setsubsubsecheadstyle{\bfseries\sffamily\slshape\raggedright}
%\setbeforesubsecskip{1.25ex plus 1ex minus .2ex }% plus 1ex minus .2ex}
%\setaftersubsecskip{-1em}%{-0.5ex plus .2ex}% plus 1ex minus .2ex}
\setsubsecindent{0pt}%0ex plus 1ex minus .2ex}
\setparaheadstyle{\bfseries\sffamily%
\raggedright}
\setcounter{secnumdepth}{2}
\setlength{\headwidth}{\textwidth}
\newcommand{\addchap}[1]{\chapter*[#1]{#1}\addcontentsline{toc}{chapter}{#1}}
\newcommand{\addsec}[1]{\section*{#1}\addcontentsline{toc}{section}{#1}}
\newcommand{\addsubsec}[1]{\subsection*{#1}\addcontentsline{toc}{subsection}{#1}}
\newcommand{\addpara}[1]{\paragraph*{#1.}\addcontentsline{toc}{subsubsection}{#1}}
\newcommand{\addparap}[1]{\paragraph*{#1}\addcontentsline{toc}{subsubsection}{#1}}

%%% Headers, footers, pagestyle
\copypagestyle{manaart}{plain}
\makeheadrule{manaart}{\headwidth}{0.5\normalrulethickness}
\makeoddhead{manaart}{%
{\footnotesize%\sffamily%
\scshape\headauthor}}{}{{\footnotesize\sffamily%
\headtitle}}
\makeoddfoot{manaart}{}{\thepage}{}
\newcommand*\autanet{\includegraphics[height=\heightof{M}]{autanet.pdf}}
\definecolor{mygray}{gray}{0.333}
\iftypodisclaim%
\ifafour\newcommand\addprintnote{\begin{picture}(0,0)%
\put(245,149){\makebox(0,0){\rotatebox{90}{\tiny\color{mygray}\textsf{This
            document is designed for screen reading and
            two-up printing on A4 or Letter paper}}}}%
\end{picture}}% A4
\else\newcommand\addprintnote{\begin{picture}(0,0)%
\put(176,112){\makebox(0,0){\rotatebox{90}{\tiny\color{mygray}\textsf{This
            document is designed for screen reading and
            two-up printing on A4 or Letter paper}}}}%
\end{picture}}\fi%afourtrue
\makeoddfoot{plain}{}{\makebox[0pt]{\thepage}\addprintnote}{}
\else
\makeoddfoot{plain}{}{\makebox[0pt]{\thepage}}{}
\fi%typodisclaimtrue
\makeoddhead{plain}{\scriptsize\reporthead}{}{}
% \copypagestyle{manainitial}{plain}
% \makeheadrule{manainitial}{\headwidth}{0.5\normalrulethickness}
% \makeoddhead{manainitial}{%
% \footnotesize\sffamily%
% \scshape\headauthor}{}{\footnotesize\sffamily%
% \headtitle}
% \makeoddfoot{manaart}{}{\thepage}{}

\pagestyle{manaart}

\setlength{\droptitle}{-3.9\onelineskip}
\pretitle{\begin{center}\LARGE\sffamily%
\bfseries}
\posttitle{\bigskip\end{center}}

\makeatletter\newcommand*{\atf}{\includegraphics[%trim=1pt 1pt 0pt 0pt,
totalheight=\heightof{@}]{atblack.png}}\makeatother
\providecommand{\affiliation}[1]{\textsl{\textsf{\footnotesize #1}}}
\providecommand{\epost}[1]{\texttt{\footnotesize\textless#1\textgreater}}
\providecommand{\email}[2]{\href{mailto:#1ZZ@#2 ((remove ZZ))}{#1\protect\atf#2}}

\preauthor{\vspace{-0.5\baselineskip}\begin{center}
\normalsize\sffamily%
\lineskip  0.5em}
\postauthor{\par\end{center}}
\predate{\DTMsetdatestyle{mydate}\begin{center}\footnotesize}
\postdate{\end{center}\vspace{-\medskipamount}}

\setfloatadjustment{figure}{\footnotesize}
\captiondelim{\quad}
\captionnamefont{\footnotesize\sffamily%
}
\captiontitlefont{\footnotesize}
%\firmlists*
\midsloppy
% handling orphan/widow lines, memman.pdf
% \clubpenalty=10000
% \widowpenalty=10000
% \raggedbottom
% Downes, memman.pdf
\clubpenalty=9996
\widowpenalty=9999
\brokenpenalty=4991
\predisplaypenalty=10000
\postdisplaypenalty=1549
\displaywidowpenalty=1602
\raggedbottom

\paragraphfootnotes\setlength{\footmarkwidth}{0em}
% \threecolumnfootnotes
%\setlength{\footmarksep}{0em}
\footmarkstyle{\textsuperscript{%\color{myred}
\scriptsize\bfseries#1}~}
%\footmarkstyle{\textsuperscript{\color{myred}\scriptsize\bfseries#1}~}
%\footmarkstyle{\textsuperscript{[#1]}~}

\selectlanguage{british}\frenchspacing

%%%%%%%%%%%%%%%%%%%%%%%%%%%%%%%%%%%%%%%%%%%%%%%%%%%%%%%%%%%%%%%%%%%%%%%%%%%%
%%% Paper's details
%%%%%%%%%%%%%%%%%%%%%%%%%%%%%%%%%%%%%%%%%%%%%%%%%%%%%%%%%%%%%%%%%%%%%%%%%%%%
\title{\propertitle}
\author{%
\hspace*{\stretch{1}}%
%% uncomment if additional authors present
% \parbox{0.5\linewidth}%\makebox[0pt][c]%
% {\protect\centering ***\\%
% \footnotesize\epost{\email{***}{***}}}%
% \hspace*{\stretch{1}}%
\parbox{0.5\linewidth}%\makebox[0pt][c]%
{\protect\centering P.G.L.  Porta Mana\\%
\footnotesize\epost{\email{pgl}{portamana.org}}}%
\hspace*{\stretch{1}}%
%\quad\href{https://orcid.org/0000-0002-6070-0784}{\protect\includegraphics[scale=0.16]{orcid_32x32.png}\textsc{orcid}:0000-0002-6070-0784}%
}

%\date{Draft of \today\ (first drafted \firstdraft)}
\date{\firstpublished; updated \updated}

%%%%%%%%%%%%%%%%%%%%%%%%%%%%%%%%%%%%%%%%%%%%%%%%%%%%%%%%%%%%%%%%%%%%%%%%%%%%
%%% Macros @@@
%%%%%%%%%%%%%%%%%%%%%%%%%%%%%%%%%%%%%%%%%%%%%%%%%%%%%%%%%%%%%%%%%%%%%%%%%%%%

% Common ones - uncomment as needed
%\providecommand{\nequiv}{\not\equiv}
%\providecommand{\coloneqq}{\mathrel{\mathop:}=}
%\providecommand{\eqqcolon}{=\mathrel{\mathop:}}
%\providecommand{\varprod}{\prod}
\newcommand*{\de}{\partialup}%partial diff
\newcommand*{\pu}{\piup}%constant pi
\newcommand*{\delt}{\deltaup}%Kronecker, Dirac
%\newcommand*{\eps}{\varepsilonup}%Levi-Civita, Heaviside
%\newcommand*{\riem}{\zetaup}%Riemann zeta
%\providecommand{\degree}{\textdegree}% degree
%\newcommand*{\celsius}{\textcelsius}% degree Celsius
%\newcommand*{\micro}{\textmu}% degree Celsius
\newcommand*{\I}{\mathrm{i}}%imaginary unit
\newcommand*{\e}{\mathrm{e}}%Neper
\newcommand*{\di}{\mathrm{d}}%differential
%\newcommand*{\Di}{\mathrm{D}}%capital differential
%\newcommand*{\planckc}{\hslash}
%\newcommand*{\avogn}{N_{\textrm{A}}}
%\newcommand*{\NN}{\bm{\mathrm{N}}}
%\newcommand*{\ZZ}{\bm{\mathrm{Z}}}
%\newcommand*{\QQ}{\bm{\mathrm{Q}}}
\newcommand*{\RR}{\bm{\mathrm{R}}}
%\newcommand*{\CC}{\bm{\mathrm{C}}}
%\newcommand*{\nabl}{\bm{\nabla}}%nabla
%\DeclareMathOperator{\lb}{lb}%base 2 log
\DeclareMathOperator{\tr}{tr}%trace
%\DeclareMathOperator{\card}{card}%cardinality
%\DeclareMathOperator{\im}{Im}%im part
%\DeclareMathOperator{\re}{Re}%re part
%\DeclareMathOperator{\sgn}{sgn}%signum
%\DeclareMathOperator{\ent}{ent}%integer less or equal to
%\DeclareMathOperator{\Ord}{O}%same order as
%\DeclareMathOperator{\ord}{o}%lower order than
%\newcommand*{\incr}{\triangle}%finite increment
\newcommand*{\defd}{\coloneqq}
\newcommand*{\defs}{\eqqcolon}
%\newcommand*{\Land}{\bigwedge}
%\newcommand*{\Lor}{\bigvee}
%\newcommand*{\lland}{\DOTSB\;\land\;}
%\newcommand*{\llor}{\DOTSB\;\lor\;}
%\newcommand*{\limplies}{\mathbin{\Rightarrow}}%implies
%\newcommand*{\suchthat}{\mid}%{\mathpunct{|}}%such that (eg in sets)
%\newcommand*{\with}{\colon}%with (list of indices)
%\newcommand*{\mul}{\times}%multiplication
%\newcommand*{\inn}{\cdot}%inner product
%\newcommand*{\dotv}{\mathord{\,\cdot\,}}%variable place
%\newcommand*{\comp}{\circ}%composition of functions
%\newcommand*{\con}{\mathbin{:}}%scal prod of tensors
%\newcommand*{\equi}{\sim}%equivalent to 
\renewcommand*{\asymp}{\simeq}%equivalent to 
%\newcommand*{\corr}{\mathrel{\hat{=}}}%corresponds to
%\providecommand{\varparallel}{\ensuremath{\mathbin{/\mkern-7mu/}}}%parallel (tentative symbol)
\renewcommand*{\le}{\leqslant}%less or equal
\renewcommand*{\ge}{\geqslant}%greater or equal
\DeclarePairedDelimiter\clcl{[}{]}
%\DeclarePairedDelimiter\clop{[}{[}
%\DeclarePairedDelimiter\opcl{]}{]}
%\DeclarePairedDelimiter\opop{]}{[}
\DeclarePairedDelimiter\abs{\lvert}{\rvert}
%\DeclarePairedDelimiter\norm{\lVert}{\rVert}
\DeclarePairedDelimiter\set{\{}{\}}
%\DeclareMathOperator{\pr}{P}%probability
\newcommand*{\pf}{\mathrm{p}}%probability
\newcommand*{\p}{\mathrm{P}}%probability
%\newcommand*{\E}{\mathrm{E}}
%\renewcommand*{\|}{\nonscript\,\vert\nonscript\;\mathopen{}}
\renewcommand*{\|}[1][]{\nonscript\,#1\vert\nonscript\;\mathopen{}}
%\DeclarePairedDelimiterX{\cond}[2]{(}{)}{#1\nonscript\,\delimsize\vert\nonscript\;\mathopen{}#2}
%\DeclarePairedDelimiterX{\condt}[2]{[}{]}{#1\nonscript\,\delimsize\vert\nonscript\;\mathopen{}#2}
%\DeclarePairedDelimiterX{\conds}[2]{\{}{\}}{#1\nonscript\,\delimsize\vert\nonscript\;\mathopen{}#2}
%\newcommand*{\+}{\lor}
%\renewcommand{\*}{\land}
\newcommand*{\sect}{\S}% Sect.~
\newcommand*{\sects}{\S\S}% Sect.~
\newcommand*{\chap}{ch.}%
\newcommand*{\chaps}{chs}%
\newcommand*{\bref}{ref.}%
\newcommand*{\brefs}{refs}%
%\newcommand*{\fn}{fn}%
\newcommand*{\eqn}{eq.}%
\newcommand*{\eqns}{eqs}%
\newcommand*{\fig}{fig.}%
\newcommand*{\figs}{figs}%
\newcommand*{\vs}{{vs}}
%\newcommand*{\etc}{{etc.}}
%\newcommand*{\ie}{{i.e.}}
%\newcommand*{\ca}{{c.}}
%\newcommand*{\eg}{{e.g.}}
\newcommand*{\foll}{{ff.}}
%\newcommand*{\viz}{{viz}}
\newcommand*{\cf}{{cf.}}
%\newcommand*{\Cf}{{Cf.}}
%\newcommand*{\vd}{{v.}}
\newcommand*{\etal}{{et al.}}
%\newcommand*{\etsim}{{et sim.}}
%\newcommand*{\ibid}{{ibid.}}
%\newcommand*{\sic}{{sic}}
%\newcommand*{\id}{\mathte{I}}%id matrix
%\newcommand*{\nbd}{\nobreakdash}%
%\newcommand*{\bd}{\hspace{0pt}}%
%\def\hy{-\penalty0\hskip0pt\relax}
%\newcommand*{\labelbis}[1]{\tag*{(\ref{#1})$_\text{r}$}}
%\newcommand*{\mathbox}[2][.8]{\parbox[t]{#1\columnwidth}{#2}}
%\newcommand*{\zerob}[1]{\makebox[0pt][l]{#1}}
\newcommand*{\tprod}{\mathop{\textstyle\prod}\nolimits}
\newcommand*{\tsum}{\mathop{\textstyle\sum}\nolimits}
\newcommand*{\tint}{\begingroup\textstyle\int\endgroup\nolimits}
%\newcommand*{\tland}{\mathop{\textstyle\bigwedge}\nolimits}
%\newcommand*{\tlor}{\mathop{\textstyle\bigvee}\nolimits}
%\newcommand*{\sprod}{\mathop{\textstyle\prod}}
%\newcommand*{\ssum}{\mathop{\textstyle\sum}}
%\newcommand*{\sint}{\begingroup\textstyle\int\endgroup}
%\newcommand*{\sland}{\mathop{\textstyle\bigwedge}}
%\newcommand*{\slor}{\mathop{\textstyle\bigvee}}
\newcommand*{\T}{^\intercal}%transpose
%%\newcommand*{\QEM}%{\textnormal{$\Box$}}%{\ding{167}}
%\newcommand*{\qem}{\leavevmode\unskip\penalty9999 \hbox{}\nobreak\hfill
%\quad\hbox{\QEM}}

%%%%%%%%%%%%%%%%%%%%%%%%%%%%%%%%%%%%%%%%%%%%%%%%%%%%%%%%%%%%%%%%%%%%%%%%%%%%
%%% Custom macros for this file @@@
%%%%%%%%%%%%%%%%%%%%%%%%%%%%%%%%%%%%%%%%%%%%%%%%%%%%%%%%%%%%%%%%%%%%%%%%%%%%
 \definecolor{notecolour}{RGB}{68,170,153}
\newcommand*{\puzzle}{{\fontencoding{U}\fontfamily{fontawesometwo}\selectfont\symbol{225}}}
%\newcommand*{\puzzle}{\maltese}
\newcommand{\mynote}[1]{ {\color{notecolour}\puzzle\ #1}}
\newcommand*{\widebar}[1]{{\mkern1.5mu\skew{2}\overline{\mkern-1.5mu#1\mkern-1.5mu}\mkern 1.5mu}}

% \newcommand{\explanation}[4][t]{%\setlength{\tabcolsep}{-1ex}
% %\smash{
% \begin{tabular}[#1]{c}#2\\[0.5\jot]\rule{1pt}{#3}\\#4\end{tabular}}%}
% \newcommand*{\ptext}[1]{\text{\small #1}}
%\DeclareMathOperator*{\argsup}{arg\,sup}
\newcommand*{\dob}{degree of belief}
\newcommand*{\dobs}{degrees of belief}
\newcommand*{\dime}{\clcl}
\newcommand*{\Le}{\textrm{L}}
\newcommand*{\Ti}{\textrm{T}}
\newcommand*{\Ma}{\textrm{M}}
\newcommand*{\Te}{\Theta}
\newcommand*{\Cu}{\textrm{I}}
\newcommand*{\Fl}{\Phi}
\newcommand*{\En}{\textrm{E}}
\newcommand*{\Xx}{\textrm{X}}
\newcommand*{\Yy}{\textrm{Y}}
\newcommand*{\Aa}{\textrm{A}}
\newcommand*{\Bb}{\textrm{B}}
\newcommand*{\Ss}{\textrm{S}}
\newcommand*{\Li}{\mathrm{L}}
\newcommand*{\ii}{\mathrm{i}}
%
\newcommand*{\yA}{\mathte{A}}
\newcommand*{\yB}{\mathte{B}}
\newcommand*{\yg}{\mathte{g}}
\newcommand*{\yT}{\mathte{T}}
\newcommand*{\yG}{\mathte{G}}
\newcommand*{\yR}{\mathte{R}}
\newcommand*{\yTa}{\bm{\varTau}}
\newcommand*{\yom}{\bm{\omega}}
\newcommand*{\yta}{\bm{\tau}}
\newcommand*{\yv}{\bm{v}}
\newcommand*{\yu}{\bm{u}}
\newcommand*{\yw}{\bm{w}}
\renewcommand*{\i}{\indices}
\newcommand*{\dex}[1][i]{\frac{\de}{\de x^{#1}}}
\newcommand*{\dix}[1][i]{\di x^{#1}}
\newcommand*{\nab}{\nabla}
\newcommand*{\yGa}{\varGamma}
%%% Custom macros end @@@

%%%%%%%%%%%%%%%%%%%%%%%%%%%%%%%%%%%%%%%%%%%%%%%%%%%%%%%%%%%%%%%%%%%%%%%%%%%%
%%% Beginning of document
%%%%%%%%%%%%%%%%%%%%%%%%%%%%%%%%%%%%%%%%%%%%%%%%%%%%%%%%%%%%%%%%%%%%%%%%%%%%
%\firmlists
\begin{document}
\captiondelim{\quad}\captionnamefont{\footnotesize}\captiontitlefont{\footnotesize}
\selectlanguage{british}\frenchspacing
\maketitle

%%%%%%%%%%%%%%%%%%%%%%%%%%%%%%%%%%%%%%%%%%%%%%%%%%%%%%%%%%%%%%%%%%%%%%%%%%%%
%%% Abstract
%%%%%%%%%%%%%%%%%%%%%%%%%%%%%%%%%%%%%%%%%%%%%%%%%%%%%%%%%%%%%%%%%%%%%%%%%%%%
\abstractrunin
\abslabeldelim{}
\renewcommand*{\abstractname}{}
\setlength{\absleftindent}{0pt}
\setlength{\absrightindent}{0pt}
\setlength{\abstitleskip}{-\absparindent}
\begin{abstract}\labelsep 0pt%
  \noindent Some notes on dimensional analysis on differential manifolds,
  with an eye on general relativity and the Einstein equation.
\\\noindent\emph{\footnotesize Note: Dear Reader
    \amp\ Peer, this manuscript is being peer-reviewed by you. Thank you.}
% \par%\\[\jot]
% \noindent
% {\footnotesize PACS: ***}\qquad%
% {\footnotesize MSC: ***}%
%\qquad{\footnotesize Keywords: ***}
\end{abstract}
\selectlanguage{british}\frenchspacing

%%%%%%%%%%%%%%%%%%%%%%%%%%%%%%%%%%%%%%%%%%%%%%%%%%%%%%%%%%%%%%%%%%%%%%%%%%%%
%%% Epigraph
%%%%%%%%%%%%%%%%%%%%%%%%%%%%%%%%%%%%%%%%%%%%%%%%%%%%%%%%%%%%%%%%%%%%%%%%%%%%
% \asudedication{\small ***}
% \vspace{\bigskipamount}
% \setlength{\epigraphwidth}{.7\columnwidth}
% %\epigraphposition{flushright}
% \epigraphtextposition{flushright}
% %\epigraphsourceposition{flushright}
% \epigraphfontsize{\footnotesize}
% \setlength{\epigraphrule}{0pt}
% %\setlength{\beforeepigraphskip}{0pt}
% %\setlength{\afterepigraphskip}{0pt}
% \epigraph{\emph{text}}{source}



%%%%%%%%%%%%%%%%%%%%%%%%%%%%%%%%%%%%%%%%%%%%%%%%%%%%%%%%%%%%%%%%%%%%%%%%%%%%
%%% BEGINNING OF MAIN TEXT
%%%%%%%%%%%%%%%%%%%%%%%%%%%%%%%%%%%%%%%%%%%%%%%%%%%%%%%%%%%%%%%%%%%%%%%%%%%%

\section{***}
\label{sec:***}

\enquote{Dimensional analysis remains a controversial and somewhat obscure
  subject. We do not attempt a complete presentation here.} \citep[Appendix
\sect~7 footnote~4]{truesdelletal1960}

\textcolor{white}{If you find this you can claim a postcard from me.}

Let's start from more general facts about dimensional analysis on differential manifolds.

For dimensional analysis I use ISO conventions and notation. I sometimes use notation such as $\yT{}_{\bullet}{}^{\bullet}$ to indicate that the tensor $\yT$ is covariant in its first slot and contravariant in its second; I call this a "co-contra-variant tensor".
\citep{aldersley1977}

\section{Coordinates}
\label{sec:coords}

From a physical point of view, a coordinate is just a function that
associates a value of a physical quantity with every event in a region (the
domain of the coordinate chart) of spacetime. Together with the other
coordinates, such function allows us to uniquely identify every event
within that region. Any physical quantity will do: the distance from
something, the time elapsed since something, an angle, an energy density,
the strength of a magnetic flux, a temperature, and so on. A coordinate can
thus have any dimensions: length $\Le$, time $\Ti$, angle $1$, energy
density $\En = \Le^{-1}\Ma\Ti^{-2}$, magnetic flux
$\Fl = \Le^{2}\Ma\Ti^{-2}\Cu^{-1}$, temperature $\Te$, and so on.

The dimensions of the coordinates don't matter, as we'll now see.

\section{Tensors}
\label{sec:tensors}

Consider a system of coordinates $(x^i)$ with dimensions $(\Xx_i)$. Each
coordinate $x^{i}$ begets a covector field (1-form) $\dix$, and the
coordinates together beget a set of dual vector fields
$\Bigl(\dex\Bigr)$. These covectors and vectors can be
used as bases for the cotangent and tangent spaces, and their tensor
products as bases for tensor tangent spaces of higher type.

The differential $\dix$ traditionally has the same dimension as $x^{i}$:
$\dim(\dix) = \Xx_{i}$, and the operator $\dex$ traditionally has the
inverse dimension: $\dim\dex = {\Xx_{i}}^{-1}$. We'll see later that
these conventions are self-consistent.

For our discussion let's now take a contra-co-tensor field
$\yA \equiv \yA\i{^{\bullet}_{\bullet}}$; the discussion generalizes to
tensors of other types in an obvious way.

The tensor $\yA$ can be expanded in terms of the basis vectors and covectors:
\begin{equation}
  \label{eq:expansion_tensor}
  \yA = A\i{^{i}_{j}}\, \dex\otimes\dix[j]
  \equiv A\i{^{0}_{0}}\,\dex[0]\otimes\dix[0] + 
  A\i{^{0}_{1}}\,\dex[0]\otimes\dix[1] + \dotsb {},
\end{equation}
where each
\begin{equation}
  A\i{^{i}_{j}} \defd  \yA\Bigl(\dix, \dex[j]\Bigr)
  \label{eq:components_def}
\end{equation}
is a component of the tensor in this coordinate system.

To make sense dimensionally, every term the sum~\eqref{eq:expansion_tensor}
must have the same dimension. This is possible only if the generic
component $A\i{^{i}_{j}}$ has dimension
\begin{equation}
  \label{eq:dim_component}
  \dim(A\i{^{i}_{j}}) = \Aa\;{\Xx_{i}}\,{\Xx_{j}}^{-1},
\end{equation}
where $\Aa$ is common to all components and is also the dimension of the
sum~\eqref{eq:expansion_tensor}. For example, suppose we're using
coordinates with dimensions
\begin{equation}
  \label{eq:example_coords}
  \dim(x^{0})=\Te,\quad
  \dim(x^{1})=\Le,\quad
  \dim(x^{2})=\Le,\quad
  \dim(x^{3})=\Le^{-1}\Ma\Ti^{-2};
\end{equation}
then the components of $\yA$ have dimensions
\begin{equation}
  \label{eq:example_components}
  \Bigl(\dim(A\i{^{i}_{j}})\Bigr) =
 \Aa\; \begin{pmatrix}
   1 & \Le^{-1}\Te & \Le^{-1}\Te & \Le\Ma^{-1}\Ti^{2}\Te
   \\
   \Le\Te^{-1} & 1 & 1 & \Le^{2}\Ma^{-1}\Ti^{2}
   \\
   \Le\Te^{-1} & 1 & 1 & \Le^{2}\Ma^{-1}\Ti^{2}
   \\
   \Le^{-1}\Ma\Ti^{-2}\Te^{-1} & \Le^{-2}\Ma\Ti^{-2} & \Le^{-2}\Ma\Ti^{-2} & 1
  \end{pmatrix}.
\end{equation}

The dimension $\Aa$ is called the \emph{absolute dimension}
\citep{dorgeloetal1946}[\chap~VI]{schouten1951_r1989} of the tensor $\yA$.
This is the intrinsic dimension of the tensor, independently of any
coordinate system, so we write
\begin{equation}
  \label{eq:abs_dim}
  \dim(\yA) = \Aa.
\end{equation}



Different coordinate systems simply lead to different dimensions of the
components of $\yA$. Formula~\eqref{eq:dim_component} for the dimensions of
the components is consistent under changes of coordinates. For example,
in coordinates  $({x'}^{k})$ with dimensions $(\Xx'_{k})$, the components
of $\yA$ are
\begin{equation}
  \label{eq:coords_change}
  {A'}\i{^{k}_{l}} = A\i{^{i}_{j}}
  \,\frac{\de {x'}^{k}}{\de x^{i}}
  \,\frac{\de {x}^{j}}{\de {x'}^{l}}
\end{equation}
and a quick check shows that
$\dim({A'}\i{^{k}_{l}}) = \Aa\,{\Xx'_{k}}\,{\Xx'_{l}}^{-1}$, consistent
with~\eqref{eq:dim_component}.

\section{Tensor operations}
\label{sec:tensor_ops}

By the reasoning of the previous section it's easy to see how various
tensor and tensor-field operations affect the absolute dimension of their
arguments. For example, the tensor product of
$\yA\i{^{\bullet}_{\bullet}}$ and $\yB\i{_{\bullet\bullet}^{\bullet}}$ can
be written as the sum
\begin{equation}
  \label{eq:tensor_prod_example}
  \yA \otimes \yB =
  A\i{^{i}_{j}}\,B\i{_{kl}^{m}}\;
  \dex[i]\otimes\dix[j]\otimes\dix[k]\otimes\dix[l]\otimes\dex[m]
\end{equation}
from which it follows that
\begin{equation}
  \label{eq:dim_comp_tensor_prod}
  \dim(A\i{^{i}_{j}}\,B\i{_{kl}^{m}}) =
  \Aa\,\Bb\;
  \Xx_{i}\,{\Xx_{j}}^{-1}\,{\Xx_{k}}^{-1}\,{\Xx_{l}}^{-1}\,\Xx_{m}
\end{equation}
with $\Aa = \dim(\yA)$ and $\Bb = \dim(\yB)$. The absolute dimension of
$\yA\otimes\yB$ is therefore $\Aa\Bb \equiv \dim(\yA)\,\dim(\yB)$.

I'll drop the adjective \enquote{absolute} when it's clear from
the context.

\begin{itemize}[wide=0pt]
\item \emph{Tensor multiplication} multiplies dimensions:
  \begin{equation}
  \dim(\yA\otimes\yB) = \dim(\yA)\dim(\yB).\label{eq:tensor_mult}
\end{equation}

\item The \emph{contraction} of the $i$th and $j$th slots (one covariant
  and one contravariant) of a tensor has the same dimension as the tensor:
  \begin{equation}
    \dim(\tr_{ij}\yA) = \dim(\yA).
    \label{eq:tensor_contr}
  \end{equation}
  Note that this only holds \emph{without} raising or lowering indices.

\item The \emph{transposition} of the $i$th and $j$th slots of a tensor has
  the same dimension as the tensor:
  \begin{equation}
    \dim(\yA^{\intercal_{ij}}) = \dim(\yA).
    \label{eq:tensor_transp}
  \end{equation}

\item The \emph{Lie bracket} of two vectors has the product of their dimensions:
  \begin{equation}
    \dim(\clcl{\yu,\yv}) =\dim(\yu)\dim(\yv).
    \label{eq:lie_der}
\end{equation}

% \item The \emph{pull-back} and \emph{push-forward} of a map $F$ between
%   manifolds don't change the dimensions of the tensors they map.
%   \begin{equation}
%   \dim(\yA\otimes\yB) = \dim(\yA)\dim(\yB).\label{eq:tensor_mult}
% \end{equation}

\item The \emph{Lie derivative} of a tensor with respect to a vector field
  has the product of the dimensions of the tensor and of the vector:
  \begin{equation}
    \dim(\Li_{\yv}\yA) =\dim(\yv)\dim(\yA).
    \label{eq:lie_der}
\end{equation}
\end{itemize}

\medskip

Regarding operations with differential forms:

\begin{itemize}[wide=0pt]
\item The \emph{exterior product} of two differential forms multiplies
  their dimensions:
  \begin{equation}
  \dim(\yom\land\yta) = \dim(\yom)\dim(\yta).\label{eq:ext_prod}
\end{equation}
  
\item The \emph{interior product} of a vector and a form multiplies their
  dimensions:
  \begin{equation}
    \dim(\ii_{\yv}\yom) =\dim(\yv)\dim(\yom).
    \label{eq:inter_prod}
\end{equation}

\item The \emph{exterior derivative} of a form has the same dimension of
  the form:
  \begin{equation}
    \dim(\di\yom) =\dim(\yom).
    \label{eq:ext_deriv}
  \end{equation}
  This can be proven using the identity
  $\di\,\ii_{\yv}+\ii_{\yv}\,\di = \Li_{\yv}$ or similar identities
  \citep[\chap~9 p.~180 Theorem~9.78]{curtisetal1985} together with
  \eqns~\eqref{eq:lie_der} and~\eqref{eq:inter_prod}.

\item The \emph{integral} of a form over a submanifold has the same dimension as
  the form:
  \begin{equation}
    \dim\bigl(\tint_{c}\yom\bigr) =\dim(\yom).
    \label{eq:integration}
  \end{equation}
\end{itemize}

\section{Connection, covariant derivative, curvature tensors}
\label{sec:connection}

Consider an arbitrary connection with covariant derivative $\nab$. For the
moment we don't assume the presence of any metric structure.

The \emph{covariant derivative} of the product $f\yv$ of a function and a vector satisfies
\citep[\sect~V.B.1 p.~300]{choquetbruhatetal1977_r1996}
\begin{equation}
  \label{eq:basic_property_covder}
  \nab(f\yv) = \di f \otimes \yv + f\nab\yv.
\end{equation}
The first summand has dimension $\dim(f)\dim(\yv)$, which must also be
the dimension of the second summand. Thus we see that
\begin{equation}
  \label{eq:dim_cov_der_vect}
  \dim(\nab\yv) = \dim(\yv).
\end{equation}
It follows that the \emph{directional covariant derivative} has dimension
\begin{equation}
  \label{eq:dim_dircov_der_vect}
  \dim(\nab_{\yu}\yv) = \dim(\yu)\dim(\yv),
\end{equation}
and by its derivation properties \citep[\sect~V.B.1
p.~303]{choquetbruhatetal1977_r1996} we see that
formula~\eqref{eq:dim_cov_der_vect} extends from vectors to 
tensors of arbitrary type.

In the coordinate system $(x^{i})$, the action of the covariant derivative
is carried by the \emph{connection coefficients} or Christoffel symbols
$(\yGa\i{^{i}_{jk}})$ defined by
\begin{equation}
  \label{eq:christoffel}
  \nab\dex[k] = \yGa\i{^{i}_{jk}}\; \dix[j]\otimes\dex[i].
\end{equation}
From this equation and the previous ones it follows that the coefficients
have dimensions
\begin{equation}
  \label{eq:dim_christoffel}
  \dim(\yGa\i{^{i}_{jk}}) = \Xx_{i}\, {\Xx_{j}}^{-1}\,{\Xx_{k}}^{-1}.
\end{equation}

The \emph{torsion} $\yTa\i{^{\bullet}_{\bullet\bullet}}$, \emph{Riemann
  curvature} $\yR\i{^{\bullet}_{\bullet\bullet\bullet}}$, and \emph{Ricci
  curvature} $\yR\i{_{\bullet\bullet}}$ tensors are
defined by
\begin{gather}
  \label{eq:torsion_eq}
\yTa(\yu,\yv) \defd \nab_{\yu}\yv - \nab_{\yv}\yu - \clcl{\yu,\yv},
\\
\yR(\yu,\yv)\yw \defd
\nab_{\yu}\nab_{\yv}\yw - \nab_{\yv}\nab_{\yu}\yw
- \nab_{\clcl{\yu,\yv}}\yw,
\\
\yR\i{_{\bullet\bullet}} \defd \tr_{13} \yR\i{^{\bullet}_{\bullet\bullet\bullet}}.
  \label{eq:riemann_eq}  
\end{gather}
From these definitions, formula~\eqref{eq:dim_cov_der_vect}, and the
dimensional effects of tensor product~\eqref{eq:tensor_mult} and
contraction~\eqref{eq:tensor_contr} we see that
\begin{equation}
  \label{eq:dim_torsion_riemann_ricci}
  \dim(\yTa\i{^{\bullet}_{\bullet\bullet}}) =
  % \label{eq:dim_riem}
   \dim(\yR\i{^{\bullet}_{\bullet\bullet\bullet}}) =
  % \label{eq:dim_ricci}
   \dim(\yR\i{_{\bullet\bullet}}) =1.
\end{equation}
The exact contra- and co-variant type used above for these tensors is very
important in these equations. If we raise any of their indices using a
metric, their dimensions will generally change.

\medskip

The formulae above are also valid if a metric is defined and the connection
is compatible with it. The connection coefficients in this case are defined
in terms of the metric tensor, but using the results of \sect*** it's easy
to see that \eqns~\eqref{eq:dim_cov_der_vect},
\eqref{eq:dim_dircov_der_vect}, \eqref{eq:dim_christoffel},
\eqref{eq:dim_torsion_riemann_ricci} still hold.

\section{Curves and integral curves}
\label{sec:curves}


Consider a curve into spacetime, $c\colon s \mapsto P(s)$, with the
parameter $s$ having dimension $\dim(s)=\Ss$.***


If we consider the manifold
as "adimensional" (if this makes sense), then the dimensions of the tangent
vector $\dot{c}$ to the curve are $\dim(\dot{c}) = [\mathrm{S}^{-1}]$. This
follows either from
$\dot{c} := \partial x^i[c(s)]/\partial s\; \partial_{x^i}$, or considering
that $\dot{c}$ can be interpreted as the push-forward of $\partial_s$, that
is, $c_*(\partial_s)$.

This has an interesting, quirky implication. Given a vector field $P\mapsto\yv(P)$ we say that $c$ is an integral curve for it if
$$
\yv[c(s)] = \dot{c}(s).
$$
But this equation is only valid if $\yv$ has dimensions $[\mathrm{S}^{-1}]$. For the general case a constant dimensional factor needs to be introduced in the equation above.

\section{Metric tensor}
\label{sec:metric}


From the above discussion we see that the component $g_{ij}$ of the metric $\yg$ has dimensions $[\mathrm{Z}\,\mathrm{X}_i\,{\mathrm{X}_j}^{-1}\,{\mathrm{X}_k}^{-1}]$, where $[\mathrm{Z}]$ are the absolute dimensions of the metric. What are these absolute dimensions?

The answer probably depends on how you see the operational meaning of the metric. Here I offer my personal point of view. We can use the metric to measure the "length" of (timelike or spacelike) paths in spacetime. The "length" of a path $c(s)$ with $s\in [a,b]$ is
$$
\int_a^b\!\!\!\mathrm{d}s\;
\sqrt{\Bigl\lvert g_{ij}[c(s)]\;\dot{c}^i(s)\,\dot{c}^j(s) \Bigr\rvert}.
$$
We see that this "length" has dimensions $[\mathrm{Z}^{1/2}]$ and not unexpectedly it doesn't depend on the dimensions of the curve parameter $s$.

If the path is timelike, this "length" can be measured by a clock having that path as worldline – it's its proper time. Thus, for me $[\mathrm{Z}^{1/2}] = [\mathrm{T}]$, a time, and therefore the absolute dimensions of the metric tensor are time squared:
$$\dim(\yg)=[\mathrm{T}^2].$$

I believe that these dimensions also make sense for spacelike paths: in this case we would have to measure the "length" by dividing it in very small pieces and using radar coordinates on each piece. So we're measuring the "length" by checking clocks, to see how long it takes for the light to bounce back: time $[\mathrm{T}]$, again.

By our usual argument it's possible to see that the Riemann curvature
tensor $\yR{}^{\bullet}{}_{\bullet\bullet\bullet}$, the Ricci tensor
$\yR_{\bullet\bullet}$, and the Einstein tensor $\yG_{\bullet\bullet}$ are
adimensional – $[1]$ – and the scalar curvature has dimensions
$[\mathrm{T}^{-2}]$. Note that the Riemann and Ricci tensors (with the
contra/co-variant type specified above) do not require a metric for their
definition, but an affine connection. They are adimensional no matter what
dimensions we give the metric. By construction the (fully co-variant)
Einstein tensor is always adimensional, too.

An important operation done with the metric:

 - "lowering an index" of a tensor multiplies its dimensions by $[\mathrm{T}^2]$, and "rising an index" multiplies them by $[\mathrm{T}^{-2}]$ (if you agree with my discussion above).
 
 
** Stress-energy-momentum tensor **

What are the absolute dimensions of the co-contra-variant stress-energy-momentum tensor $\yT{}_{\bullet}{}^{\bullet}$? We must look for an operational meaning here too. I'll try to  sketch an informal argument that reflects my point of view. The argument can be made more rigorous but that would take too long to do here.

The dynamics equation $\nabla\cdot\yT=0$ holds in general-relativistic (thermo)mechanics, and also in Newtonian (thermo)mechanics when no body forces and no body heating are present. In Newtonian mechanics it's the formal combination of the balances of momentum density and energy density – which incidentally have the same dimensions $[\mathrm{M}\,\mathrm{L}^{-1}\,\mathrm{T}^{-3}]$, energy/(volume × time). 

The divergence of the stress-energy-momentum gives us a 4-force density, just like the 3-divergence of the stress gives us a force density. Please check Misner \amp al (1973), chap. 14, for a very interesting discussion of these matters, and also Eckart (1940) and Burke (1980, 1987).

Further, the 4-force is an object that, integrated over a path, gives us an energy density (cf Milne 1951 chap. IV, and Burke again). The integral of a force in Newtonian mechanics is the work done by the force. In general-relativistic mechanics, the timelike component of the 4-force additionally gives us the increase in energy owing to heating (Eckart 1940).

So $\nabla\cdot\yT\equiv T{}_i{}^j{}_{;\,j}\;\mathrm{d}x^i$ has the dimensions of energy density, $[\mathrm{M}\,\mathrm{L}^{-1}\mathrm{T}^{-2}]$. The *co-contra-variant* stress-energy-momentum $\yT{}_{\bullet}{}^{\bullet}$ has therefore the same dimensions. But the *co-co-variant* tensor, obtained by contraction with the metric, $\yT_{\bullet\bullet} \equiv \yT\cdot \yg$, has dimensions of energy density times squared time: $[\mathrm{M}\,\mathrm{L}^{-1}]$, a mass over length.

Einstein's constant $\kappa$ therefore relates a dimensionless quantity and a mass over length:
$$\yG_{\bullet\bullet} = \kappa \yT_{\bullet\bullet}\;.$$
Its dimension must  be $[\mathrm{M}^{-1}\,\mathrm{L}]$, and it's easily seen that these are the dimensions of $G/c^2$. So I'm one of those people (like Fock 1964 p. 199) who define
$$
\kappa = 8\pi G/c^2.
$$


** References **

 - Burke (1980): *Spacetime, Geometry, Cosmology* (University Science Books)
 - Burke (1987): *Applied Differential Geometry* (Cambridge)
 - [Eckart (1940)]: *The thermodynamics of irreversible processes. III. Relativistic theory of the simple fluid*, Phys. Rev. 58, 919.
 - Fock (1964): *The Theory of Space, Time and Gravitation* (Pergamon)
 - Misner, Thorne, Wheeler (1973): *Gravitation* (Freeman)
 - Schouten (1989): *Tensor Analysis for Physicists* (Dover, 2nd ed.)


%%%% examples use empheq
%   \begin{empheq}[left={\mathllap{\begin{aligned}    \de\yF_{\yc}/\de\yp&=0\text{:} \\
%         \de\yF_{\yc}/\de\ym&=0\text{:}\\ \de\yF_{\yc}/\de\yl&=0\text{:}\end{aligned}}\qquad}\empheqlbrace]{align}
%     \label{eq:con_p}
% %    \de\yF_{\yc}/\de\yp &\equiv
%     -\ln\yp + \ln\yq + \yl\yM + \ym\yu &=0,\\
%     \label{eq:con_u}
% %    \de\yF_{\yc}/\de\ym &\equiv
%     \yu\yp-1 &=0,\\
%     \label{eq:con_l}
%     %\de\yF_{\yc}/\de\yl &\equiv
%     \yM\yp-\yc &=0.
%   \end{empheq}
%%%%
% \begin{empheq}[box=\widefbox]{equation}
%   \label{eq:maxent_question}
%   \p\bigl[\yE{N+1}{k} \bigcond \tsum\yo\yf{N}\in\yA, \yM\bigr] = \mathord{?}
% \end{empheq}



% \[
%   \begin{tikzcd}
%       M_{n,n}(\CC) \arrow{r}{R'_{a}(\Hat{U})} & M_{n,n}(\CC)
%     \\
%     L(\mathcal{H}) \arrow{r}{\Hat{U}} \arrow[swap]{d}{R_*}\arrow[swap]{u}{R'_*} & L(\mathcal{H}) \arrow{d}{R_*}\arrow{u}{R'_*} \\
%       M_{n,n}(\CC) \arrow{r}{R_{a}(\Hat{U})} & M_{n,n}(\CC)
%   \end{tikzcd}
% \]

% \[
%   \begin{tikzcd}
%       \CC^n \arrow{r}{R'_*(A)} & \CC^n
%     \\
%     \mathcal{H} \arrow{r}{A} \arrow[swap]{d}{R}\arrow[swap]{u}{R'} & \mathcal{H} \arrow{d}{R}\arrow{u}{R'} \\
%       \CC^n \arrow{r}{R_*(A)} & \CC^n
%   \end{tikzcd}
% \]


% \[
%   \begin{tikzcd}
%     \mathcal{H} \arrow{r}{A} \arrow[swap]{d}{R} & \mathcal{H} \arrow{d}{R} \\
%       \CC^n \arrow{r}{R_*(A)} & \CC^n
%   \end{tikzcd}
% \]

%%\setlength{\intextsep}{0.5ex}% with wrapfigure
%\begin{figure}[p!]%{r}{0.4\linewidth} % with wrapfigure
%  \centering\includegraphics[trim={12ex 0 18ex 0},clip,width=\linewidth]{maxent_saddle.png}\\
%\caption{caption}\label{fig:comparison_a5}
%\end{figure}% exp_family_maxent.nb


%%%%%%%%%%%%%%%%%%%%%%%%%%%%%%%%%%%%%%%%%%%%%%%%%%%%%%%%%%%%%%%%%%%%%%%%%%%%
%%% Acknowledgements
%%%%%%%%%%%%%%%%%%%%%%%%%%%%%%%%%%%%%%%%%%%%%%%%%%%%%%%%%%%%%%%%%%%%%%%%%%%% 
\iffalse
\begin{acknowledgements}
  \ldots to Mari \amp\ Miri for continuous encouragement and affection, and
  to Buster Keaton and Saitama for filling life with awe and inspiration.
  To the developers and maintainers of \LaTeX, Emacs, AUC\TeX, Open Science
  Framework, R, Python, Inkscape, Sci-Hub for making a free and impartial
  scientific exchange possible.
%\rotatebox{15}{P}\rotatebox{5}{I}\rotatebox{-10}{P}\rotatebox{10}{\reflectbox{P}}\rotatebox{-5}{O}.
%\sourceatright{\autanet}
\mbox{}\hfill\autanet
\end{acknowledgements}
\fi

%%%%%%%%%%%%%%%%%%%%%%%%%%%%%%%%%%%%%%%%%%%%%%%%%%%%%%%%%%%%%%%%%%%%%%%%%%%%
%%% Appendices
%%%%%%%%%%%%%%%%%%%%%%%%%%%%%%%%%%%%%%%%%%%%%%%%%%%%%%%%%%%%%%%%%%%%%%%%%%%% 
\clearpage
% %\renewcommand*{\appendixpagename}{Appendix}
% %\renewcommand*{\appendixname}{Appendix}
% %\appendixpage
% \appendix

%%%%%%%%%%%%%%%%%%%%%%%%%%%%%%%%%%%%%%%%%%%%%%%%%%%%%%%%%%%%%%%%%%%%%%%%%%%%
%%% Bibliography
%%%%%%%%%%%%%%%%%%%%%%%%%%%%%%%%%%%%%%%%%%%%%%%%%%%%%%%%%%%%%%%%%%%%%%%%%%%% 
\defbibnote{prenote}{{\footnotesize (\enquote{de $X$} is listed under D,
    \enquote{van $X$} under V, and so on, regardless of national
    conventions.)\par}}
% \defbibnote{postnote}{\par\medskip\noindent{\footnotesize% Note:
%     \arxivp \mparcp \philscip \biorxivp}}

\printbibliography[prenote=prenote%,postnote=postnote
]

\end{document}

%%%%%%%%%%%%%%%%%%%%%%%%%%%%%%%%%%%%%%%%%%%%%%%%%%%%%%%%%%%%%%%%%%%%%%%%%%%%
%%% Cut text (won't be compiled)
%%%%%%%%%%%%%%%%%%%%%%%%%%%%%%%%%%%%%%%%%%%%%%%%%%%%%%%%%%%%%%%%%%%%%%%%%%%% 


%%% Local Variables: 
%%% mode: LaTeX
%%% TeX-PDF-mode: t
%%% TeX-master: t
%%% End: 
