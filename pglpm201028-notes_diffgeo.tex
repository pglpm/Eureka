\pdfoutput=1
%% Author: PGL  Porta Mana
%% Created: 2015-05-01T20:53:34+0200
%% Last-Updated: 2022-01-25T09:58:31+0100
%%%%%%%%%%%%%%%%%%%%%%%%%%%%%%%%%%%%%%%%%%%%%%%%%%%%%%%%%%%%%%%%%%%%%%%%%%%%
\newif\ifarxiv
\arxivfalse
\ifarxiv\pdfmapfile{+classico.map}\fi
\newif\ifafour
\afourfalse% true = A4, false = A5
\newif\iftypodisclaim % typographical disclaim on the side
\typodisclaimtrue
\newcommand*{\memfontfamily}{zplx}
\newcommand*{\memfontpack}{newpxtext}
\documentclass[\ifafour a4paper,12pt,\else a5paper,10pt,\fi%extrafontsizes,%
onecolumn,oneside,article,%french,italian,german,swedish,latin,
british%
]{memoir}
\newcommand*{\firstdraft}{28 October 2020}
\newcommand*{\firstpublished}{\firstdraft}
\newcommand*{\updated}{\ifarxiv***\else\today\fi}
\newcommand*{\propertitle}{Notes on multivector algebra\\ on differential manifolds%\\{\large ***}%
}% title uses LARGE; set Large for smaller
\newcommand*{\pdftitle}{\propertitle}
\newcommand*{\headtitle}{Multivector algebra}
\newcommand*{\pdfauthor}{P.G.L.  Porta Mana}
\newcommand*{\headauthor}{Porta Mana}
\newcommand*{\reporthead}{\iftrue\else Open Science Framework \href{https://doi.org/10.31219/osf.io/***}{\textsc{doi}:10.31219/osf.io/***}\fi}% Report number

%%%%%%%%%%%%%%%%%%%%%%%%%%%%%%%%%%%%%%%%%%%%%%%%%%%%%%%%%%%%%%%%%%%%%%%%%%%%
%%% Calls to packages (uncomment as needed)
%%%%%%%%%%%%%%%%%%%%%%%%%%%%%%%%%%%%%%%%%%%%%%%%%%%%%%%%%%%%%%%%%%%%%%%%%%%%

%\usepackage{pifont}

%\usepackage{fontawesome}

\usepackage[T1]{fontenc} 
\input{glyphtounicode} \pdfgentounicode=1

\usepackage[utf8]{inputenx}

%\usepackage{newunicodechar}
% \newunicodechar{Ĕ}{\u{E}}
% \newunicodechar{ĕ}{\u{e}}
% \newunicodechar{Ĭ}{\u{I}}
% \newunicodechar{ĭ}{\u{\i}}
% \newunicodechar{Ŏ}{\u{O}}
% \newunicodechar{ŏ}{\u{o}}
% \newunicodechar{Ŭ}{\u{U}}
% \newunicodechar{ŭ}{\u{u}}
% \newunicodechar{Ā}{\=A}
% \newunicodechar{ā}{\=a}
% \newunicodechar{Ē}{\=E}
% \newunicodechar{ē}{\=e}
% \newunicodechar{Ī}{\=I}
% \newunicodechar{ī}{\={\i}}
% \newunicodechar{Ō}{\=O}
% \newunicodechar{ō}{\=o}
% \newunicodechar{Ū}{\=U}
% \newunicodechar{ū}{\=u}
% \newunicodechar{Ȳ}{\=Y}
% \newunicodechar{ȳ}{\=y}

\newcommand*{\bmmax}{0} % reduce number of bold fonts, before font packages
\newcommand*{\hmmax}{0} % reduce number of heavy fonts, before font packages

\usepackage{textcomp}

%\usepackage[normalem]{ulem}% package for underlining
% \makeatletter
% \def\ssout{\bgroup \ULdepth=-.35ex%\UL@setULdepth
%  \markoverwith{\lower\ULdepth\hbox
%    {\kern-.03em\vbox{\hrule width.2em\kern1.2\p@\hrule}\kern-.03em}}%
%  \ULon}
% \makeatother

\usepackage{amsmath}

\usepackage{mathtools}
%\addtolength{\jot}{\jot} % increase spacing in multiline formulae
\setlength{\multlinegap}{0pt}

%\usepackage{empheq}% automatically calls amsmath and mathtools
%\newcommand*{\widefbox}[1]{\fbox{\hspace{1em}#1\hspace{1em}}}

%%%% empheq above seems more versatile than these:
%\usepackage{fancybox}
%\usepackage{framed}

% \usepackage[misc]{ifsym} % for dice
% \newcommand*{\diceone}{{\scriptsize\Cube{1}}}

\usepackage{amssymb}

\usepackage{amsxtra}

\usepackage{tensor}

\usepackage[main=british]{babel}\selectlanguage{british}
%\newcommand*{\langnohyph}{\foreignlanguage{nohyphenation}}
\newcommand{\langnohyph}[1]{\begin{hyphenrules}{nohyphenation}#1\end{hyphenrules}}

\usepackage[autostyle=false,autopunct=false,english=british]{csquotes}
\setquotestyle{american}
\newcommand*{\defquote}[1]{`\,#1\,'}

% \makeatletter
% \renewenvironment{quotation}%
%                {\list{}{\listparindent 1.5em%
%                         \itemindent    \listparindent
%                         \rightmargin=1em   \leftmargin=1em
%                         \parsep        \z@ \@plus\p@}%
%                 \item[]\footnotesize}%
%                 {\endlist}
% \makeatother                


\usepackage{amsthm}
%% from https://tex.stackexchange.com/a/404680/97039
\makeatletter
\def\@endtheorem{\endtrivlist}
\makeatother

\newcommand*{\QED}{\textsc{q.e.d.}}
\renewcommand*{\qedsymbol}{\QED}
\theoremstyle{remark}
\newtheorem{note}{Note}
\newtheorem*{remark}{Note}
\newtheoremstyle{innote}{\parsep}{\parsep}{\footnotesize}{}{}{}{0pt}{}
\theoremstyle{innote}
\newtheorem*{innote}{}

\usepackage[shortlabels,inline]{enumitem}
\SetEnumitemKey{para}{itemindent=\parindent,leftmargin=0pt,listparindent=\parindent,parsep=0pt,itemsep=\topsep}
% \begin{asparaenum} = \begin{enumerate}[para]
% \begin{inparaenum} = \begin{enumerate*}
\setlist{itemsep=0pt,topsep=\parsep}
\setlist[enumerate,2]{label=\alph*.}
\setlist[enumerate]{label=\arabic*.,leftmargin=1.5\parindent}
\setlist[itemize]{leftmargin=1.5\parindent}
\setlist[description]{leftmargin=1.5\parindent}
% old alternative:
% \setlist[enumerate,2]{label=\alph*.}
% \setlist[enumerate]{leftmargin=\parindent}
% \setlist[itemize]{leftmargin=\parindent}
% \setlist[description]{leftmargin=\parindent}

\usepackage[babel,theoremfont,largesc]{newpxtext}

\usepackage[bigdelims,nosymbolsc%,smallerops % probably arXiv doesn't have it
]{newpxmath}
%\useosf
%\linespread{1.083}%
%\linespread{1.05}% widely used
\linespread{1.1}% best for text with maths
%% smaller operators for old version of newpxmath
\makeatletter
\def\re@DeclareMathSymbol#1#2#3#4{%
    \let#1=\undefined
    \DeclareMathSymbol{#1}{#2}{#3}{#4}}
%\re@DeclareMathSymbol{\bigsqcupop}{\mathop}{largesymbols}{"46}
%\re@DeclareMathSymbol{\bigodotop}{\mathop}{largesymbols}{"4A}
\re@DeclareMathSymbol{\bigoplusop}{\mathop}{largesymbols}{"4C}
\re@DeclareMathSymbol{\bigotimesop}{\mathop}{largesymbols}{"4E}
\re@DeclareMathSymbol{\sumop}{\mathop}{largesymbols}{"50}
\re@DeclareMathSymbol{\prodop}{\mathop}{largesymbols}{"51}
\re@DeclareMathSymbol{\bigcupop}{\mathop}{largesymbols}{"53}
\re@DeclareMathSymbol{\bigcapop}{\mathop}{largesymbols}{"54}
%\re@DeclareMathSymbol{\biguplusop}{\mathop}{largesymbols}{"55}
\re@DeclareMathSymbol{\bigwedgeop}{\mathop}{largesymbols}{"56}
\re@DeclareMathSymbol{\bigveeop}{\mathop}{largesymbols}{"57}
%\re@DeclareMathSymbol{\bigcupdotop}{\mathop}{largesymbols}{"DF}
%\re@DeclareMathSymbol{\bigcapplusop}{\mathop}{largesymbolsPXA}{"00}
%\re@DeclareMathSymbol{\bigsqcupplusop}{\mathop}{largesymbolsPXA}{"02}
%\re@DeclareMathSymbol{\bigsqcapplusop}{\mathop}{largesymbolsPXA}{"04}
%\re@DeclareMathSymbol{\bigsqcapop}{\mathop}{largesymbolsPXA}{"06}
\re@DeclareMathSymbol{\bigtimesop}{\mathop}{largesymbolsPXA}{"10}
%\re@DeclareMathSymbol{\coprodop}{\mathop}{largesymbols}{"60}
%\re@DeclareMathSymbol{\varprod}{\mathop}{largesymbolsPXA}{16}
\makeatother
%%
%% With euler font cursive for Greek letters - the [1] means 100% scaling
\DeclareFontFamily{U}{egreek}{\skewchar\font'177}%
\DeclareFontShape{U}{egreek}{m}{n}{<-6>s*[1]eurm5 <6-8>s*[1]eurm7 <8->s*[1]eurm10}{}%
\DeclareFontShape{U}{egreek}{m}{it}{<->s*[1]eurmo10}{}%
\DeclareFontShape{U}{egreek}{b}{n}{<-6>s*[1]eurb5 <6-8>s*[1]eurb7 <8->s*[1]eurb10}{}%
\DeclareFontShape{U}{egreek}{b}{it}{<->s*[1]eurbo10}{}%
\DeclareSymbolFont{egreeki}{U}{egreek}{m}{it}%
\SetSymbolFont{egreeki}{bold}{U}{egreek}{b}{it}% from the amsfonts package
\DeclareSymbolFont{egreekr}{U}{egreek}{m}{n}%
\SetSymbolFont{egreekr}{bold}{U}{egreek}{b}{n}% from the amsfonts package
% Take also \sum, \prod, \coprod symbols from Euler fonts
\DeclareFontFamily{U}{egreekx}{\skewchar\font'177}
\DeclareFontShape{U}{egreekx}{m}{n}{%
       <-7.5>s*[0.9]euex7%
    <7.5-8.5>s*[0.9]euex8%
    <8.5-9.5>s*[0.9]euex9%
    <9.5->s*[0.9]euex10%
}{}
\DeclareSymbolFont{egreekx}{U}{egreekx}{m}{n}
\DeclareMathSymbol{\sumop}{\mathop}{egreekx}{"50}
\DeclareMathSymbol{\prodop}{\mathop}{egreekx}{"51}
\DeclareMathSymbol{\coprodop}{\mathop}{egreekx}{"60}
\makeatletter
\def\sum{\DOTSI\sumop\slimits@}
\def\prod{\DOTSI\prodop\slimits@}
\def\coprod{\DOTSI\coprodop\slimits@}
\makeatother
\input{definegreek.tex}% Greek letters not usually given in LaTeX.

%\usepackage%[scaled=0.9]%
%{classico}%  Optima as sans-serif font
\renewcommand\sfdefault{uop}
\DeclareMathAlphabet{\mathsf}  {T1}{\sfdefault}{m}{sl}
\SetMathAlphabet{\mathsf}{bold}{T1}{\sfdefault}{b}{sl}
%\newcommand*{\mathte}[1]{\textbf{\textit{\textsf{#1}}}}
% Upright sans-serif math alphabet
% \DeclareMathAlphabet{\mathsu}  {T1}{\sfdefault}{m}{n}
% \SetMathAlphabet{\mathsu}{bold}{T1}{\sfdefault}{b}{n}

% DejaVu Mono as typewriter text
\usepackage[scaled=0.84]{DejaVuSansMono}

\usepackage{mathdots}

\usepackage[usenames]{xcolor}
% Tol (2012) colour-blind-, print-, screen-friendly colours, alternative scheme; Munsell terminology
\definecolor{mypurpleblue}{RGB}{68,119,170}
\definecolor{myblue}{RGB}{102,204,238}
\definecolor{mygreen}{RGB}{34,136,51}
\definecolor{myyellow}{RGB}{204,187,68}
\definecolor{myred}{RGB}{238,102,119}
\definecolor{myredpurple}{RGB}{170,51,119}
\definecolor{mygrey}{RGB}{187,187,187}
% Tol (2012) colour-blind-, print-, screen-friendly colours; Munsell terminology
% \definecolor{lbpurple}{RGB}{51,34,136}
% \definecolor{lblue}{RGB}{136,204,238}
% \definecolor{lbgreen}{RGB}{68,170,153}
% \definecolor{lgreen}{RGB}{17,119,51}
% \definecolor{lgyellow}{RGB}{153,153,51}
% \definecolor{lyellow}{RGB}{221,204,119}
% \definecolor{lred}{RGB}{204,102,119}
% \definecolor{lpred}{RGB}{136,34,85}
% \definecolor{lrpurple}{RGB}{170,68,153}
\definecolor{lgrey}{RGB}{221,221,221}
%\newcommand*\mycolourbox[1]{%
%\colorbox{mygrey}{\hspace{1em}#1\hspace{1em}}}
\colorlet{shadecolor}{lgrey}

\usepackage{bm}

\usepackage{microtype}

\usepackage[backend=biber,mcite,%subentry,
citestyle=authoryear-comp,bibstyle=pglpm-authoryear,autopunct=false,sorting=ny,sortcites=false,natbib=false,maxcitenames=2,maxbibnames=8,minbibnames=8,giveninits=true,uniquename=false,uniquelist=false,maxalphanames=1,block=space,hyperref=true,defernumbers=false,useprefix=true,sortupper=false,language=british,parentracker=false]{biblatex}
\DeclareSortingTemplate{ny}{\sort{\field{sortname}\field{author}\field{editor}}\sort{\field{year}}}
\iffalse\makeatletter%%% replace parenthesis with brackets
\newrobustcmd*{\parentexttrack}[1]{%
  \begingroup
  \blx@blxinit
  \blx@setsfcodes
  \blx@bibopenparen#1\blx@bibcloseparen
  \endgroup}
\AtEveryCite{%
  \let\parentext=\parentexttrack%
  \let\bibopenparen=\bibopenbracket%
  \let\bibcloseparen=\bibclosebracket}
\makeatother\fi
\DefineBibliographyExtras{british}{\def\finalandcomma{\addcomma}}
\renewcommand*{\finalnamedelim}{\addspace\amp\space}
% \renewcommand*{\finalnamedelim}{\addcomma\space}
\renewcommand*{\textcitedelim}{\addcomma\space}
% \setcounter{biburlnumpenalty}{1} % to allow url breaks anywhere
% \setcounter{biburlucpenalty}{0}
% \setcounter{biburllcpenalty}{1}
\DeclareDelimFormat{multicitedelim}{\addsemicolon\addspace\space}
\DeclareDelimFormat{compcitedelim}{\addsemicolon\addspace\space}
\DeclareDelimFormat{postnotedelim}{\addspace}
\ifarxiv\else\addbibresource{portamanabib.bib}\fi
\renewcommand{\bibfont}{\footnotesize}
%\appto{\citesetup}{\footnotesize}% smaller font for citations
\defbibheading{bibliography}[\bibname]{\section*{#1}\addcontentsline{toc}{section}{#1}%\markboth{#1}{#1}
}
\newcommand*{\citep}{\footcites}
\newcommand*{\citey}{\footcites}%{\parencites*}
\newcommand*{\ibid}{\unspace\addtocounter{footnote}{-1}\footnotemark{}}
%\renewcommand*{\cite}{\parencite}
%\renewcommand*{\cites}{\parencites}
\providecommand{\href}[2]{#2}
\providecommand{\eprint}[2]{\texttt{\href{#1}{#2}}}
\newcommand*{\amp}{\&}
% \newcommand*{\citein}[2][]{\textnormal{\textcite[#1]{#2}}%\addtocategory{extras}{#2}
% }
\newcommand*{\citein}[2][]{\textnormal{\textcite[#1]{#2}}%\addtocategory{extras}{#2}
}
\newcommand*{\citebi}[2][]{\textcite[#1]{#2}%\addtocategory{extras}{#2}
}
\newcommand*{\subtitleproc}[1]{}
\newcommand*{\chapb}{ch.}
%
%\def\UrlOrds{\do\*\do\-\do\~\do\'\do\"\do\-}%
\def\myUrlOrds{\do\0\do\1\do\2\do\3\do\4\do\5\do\6\do\7\do\8\do\9\do\a\do\b\do\c\do\d\do\e\do\f\do\g\do\h\do\i\do\j\do\k\do\l\do\m\do\n\do\o\do\p\do\q\do\r\do\s\do\t\do\u\do\v\do\w\do\x\do\y\do\z\do\A\do\B\do\C\do\D\do\E\do\F\do\G\do\H\do\I\do\J\do\K\do\L\do\M\do\N\do\O\do\P\do\Q\do\R\do\S\do\T\do\U\do\V\do\W\do\X\do\Y\do\Z}%
\makeatletter
%\g@addto@macro\UrlSpecials{\do={\newline}}
\g@addto@macro{\UrlBreaks}{\myUrlOrds}
\makeatother
\newcommand*{\arxiveprint}[1]{%
\href{https://arxiv.org/abs/#1}{arXiv:\allowbreak\nolinkurl{#1}}%
}
\newcommand*{\mparceprint}[1]{%
\href{http://www.ma.utexas.edu/mp_arc-bin/mpa?yn=#1}{mp\_arc:\allowbreak\nolinkurl{#1}}%
}
\newcommand*{\haleprint}[1]{%
\href{https://hal.archives-ouvertes.fr/#1}{\textsc{hal}:\allowbreak\nolinkurl{#1}}%
}
\newcommand*{\philscieprint}[1]{%
\href{http://philsci-archive.pitt.edu/archive/#1}{PhilSci:\allowbreak\nolinkurl{#1}}%
}
\newcommand*{\doi}[1]{%
\href{https://doi.org/#1}{\textsc{doi}:\allowbreak\nolinkurl{#1}}%
}
\newcommand*{\biorxiveprint}[1]{%
bioRxiv \doi{10.1101/#1}%
}
\newcommand*{\osfeprint}[1]{%
Open Science Framework \doi{10.31219/osf.io/#1}%
}

\usepackage{graphicx}

%\usepackage{wrapfig}

%\usepackage{tikz-cd}

\PassOptionsToPackage{hyphens}{url}\usepackage[hypertexnames=false,pdfencoding=unicode,psdextra]{hyperref}

\usepackage[depth=4]{bookmark}
\hypersetup{colorlinks=true,bookmarksnumbered,pdfborder={0 0 0.25},citebordercolor={0.2667 0.4667 0.6667},citecolor=mypurpleblue,linkbordercolor={0.6667 0.2 0.4667},linkcolor=myredpurple,urlbordercolor={0.1333 0.5333 0.2},urlcolor=mygreen,breaklinks=true,pdftitle={\pdftitle},pdfauthor={\pdfauthor}}
% \usepackage[vertfit=local]{breakurl}% only for arXiv
\providecommand*{\urlalt}{\href}

\usepackage[british]{datetime2}
\DTMnewdatestyle{mydate}%
{% definitions
\renewcommand*{\DTMdisplaydate}[4]{%
\number##3\ \DTMenglishmonthname{##2} ##1}%
\renewcommand*{\DTMDisplaydate}{\DTMdisplaydate}%
}
\DTMsetdatestyle{mydate}

%%%%%%%%%%%%%%%%%%%%%%%%%%%%%%%%%%%%%%%%%%%%%%%%%%%%%%%%%%%%%%%%%%%%%%%%%%%%
%%% Layout. I do not know on which kind of paper the reader will print the
%%% paper on (A4? letter? one-sided? double-sided?). So I choose A5, which
%%% provides a good layout for reading on screen and save paper if printed
%%% two pages per sheet. Average length line is 66 characters and page
%%% numbers are centred.
%%%%%%%%%%%%%%%%%%%%%%%%%%%%%%%%%%%%%%%%%%%%%%%%%%%%%%%%%%%%%%%%%%%%%%%%%%%%
\ifafour\setstocksize{297mm}{210mm}%{*}% A4
\else\setstocksize{210mm}{5.5in}%{*}% 210x139.7
\fi
\settrimmedsize{\stockheight}{\stockwidth}{*}
\setlxvchars[\normalfont] %313.3632pt for a 66-characters line
\setxlvchars[\normalfont]
% \setlength{\trimtop}{0pt}
% \setlength{\trimedge}{\stockwidth}
% \addtolength{\trimedge}{-\paperwidth}
%\settrims{0pt}{0pt}
% The length of the normalsize alphabet is 133.05988pt - 10 pt = 26.1408pc
% The length of the normalsize alphabet is 159.6719pt - 12pt = 30.3586pc
% Bringhurst gives 32pc as boundary optimal with 69 ch per line
% The length of the normalsize alphabet is 191.60612pt - 14pt = 35.8634pc
\ifafour\settypeblocksize{*}{32pc}{1.618} % A4
%\setulmargins{*}{*}{1.667}%gives 5/3 margins % 2 or 1.667
\else\settypeblocksize{*}{26pc}{1.618}% nearer to a 66-line newpx and preserves GR
\fi
\setulmargins{*}{*}{1}%gives equal margins
\setlrmargins{*}{*}{*}
\setheadfoot{\onelineskip}{2.5\onelineskip}
\setheaderspaces{*}{2\onelineskip}{*}
\setmarginnotes{2ex}{10mm}{0pt}
\checkandfixthelayout[nearest]
%%% End layout
%% this fixes missing white spaces
%\pdfmapline{+dummy-space <dummy-space.pfb}
%\pdfinterwordspaceon% seems to add a white margin to Sumatrapdf

%%% Sectioning
\newcommand*{\asudedication}[1]{%
{\par\centering\textit{#1}\par}}
\newenvironment{acknowledgements}{\section*{Thanks}\addcontentsline{toc}{section}{Thanks}}{\par}
\makeatletter\renewcommand{\appendix}{\par
  \bigskip{\centering
   \interlinepenalty \@M
   \normalfont
   \printchaptertitle{\sffamily\appendixpagename}\par}
  \setcounter{section}{0}%
  \gdef\@chapapp{\appendixname}%
  \gdef\thesection{\@Alph\c@section}%
  \anappendixtrue}\makeatother
\counterwithout{section}{chapter}
\setsecnumformat{\upshape\csname the#1\endcsname\quad}
\setsecheadstyle{\large\bfseries\sffamily%
\centering}
\setsubsecheadstyle{\bfseries\sffamily%
\raggedright}
%\setbeforesecskip{-1.5ex plus 1ex minus .2ex}% plus 1ex minus .2ex}
%\setaftersecskip{1.3ex plus .2ex }% plus 1ex minus .2ex}
%\setsubsubsecheadstyle{\bfseries\sffamily\slshape\raggedright}
%\setbeforesubsecskip{1.25ex plus 1ex minus .2ex }% plus 1ex minus .2ex}
%\setaftersubsecskip{-1em}%{-0.5ex plus .2ex}% plus 1ex minus .2ex}
\setsubsecindent{0pt}%0ex plus 1ex minus .2ex}
\setparaheadstyle{\bfseries\sffamily%
\raggedright}
\setcounter{secnumdepth}{2}
\setlength{\headwidth}{\textwidth}
\newcommand{\addchap}[1]{\chapter*[#1]{#1}\addcontentsline{toc}{chapter}{#1}}
\newcommand{\addsec}[1]{\section*{#1}\addcontentsline{toc}{section}{#1}}
\newcommand{\addsubsec}[1]{\subsection*{#1}\addcontentsline{toc}{subsection}{#1}}
\newcommand{\addpara}[1]{\paragraph*{#1.}\addcontentsline{toc}{subsubsection}{#1}}
\newcommand{\addparap}[1]{\paragraph*{#1}\addcontentsline{toc}{subsubsection}{#1}}

%%% Headers, footers, pagestyle
\copypagestyle{manaart}{plain}
\makeheadrule{manaart}{\headwidth}{0.5\normalrulethickness}
\makeoddhead{manaart}{%
{\footnotesize%\sffamily%
\scshape\headauthor}}{}{{\footnotesize\sffamily%
\headtitle}}
\makeoddfoot{manaart}{}{\thepage}{}
\newcommand*\autanet{\includegraphics[height=\heightof{M}]{autanet.pdf}}
\definecolor{mygray}{gray}{0.333}
\iftypodisclaim%
\ifafour\newcommand\addprintnote{\begin{picture}(0,0)%
\put(245,149){\makebox(0,0){\rotatebox{90}{\tiny\color{mygray}\textsf{This
            document is designed for screen reading and
            two-up printing on A4 or Letter paper}}}}%
\end{picture}}% A4
\else\newcommand\addprintnote{\begin{picture}(0,0)%
\put(176,112){\makebox(0,0){\rotatebox{90}{\tiny\color{mygray}\textsf{This
            document is designed for screen reading and
            two-up printing on A4 or Letter paper}}}}%
\end{picture}}\fi%afourtrue
\makeoddfoot{plain}{}{\makebox[0pt]{\thepage}\addprintnote}{}
\else
\makeoddfoot{plain}{}{\makebox[0pt]{\thepage}}{}
\fi%typodisclaimtrue
\makeoddhead{plain}{\scriptsize\reporthead}{}{}
% \copypagestyle{manainitial}{plain}
% \makeheadrule{manainitial}{\headwidth}{0.5\normalrulethickness}
% \makeoddhead{manainitial}{%
% \footnotesize\sffamily%
% \scshape\headauthor}{}{\footnotesize\sffamily%
% \headtitle}
% \makeoddfoot{manaart}{}{\thepage}{}

\pagestyle{manaart}

\setlength{\droptitle}{-3.9\onelineskip}
\pretitle{\begin{center}\LARGE\sffamily%
\bfseries}
\posttitle{\bigskip\end{center}}

\makeatletter\newcommand*{\atf}{\includegraphics[totalheight=\heightof{@}]{atblack.png}}\makeatother
\providecommand{\affiliation}[1]{\textsl{\textsf{\footnotesize #1}}}
\providecommand{\epost}[1]{\texttt{\footnotesize\textless#1\textgreater}}
\providecommand{\email}[2]{\href{mailto:#1ZZ@#2 ((remove ZZ))}{#1\protect\atf#2}}
%\providecommand{\email}[2]{\href{mailto:#1@#2}{#1@#2}}

\preauthor{\vspace{-0.5\baselineskip}\begin{center}
\normalsize\sffamily%
\lineskip  0.5em}
\postauthor{\par\end{center}}
\predate{\DTMsetdatestyle{mydate}\begin{center}\footnotesize}
\postdate{\end{center}\vspace{-\medskipamount}}

\setfloatadjustment{figure}{\footnotesize}
\captiondelim{\quad}
\captionnamefont{\footnotesize\sffamily%
}
\captiontitlefont{\footnotesize}
%\firmlists*
\midsloppy
% handling orphan/widow lines, memman.pdf
% \clubpenalty=10000
% \widowpenalty=10000
% \raggedbottom
% Downes, memman.pdf
\clubpenalty=9996
\widowpenalty=9999
\brokenpenalty=4991
\predisplaypenalty=10000
\postdisplaypenalty=1549
\displaywidowpenalty=1602
\raggedbottom

\paragraphfootnotes
\setlength{\footmarkwidth}{2ex}
% \threecolumnfootnotes
%\setlength{\footmarksep}{0em}
\footmarkstyle{\textsuperscript{%\color{myred}
\scriptsize\bfseries#1}~}
%\footmarkstyle{\textsuperscript{\color{myred}\scriptsize\bfseries#1}~}
%\footmarkstyle{\textsuperscript{[#1]}~}

\selectlanguage{british}\frenchspacing

%%%%%%%%%%%%%%%%%%%%%%%%%%%%%%%%%%%%%%%%%%%%%%%%%%%%%%%%%%%%%%%%%%%%%%%%%%%%
%%% Paper's details
%%%%%%%%%%%%%%%%%%%%%%%%%%%%%%%%%%%%%%%%%%%%%%%%%%%%%%%%%%%%%%%%%%%%%%%%%%%%
\title{\propertitle}
\author{%
\hspace*{\stretch{1}}%
%% uncomment if additional authors present
% \parbox{0.5\linewidth}%\makebox[0pt][c]%
% {\protect\centering ***\\%
% \footnotesize\epost{\email{***}{***}}}%
% \hspace*{\stretch{1}}%
\parbox{0.75\linewidth}%\makebox[0pt][c]%
{\protect\centering P.G.L.  Porta Mana  \href{https://orcid.org/0000-0002-6070-0784}{\protect\includegraphics[scale=0.16]{orcid_32x32.png}}\\%
\footnotesize\epost{\email{pgl}{portamana.org}}}%
%% uncomment if additional authors present
% \hspace*{\stretch{1}}%
% \parbox{0.5\linewidth}%\makebox[0pt][c]%
% {\protect\centering ***\\%
% \footnotesize\epost{\email{***}{***}}}%
\hspace*{\stretch{1}}%
}

%\date{Draft of \today\ (first drafted \firstdraft)}
\date{\firstpublished; updated \updated}

%%%%%%%%%%%%%%%%%%%%%%%%%%%%%%%%%%%%%%%%%%%%%%%%%%%%%%%%%%%%%%%%%%%%%%%%%%%%
%%% Macros @@@
%%%%%%%%%%%%%%%%%%%%%%%%%%%%%%%%%%%%%%%%%%%%%%%%%%%%%%%%%%%%%%%%%%%%%%%%%%%%

% Common ones - uncomment as needed
%\providecommand{\nequiv}{\not\equiv}
%\providecommand{\coloneqq}{\mathrel{\mathop:}=}
%\providecommand{\eqqcolon}{=\mathrel{\mathop:}}
%\providecommand{\varprod}{\prod}
\newcommand*{\de}{\partialup}%partial diff
\newcommand*{\pu}{\piup}%constant pi
\newcommand*{\delt}{\deltaup}%Kronecker, Dirac
%\newcommand*{\eps}{\varepsilonup}%Levi-Civita, Heaviside
%\newcommand*{\riem}{\zetaup}%Riemann zeta
%\providecommand{\degree}{\textdegree}% degree
%\newcommand*{\celsius}{\textcelsius}% degree Celsius
%\newcommand*{\micro}{\textmu}% degree Celsius
\newcommand*{\I}{\mathrm{i}}%imaginary unit
\newcommand*{\e}{\mathrm{e}}%Neper
\newcommand*{\di}{\mathrm{d}}%differential
%\newcommand*{\Di}{\mathrm{D}}%capital differential
%\newcommand*{\planckc}{\hslash}
%\newcommand*{\avogn}{N_{\textrm{A}}}
%\newcommand*{\NN}{\bm{\mathrm{N}}}
%\newcommand*{\ZZ}{\bm{\mathrm{Z}}}
%\newcommand*{\QQ}{\bm{\mathrm{Q}}}
\newcommand*{\RR}{\bm{\mathrm{R}}}
%\newcommand*{\CC}{\bm{\mathrm{C}}}
%\newcommand*{\nabl}{\bm{\nabla}}%nabla
%\DeclareMathOperator{\lb}{lb}%base 2 log
\DeclareMathOperator{\tr}{tr}%trace
%\DeclareMathOperator{\card}{card}%cardinality
%\DeclareMathOperator{\im}{Im}%im part
%\DeclareMathOperator{\re}{Re}%re part
%\DeclareMathOperator{\sgn}{sgn}%signum
%\DeclareMathOperator{\ent}{ent}%integer less or equal to
%\DeclareMathOperator{\Ord}{O}%same order as
%\DeclareMathOperator{\ord}{o}%lower order than
%\newcommand*{\incr}{\triangle}%finite increment
\newcommand*{\defd}{\coloneqq}
\newcommand*{\defs}{\eqqcolon}
%\newcommand*{\Land}{\bigwedge}
%\newcommand*{\Lor}{\bigvee}
%\newcommand*{\lland}{\DOTSB\;\land\;}
%\newcommand*{\llor}{\DOTSB\;\lor\;}
%\newcommand*{\limplies}{\mathbin{\Rightarrow}}%implies
%\newcommand*{\suchthat}{\mid}%{\mathpunct{|}}%such that (eg in sets)
%\newcommand*{\with}{\colon}%with (list of indices)
%\newcommand*{\mul}{\times}%multiplication
%\newcommand*{\inn}{\cdot}%inner product
%\newcommand*{\dotv}{\mathord{\,\cdot\,}}%variable place
%\newcommand*{\comp}{\circ}%composition of functions
%\newcommand*{\con}{\mathbin{:}}%scal prod of tensors
%\newcommand*{\equi}{\sim}%equivalent to 
\renewcommand*{\asymp}{\simeq}%equivalent to 
%\newcommand*{\corr}{\mathrel{\hat{=}}}%corresponds to
%\providecommand{\varparallel}{\ensuremath{\mathbin{/\mkern-7mu/}}}%parallel (tentative symbol)
\renewcommand*{\le}{\leqslant}%less or equal
\renewcommand*{\ge}{\geqslant}%greater or equal
%\DeclarePairedDelimiter\clcl{[}{]}
%\DeclarePairedDelimiter\clop{[}{[}
%\DeclarePairedDelimiter\opcl{]}{]}
%\DeclarePairedDelimiter\opop{]}{[}
\DeclarePairedDelimiter\abs{\lvert}{\rvert}
%\DeclarePairedDelimiter\norm{\lVert}{\rVert}
\DeclarePairedDelimiter\set{\{}{\}} %}
%\DeclareMathOperator{\pr}{P}%probability
\newcommand*{\p}{\mathrm{p}}%probability
\renewcommand*{\P}{\mathrm{P}}%probability
%\newcommand*{\E}{\mathrm{E}}
%% The "\:" space is chosen to correctly separate inner binary and external rels
\renewcommand*{\|}[1][]{\nonscript\:#1\vert\nonscript\:\mathopen{}}
%\DeclarePairedDelimiterX{\cp}[2]{(}{)}{#1\nonscript\:\delimsize\vert\nonscript\:\mathopen{}#2}
%\DeclarePairedDelimiterX{\ct}[2]{[}{]}{#1\nonscript\;\delimsize\vert\nonscript\:\mathopen{}#2}
%\DeclarePairedDelimiterX{\cs}[2]{\{}{\}}{#1\nonscript\:\delimsize\vert\nonscript\:\mathopen{}#2}
%\newcommand*{\+}{\lor}
%\renewcommand{\*}{\land}
%% symbol = for equality statements within probabilities
%% from https://tex.stackexchange.com/a/484142/97039
% \newcommand*{\eq}{\mathrel{\!=\!}}
% \let\texteq\=
% \renewcommand*{\=}{\TextOrMath\texteq\eq}
% \newcommand*{\eq}[1][=]{\mathrel{\!#1\!}}
\newcommand*{\mo}[1][=]{\mathord{\,#1\,}}
%%
\newcommand*{\sect}{\S}% Sect.~
\newcommand*{\sects}{\S\S}% Sect.~
\newcommand*{\chap}{ch.}%
\newcommand*{\chaps}{chs}%
\newcommand*{\bref}{ref.}%
\newcommand*{\brefs}{refs}%
%\newcommand*{\fn}{fn}%
\newcommand*{\eqn}{eq.}%
\newcommand*{\eqns}{eqs}%
\newcommand*{\fig}{fig.}%
\newcommand*{\figs}{figs}%
\newcommand*{\vs}{{vs}}
\newcommand*{\eg}{{e.g.}}
\newcommand*{\etc}{{etc.}}
\newcommand*{\ie}{{i.e.}}
%\newcommand*{\ca}{{c.}}
\newcommand*{\foll}{{ff.}}
%\newcommand*{\viz}{{viz}}
\newcommand*{\cf}{{cf.}}
%\newcommand*{\Cf}{{Cf.}}
%\newcommand*{\vd}{{v.}}
\newcommand*{\etal}{{et al.}}
%\newcommand*{\etsim}{{et sim.}}
%\newcommand*{\ibid}{{ibid.}}
%\newcommand*{\sic}{{sic}}
%\newcommand*{\id}{\mathte{I}}%id matrix
%\newcommand*{\nbd}{\nobreakdash}%
%\newcommand*{\bd}{\hspace{0pt}}%
%\def\hy{-\penalty0\hskip0pt\relax}
%\newcommand*{\labelbis}[1]{\tag*{(\ref{#1})$_\text{r}$}}
%\newcommand*{\mathbox}[2][.8]{\parbox[t]{#1\columnwidth}{#2}}
%\newcommand*{\zerob}[1]{\makebox[0pt][l]{#1}}
\newcommand*{\tprod}{\mathop{\textstyle\prod}\nolimits}
\newcommand*{\tsum}{\mathop{\textstyle\sum}\nolimits}
%\newcommand*{\tint}{\begingroup\textstyle\int\endgroup\nolimits}
%\newcommand*{\tland}{\mathop{\textstyle\bigwedge}\nolimits}
%\newcommand*{\tlor}{\mathop{\textstyle\bigvee}\nolimits}
%\newcommand*{\sprod}{\mathop{\textstyle\prod}}
%\newcommand*{\ssum}{\mathop{\textstyle\sum}}
%\newcommand*{\sint}{\begingroup\textstyle\int\endgroup}
%\newcommand*{\sland}{\mathop{\textstyle\bigwedge}}
%\newcommand*{\slor}{\mathop{\textstyle\bigvee}}
\newcommand*{\T}{^\transp}%transpose
%%\newcommand*{\QEM}%{\textnormal{$\Box$}}%{\ding{167}}
%\newcommand*{\qem}{\leavevmode\unskip\penalty9999 \hbox{}\nobreak\hfill
%\quad\hbox{\QEM}}

%%%%%%%%%%%%%%%%%%%%%%%%%%%%%%%%%%%%%%%%%%%%%%%%%%%%%%%%%%%%%%%%%%%%%%%%%%%%
%%% Custom macros for this file @@@
%%%%%%%%%%%%%%%%%%%%%%%%%%%%%%%%%%%%%%%%%%%%%%%%%%%%%%%%%%%%%%%%%%%%%%%%%%%%
\definecolor{notecolour}{RGB}{68,170,153}
%\newcommand*{\puzzle}{\maltese}
\newcommand*{\puzzle}{{\fontencoding{U}\fontfamily{fontawesometwo}\selectfont\symbol{225}}}
\newcommand*{\wrench}{{\fontencoding{U}\fontfamily{fontawesomethree}\selectfont\symbol{114}}}
\newcommand*{\pencil}{{\fontencoding{U}\fontfamily{fontawesometwo}\selectfont\symbol{210}}}
\newcommand{\mynote}[1]{ {\color{notecolour}\puzzle\ #1}}

\newcommand*{\widebar}[1]{{\mkern1.5mu\skew{2}\overline{\mkern-1.5mu#1\mkern-1.5mu}\mkern 1.5mu}}

% \newcommand{\explanation}[4][t]{%\setlength{\tabcolsep}{-1ex}
% %\smash{
% \begin{tabular}[#1]{c}#2\\[0.5\jot]\rule{1pt}{#3}\\#4\end{tabular}}%}
% \newcommand*{\ptext}[1]{\text{\small #1}}
%\DeclareMathOperator*{\argsup}{arg\,sup}
\DeclareMathOperator{\rank}{rank}
\DeclareMathOperator{\sgn}{sgn}
\newcommand*{\ii}{\mathbin{\rfloor}}
\newcommand*{\rii}{\mathbin{\lfloor}}
\newcommand*{\ts}[1][txyz]{\underset{#1}{1}}
\newcommand*{\dob}{degree of belief}
\newcommand*{\dobs}{degrees of belief}
\newcommand*{\hx}{\hat{x}}
\newcommand*{\lx}{\bar{x}}
% from https://tex.stackexchange.com/a/424252/97039
\makeatletter
\newcommand*{\q}{}% Check if undefined
\DeclareRobustCommand*{\q}{%
  \mathord{\mathpalette\bigcdot@{}}% changed mathbin to mathord
}
\newcommand*{\bigcdot@scalefactor}{0.7}
\newcommand*{\bigcdot@widthfactor}{1.5}
\newcommand*{\bigcdot@}[2]{%
  % #1: math style
  % #2: unused
  \sbox0{$#1\vcenter{}$}% math axis
  \sbox2{$#1\cdot\m@th$}%
  \hbox to \bigcdot@widthfactor\wd2{%
    \hfil
    \raise\ht0\hbox{%
      \scalebox{\bigcdot@scalefactor}{%
        \lower\ht0\hbox{$#1\bullet\m@th$}%
      }%
    }%
    \hfil
  }%
}
\makeatother
\renewcommand*{\i}{\indices}
\newcommand*{\mino}[2][]{\overset{#1}{#2}}
%%% Custom macros end @@@

%%%%%%%%%%%%%%%%%%%%%%%%%%%%%%%%%%%%%%%%%%%%%%%%%%%%%%%%%%%%%%%%%%%%%%%%%%%%
%%% Beginning of document
%%%%%%%%%%%%%%%%%%%%%%%%%%%%%%%%%%%%%%%%%%%%%%%%%%%%%%%%%%%%%%%%%%%%%%%%%%%%
%\firmlists
\begin{document}
\captiondelim{\quad}\captionnamefont{\footnotesize}\captiontitlefont{\footnotesize}
\selectlanguage{british}\frenchspacing
\maketitle

%%%%%%%%%%%%%%%%%%%%%%%%%%%%%%%%%%%%%%%%%%%%%%%%%%%%%%%%%%%%%%%%%%%%%%%%%%%%
%%% Abstract
%%%%%%%%%%%%%%%%%%%%%%%%%%%%%%%%%%%%%%%%%%%%%%%%%%%%%%%%%%%%%%%%%%%%%%%%%%%%
\abstractrunin
\abslabeldelim{}
\renewcommand*{\abstractname}{}
\setlength{\absleftindent}{0pt}
\setlength{\absrightindent}{0pt}
\setlength{\abstitleskip}{-\absparindent}
% \begin{abstract}\labelsep 0pt%
%   \noindent ***
% % \\\noindent\emph{\footnotesize Note: Dear Reader
% %     \amp\ Peer, this manuscript is being peer-reviewed by you. Thank you.}
% % \par%\\[\jot]
% % \noindent
% % {\footnotesize PACS: ***}\qquad%
% % {\footnotesize MSC: ***}%
% %\qquad{\footnotesize Keywords: ***}
% \end{abstract}
\selectlanguage{british}\frenchspacing

%%%%%%%%%%%%%%%%%%%%%%%%%%%%%%%%%%%%%%%%%%%%%%%%%%%%%%%%%%%%%%%%%%%%%%%%%%%%
%%% Epigraph
%%%%%%%%%%%%%%%%%%%%%%%%%%%%%%%%%%%%%%%%%%%%%%%%%%%%%%%%%%%%%%%%%%%%%%%%%%%%
% \asudedication{\small ***}
% \vspace{\bigskipamount}
% \setlength{\epigraphwidth}{.7\columnwidth}
% %\epigraphposition{flushright}
% \epigraphtextposition{flushright}
% %\epigraphsourceposition{flushright}
% \epigraphfontsize{\footnotesize}
% \setlength{\epigraphrule}{0pt}
% %\setlength{\beforeepigraphskip}{0pt}
% %\setlength{\afterepigraphskip}{0pt}
% \epigraph{\emph{text}}{source}



%%%%%%%%%%%%%%%%%%%%%%%%%%%%%%%%%%%%%%%%%%%%%%%%%%%%%%%%%%%%%%%%%%%%%%%%%%%%
%%% BEGINNING OF MAIN TEXT
%%%%%%%%%%%%%%%%%%%%%%%%%%%%%%%%%%%%%%%%%%%%%%%%%%%%%%%%%%%%%%%%%%%%%%%%%%%%

\section{Motivation}
\label{sec:motivation}

The idea is to build tensors not from the two spaces of vectors and
covectors, but from the $2N^{2}$ spaces of multivectors and multicovectors
with their possible straight and twisted orientations.


The exterior algebra is an algebra independent of the tensor one, and it
expresses very intuitive geometric relations
\citep[\cf][]{deschamps1970,deschamps1981}.

The fact that it is independent of the tensor algebra is clear from the
fact that we can establish several inequivalent relations between the tensor and
exterior products, none of them being canonical.


Antisymmetrizer $A$ (a projection):
\begin{equation}
  \label{eq:antisymmetrizer}
  AT \defd \frac{1}{(\deg T)!} \sum_{\pi}\sgn(\pi)\ T\circ\pi
\end{equation}
\textcite{abrahametal1983_r1988}, \textcite{choquetbruhatetal1977_r1996},
\textcite{bossavit1991} use this relation:
\begin{equation}
  \label{eq:wedge_marsden}
  \begin{split}
    \alpha \land \beta &\equiv
    \frac{(\deg \alpha+\deg \beta)!}{(\deg \alpha)!\ (\deg \beta)!}\ 
    A(\alpha \otimes \beta)
  \\
  &\equiv \frac{1}{(\deg \alpha)!\ (\deg \beta)!}
  \sum_{\pi}\sgn(\pi)\ (\alpha \otimes \beta)\circ\pi \ ,
\end{split}
\end{equation}
but relations with different multiplicative factors are also possible.

It's best to define the exterior product intrinsically, with its
multilinear, associative, and graded-commutative properties.


\section{Multivector tensor algebra}
\label{sec:multivector_tensor}

Multivector algebra is based on an affine space, that is, a space on which
the notion of distant parallelism is defined \citep{portamana2011_r2019}.
We take the notion and properties of affine space as understood. We denote
by $N$ the dimension of the base affine space.

The primitive notions on which multivector algebra is constructed are those
of subspace, complementary subspace, inner and outer orientation, and magnitude.


\subsection{Space and complement space}
\label{sec:complement}

We take the notion of subspace as understood as well, and will usually omit
the \enquote{sub-} prefix for brevity. A better name for it would be
\enquote{flat}. We call \enquote{$n$-space} a subspace of dimension $n$. A
0-space is a point, a 1-space is a straight line, a 2-space is a plane, and
so on.

The complement to an $n$-space is easier to picture and intuitively grasp
than to define. We can define it as the set of spaces parallel to that
space. It can be depicted as an $(N-n)$-space that crosses the $n$-space,
with the understanding that the specific choice of such $(N-n)$-space is
unimportant.


\subsection{Inner and outer orientations}
\label{sec:orientation}

The orientation of a space is also a notion easier to grasp than to define.
There are two kinds of orientations of a space: inner and outer.

The inner orientation of an $n$-space can be defined or identified by
giving a $+$ or $-$ sign to a point on the $n$-space, and then choosing an
ordered sequence of other $n$ points such that the whole set of $n+1$
points spans the $n$-space. The choice of such $n+1$ points is unimportant,
modulo an affine transformation. From this it follows that every space can
only have one out of two distinct inner orientations.

A 0-space, that is a point, simply has a $+$ or $-$ sign as its inner
orientation. Let's call this a \enquote{signed point}. The inner
orientation of a 1-space, that is a line, is determined by an ordered
sequence of two points, the first of which is signed. If the signed point
is $-$ we can identify the inner orientation as traversing the line from
the first (signed) to the second; if the signed point is $-$, we traverse
from the second to the first. The inner orientation of a 2-space, that is a
plane, is determined by a ordered sequence of three non-collinear points,
the first of which is signed. If the signed point is $+$ we traverse the
points so first to last, intuitively determining a sense of rotation of the
plane. Similarly the inner orientation of a 3-space can be understood in
terms of the familiar right-hand or left-hand screw senses.

The outer orientation of an $n$-space corresponds to an inner orientation
of its complement. This means that the outer orientation depends on the
dimension of the base space on which we work.





\section{Transformation of components of multi-vectors and -covectors under
  changes of coordinate}
\label{sec:comp_transf}

In the following, square brackets $[\dotso]$ will always only indicate a matrix,
including row- or column-matrices. They will not be used to delimit
function arguments or for grouping, to avoid confusion.

\bigskip

Consider coordinate systems $\lx^{a}$ and $\hx^{a}$ on an $(N-1)$-dimensional
manifold. The tangent map from the first to the second chart domain is
$\bm{M} = (M\i{^{a}_{b}}) \defd \Bigl(\frac{\de \hx^{a}}{\de \lx^{b}}\Bigr)$,
so that, in matrix notation,
\begin{align}
  \label{eq:base_coord_vec_tr}
  [\de_{\lx^{a}}] &=  [\de_{\hx^{b}}]\ \bm{M}
  &
    [\de_{\hx^{a}}] &= [\de_{\lx_{b}}]\ \bm{M}^{-1}
  \\
  \label{eq:base_coord_covec_tr}
  [\di \lx^{a}] &= \bm{M}^{-1}\ [\di\hx^{b}]
  &
    [\di\hx^{a}] &= \bm{M}\ [\di \lx^{b}]
\end{align}
with the convention that every collection of objects (vectors or
components) with lower indices is to be considered as a row matrix; and
with upper indices, as a column matrix.

The components $\bar{v}^{a}$ and $\hat{v}^{a}$ of a vector $\bm{v}$, and the
components $\bar{\omega}_{a}$ and $\hat{\omega}_{a}$ of a covector $\bm{\omega}$
in the two coordinate systems are related by
\begin{align}
  \label{eq:comp_vec_tr}
  [\hat{v}^{a}] &= \bm{M}\ [\bar{v}^{b}]
  &
    [\bar{v}^{a}] &= \bm{M}^{-1}\ [\hat{v}^{b}]
  \\
  \label{eq:comp_covec_tr}
  [\hat{\omega}_{a}] &= [\bar{\omega}_{b}]\ \bm{M}^{-1}
  &
    [\bar{\omega}_{a}] &= [\hat{\omega}_{b}]\ \bm{M}
\end{align}
These equations follow from the fact that, with the matrix-notation
convention above,
\begin{equation}
  \label{eq:matrix_comps}
  \bm{v} = [\de_{\hx^{a}}]\ [\hat{v}^{a}] = [\de_{\lx^{a}}]\ [\bar{v}^{a}]
  \ ,
  \qquad
  \bm{\omega} = [\hat{\omega}_{a}]\ [\di\hx^{a}] = [\bar{\omega}_{a}]\
  [\di\lx^{a}]
  \ .
\end{equation}

\bigskip

Now consider the volume elements in the two coordinate systems, let us
denote them $\bar{\bm{\tau}} \defd \bigwedge_{i} \di \lx^{i}$ and $\hat{\bm{\tau}}
\defd \bigwedge_{i} \di \hx^{i}$. They are related by
\begin{equation}
  \label{eq:base_coord_vol_tr}
  \hat{\bm{\tau}} = \det(\bm{M})\ \bar{\bm{\tau}} \ .
\end{equation}

The components $\bar{q}$, $\hat{q}$ of an $N$-form in the two coordinate system, in
terms of the corresponding (degenerate) bases of $N$-forms, are therefore
related by
\begin{equation}
  \label{eq:comp_vol_tr}
  \bar{q} = \hat{q}\ \det(\bm{M}) \ .
\end{equation}
This can be understood as a multiplication of $\hat{q}$ by a 1-by-1 matrix, namely
the (only) \emph{minor} \citep[\sect~0.7.1]{hornetal1985_r2013} of order
$N$ of $\bm{M}$. Let us generalize this.

\bigskip

Let $I$ and $J$ denote ordered, non-repeated multi-indices, for example
$I= (0,2,3)$. Denote by
\begin{gather}
  \label{eq:minor_det}
  \mino[r]{\bm{M}} \defd
  \text{matrix of minors of $\bm{M}$ of order $r$}
  \\
  \label{eq:comp_minor_det}
 \mino{M}\i{^{I}_{J}} \defd
  \text{minor of $\bm{M}$ obtained keeping rows $I$ and columns $J$}
\end{gather}
Note in particular that $\mino[1]{\bm{M}} = \bm{M}$ and
$\mino[N]{\bm{M}}=M\i{^{0\dotso N}_{0\dotso N}} = \det(\bm{M})$.

Also denote, for a multi-index $I = (I_{1}, \dotsc, I_{r})$ of order r,
\begin{equation}
  \label{eq:bases_cov_multitind}
  \di\lx^{I} \defd \di\lx^{I_{1}} \land \dotsb \land \di\lx^{I_{r}}
\end{equation}
and similarly for the basis elements of multivectors.


\bigskip



Now consider an $r$-covector $\omega$. It can be written in terms of the bases
of $r$-covectors in the two coordinate systems:
\begin{align}
  \label{eq:2omega_bases}
  \omega = \bar{\omega}_{I}\ \di \lx^{I} = \hat{\omega}_{I}\ \di\hx^{I} \ ,
  \qquad\text{\small with $I$ ranging over multiindices of order $r$}
\end{align}
Then the components are related by the following formula:
\begin{equation}
  \label{eq:components_multicov_transf}
  \hat{\omega}_{I} = \det\nolimits\i{^{J}_{I}}(\bm{M})\ \bar{\omega}_{J}
\end{equation}

Let us see a concrete example of why this is true, on a 4-dimensional
manifold. Consider the component $\bar{\omega}_{0\,1}$ of a 2-form:
\begin{multline}
  \label{eq:coord_2form_01}
  \bar{\omega}_{0\,1}\ \di\lx^{0} \land \di\lx^{1} =
  \bar{\omega}_{0\,1}\
  (M\i{^{0}_{i}}\ \di\hx^{i}) \land (M\i{^{1}_{j}}\ \di\hx^{j})={}\\
  \bar{\omega}_{0\,1}\ 
  \begin{aligned}[t]
  \bigl[&(M\i{^{0}_{0}}\ M\i{^{1}_{1}} - M\i{^{1}_{0}}\ M\i{^{0}_{1}})\
  \di\hx^{0\,1} +{}\\
  &(M\i{^{0}_{0}}\ M\i{^{1}_{2}} - M\i{^{1}_{0}}\ M\i{^{0}_{2}})\
  \di\hx^{0\,2} +{}\\
  &(M\i{^{0}_{0}}\ M\i{^{1}_{3}} - M\i{^{1}_{0}}\ M\i{^{0}_{3}})\
  \di\hx^{0\,3} +{}\\
  &(M\i{^{0}_{1}}\ M\i{^{1}_{2}} - M\i{^{1}_{1}}\ M\i{^{0}_{2}})\
  \di\hx^{1\,2} +{}\\
  &(M\i{^{0}_{1}}\ M\i{^{1}_{3}} - M\i{^{1}_{1}}\ M\i{^{0}_{3}})\
  \di\hx^{1\,3} +{}\\
  &(M\i{^{0}_{2}}\ M\i{^{1}_{3}} - M\i{^{1}_{2}}\ M\i{^{0}_{3}})\
  \di\hx^{2\,3} \bigr] = {}
\end{aligned}\\
\bar{\omega}_{0\,1}\ \det\nolimits\i{^{0\,1}_{I}}(\bm{M})\ \di\hx^{I}
\end{multline}
Analogous results can be obtained for all remaining components, and summing
them we find
\begin{equation}
  \label{eq:components_multicov_transf_example2}
  \bar{\omega}_{J}\ \di\lx^{J} =
  \bar{\omega}_{J}\ \det\nolimits\i{^{J}_{I}}(\bm{M})\ \di\hx^{I}
\end{equation}


\section{Inner or dual or dot product}
\label{sec:inner_product}

For a vector $u$ and covector $\omega$ with $\deg u \le \deg\omega$ it's
defined as
\begin{equation}
  \label{eq:inner_prod}
  \begin{gathered}
    u \ii \omega \defd \omega(u) \quad\text{if } \deg u=\deg\omega
\\
(u \ii \omega)(v) \defd (u\land v)\ii \omega \equiv \omega(u \land v)
 \quad\text{if } \deg u < \deg\omega
  \end{gathered}
\end{equation}
It's possible to define an inner product from the right side, but it gives
the same result as above except for a sign:
\begin{equation}
  \label{eq:right_inner_prod}
\begin{gathered}
  \begin{aligned}
  (\omega \rii u)(v) \defd (v\land u)\ii \omega
  &\equiv (-1)^{\deg u\ \deg v}\ (u\land v)\ii \omega
  \\&\equiv (-1)^{\deg u\ \deg v}\ (u \ii \omega)(v)
  \end{aligned}
  \\[\jot]
  {}\implies
\omega \rii u \equiv (-1)^{\deg u\ (\deg\omega-\deg u)}\ 
u \ii \omega \equiv (-1)^{\deg u\ (\deg\omega-1)}\ 
u \ii \omega
\end{gathered}
\end{equation}
For 1-vectors in particular:
\begin{equation}
  \label{eq:right_inner_equiv_1vector}
  \omega \rii u \equiv (-1)^{\deg\omega-1}\ 
  u \ii \omega
  \qquad\text{if }\deg u = 1
\end{equation}

Also
\begin{equation}
  \label{eq:ext_inner_prod}
  \begin{gathered}
    u \rii \omega \defd \omega(u) \equiv u \ii \omega \quad\text{if } \deg u=\deg\omega
\\
(u \rii \omega)(\xi) \defd u\rii(\xi\land \omega) \equiv (\xi\land\omega)(u)
\quad\text{if } \deg u > \deg\omega
\end{gathered}
\end{equation}
and
\begin{equation}
  \label{eq:ext_right_inner_prod}
\begin{gathered}
  \begin{aligned}
    (\omega\ii u)(\xi) \defd
    u\ii (\omega \land \xi)
  &\equiv (-1)^{\deg \omega\ \deg \xi}\ u\ii (\xi \land \omega)
  \\&\equiv (-1)^{\deg \omega\ \deg \xi}\ (u \rii \omega)(\xi)
  \end{aligned}
  \\[\jot]
  {}\implies
  \omega\ii u \equiv (-1)^{\deg \omega\ (\deg v-1)}\ 
u \rii \omega
\end{gathered}
\end{equation}

The lower hook in \enquote{$\ii$} and \enquote{$\rii$} is useful to denote
the object with lower degree, to know how to apply the sign in the
graded-commutativity property and to know what kind of object --
multivector or multicovector -- one obtains.

If we define the degree of vectors to be negative, we can say that
$\alpha \ii \beta$ yields an object of degree
$\deg(\alpha\ii\beta)=\deg\alpha+\deg\beta$, no matter whether $\alpha$ is
a vector and $\beta$ a covector or vice versa. With this convention we
could use the more compact dot-notation
\citep[\cf][\sect~F.I.267]{truesdelletal1960}
\begin{equation}
  \label{eq:inner_dot}
  \begin{gathered}
    \alpha \cdot \beta
    \\
    \text{with}\qquad
    \deg(\alpha \cdot \beta) = \deg\alpha+\deg\beta
    \\
    \beta \cdot \alpha =
    (-1)^{\min\set{\abs{\deg \alpha},\abs{\deg\beta}}\ (\deg\alpha+\deg\beta)}\
    \alpha \cdot \beta \ .
\end{gathered}
\end{equation}

But it doesn't make much sense to use a unique symbol, because it would not
represent an associative operation (unlike the wedge).

The inner product with a 1-vector or a 1-covector is a graded derivation.

\section{Tensor products and equivalent objects}
\label{sec:tensor_equivalent}

When we take tensor products of exterior objects, some special objects and
some canonical correspondences between different classes of objects appear.

Take for example a non-zero $N$-covector $\gamma$. The tensor product
$\gamma \otimes \gamma^{-1}$ is an $N$-vector-valued $N$-covector. This
object is independent of the specific $\gamma$ we chose. It has only one
non-zero component, of value 1, which is invariant under basis or
coordinate changes \citep[\sect~II.8 p.~29 bottom]{schouten1951_r1989}. It
has properties similar to those of a scalar, and the tensor space of
$N$-vector-valued $N$-covectors is similar to those of scalars
(\textcite{schouten1951_r1989} \sect~II.8 p.~29: \enquote{This is not a new
  geometric conception, only a new notation enabling us to get rid of a lot
  of indices}).

The inner product of a $p$-vector $u$ with the object above is an
$N$-vector-valued $(N-p)$-covector:
\begin{equation}
  \label{eq:hypervolume-valued-covector}
  u \cdot \gamma \otimes \gamma^{-1} = \omega \otimes \gamma^{-1}
  \qquad \text{with } \omega = u \cdot \gamma \ .
\end{equation}
The space of $p$-vectors is therefore equivalent, in a canonical way, to
the space of $N$-vector-valued $(N-p)$-covectors. The independent
components transform in the same way under a change of basis/coordinates;
see the example in \textcite{schouten1951_r1989} \sect~II.8 p.~30 bottom.

As Schouten \citey[\sect~II.8 p.~30]{schouten1951_r1989} says,
\enquote{Hence the geometrical meanings of corresponding quantities do not
  differ. There is only a difference in notation. \textelp{} The use of
  $\Delta$-densities is sometimes convenient; \textelp{} the formulae
  contain less indices}.

With the coordinate-free (and index-free) approach the use of such
quantities offers no advantages.


\section{Inner product with $N$-covector}
\label{sec:inner_volume}

We can use several possible conventions in defining an inner product of
multivectors and multicovectors. One basic requirement is that the usual
inner product, which has the recursive property
\begin{equation}
  \label{eq:basic_inner_prod}
  u \cdot (\omega\land \xi) = (u \cdot \omega)\land \xi 
\end{equation}
for every 1-vector $u$, be respected. This property says that the first
slots of the covector on the right are combined first. Another basic
requirement is that
\begin{equation}
  \label{eq:basic_inner_prod_volumes}
  (u \land v \land \dotsb) \cdot (\omega\land \xi \land \dots) = (u \cdot
  \omega)\ (v \cdot \xi) \dotsm
\end{equation}
for the same number of 1-vectors and 1-covectors.


There are two main decisions in the extension: what to do if the vector and
covector are swapped, and what to do if the vector has higher order than
the covector. Reasonable alternatives for the first are
\begin{subequations}
    \label{eq:inner_swap_alternative}
  \begin{gather}
    \label{eq:inner_swap_alternative_right}
    (\omega \land \xi) \cdot u \defd \omega\land (\xi \cdot u)
    \quad\text{recursively}
    \\\shortintertext{or}
    \label{eq:inner_swap_alternative_left}
    (\omega \land \xi) \cdot u \defd
   u \cdot (\omega \land \xi) = (u \cdot \omega)\land \xi \ ,
  \end{gather}
\end{subequations}
for every 1-vector $u$. Reasonable alternatives for the second are
\begin{subequations}
    \label{eq:inner_multiv_alternative}
  \begin{gather}
    \label{eq:inner_multiv_alternative_right}
   (u \land v) \cdot \omega = u \land (v \cdot \omega)
    \quad\text{recursively}
    \\\shortintertext{or}
    \label{eq:inner_multiv_alternative_left}
   (u \land v) \cdot \omega \defd (u \cdot \omega) \land v
  \end{gather}
\end{subequations}
for every 1-covector $\omega$.

In other words, we must choose among
\begin{enumerate}[label=(I\arabic*)]
\item \emph{the adjacent maximal sets of slots of the two terms of the
    inner product should always be combined} and \emph{the two terms of the
    inner product should be left in the order they are};
\item \emph{the adjacent maximal sets of slots of the two terms of the
    inner product should always be combined} and \emph{the multivector
    should always be put on the left of the multicovector as a first
    thing};
\item \emph{the first maximal sets of slots of the two terms of the inner
    product should always be combined}.
\end{enumerate}

Let's examine the consequences of such choices in particular cases:
\begin{align}
  \label{eq:examples_choices_inner_two_x1}
 \de_{x} \cdot (\di x \land \di y)  &=
  \begin{cases}
    \di y &\text{I1}\\
    \di y &\text{I2}\\
    \di y &\text{I3}\\
  \end{cases}
  \\
  \label{eq:examples_choices_inner_two_x2}
  (\di x \land \di y) \cdot \de_{x} &=
  \begin{cases}
    -\di y &\text{I1}\\
    \di y &\text{I2}\\
    \di y &\text{I3}\\
  \end{cases}
  \\
%   \label{eq:examples_choices_inner_two_x3}
%   (\de_{x} \land \de_{y}) \cdot \di x &=
%   \begin{cases}
%     -\de_{y} &\text{I1}\\
%     \de_{y} &\text{I2}\\
%     \de_{y} &\text{I3}\\
%   \end{cases}
% \\
%   \label{eq:examples_choices_inner_two_x4}
%  \di x \cdot (\de_{x} \land \de_{y}) &=
%   \begin{cases}
%     \de_{y} &\text{I1}\\
%     -\de_{y} &\text{I2}\\
%     \de_{y} &\text{I3}\\
%   \end{cases}
% \end{align}
% \begin{align}
%   \label{eq:examples_choices_inner_two_y1}
%  \de_{y} \cdot (\di x \land \di y)  &=
%   \begin{cases}
%     -\di x &\text{I1}\\
%     -\di x &\text{I2}\\
%     -\di x &\text{I3}\\
%   \end{cases}
%   \\
%   \label{eq:examples_choices_inner_two_y2}
%   (\di x \land \di y) \cdot \de_{y} &=
%   \begin{cases}
%     \di x &\text{I1}\\
%     -\di x &\text{I2}\\
%     -\di x &\text{I3}\\
%   \end{cases}
%   \\
  \label{eq:examples_choices_inner_two_y3}
  (\de_{x} \land \de_{y}) \cdot \di y &=
  \begin{cases}
    \de_{x} &\text{I1}\\
    \de_{x} &\text{I2}\\
    -\de_{x} &\text{I3}\\
  \end{cases}
\\
  \label{eq:examples_choices_inner_two_y4}
 \di y \cdot (\de_{x} \land \de_{y}) &=
  \begin{cases}
    -\de_{x} &\text{I1}\\
   \de_{x} &\text{I2}\\
    -\de_{x} &\text{I3}\\
  \end{cases}
\end{align}
\begin{align}
  \label{eq:examples_choices_inner_three}
 \de_{x} \cdot (\di x \land \di y \land \di z) &=
  \begin{cases}
    \di y \land \di z &\text{I1}\\
    \di y \land \di z &\text{I2}\\
    \di y \land \di z &\text{I3}\\
  \end{cases}
  \\
  (\di x \land \di y \land \di z) \cdot \de_{x} &=
  \begin{cases}
    \di y \land \di z &\text{I1}\\
    \di y \land \di z &\text{I2}\\
    \di y \land \di z &\text{I3}\\
  \end{cases}
  \\
  (\de_{x} \land \de_{y} \land \de_{z}) \cdot (\di y \land \di z) &=
  \begin{cases}
    \de_{x}  &\text{I1}\\
    \de_{x}  &\text{I2}\\
    \de_{x}  &\text{I3}\\
  \end{cases}
  \\
  (\di y \land \di z) \cdot (\de_{x} \land \de_{y} \land \de_{z})  &=
  \begin{cases}
    \de_{x} &\text{I1}\\
    \de_{x} &\text{I2}\\
    \de_{x} &\text{I3}\\
  \end{cases}
\end{align}


Let's explore some consequences of the three alternatives.

Under the third alternative (I3), in even dimensions the inner products
with the straight volume element $\gamma$ and then with its inverse
$\gamma^{-1}$ becomes an anti-involution, see
\eqns~\eqref{eq:examples_choices_inner_two_x1} and
\eqref{eq:examples_choices_inner_two_y4}; whereas in odd dimensions it's an
involution. This may be confusing and cumbersome when one is studying a
space with an unspecified dimension.

Under the first alternative (I1), in even dimensions the inner product with
the straight volume element and then its inverse may be an involution,
provided that the product happens from opposite sides. For a tensor product
this would lead to the equalities involving transposition, for example
\begin{gather}
  \label{eq:example_invol_transp}
  u \otimes v =
  \{ \gamma^{-1} \cdot
  [\gamma \cdot (u \otimes v) \cdot \gamma]\T
  \cdot \gamma^{-1} \}\T
  \qquad\text{(I1)}
  \\\shortintertext{or}
  u \otimes v =
  \{ \gamma^{-1} \cdot
  [(u \otimes v) \cdot \gamma]\T 
\}\T  \qquad\text{(I1)},
\end{gather}
whereas in odd dimensions such transposition wouldn't be necessary. This
could perhaps be obviated by using two pairs of symbols such as
\enquote{$\ii$} and \enquote{$\rii$} with $a \rii b \defd b \ii a$. But
again in odd dimensions such distinction would often be unnecessary.

The second alternative (I2) seems to lead to the least cumbersome
consequences, valid in even and odd dimensions alike.

So we can define the inner product starting from 1-vectors and 1-covectors as
\begin{equation}
  \label{eq:inner_prod_final_def}
  \begin{gathered}
    u \cdot \omega \equiv \omega \cdot u \defd \omega(u)
    \\
    (u \land v) \cdot (\omega \land \xi \land \zeta) \equiv
    (\omega \land \xi \land \zeta) \cdot (u \land v)
    \defd (u \cdot \omega)\ (v \cdot \xi)\ \zeta
    \\
    (u \land v \land w) \cdot (\omega \land \xi) \equiv
    (\omega \land \xi) \cdot (u \land v \land w)
    \defd u\ (v \cdot \omega)\ (w \cdot \xi)
\end{gathered}
\end{equation}
and generalizing.

\section{Orientation 2}
\label{sec:orientation2}

One way to define \emph{inner} orientation is inductively as follows.

The orientation of a point, a 0-flat, is $+$ or $-$.

The orientation of an $r$-flat bounded by pairs of parallel $(r-1)$-flats
is determined by giving an orientation of the $(r-1)$-flats in such a way
that parallel pairs have opposite orientations, and the orientations of the
common $(r-2)$-flats cancel each other.

So the orientation of a line is determined by assigning $+$ and $-$ signs
to its boundary points. The orientation of a parallelogram is determined by
assigning opposite orientations to its opposite sides. The orientation of a
volume is effectively given in terms of that of its boundary. This is
equivalent to the definition in Tonti \citep[\chap~3]{tonti2013}. The
orientation in terms of circulation is given by the orientation of the
boundary at one of its points, followed by that of the inward normal.

Because of the latter correspondence this definition of orientation seems
to give a more straightforward understanding of the connection between
inner and outer orientations.

\medskip

To define \emph{outer} orientation we proceed inductively in a similar way
as follows.

The outer orientation of a point is the \emph{inner} orientation of an
$N$-flat (volume) containing that point.

The orientation of an $r$-flat bounded by pairs of parallel $(r-1)$-flats
is determined by giving an orientation of the $(r-1)$-flats in such a way
that parallel pairs have opposite orientations, and the orientations of the
common $(r-2)$-flats cancel each other.

According to this definition and graphical conventions, an $N$-flat with an
outer orientation $+$ is indicated by inward-pointing arrows crossing its
boundary.

\medskip

It turns out that we can give a new, dual definition for the inner
orientation of a point: the inner orientation of a point is the
\emph{outer} orientation of an $N$-flat (volume) containing that point. The
two definitions consistently lead to each other.




\clearpage

\bigskip\mynote{Below: old text}

If $\gamma$ is a non-zero $N$-covector (hypervolume covector), so that
$\gamma^{-1}$ is its dual $N$-vector, we have
\begin{gather}
  \label{eq:inner_volume}
  \gamma^{-1} \cdot (u\cdot \gamma) = (\gamma \cdot u) \cdot \gamma^{-1} = u
  \qquad
  \gamma \cdot (\omega \cdot \gamma^{-1}) =
  (\gamma^{-1} \cdot \omega) \cdot \gamma = \omega
\end{gather}
for any multivector $u$ and multicovector $\omega$ \citep[\cf][\sect~II.7
p.~28]{schouten1951_r1989}. According to~\eqref{eq:right_inner_prod},
\begin{equation}
  \label{eq:sign_inner_volume}
  \gamma\cdot u = (-1)^{\deg(u)\ (N-1)}\ u \cdot \gamma
\end{equation}
and analogously for the dual case.

Therefore
$\gamma^{-1} \cdot(\gamma \cdot u) = (-1)^{\deg(u)\ (N-1)}\ u \ne u =
(\gamma^{-1} \cdot \gamma) \cdot u$, which shows that the inner product
is non-associative in general.

\section{Star operator}
\label{sec:star_operator}

\mynote{This section has mistakes}

Comparison of definitions of star operator:

Denote with $\widebar{\omega}$ the $p$-vector obtained by rising all slots
of $\omega$ with a metric $g$, and with $\underbar{v}$ the reverse
operation. Let $\gamma$ be the volume element induced by the metric.

 \textcite[\sect~V.A.4]{choquetbruhatetal1977_r1996}:
 \begin{equation}
   \label{eq:starop_choquet}
   *\omega \defd \widebar{\omega}\cdot \gamma
 \end{equation}
Applying twice:
\begin{multline}
  \label{eq:double_star}
  **\omega = \widebar{\widebar{\omega}\cdot \gamma} \cdot \gamma =
  (\omega \cdot \gamma^{-1}) \cdot \gamma ={}\\
  (-1)^{\deg(\omega)\ (N-1)} (\gamma^{-1} \cdot \omega) \cdot \gamma =
  (-1)^{\deg(\omega)\ (N-1)} \omega
\end{multline}
So $*^{-1} = (-1)^{\deg(\omega)\ (N-1)}\ *$. Compare with
\textcite[\sect~4.2 Ex.~71]{bossavit1991}.


\textcolor{white}{If you find this you can claim a postcard from me.}



\section{Twisted scalars, vectors, covectors}
\label{sec:twisted}

A twisted scalar is a positive number with an associated outer orientation.
We can specify such orientation locally for example by giving an ordered
list of coordinate functions. Denote such a unit twisted scalar by
\begin{equation}
  \label{eq:twisted_scalar}
  \ts
\end{equation}
It satisfies
\begin{equation}
  \label{eq:twisted_scalar_properties}
  \begin{gathered}
    \ts \cdot \ts = 1
    \\
    a \cdot \ts = \underset{txyz}{a} \text{ for any scalar $a$}
    \\
    \underset{txyz}{-1} = \ts[xtyz] \text{ or any other odd permutation}
\end{gathered}
\end{equation}



$\ts$

$$a \land b$$

% \clearpage

% $\p(O \| M \land S)
% = \tr\bigl(\bm{\varLambda}_{O,M} \bm{\rho}_{S}\bigr)
%  = \langle \psi_{O} \vert \psi_{S} \rangle$

% \clearpage

%%%% examples use empheq
%   \begin{empheq}[left={\mathllap{\begin{aligned}    \de\yF_{\yc}/\de\yp&=0\text{:} \\
%         \de\yF_{\yc}/\de\ym&=0\text{:}\\ \de\yF_{\yc}/\de\yl&=0\text{:}\end{aligned}}\qquad}\empheqlbrace]{align}
%     \label{eq:con_p}
% %    \de\yF_{\yc}/\de\yp &\equiv
%     -\ln\yp + \ln\yq + \yl\yM + \ym\yu &=0,\\
%     \label{eq:con_u}
% %    \de\yF_{\yc}/\de\ym &\equiv
%     \yu\yp-1 &=0,\\
%     \label{eq:con_l}
%     %\de\yF_{\yc}/\de\yl &\equiv
%     \yM\yp-\yc &=0.
%   \end{empheq}
%%%%
% \begin{empheq}[box=\widefbox]{equation}
%   \label{eq:maxent_question}
%   \p\bigl[\yE{N+1}{k} \bigcond \tsum\yo\yf{N}\in\yA, \yM\bigr] = \mathord{?}
% \end{empheq}



% \[
%   \begin{tikzcd}
%       M_{n,n}(\CC) \arrow{r}{R'_{a}(\Hat{U})} & M_{n,n}(\CC)
%     \\
%     L(\mathcal{H}) \arrow{r}{\Hat{U}} \arrow[swap]{d}{R_*}\arrow[swap]{u}{R'_*} & L(\mathcal{H}) \arrow{d}{R_*}\arrow{u}{R'_*} \\
%       M_{n,n}(\CC) \arrow{r}{R_{a}(\Hat{U})} & M_{n,n}(\CC)
%   \end{tikzcd}
% \]

% \[
%   \begin{tikzcd}
%       \CC^n \arrow{r}{R'_*(A)} & \CC^n
%     \\
%     \mathcal{H} \arrow{r}{A} \arrow[swap]{d}{R}\arrow[swap]{u}{R'} & \mathcal{H} \arrow{d}{R}\arrow{u}{R'} \\
%       \CC^n \arrow{r}{R_*(A)} & \CC^n
%   \end{tikzcd}
% \]


% \[
%   \begin{tikzcd}
%     \mathcal{H} \arrow{r}{A} \arrow[swap]{d}{R} & \mathcal{H} \arrow{d}{R} \\
%       \CC^n \arrow{r}{R_*(A)} & \CC^n
%   \end{tikzcd}
% \]

%%\setlength{\intextsep}{0ex}% with wrapfigure
%%\setlength{\columnsep}{0ex}% with wrapfigure
%\begin{figure}[p!]% with figure
%\begin{wrapfigure}{r}{0.4\linewidth} % with wrapfigure
%  \centering\includegraphics[trim={12ex 0 18ex 0},clip,width=\linewidth]{maxent_saddle.png}\\
%\caption{caption}\label{fig:comparison_a5}
%\end{figure}% exp_family_maxent.nb


%%%%%%%%%%%%%%%%%%%%%%%%%%%%%%%%%%%%%%%%%%%%%%%%%%%%%%%%%%%%%%%%%%%%%%%%%%%%
%%% Acknowledgements
%%%%%%%%%%%%%%%%%%%%%%%%%%%%%%%%%%%%%%%%%%%%%%%%%%%%%%%%%%%%%%%%%%%%%%%%%%%% 
\iffalse
\begin{acknowledgements}
  \ldots to Mari \amp\ Miri for continuous encouragement and affection, and
  to Buster Keaton and Saitama for filling life with awe and inspiration.
  To the developers and maintainers of \LaTeX, Emacs, AUC\TeX, Open Science
  Framework, R, Python, Inkscape, Sci-Hub for making a free and impartial
  scientific exchange possible.
%\rotatebox{15}{P}\rotatebox{5}{I}\rotatebox{-10}{P}\rotatebox{10}{\reflectbox{P}}\rotatebox{-5}{O}.
%\sourceatright{\autanet}
\mbox{}\hfill\autanet
\end{acknowledgements}
\fi

%%%%%%%%%%%%%%%%%%%%%%%%%%%%%%%%%%%%%%%%%%%%%%%%%%%%%%%%%%%%%%%%%%%%%%%%%%%%
%%% Appendices
%%%%%%%%%%%%%%%%%%%%%%%%%%%%%%%%%%%%%%%%%%%%%%%%%%%%%%%%%%%%%%%%%%%%%%%%%%%% 
%\clearpage
\bigskip
% %\renewcommand*{\appendixpagename}{Appendix}
% %\renewcommand*{\appendixname}{Appendix}
% %\appendixpage
% \appendix

%%%%%%%%%%%%%%%%%%%%%%%%%%%%%%%%%%%%%%%%%%%%%%%%%%%%%%%%%%%%%%%%%%%%%%%%%%%%
%%% Bibliography
%%%%%%%%%%%%%%%%%%%%%%%%%%%%%%%%%%%%%%%%%%%%%%%%%%%%%%%%%%%%%%%%%%%%%%%%%%%% 
\renewcommand*{\finalnamedelim}{\addcomma\space}
\defbibnote{prenote}{{\footnotesize (\enquote{de $X$} is listed under D,
    \enquote{van $X$} under V, and so on, regardless of national
    conventions.)\par}}
% \defbibnote{postnote}{\par\medskip\noindent{\footnotesize% Note:
%     \arxivp \mparcp \philscip \biorxivp}}

\printbibliography[prenote=prenote%,postnote=postnote
]

\end{document}

%%%%%%%%%%%%%%%%%%%%%%%%%%%%%%%%%%%%%%%%%%%%%%%%%%%%%%%%%%%%%%%%%%%%%%%%%%%%
%%% Cut text (won't be compiled)
%%%%%%%%%%%%%%%%%%%%%%%%%%%%%%%%%%%%%%%%%%%%%%%%%%%%%%%%%%%%%%%%%%%%%%%%%%%% 


%%% Local Variables: 
%%% mode: LaTeX
%%% TeX-PDF-mode: t
%%% TeX-master: t
%%% End: 
