\pdfoutput=1
%% Author: PGL  Porta Mana
%% Created: 2015-05-01T20:53:34+0200
%% Last-Updated: 2022-07-22T15:43:13+0200
%%%%%%%%%%%%%%%%%%%%%%%%%%%%%%%%%%%%%%%%%%%%%%%%%%%%%%%%%%%%%%%%%%%%%%%%%%%%
\newif\ifarxiv
\arxivfalse
\iftrue\pdfmapfile{+classico.map}\fi
\newif\ifafour
\afourfalse% true = A4, false = A5
\newif\iftypodisclaim % typographical disclaim on the side
\typodisclaimtrue
\newcommand*{\memfontfamily}{zplx}
\newcommand*{\memfontpack}{newpxtext}
\documentclass[\ifafour a4paper,12pt,\else a5paper,10pt,\fi%extrafontsizes,%
onecolumn,oneside,article,%french,italian,german,swedish,latin,
british%
]{memoir}
\newcommand*{\firstdraft}{20 July 2022}
\newcommand*{\firstpublished}{\firstdraft}
\newcommand*{\updated}{\ifarxiv***\else\today\fi}
\newcommand*{\propertitle}{Kinematics and dynamics\\ from a modern perspective%\\{\large ***}%
}% title uses LARGE; set Large for smaller
\newcommand*{\pdftitle}{\propertitle}
\newcommand*{\headtitle}{Modern kinematics and dynamics}
\newcommand*{\pdfauthor}{P.G.L.  Porta Mana}
\newcommand*{\headauthor}{Porta Mana}
\newcommand*{\reporthead}{\ifarxiv\else Open Science Framework \href{https://doi.org/10.31219/osf.io/***}{\textsc{doi}:10.31219/osf.io/***}\fi}% Report number

%%%%%%%%%%%%%%%%%%%%%%%%%%%%%%%%%%%%%%%%%%%%%%%%%%%%%%%%%%%%%%%%%%%%%%%%%%%%
%%% Calls to packages (uncomment as needed)
%%%%%%%%%%%%%%%%%%%%%%%%%%%%%%%%%%%%%%%%%%%%%%%%%%%%%%%%%%%%%%%%%%%%%%%%%%%%

%\usepackage{pifont}

%\usepackage{fontawesome}

\usepackage[T1]{fontenc} 
\input{glyphtounicode} \pdfgentounicode=1

\usepackage[utf8]{inputenx}

%\usepackage{newunicodechar}
% \newunicodechar{Ĕ}{\u{E}}
% \newunicodechar{ĕ}{\u{e}}
% \newunicodechar{Ĭ}{\u{I}}
% \newunicodechar{ĭ}{\u{\i}}
% \newunicodechar{Ŏ}{\u{O}}
% \newunicodechar{ŏ}{\u{o}}
% \newunicodechar{Ŭ}{\u{U}}
% \newunicodechar{ŭ}{\u{u}}
% \newunicodechar{Ā}{\=A}
% \newunicodechar{ā}{\=a}
% \newunicodechar{Ē}{\=E}
% \newunicodechar{ē}{\=e}
% \newunicodechar{Ī}{\=I}
% \newunicodechar{ī}{\={\i}}
% \newunicodechar{Ō}{\=O}
% \newunicodechar{ō}{\=o}
% \newunicodechar{Ū}{\=U}
% \newunicodechar{ū}{\=u}
% \newunicodechar{Ȳ}{\=Y}
% \newunicodechar{ȳ}{\=y}

\newcommand*{\bmmax}{0} % reduce number of bold fonts, before font packages
\newcommand*{\hmmax}{0} % reduce number of heavy fonts, before font packages

\usepackage{textcomp}

%\usepackage[normalem]{ulem}% package for underlining
% \makeatletter
% \def\ssout{\bgroup \ULdepth=-.35ex%\UL@setULdepth
%  \markoverwith{\lower\ULdepth\hbox
%    {\kern-.03em\vbox{\hrule width.2em\kern1.2\p@\hrule}\kern-.03em}}%
%  \ULon}
% \makeatother

\usepackage{amsmath}

\usepackage{mathtools}
%\addtolength{\jot}{\jot} % increase spacing in multiline formulae
\setlength{\multlinegap}{0pt}

%\usepackage{empheq}% automatically calls amsmath and mathtools
%\newcommand*{\widefbox}[1]{\fbox{\hspace{1em}#1\hspace{1em}}}

%%%% empheq above seems more versatile than these:
%\usepackage{fancybox}
%\usepackage{framed}

% \usepackage[misc]{ifsym} % for dice
% \newcommand*{\diceone}{{\scriptsize\Cube{1}}}

\usepackage{amssymb}

\usepackage{amsxtra}

\usepackage[main=british]{babel}\selectlanguage{british}
%\newcommand*{\langnohyph}{\foreignlanguage{nohyphenation}}
\newcommand{\langnohyph}[1]{\begin{hyphenrules}{nohyphenation}#1\end{hyphenrules}}

\usepackage[autostyle=false,autopunct=false,english=british]{csquotes}
\setquotestyle{american}
\newcommand*{\defquote}[1]{`\,#1\,'}

% \makeatletter
% \renewenvironment{quotation}%
%                {\list{}{\listparindent 1.5em%
%                         \itemindent    \listparindent
%                         \rightmargin=1em   \leftmargin=1em
%                         \parsep        \z@ \@plus\p@}%
%                 \item[]\footnotesize}%
%                 {\endlist}
% \makeatother                


\usepackage{amsthm}
%% from https://tex.stackexchange.com/a/404680/97039
\makeatletter
\def\@endtheorem{\endtrivlist}
\makeatother

\newcommand*{\QED}{\textsc{q.e.d.}}
\renewcommand*{\qedsymbol}{\QED}
\theoremstyle{remark}
\newtheorem{note}{Note}
\newtheorem*{remark}{Note}
\newtheoremstyle{innote}{\parsep}{\parsep}{\footnotesize}{}{}{}{0pt}{}
\theoremstyle{innote}
\newtheorem*{innote}{}

\usepackage[shortlabels,inline]{enumitem}
\SetEnumitemKey{para}{itemindent=\parindent,leftmargin=0pt,listparindent=\parindent,parsep=0pt,itemsep=\topsep}
% \begin{asparaenum} = \begin{enumerate}[para]
% \begin{inparaenum} = \begin{enumerate*}
\setlist{itemsep=0pt,topsep=\parsep}
\setlist[enumerate,2]{label=(\roman*)}
\setlist[enumerate]{label=(\alph*),leftmargin=1.5\parindent}
\setlist[itemize]{leftmargin=1.5\parindent}
\setlist[description]{leftmargin=1.5\parindent}
% old alternative:
% \setlist[enumerate,2]{label=\alph*.}
% \setlist[enumerate]{leftmargin=\parindent}
% \setlist[itemize]{leftmargin=\parindent}
% \setlist[description]{leftmargin=\parindent}

\usepackage[babel,theoremfont,largesc]{newpxtext}

% For Baskerville see https://ctan.org/tex-archive/fonts/baskervillef?lang=en
% and http://mirrors.ctan.org/fonts/baskervillef/doc/baskervillef-doc.pdf
% \usepackage[p]{baskervillef}
% \usepackage[varqu,varl,var0]{inconsolata}
% \usepackage[scale=.95,type1]{cabin}
% \usepackage[baskerville,vvarbb]{newtxmath}
% \usepackage[cal=boondoxo]{mathalfa}


\usepackage[bigdelims,nosymbolsc%,smallerops % probably arXiv doesn't have it
]{newpxmath}
%\useosf
%\linespread{1.083}%
%\linespread{1.05}% widely used
\linespread{1.1}% best for text with maths
%% smaller operators for old version of newpxmath
\makeatletter
\def\re@DeclareMathSymbol#1#2#3#4{%
    \let#1=\undefined
    \DeclareMathSymbol{#1}{#2}{#3}{#4}}
%\re@DeclareMathSymbol{\bigsqcupop}{\mathop}{largesymbols}{"46}
%\re@DeclareMathSymbol{\bigodotop}{\mathop}{largesymbols}{"4A}
\re@DeclareMathSymbol{\bigoplusop}{\mathop}{largesymbols}{"4C}
\re@DeclareMathSymbol{\bigotimesop}{\mathop}{largesymbols}{"4E}
\re@DeclareMathSymbol{\sumop}{\mathop}{largesymbols}{"50}
\re@DeclareMathSymbol{\prodop}{\mathop}{largesymbols}{"51}
\re@DeclareMathSymbol{\bigcupop}{\mathop}{largesymbols}{"53}
\re@DeclareMathSymbol{\bigcapop}{\mathop}{largesymbols}{"54}
%\re@DeclareMathSymbol{\biguplusop}{\mathop}{largesymbols}{"55}
\re@DeclareMathSymbol{\bigwedgeop}{\mathop}{largesymbols}{"56}
\re@DeclareMathSymbol{\bigveeop}{\mathop}{largesymbols}{"57}
%\re@DeclareMathSymbol{\bigcupdotop}{\mathop}{largesymbols}{"DF}
%\re@DeclareMathSymbol{\bigcapplusop}{\mathop}{largesymbolsPXA}{"00}
%\re@DeclareMathSymbol{\bigsqcupplusop}{\mathop}{largesymbolsPXA}{"02}
%\re@DeclareMathSymbol{\bigsqcapplusop}{\mathop}{largesymbolsPXA}{"04}
%\re@DeclareMathSymbol{\bigsqcapop}{\mathop}{largesymbolsPXA}{"06}
\re@DeclareMathSymbol{\bigtimesop}{\mathop}{largesymbolsPXA}{"10}
%\re@DeclareMathSymbol{\coprodop}{\mathop}{largesymbols}{"60}
%\re@DeclareMathSymbol{\varprod}{\mathop}{largesymbolsPXA}{16}
\makeatother
%%
%% With euler font cursive for Greek letters - the [1] means 100% scaling
\DeclareFontFamily{U}{egreek}{\skewchar\font'177}%
\DeclareFontShape{U}{egreek}{m}{n}{<-6>s*[1]eurm5 <6-8>s*[1]eurm7 <8->s*[1]eurm10}{}%
\DeclareFontShape{U}{egreek}{m}{it}{<->s*[1]eurmo10}{}%
\DeclareFontShape{U}{egreek}{b}{n}{<-6>s*[1]eurb5 <6-8>s*[1]eurb7 <8->s*[1]eurb10}{}%
\DeclareFontShape{U}{egreek}{b}{it}{<->s*[1]eurbo10}{}%
\DeclareSymbolFont{egreeki}{U}{egreek}{m}{it}%
\SetSymbolFont{egreeki}{bold}{U}{egreek}{b}{it}% from the amsfonts package
\DeclareSymbolFont{egreekr}{U}{egreek}{m}{n}%
\SetSymbolFont{egreekr}{bold}{U}{egreek}{b}{n}% from the amsfonts package
% Take also \sum, \prod, \coprod symbols from Euler fonts
\DeclareFontFamily{U}{egreekx}{\skewchar\font'177}
\DeclareFontShape{U}{egreekx}{m}{n}{%
       <-7.5>s*[0.9]euex7%
    <7.5-8.5>s*[0.9]euex8%
    <8.5-9.5>s*[0.9]euex9%
    <9.5->s*[0.9]euex10%
}{}
\DeclareSymbolFont{egreekx}{U}{egreekx}{m}{n}
\DeclareMathSymbol{\sumop}{\mathop}{egreekx}{"50}
\DeclareMathSymbol{\prodop}{\mathop}{egreekx}{"51}
\DeclareMathSymbol{\coprodop}{\mathop}{egreekx}{"60}
\makeatletter
\def\sum{\DOTSI\sumop\slimits@}
\def\prod{\DOTSI\prodop\slimits@}
\def\coprod{\DOTSI\coprodop\slimits@}
\makeatother
\input{definegreek.tex}% Greek letters not usually given in LaTeX.

%\usepackage%[scaled=0.9]%
%{classico}%  Optima as sans-serif font
\renewcommand\sfdefault{uop}
\DeclareMathAlphabet{\mathsf}  {T1}{\sfdefault}{m}{sl}
\SetMathAlphabet{\mathsf}{bold}{T1}{\sfdefault}{b}{sl}
%\newcommand*{\mathte}[1]{\textbf{\textit{\textsf{#1}}}}
% Upright sans-serif math alphabet
% \DeclareMathAlphabet{\mathsu}  {T1}{\sfdefault}{m}{n}
% \SetMathAlphabet{\mathsu}{bold}{T1}{\sfdefault}{b}{n}

% DejaVu Mono as typewriter text
\usepackage[scaled=0.84]{DejaVuSansMono}

\usepackage{mathdots}

\usepackage[usenames]{xcolor}
% Tol (2012) colour-blind-, print-, screen-friendly colours, alternative scheme; Munsell terminology
\definecolor{bluepurple}{RGB}{68,119,170}
\definecolor{blue}{RGB}{102,204,238}
\definecolor{green}{RGB}{34,136,51}
\definecolor{yellow}{RGB}{204,187,68}
\definecolor{red}{RGB}{238,102,119}
\definecolor{redpurple}{RGB}{170,51,119}
\definecolor{grey}{RGB}{187,187,187}
% Tol (2012) colour-blind-, print-, screen-friendly colours; Munsell terminology
% \definecolor{lbpurple}{RGB}{51,34,136}
% \definecolor{lblue}{RGB}{136,204,238}
% \definecolor{lbgreen}{RGB}{68,170,153}
% \definecolor{lgreen}{RGB}{17,119,51}
% \definecolor{lgyellow}{RGB}{153,153,51}
% \definecolor{lyellow}{RGB}{221,204,119}
% \definecolor{lred}{RGB}{204,102,119}
% \definecolor{lpred}{RGB}{136,34,85}
% \definecolor{lrpurple}{RGB}{170,68,153}
\definecolor{lgrey}{RGB}{221,221,221}
%\newcommand*\mycolourbox[1]{%
%\colorbox{grey}{\hspace{1em}#1\hspace{1em}}}
\colorlet{shadecolor}{lgrey}

\usepackage{bm}

\usepackage{microtype}

\usepackage[backend=biber,mcite,%subentry,
citestyle=authoryear-comp,bibstyle=pglpm-authoryear,autopunct=false,sorting=ny,sortcites=false,natbib=false,maxcitenames=2,maxbibnames=8,minbibnames=8,giveninits=true,uniquename=false,uniquelist=false,maxalphanames=1,block=space,hyperref=true,defernumbers=false,useprefix=true,sortupper=false,language=british,parentracker=false]{biblatex}
\DeclareSortingTemplate{ny}{\sort{\field{sortname}\field{author}\field{editor}}\sort{\field{year}}}
\DeclareFieldFormat{postnote}{#1}
\iffalse\makeatletter%%% replace parenthesis with brackets
\newrobustcmd*{\parentexttrack}[1]{%
  \begingroup
  \blx@blxinit
  \blx@setsfcodes
  \blx@bibopenparen#1\blx@bibcloseparen
  \endgroup}
\AtEveryCite{%
  \let\parentext=\parentexttrack%
  \let\bibopenparen=\bibopenbracket%
  \let\bibcloseparen=\bibclosebracket}
\makeatother\fi
\DefineBibliographyExtras{british}{\def\finalandcomma{\addcomma}}
\renewcommand*{\finalnamedelim}{\addspace\amp\space}
% \renewcommand*{\finalnamedelim}{\addcomma\space}
\renewcommand*{\textcitedelim}{\addcomma\space}
% \setcounter{biburlnumpenalty}{1} % to allow url breaks anywhere
% \setcounter{biburlucpenalty}{0}
% \setcounter{biburllcpenalty}{1}
\DeclareDelimFormat{multicitedelim}{\addsemicolon\addspace\space}
\DeclareDelimFormat{compcitedelim}{\addsemicolon\addspace\space}
\DeclareDelimFormat{postnotedelim}{\addspace}
\ifarxiv\else\addbibresource{portamanabib.bib}\fi
\renewcommand{\bibfont}{\footnotesize}
%\appto{\citesetup}{\footnotesize}% smaller font for citations
\defbibheading{bibliography}[\bibname]{\section*{#1}\addcontentsline{toc}{section}{#1}%\markboth{#1}{#1}
}
\newcommand*{\citep}{\footcites}
\newcommand*{\citey}{\footcites}%{\parencites*}
\newcommand*{\ibid}{\unspace\addtocounter{footnote}{-1}\footnotemark{}}
%\renewcommand*{\cite}{\parencite}
%\renewcommand*{\cites}{\parencites}
\providecommand{\href}[2]{#2}
\providecommand{\eprint}[2]{\texttt{\href{#1}{#2}}}
\newcommand*{\amp}{\&}
% \newcommand*{\citein}[2][]{\textnormal{\textcite[#1]{#2}}%\addtocategory{extras}{#2}
% }
\newcommand*{\citein}[2][]{\textnormal{\textcite[#1]{#2}}%\addtocategory{extras}{#2}
}
\newcommand*{\citebi}[2][]{\textcite[#1]{#2}%\addtocategory{extras}{#2}
}
\newcommand*{\subtitleproc}[1]{}
\newcommand*{\chapb}{ch.}
%
%\def\UrlOrds{\do\*\do\-\do\~\do\'\do\"\do\-}%
\def\myUrlOrds{\do\0\do\1\do\2\do\3\do\4\do\5\do\6\do\7\do\8\do\9\do\a\do\b\do\c\do\d\do\e\do\f\do\g\do\h\do\i\do\j\do\k\do\l\do\m\do\n\do\o\do\p\do\q\do\r\do\s\do\t\do\u\do\v\do\w\do\x\do\y\do\z\do\A\do\B\do\C\do\D\do\E\do\F\do\G\do\H\do\I\do\J\do\K\do\L\do\M\do\N\do\O\do\P\do\Q\do\R\do\S\do\T\do\U\do\V\do\W\do\X\do\Y\do\Z}%
\makeatletter
%\g@addto@macro\UrlSpecials{\do={\newline}}
\g@addto@macro{\UrlBreaks}{\myUrlOrds}
\makeatother
\newcommand*{\arxiveprint}[1]{%
arXiv \doi{10.48550/arXiv.#1}%
}
\newcommand*{\mparceprint}[1]{%
\href{http://www.ma.utexas.edu/mp_arc-bin/mpa?yn=#1}{mp\_arc:\allowbreak\nolinkurl{#1}}%
}
\newcommand*{\haleprint}[1]{%
\href{https://hal.archives-ouvertes.fr/#1}{\textsc{hal}:\allowbreak\nolinkurl{#1}}%
}
\newcommand*{\philscieprint}[1]{%
\href{http://philsci-archive.pitt.edu/archive/#1}{PhilSci:\allowbreak\nolinkurl{#1}}%
}
\newcommand*{\doi}[1]{%
\href{https://doi.org/#1}{\textsc{doi}:\allowbreak\nolinkurl{#1}}%
}
\newcommand*{\biorxiveprint}[1]{%
bioRxiv \doi{10.1101/#1}%
}
\newcommand*{\osfeprint}[1]{%
Open Science Framework \doi{10.31219/osf.io/#1}%
}

\usepackage{graphicx}

%\usepackage{wrapfig}

%\usepackage{tikz-cd}

\PassOptionsToPackage{hyphens}{url}\usepackage[hypertexnames=false,pdfencoding=unicode,psdextra]{hyperref}

\usepackage[depth=4]{bookmark}
\hypersetup{colorlinks=true,bookmarksnumbered,pdfborder={0 0 0.25},citebordercolor={0.2667 0.4667 0.6667},citecolor=bluepurple,linkbordercolor={0.6667 0.2 0.4667},linkcolor=redpurple,urlbordercolor={0.1333 0.5333 0.2},urlcolor=green,breaklinks=true,pdftitle={\pdftitle},pdfauthor={\pdfauthor}}
% \usepackage[vertfit=local]{breakurl}% only for arXiv
\providecommand*{\urlalt}{\href}

\usepackage[british]{datetime2}
\DTMnewdatestyle{mydate}%
{% definitions
\renewcommand*{\DTMdisplaydate}[4]{%
\number##3\ \DTMenglishmonthname{##2} ##1}%
\renewcommand*{\DTMDisplaydate}{\DTMdisplaydate}%
}
\DTMsetdatestyle{mydate}

%%%%%%%%%%%%%%%%%%%%%%%%%%%%%%%%%%%%%%%%%%%%%%%%%%%%%%%%%%%%%%%%%%%%%%%%%%%%
%%% Layout. I do not know on which kind of paper the reader will print the
%%% paper on (A4? letter? one-sided? double-sided?). So I choose A5, which
%%% provides a good layout for reading on screen and save paper if printed
%%% two pages per sheet. Average length line is 66 characters and page
%%% numbers are centred.
%%%%%%%%%%%%%%%%%%%%%%%%%%%%%%%%%%%%%%%%%%%%%%%%%%%%%%%%%%%%%%%%%%%%%%%%%%%%
\ifafour\setstocksize{297mm}{210mm}%{*}% A4
\else\setstocksize{210mm}{5.5in}%{*}% 210x139.7
\fi
\settrimmedsize{\stockheight}{\stockwidth}{*}
\setlxvchars[\normalfont] %313.3632pt for a 66-characters line
\setxlvchars[\normalfont]
% \setlength{\trimtop}{0pt}
% \setlength{\trimedge}{\stockwidth}
% \addtolength{\trimedge}{-\paperwidth}
%\settrims{0pt}{0pt}
% The length of the normalsize alphabet is 133.05988pt - 10 pt = 26.1408pc
% The length of the normalsize alphabet is 159.6719pt - 12pt = 30.3586pc
% Bringhurst gives 32pc as boundary optimal with 69 ch per line
% The length of the normalsize alphabet is 191.60612pt - 14pt = 35.8634pc
\ifafour\settypeblocksize{*}{32pc}{1.618} % A4
%\setulmargins{*}{*}{1.667}%gives 5/3 margins % 2 or 1.667
\else\settypeblocksize{*}{26pc}{1.618}% nearer to a 66-line newpx and preserves GR
\fi
\setulmargins{*}{*}{1}%gives equal margins
\setlrmargins{*}{*}{*}
\setheadfoot{\onelineskip}{2.5\onelineskip}
\setheaderspaces{*}{2\onelineskip}{*}
\setmarginnotes{2ex}{10mm}{0pt}
\checkandfixthelayout[nearest]
%%% End layout
%% this fixes missing white spaces
%\pdfmapline{+dummy-space <dummy-space.pfb}
%\pdfinterwordspaceon% seems to add a white margin to Sumatrapdf

%%% Sectioning
\newcommand*{\asudedication}[1]{%
{\par\centering\textit{#1}\par}}
\newenvironment{acknowledgements}{\section*{Thanks}\addcontentsline{toc}{section}{Thanks}}{\par}
\makeatletter\renewcommand{\appendix}{\par
  \bigskip{\centering
   \interlinepenalty \@M
   \normalfont
   \printchaptertitle{\sffamily\appendixpagename}\par}
  \setcounter{section}{0}%
  \gdef\@chapapp{\appendixname}%
  \gdef\thesection{\@Alph\c@section}%
  \anappendixtrue}\makeatother
\counterwithout{section}{chapter}
\setsecnumformat{\upshape\csname the#1\endcsname\quad}
\setsecheadstyle{\large\bfseries\sffamily%
\centering}
\setsubsecheadstyle{\bfseries\sffamily%
\raggedright}
%\setbeforesecskip{-1.5ex plus 1ex minus .2ex}% plus 1ex minus .2ex}
%\setaftersecskip{1.3ex plus .2ex }% plus 1ex minus .2ex}
%\setsubsubsecheadstyle{\bfseries\sffamily\slshape\raggedright}
%\setbeforesubsecskip{1.25ex plus 1ex minus .2ex }% plus 1ex minus .2ex}
%\setaftersubsecskip{-1em}%{-0.5ex plus .2ex}% plus 1ex minus .2ex}
\setsubsecindent{0pt}%0ex plus 1ex minus .2ex}
\setparaheadstyle{\bfseries\sffamily%
\raggedright}
\setcounter{secnumdepth}{2}
\setlength{\headwidth}{\textwidth}
\newcommand{\addchap}[1]{\chapter*[#1]{#1}\addcontentsline{toc}{chapter}{#1}}
\newcommand{\addsec}[1]{\section*{#1}\addcontentsline{toc}{section}{#1}}
\newcommand{\addsubsec}[1]{\subsection*{#1}\addcontentsline{toc}{subsection}{#1}}
\newcommand{\addpara}[1]{\paragraph*{#1.}\addcontentsline{toc}{subsubsection}{#1}}
\newcommand{\addparap}[1]{\paragraph*{#1}\addcontentsline{toc}{subsubsection}{#1}}

%%% Headers, footers, pagestyle
\copypagestyle{manaart}{plain}
\makeheadrule{manaart}{\headwidth}{0.5\normalrulethickness}
\makeoddhead{manaart}{%
{\footnotesize%\sffamily%
\scshape\headauthor}}{}{{\footnotesize\sffamily%
\headtitle}}
\makeoddfoot{manaart}{}{\thepage}{}
\newcommand*\autanet{\includegraphics[height=\heightof{M}]{autanet.pdf}}
\definecolor{mygray}{gray}{0.333}
\iftypodisclaim%
\ifafour\newcommand\addprintnote{\begin{picture}(0,0)%
\put(245,149){\makebox(0,0){\rotatebox{90}{\tiny\color{mygray}\textsf{This
            document is designed for screen reading and
            two-up printing on A4 or Letter paper}}}}%
\end{picture}}% A4
\else\newcommand\addprintnote{\begin{picture}(0,0)%
\put(176,112){\makebox(0,0){\rotatebox{90}{\tiny\color{mygray}\textsf{This
            document is designed for screen reading and
            two-up printing on A4 or Letter paper}}}}%
\end{picture}}\fi%afourtrue
\makeoddfoot{plain}{}{\makebox[0pt]{\thepage}\addprintnote}{}
\else
\makeoddfoot{plain}{}{\makebox[0pt]{\thepage}}{}
\fi%typodisclaimtrue
\makeoddhead{plain}{\scriptsize\reporthead}{}{}
% \copypagestyle{manainitial}{plain}
% \makeheadrule{manainitial}{\headwidth}{0.5\normalrulethickness}
% \makeoddhead{manainitial}{%
% \footnotesize\sffamily%
% \scshape\headauthor}{}{\footnotesize\sffamily%
% \headtitle}
% \makeoddfoot{manaart}{}{\thepage}{}

\pagestyle{manaart}

\setlength{\droptitle}{-3.9\onelineskip}
\pretitle{\begin{center}\LARGE\sffamily%
\bfseries}
\posttitle{\bigskip\end{center}}

\makeatletter\newcommand*{\atf}{\includegraphics[totalheight=\heightof{@}]{atblack.png}}\makeatother
\providecommand{\affiliation}[1]{\textsl{\textsf{\footnotesize #1}}}
\providecommand{\epost}[1]{\texttt{\footnotesize\textless#1\textgreater}}
\providecommand{\email}[2]{\href{mailto:#1ZZ@#2 ((remove ZZ))}{#1\protect\atf#2}}
%\providecommand{\email}[2]{\href{mailto:#1@#2}{#1@#2}}

\preauthor{\vspace{-0.5\baselineskip}\begin{center}
\normalsize\sffamily%
\lineskip  0.5em}
\postauthor{\par\end{center}}
\predate{\DTMsetdatestyle{mydate}\begin{center}\footnotesize}
\postdate{\end{center}\vspace{-\medskipamount}}

\setfloatadjustment{figure}{\footnotesize}
\captiondelim{\quad}
\captionnamefont{\footnotesize\sffamily%
}
\captiontitlefont{\footnotesize}
%\firmlists*
\midsloppy
% handling orphan/widow lines, memman.pdf
% \clubpenalty=10000
% \widowpenalty=10000
% \raggedbottom
% Downes, memman.pdf
\clubpenalty=9996
\widowpenalty=9999
\brokenpenalty=4991
\predisplaypenalty=10000
\postdisplaypenalty=1549
\displaywidowpenalty=1602
\raggedbottom

\paragraphfootnotes
\setlength{\footmarkwidth}{2ex}
% \threecolumnfootnotes
%\setlength{\footmarksep}{0em}
\footmarkstyle{\textsuperscript{%\color{red}
\scriptsize\bfseries#1}~}
%\footmarkstyle{\textsuperscript{\color{red}\scriptsize\bfseries#1}~}
%\footmarkstyle{\textsuperscript{[#1]}~}

\selectlanguage{british}\frenchspacing

\definecolor{notecolour}{RGB}{68,170,153}
%\newcommand*{\puzzle}{\maltese}
\newcommand*{\puzzle}{{\fontencoding{U}\fontfamily{fontawesometwo}\selectfont\symbol{225}}}
\newcommand*{\wrench}{{\fontencoding{U}\fontfamily{fontawesomethree}\selectfont\symbol{114}}}
\newcommand*{\pencil}{{\fontencoding{U}\fontfamily{fontawesometwo}\selectfont\symbol{210}}}
\newcommand{\mynotew}[1]{{\footnotesize\color{notecolour}\wrench\ #1}}
\newcommand{\mynotep}[1]{{\footnotesize\color{notecolour}\pencil\ #1}}
\newcommand{\mynotez}[1]{{\footnotesize\color{notecolour}\puzzle\ #1}}

%%%%%%%%%%%%%%%%%%%%%%%%%%%%%%%%%%%%%%%%%%%%%%%%%%%%%%%%%%%%%%%%%%%%%%%%%%%%
%%% Paper's details
%%%%%%%%%%%%%%%%%%%%%%%%%%%%%%%%%%%%%%%%%%%%%%%%%%%%%%%%%%%%%%%%%%%%%%%%%%%%
\title{\propertitle}
\author{%
\hspace*{\stretch{1}}%
%% uncomment if additional authors present
% \parbox{0.5\linewidth}%\makebox[0pt][c]%
% {\protect\centering ***\\%
% \footnotesize\epost{\email{***}{***}}}%
% \hspace*{\stretch{1}}%
\parbox{1\linewidth}%\makebox[0pt][c]%
{\protect\centering P.G.L.  Porta Mana  \href{https://orcid.org/0000-0002-6070-0784}{\protect\includegraphics[scale=0.16]{orcid_32x32.png}}\\\footnotesize
Western Norway University of Applied Sciences%
\quad\epost{\email{pgl}{portamana.org}}}%
% Mohn Medical Imaging and Visualization Centre, Dept of Computer science, Electrical Engineering and Mathematical Sciences, Western Norway University of Applied Sciences, Bergen, Norway
%% uncomment if additional authors present
% \hspace*{\stretch{1}}%
% \parbox{0.5\linewidth}%\makebox[0pt][c]%
% {\protect\centering ***\\%
% \footnotesize\epost{\email{***}{***}}}%
\hspace*{\stretch{1}}%
}

%\date{Draft of \today\ (first drafted \firstdraft)}
\date{\firstpublished; updated \updated}

%%%%%%%%%%%%%%%%%%%%%%%%%%%%%%%%%%%%%%%%%%%%%%%%%%%%%%%%%%%%%%%%%%%%%%%%%%%%
%%% Macros @@@
%%%%%%%%%%%%%%%%%%%%%%%%%%%%%%%%%%%%%%%%%%%%%%%%%%%%%%%%%%%%%%%%%%%%%%%%%%%%

% Common ones - uncomment as needed
%\providecommand{\nequiv}{\not\equiv}
%\providecommand{\coloneqq}{\mathrel{\mathop:}=}
%\providecommand{\eqqcolon}{=\mathrel{\mathop:}}
%\providecommand{\varprod}{\prod}
\newcommand*{\de}{\partialup}%partial diff
\newcommand*{\pu}{\piup}%constant pi
\newcommand*{\delt}{\deltaup}%Kronecker, Dirac
%\newcommand*{\eps}{\varepsilonup}%Levi-Civita, Heaviside
%\newcommand*{\riem}{\zetaup}%Riemann zeta
%\providecommand{\degree}{\textdegree}% degree
%\newcommand*{\celsius}{\textcelsius}% degree Celsius
%\newcommand*{\micro}{\textmu}% degree Celsius
\newcommand*{\I}{\mathrm{i}}%imaginary unit
\newcommand*{\e}{\mathrm{e}}%Neper
\newcommand*{\di}{\mathrm{d}}%differential
%\newcommand*{\Di}{\mathrm{D}}%capital differential
%\newcommand*{\planckc}{\hslash}
%\newcommand*{\avogn}{N_{\textrm{A}}}
%\newcommand*{\NN}{\bm{\mathrm{N}}}
%\newcommand*{\ZZ}{\bm{\mathrm{Z}}}
%\newcommand*{\QQ}{\bm{\mathrm{Q}}}
\newcommand*{\RR}{\bm{\mathrm{R}}}
%\newcommand*{\CC}{\bm{\mathrm{C}}}
%\newcommand*{\nabl}{\bm{\nabla}}%nabla
%\DeclareMathOperator{\lb}{lb}%base 2 log
%\DeclareMathOperator{\tr}{tr}%trace
%\DeclareMathOperator{\card}{card}%cardinality
%\DeclareMathOperator{\im}{Im}%im part
%\DeclareMathOperator{\re}{Re}%re part
%\DeclareMathOperator{\sgn}{sgn}%signum
%\DeclareMathOperator{\ent}{ent}%integer less or equal to
%\DeclareMathOperator{\Ord}{O}%same order as
%\DeclareMathOperator{\ord}{o}%lower order than
%\newcommand*{\incr}{\triangle}%finite increment
\newcommand*{\defd}{\coloneqq}
\newcommand*{\defs}{\eqqcolon}
%\newcommand*{\Land}{\bigwedge}
%\newcommand*{\Lor}{\bigvee}
%\newcommand*{\lland}{\DOTSB\;\land\;}
%\newcommand*{\llor}{\DOTSB\;\lor\;}
\newcommand*{\limplies}{\mathbin{\Rightarrow}}%implies
%\newcommand*{\suchthat}{\mid}%{\mathpunct{|}}%such that (eg in sets)
%\newcommand*{\with}{\colon}%with (list of indices)
%\newcommand*{\mul}{\times}%multiplication
%\newcommand*{\inn}{\cdot}%inner product
%\newcommand*{\dotv}{\mathord{\,\cdot\,}}%variable place
%\newcommand*{\comp}{\circ}%composition of functions
%\newcommand*{\con}{\mathbin{:}}%scal prod of tensors
%\newcommand*{\equi}{\sim}%equivalent to 
\renewcommand*{\asymp}{\simeq}%equivalent to 
%\newcommand*{\corr}{\mathrel{\hat{=}}}%corresponds to
%\providecommand{\varparallel}{\ensuremath{\mathbin{/\mkern-7mu/}}}%parallel (tentative symbol)
\renewcommand*{\le}{\leqslant}%less or equal
\renewcommand*{\ge}{\geqslant}%greater or equal
\DeclarePairedDelimiter\clcl{[}{]}
%\DeclarePairedDelimiter\clop{[}{[}
%\DeclarePairedDelimiter\opcl{]}{]}
%\DeclarePairedDelimiter\opop{]}{[}
\DeclarePairedDelimiter\abs{\lvert}{\rvert}
%\DeclarePairedDelimiter\norm{\lVert}{\rVert}
\DeclarePairedDelimiter\set{\{}{\}} %}
%\DeclareMathOperator{\pr}{P}%probability
\newcommand*{\p}{\mathrm{p}}%probability
\renewcommand*{\P}{\mathrm{P}}%probability
%\newcommand*{\E}{\mathrm{E}}
%% The "\:" space is chosen to correctly separate inner binary and external rels
\renewcommand*{\|}[1][]{\nonscript\:#1\vert\nonscript\:\mathopen{}}
%\DeclarePairedDelimiterX{\cp}[2]{(}{)}{#1\nonscript\:\delimsize\vert\nonscript\:\mathopen{}#2}
%\DeclarePairedDelimiterX{\ct}[2]{[}{]}{#1\nonscript\;\delimsize\vert\nonscript\:\mathopen{}#2}
%\DeclarePairedDelimiterX{\cs}[2]{\{}{\}}{#1\nonscript\:\delimsize\vert\nonscript\:\mathopen{}#2}
%\newcommand*{\+}{\lor}
%\renewcommand{\*}{\land}
%% symbol = for equality statements within probabilities
%% from https://tex.stackexchange.com/a/484142/97039
% \newcommand*{\eq}{\mathrel{\!=\!}}
% \let\texteq\=
% \renewcommand*{\=}{\TextOrMath\texteq\eq}
% \newcommand*{\eq}[1][=]{\mathrel{\!#1\!}}
\newcommand*{\mo}[1][=]{\mathord{\,#1\,}}
%%
\newcommand*{\sect}{\S}% Sect.~
\newcommand*{\sects}{\S\S}% Sect.~
\newcommand*{\chap}{ch.}%
\newcommand*{\chaps}{chs}%
\newcommand*{\bref}{ref.}%
\newcommand*{\brefs}{refs}%
%\newcommand*{\fn}{fn}%
\newcommand*{\eqn}{eq.}%
\newcommand*{\eqns}{eqs}%
\newcommand*{\fig}{fig.}%
\newcommand*{\figs}{figs}%
\newcommand*{\vs}{{vs}}
\newcommand*{\eg}{{e.g.}}
\newcommand*{\etc}{{etc.}}
\newcommand*{\ie}{{i.e.}}
%\newcommand*{\ca}{{c.}}
\newcommand*{\foll}{{ff.}}
%\newcommand*{\viz}{{viz}}
\newcommand*{\cf}{{cf.}}
%\newcommand*{\Cf}{{Cf.}}
%\newcommand*{\vd}{{v.}}
\newcommand*{\etal}{{et al.}}
%\newcommand*{\etsim}{{et sim.}}
%\newcommand*{\ibid}{{ibid.}}
%\newcommand*{\sic}{{sic}}
%\newcommand*{\id}{\mathte{I}}%id matrix
%\newcommand*{\nbd}{\nobreakdash}%
%\newcommand*{\bd}{\hspace{0pt}}%
%\def\hy{-\penalty0\hskip0pt\relax}
%\newcommand*{\labelbis}[1]{\tag*{(\ref{#1})$_\text{r}$}}
%\newcommand*{\mathbox}[2][.8]{\parbox[t]{#1\columnwidth}{#2}}
%\newcommand*{\zerob}[1]{\makebox[0pt][l]{#1}}
\newcommand*{\tprod}{\mathop{\textstyle\prod}\nolimits}
\newcommand*{\tsum}{\mathop{\textstyle\sum}\nolimits}
%\newcommand*{\tint}{\begingroup\textstyle\int\endgroup\nolimits}
%\newcommand*{\tland}{\mathop{\textstyle\bigwedge}\nolimits}
%\newcommand*{\tlor}{\mathop{\textstyle\bigvee}\nolimits}
%\newcommand*{\sprod}{\mathop{\textstyle\prod}}
%\newcommand*{\ssum}{\mathop{\textstyle\sum}}
%\newcommand*{\sint}{\begingroup\textstyle\int\endgroup}
%\newcommand*{\sland}{\mathop{\textstyle\bigwedge}}
%\newcommand*{\slor}{\mathop{\textstyle\bigvee}}
%\newcommand*{\T}{^\transp}%transpose
%%\newcommand*{\QEM}%{\textnormal{$\Box$}}%{\ding{167}}
%\newcommand*{\qem}{\leavevmode\unskip\penalty9999 \hbox{}\nobreak\hfill
%\quad\hbox{\QEM}}

%%%%%%%%%%%%%%%%%%%%%%%%%%%%%%%%%%%%%%%%%%%%%%%%%%%%%%%%%%%%%%%%%%%%%%%%%%%%
%%% Custom macros for this file @@@
%%%%%%%%%%%%%%%%%%%%%%%%%%%%%%%%%%%%%%%%%%%%%%%%%%%%%%%%%%%%%%%%%%%%%%%%%%%%

\newcommand*{\widebar}[1]{{\mkern1.5mu\skew{2}\overline{\mkern-1.5mu#1\mkern-1.5mu}\mkern 1.5mu}}

% \newcommand{\explanation}[4][t]{%\setlength{\tabcolsep}{-1ex}
% %\smash{
% \begin{tabular}[#1]{c}#2\\[0.5\jot]\rule{1pt}{#3}\\#4\end{tabular}}%}
% \newcommand*{\ptext}[1]{\text{\small #1}}
%\DeclareMathOperator*{\argsup}{arg\,sup}
\newcommand*{\dob}{degree of belief}
\newcommand*{\dobs}{degrees of belief}
%\newcommand*{\tw}[2]{\underbracket[0pt][0pt]{#1}_{\scriptscriptstyle #2}}
%\newcommand*{\ti}{\tw{\di}}
%\newcommand*{\te}{\tw{\de}}
\newcommand*{\tw}[2][\scriptstyle\sim]{\smash[b]{\underset{\mathclap{{}^{#1}}}{#2}}}
\newcommand*{\ti}[1][\scriptstyle\sim]{\tw[#1]{\di}}
\newcommand*{\te}[1][\scriptstyle\sim]{\tw[#1]{\de}}

%%% Custom macros end @@@

%%%%%%%%%%%%%%%%%%%%%%%%%%%%%%%%%%%%%%%%%%%%%%%%%%%%%%%%%%%%%%%%%%%%%%%%%%%%
%%% Beginning of document
%%%%%%%%%%%%%%%%%%%%%%%%%%%%%%%%%%%%%%%%%%%%%%%%%%%%%%%%%%%%%%%%%%%%%%%%%%%%
%\firmlists
\begin{document}
\captiondelim{\quad}\captionnamefont{\footnotesize}\captiontitlefont{\footnotesize}
\selectlanguage{british}\frenchspacing
\maketitle

%%%%%%%%%%%%%%%%%%%%%%%%%%%%%%%%%%%%%%%%%%%%%%%%%%%%%%%%%%%%%%%%%%%%%%%%%%%%
%%% Abstract
%%%%%%%%%%%%%%%%%%%%%%%%%%%%%%%%%%%%%%%%%%%%%%%%%%%%%%%%%%%%%%%%%%%%%%%%%%%%
\abstractrunin
\abslabeldelim{}
\renewcommand*{\abstractname}{}
\setlength{\absleftindent}{0pt}
\setlength{\absrightindent}{0pt}
\setlength{\abstitleskip}{-\absparindent}
\begin{abstract}\labelsep 0pt%
  \noindent The kinematics of mechanical, thermodynamic, and electromagnetic phenomena is developed in such a way as to be used for Newtonian, Lorentzian, and general relativity. The same is done, as much as possible, for their dynamics as well
% \\\noindent\emph{\footnotesize Note: Dear Reader
%     \amp\ Peer, this manuscript is being peer-reviewed by you. Thank you.}
% \par%\\[\jot]
% \noindent
% {\footnotesize PACS: ***}\qquad%
% {\footnotesize MSC: ***}%
%\qquad{\footnotesize Keywords: ***}
\end{abstract}
\selectlanguage{british}\frenchspacing

%%%%%%%%%%%%%%%%%%%%%%%%%%%%%%%%%%%%%%%%%%%%%%%%%%%%%%%%%%%%%%%%%%%%%%%%%%%%
%%% Epigraph
%%%%%%%%%%%%%%%%%%%%%%%%%%%%%%%%%%%%%%%%%%%%%%%%%%%%%%%%%%%%%%%%%%%%%%%%%%%%
% \asudedication{\small ***}
% \vspace{\bigskipamount}
% \setlength{\epigraphwidth}{.7\columnwidth}
% %\epigraphposition{flushright}
% \epigraphtextposition{flushright}
% %\epigraphsourceposition{flushright}
% \epigraphfontsize{\footnotesize}
% \setlength{\epigraphrule}{0pt}
% %\setlength{\beforeepigraphskip}{0pt}
% %\setlength{\afterepigraphskip}{0pt}
% \epigraph{\emph{text}}{source}



%%%%%%%%%%%%%%%%%%%%%%%%%%%%%%%%%%%%%%%%%%%%%%%%%%%%%%%%%%%%%%%%%%%%%%%%%%%%
%%% BEGINNING OF MAIN TEXT
%%%%%%%%%%%%%%%%%%%%%%%%%%%%%%%%%%%%%%%%%%%%%%%%%%%%%%%%%%%%%%%%%%%%%%%%%%%%

\section{Scribbles and memos}
\label{sec:memos}

% \[ \color{bluepurple}\bm{a} \color{redpurple}\mathbin{\bm{\land}}\color{bluepurple} \bm{b} \]

\subsection{Twisted objects}
\label{sec:twisted}

Denote the wedge product by juxtaposition: $\di t \land \di x \defs \di t\, \di x$ and so on; abbreviate $\di t\, \di x \defs \di tx$ and so on; and denote twisted differential forms by $\ti$.

Where an ordered set of coordinate functions $(t,x,y,z)$ is chosen, the twisted unit $\tw{1}$ is defined. It has unit magnitude and outer-orientation $txyz$, and the property $\tw{1}\cdot\tw{1}=1$. In general it is only defined in chart domain, where coordinate functions can be defined, but not globally. We obtain twisted vectors and covectors by multiplying their non-twisted counterparts by the twisted unit. For a vector or covector $\omega$ we have that the orientation of its twisted counterpart is such that \autocites[\eqn~(28.1)]{burke1985_r1987}
\begin{equation}
  \label{eq:orientation_twisted_cov}
  \set[\big]{\set{\tw{\omega}}, \set{\omega}} = \set[\big]{\tw{1}}\ .
\end{equation}
Said otherwise, in the product $\tw{1} \cdot \omega$ the \emph{right} side of the orientation of $\tw{1}$ cancels out with the orientation of $\omega$. This rule must be respected even if we invert the product order, so $\tw{1} \cdot \omega \equiv \omega \cdot \tw{1}$.

Multiplying a twisted vector or covector by the twisted unit, we obtain its non-twisted counterpart. The resulting orientation is obtained by cancelling out the orientation of the twisted object and the \emph{left} side of the orientation of $\tw{1}$.

We have:
\begin{gather}
  \set[\big]{\tw{1}} = txyz \ ;
  \\[\jot]
  \set[\big]{\ti t} = -xyz \ , \quad
  \set[\big]{\ti x} = tyz \ , \quad
  \set[\big]{\ti y} = tzx \ , \quad
  \set[\big]{\ti z} = txy \ ;
  \\[\jot]
  \set[\big]{\ti tx} = yz \ , \quad
  \set[\big]{\ti ty} = zx \ , \quad
  \set[\big]{\ti tz} = xy \ , \\
  \set[\big]{\ti xy} = tz \ , \quad
  \set[\big]{\ti yz} = tx \ , \quad
  \set[\big]{\ti zx} = ty \ ;
  \\[\jot]
  \set[\big]{\ti tyz} = -x \ , \quad
  \set[\big]{\ti tzx} = -y \ , \quad
  \set[\big]{\ti txy} = -z \ , \quad
  \set[\big]{\ti xyz} = t \ ;
  \\[\jot]
  \set[\big]{\ti txyz} = +1 \ .
\end{gather}
The minus signs appear in the odd ranks when we have $t$ and an even number of other coordinates after the \enquote{$\ti$}. These minus signs flip if we keep $t$ always to the right, with orientation $xyzt$.

Note that considering, say, the \emph{function} $x$, we have
\begin{equation}
  \label{eq:function_t_twisted}
  \set[\big]{\tw{x}} =
  \begin{cases}
    \set{txyz} & \text{if } x>0 \ , \\
    -\set{txyz} & \text{if } x<0 \ .
  \end{cases}
\end{equation}


\subsection{Charge and current densities}
\label{sec:charge_current}

We can represent charge density and current density in one geometrical entity. Consider an ordered coordinate system $(t, x, y, z)$. The charge-current density is
\begin{equation}
  \label{eq:charge-current}
  \begin{split}
\bm{Q} &\defd  \rho\ \ti xyz
- j_{x}\ \ti tyz - j_{y}\ \ti tzx - j_{z}\ \ti txy
  \\[\jot]
  &\equiv\rho\ \ti xyz -
  \di t\ \bigl(j_{x}\ \ti yz + j_{y}\ \ti zx + j_{z}\ \ti xy\bigr) \ .
\end{split}
\end{equation}
It has the dimensions of charge (current${}\cdot{}$time). The minus signs appear so that the $x$-component of the current, for example, is positive when $j_{x} > 0$ and so on.

This object automatically give us net volume charge when integrated over a three-dimensional region at constant time, or the net flux of charge when integrated over a two-dimensional surface -- possibly even moving -- over a lapse of time. Consider for example a 3-volume $V$ at constant time, having \emph{outer} orientation in the positive $t$ direction and parameterized by
\begin{gather}
  \label{eq:3-volume_space}
  (u, v, w) \mapsto (t_{0}, u, v, w) \ .
\\
  \shortintertext{On it, the basis twisted 1-forms map to \autocites[p.~192]{burke1985_r1987}}
  \begin{gathered}
  \ti[-xyz] t\bigr\rvert_{V} = 0\ , \quad
  \ti[tyz] x\bigr\rvert_{V} = \ti[vw] u\ , \quad
  \ti[tzx] y\bigr\rvert_{V} = \ti[wu] v\ , \quad
  \ti[txy] z\bigr\rvert_{V} = \ti[uv] w\ ,
  \\[2\jot]
  \ti[t]xyz\bigr\rvert_{V} = \ti[+]uvw \ .
\end{gathered}
\end{gather}
Then we have
\begin{equation}
  \label{eq:charge-current_on_3volume}
  \bigl(\rho\ \ti xyz
  - j_{x}\ \ti tyz - j_{y}\ \ti tzx - j_{z}\ \ti txy \bigr)\bigr\rvert_{V} =
  \rho\ \ti[+] uvw
\end{equation}
and the current density gives no contribution.

The charge-current density also correctly transform under coordinate changes. Consider for example
\begin{gather}
  \label{eq:coord_change_galilei}
  (t', x', y', z') = (t, x-vt, y, z) \ ,
  \qquad
  (t, x, y, z) = (t', x'+vt', y', z') \ ,
  \\
  \shortintertext{for which}
  \begin{gathered}
  \di t = \di t' \ , \quad
  \di y = \di y' \ , \quad
  \di z = \di z' \ , \quad
  \di x = \di x' + v\,\di t' \ ;
\end{gathered}
\\[\jot]
\begin{aligned}
  \di x\, \di y\, \di z &=
  (\di x' + v\,\di t')\,\di y'\,\di z' =
  \di x'y'z' + v\,\di t'y'z' \ ,
  \\[\jot]
  \di t\, \di z\, \di x &=
  \di t'\, \di z'\, (\di x' + v\,\di t') =
  \di t'\, \di z'\, \di x' \ ,
  \\[\jot]
  \di t\, \di x\, \di y &=
  \di t'\, (\di x' + v\,\di t')\, \di y' =
  \di t'\, \di x'\, \di y' \ .
\end{aligned}
\end{gather}
The charge-current density can then be rewritten as
\begin{equation}
  \label{eq:charge-current_coordtransf}
  \begin{split}
\bm{Q} &=  \rho\ \ti xyz
  - j_{x}\ \ti tyz - j_{y}\ \ti tzx - j_{z}\ \ti txy
  \\[2\jot]
  &=\rho\ \ti x'y'z'
  - (j_{x} - \rho\ v)\ \ti t'y'z'
  - j_{y}\ \ti t'z'x' - j_{z}\ \ti t'x'y' \ ,
\end{split}
\end{equation}
which is indeed the correct transformation for the charge density and the $x$-component of the current density \autocites[\eqn~(5.8)]{kovetz2000}. It is important to note that we did not make any assumptions regarding spacetime symmetries and metric. The transformation~\eqref{eq:coord_change_galilei} is a Galilei boost between Galileian inertial frames, if we assume Newtonian relativity, and a non-symmetry preserving coordinate transformation in Lorentzian or general relativity. So there is no contradiction with any of these theories. If we assume that Lorentzian relativity holds and $(t,x,y,z)$ is a Lorentzian inertial frame, then a metric-preserving transformation would instead be
\begin{equation}
  \label{eq:coord_charge_lorentz}
  \begin{gathered}
  (t', x', y', z') =
  \Bigl((t-x\,v/c^{2})/\gamma,\ (x-vt)/\gamma,\ y,\ z \Bigr) \ ,
  \\ \gamma \defd \sqrt{1-v^{2}/c^{2}}\ ,
\end{gathered}
\end{equation}
and a calculation similar to the previous one shows that the components of the charge-density would again transform as expected  \autocites[\eqns~(12.17)--(12.18)]{kovetz2000}.

The law of charge conservation is simply expressed by
\begin{equation}
  \label{eq:charge_conservation}
  \di\bm{Q} = 0 \ ,
\end{equation}
which leads to, considering permutations and antisymmetry,
\begin{equation}
  \label{eq:charge_cons_calculated}
  \begin{split}
0=\di\bm{Q} &= \de_{t}\rho\ \ti txyz 
  -\de_{x}j_{x}\ \ti xtyz
  -\de_{y}j_{y}\ \ti ytzx
  -\de_{x}j_{x}\ \ti ztxy
  \\
  &=\bigl(\de_{t}\rho + \de_{x}j_{x} + \de_{y}j_{y} + \de_{z}j_{z} \bigr)\
  \ti txyz \ ,
\end{split}
\end{equation}
implying the familiar \autocites[\eqn~(1.14)]{kovetz2000}
\begin{equation}
  \label{eq:familiar_charge_cons}
  \de_{t}\rho + \de_{x}j_{x} + \de_{y}j_{y} + \de_{z}j_{z} = 0 \ ,
\end{equation}
but now shown to be \emph{valid in any coordinate system}.

All we have done in this section holds also for mass density and mass flux. It is important to keep mass flux and momentum separate, as they are not the same in Lorentzian and general relativity.

\subsection{Electromagnetic field}
\label{sec:electromagnetic_field}

We can represent the electric field and magnetic flux in one geometrical entity as well:
\begin{equation}
  \label{eq:Faraday}
  \begin{split}
    \bm{F} &\defd
    B_{x}\ \di yz + B_{y}\ \di zx + B_{z}\ \di xy
  - E_{x}\ \di tx - E_{y}\ \di ty - E_{z}\ \di tz
    \\
    &\equiv
    B_{x}\ \di yz + B_{y}\ \di zx + B_{z}\ \di xy -
    \di t\ \bigl(E_{x}\ \di x + E_{y}\ \di y + E_{z}\ \di z \bigr)\ .
\end{split}
\end{equation}

This object automatically gives us the net magnetic flux, when integrated on a surface at a chosen time, or the time-integrated voltage, when integrated on a curve over a lapse of time. 


Under the coordinate transformation~\eqref{eq:coord_change_galilei} we have
\begin{equation}
  \label{eq:transf_2form_galilei}
  \begin{gathered}
    \di t\,\di x = \di t'\,\di x' \ ,\quad
    \di t\,\di y = \di t'\,\di y' \ ,\quad
    \di t\,\di z = \di t'\,\di z' \ ,
    \\
    \di y\,\di z = \di y'\,\di z' \ ,\\
    \di z\,\di x = \di z'\,\di x' - v\,\di t'\,\di z'\ ,\quad
    \di x\,\di y = \di x'\,\di y' + v\,\di t'\,\di y' \ ,
  \end{gathered}
\end{equation}
and
\begin{equation}
  \label{eq:transf_faraday}
  \begin{split}
    \bm{F} &= 
    \!\begin{aligned}[t]
    &B_{x}\ \di y'z' + B_{y}\ (\di z'x' - v\,\di t'z')
    + B_{z}\ (\di x'y' + v\,\di t'y') \\
    &\quad {}- E_{x}\ \di t'x' - E_{y}\ \di t'y' - E_{z}\ \di t'z'
  \end{aligned}
  \\
    &\equiv
    \!\begin{aligned}[t]
    &B_{x}\ \di y'z' + B_{y}\ \di z'x' + B_{z}\ \di x'y' \\
    &\quad {} - E_{x}\ \di t'x' 
   - (E_{y} - v\,B_{z})\ \di t'y' - (E_{z} + v\,B_{y})\ \di t'z' \ .
  \end{aligned}
\end{split}
\end{equation}
which is again as expected in the Galileian case \autocites[\eqn~(11.3)]{kovetz2000}.

The conservation of electromagnetic flux is simply expressed by
\begin{equation}
  \label{eq:cons_EM}
  \di \bm{F} = 0 \ ,
\end{equation}
which in coordinates becomes, keeping only terms that will not vanish owing to antisymmetry,
\begin{equation}
  \label{eq:cons_EM_coords}
  \begin{split}
    0 = \di\bm{F} &=
    \!\begin{aligned}[t]
     & (\de_{t}B_{x}\ \di t + \de_{x}B_{x}\ \di x)\ \di yz\\
    &{}+ (\de_{t}B_{y}\ \di t + \de_{y}B_{y}\ \di y)\ \di zx\\
    &{}+  (\de_{t}B_{z}\ \di t + \de_{z}B_{z}\ \di z)\ \di xy\\
    &{}- (\de_{y}E_{x}\ \di y + \de_{z}E_{x}\ \di z)\ \di tx\\
    &{}- (\de_{z}E_{y}\ \di z + \de_{x}E_{y}\ \di x)\ \di ty\\
    &{}- (\de_{x}E_{z}\ \di x + \de_{y}E_{z}\ \di y)\ \di tz
    \end{aligned}
  \\[\jot]
  &=
  \!\begin{aligned}[t]
&(\de_{x}B_{x} + \de_{y}B_{y} + \de_{z}B_{z})\ \di xyz \\
  &{}+ (\de_{t}B_{x} + \de_{y}E_{z} - \de_{z}E_{y})\ \di tyz \\
  &{}+ (\de_{t}B_{y} + \de_{z}E_{x} - \de_{x}E_{z})\ \di tzx \\
  &{}+ (\de_{t}B_{z} + \de_{x}E_{y} - \de_{y}E_{x})\ \di txy \ ,
\end{aligned}
  \end{split}
\end{equation}
where all four components must vanish, implying the familiar
\begin{equation}
  \label{eq:cons_EM_familiar}
  \begin{aligned}
    \de_{x}B_{x} + \de_{y}B_{y} + \de_{z}B_{z} &= 0 \\[\jot]
  \de_{t}B_{x} + \de_{y}E_{z} - \de_{z}E_{y} &=0 \\
  \de_{t}B_{y} + \de_{z}E_{x} - \de_{x}E_{z} &=0 \\
  \de_{t}B_{z} + \de_{x}E_{y} - \de_{y}E_{x} &=0 \ .
\end{aligned}
\end{equation}
Also these equations are \emph{valid in any coordinate system}.



\iffalse
\section{Orientation choices}
\label{sec:orientation_choices}

Let's examine several conventions, keeping in mind the rules found in  \cites[\eqns~(5)--(11)]{burke1983}:
\begin{enumerate}[label=(\roman*),para]
\item (3+1)D, order $(t,x,y,z)$:\\
  $\set[\big]{\ti x} = tyz$,\;  $\set[\big]{\ti t} = -xyz$,\;
  $\set[\big]{\ti tx} = yz$,\;  $\set[\big]{\ti xy } = tz$,\;
  $\set[\big]{\ti xyz}=t$.
\item (2+1)D, order $(t,x,y)$:\\
  $\set[\big]{\ti x} = -ty$,\;  $\set[\big]{\ti t} = xy$,\;
  $\set[\big]{\ti tx} = y$,\;  $\set[\big]{\ti xy } = t$.
\item (3+1)D, order $(x,y,z,t)$:\\
  $\set[\big]{\ti x} = -yzt$,\;  $\set[\big]{\ti t} = xyz$,\;
  $\set[\big]{\ti xt} = yz$,\;  $\set[\big]{\ti xy } = zt$,\;
    $\set[\big]{\ti xyz}=-t$.
\item (2+1)D, order $(x,y,t)$:\\
  $\set[\big]{\ti x} = yt$,\;  $\set[\big]{\ti t} = xy$,\;
  $\set[\big]{\ti xt} = -y$,\;  $\set[\big]{\ti xy } = t$.
\end{enumerate}
It is convenient to keep $t$ factors always to either right or left, because this allows us to factor them in sums without worries about signs. One thing to be noted with this general choice is that we get either $\set[\big]{\ti t}=-xyz$ or $\set[\big]{\ti xyz}=-t$: one minus sign will appear in either case.
\begin{equation}
  \label{eq:table_orientation}
  \begin{aligned}
    3+1,&\ txyz &    2+1,&\ txy &
    3+1,&\ xyzt &    2+1,&\ xyt 
\\[\jot]
    \set[\big]{\ti t} &= -xyz & \set[\big]{\ti t} &= xy &
    \set[\big]{\ti t} &= xyz & \set[\big]{\ti t} &= xy 
\\
    \set[\big]{\ti x} &= tyz & \set[\big]{\ti x} &= -ty &
    \set[\big]{\ti x} &= -yzt & \set[\big]{\ti x} &= yt 
\\
    \set[\big]{\ti y} &= tzx & \set[\big]{\ti y} &= tx &
    \set[\big]{\ti y} &= -zxt & \set[\big]{\ti y} &= -xt 
\\[\jot]
    \set[\big]{\ti tx} &= yz & \set[\big]{\ti tx} &= y &
    \set[\big]{\ti xt} &= yz & \set[\big]{\ti xt} &= -y 
\\
    \set[\big]{\ti ty} &= zx & \set[\big]{\ti ty} &= -x &
    \set[\big]{\ti yt} &= zx & \set[\big]{\ti yt} &= x 
\\
    \set[\big]{\ti xy } &= tz & \set[\big]{\ti xy } &= t &
    \set[\big]{\ti xy } &= zt & \set[\big]{\ti xy } &= t 
\\
    \set[\big]{\ti zx } &= ty & & &
    \set[\big]{\ti zx } &= ty & & 
\\[\jot]
    \set[\big]{\ti xyz}&=t & & &
    \set[\big]{\ti xyz}&=-t & & 
\\
    \set[\big]{\ti txy}&=-z & & &
    \set[\big]{\ti xyt}&=z & & 
\\[\jot]
    \set[\big]{\gamma\cdot\de_{t}} &= -t &
    \set[\big]{\gamma\cdot\de_{t}} &= t &
    \set[\big]{\gamma\cdot\de_{t}} &= -t &
    \set[\big]{\gamma\cdot\de_{t}} &= t 
\\
    \set[\big]{\gamma\cdot\de_{x}} &= -x &
    \set[\big]{\gamma\cdot\de_{x}} &= x &
    \set[\big]{\gamma\cdot\de_{x}} &= -x &
    \set[\big]{\gamma\cdot\de_{x}} &= x 
\\
    \set[\big]{\gamma\cdot\de_{tx}} &= tx &
    \set[\big]{\gamma\cdot\de_{tx}} &= tx &
    \set[\big]{\gamma\cdot\de_{xt}} &= xt &
    \set[\big]{\gamma\cdot\de_{xt}} &= xt 
\\
    \set[\big]{\gamma\cdot\de_{xy}} &= xy &
    \set[\big]{\gamma\cdot\de_{xy}} &= xy &
    \set[\big]{\gamma\cdot\de_{xy}} &= xy &
    \set[\big]{\gamma\cdot\de_{xt}} &= xy 
\\
    \set[\big]{\gamma\cdot\de_{xyz}} &= -xyz & & &
    \set[\big]{\gamma\cdot\de_{xyz}} &= -xyz & & 
\\
    %
    \set[\big]{\de_{t}\cdot\gamma} &= t &
    \set[\big]{\de_{t}\cdot\gamma} &= t &
    \set[\big]{\de_{t}\cdot\gamma} &= t &
    \set[\big]{\de_{t}\cdot\gamma} &= t 
\\
    \set[\big]{\de_{x}\cdot\gamma} &= x &
    \set[\big]{\de_{x}\cdot\gamma} &= x &
    \set[\big]{\de_{x}\cdot\gamma} &= x &
    \set[\big]{\de_{x}\cdot\gamma} &= x 
\\
    \set[\big]{\de_{tx}\cdot\gamma} &= tx &
    \set[\big]{\de_{tx}\cdot\gamma} &= tx &
    \set[\big]{\de_{xt}\cdot\gamma} &= xt &
    \set[\big]{\de_{xt}\cdot\gamma} &= xt 
\\
    \set[\big]{\de_{xy}\cdot\gamma} &= xy &
    \set[\big]{\de_{xy}\cdot\gamma} &= xy &
    \set[\big]{\de_{xy}\cdot\gamma} &= xy &
    \set[\big]{\de_{xt}\cdot\gamma} &= xy 
\\
    \set[\big]{\de_{xyz}\cdot\gamma} &= xyz & & &
    \set[\big]{\de_{xyz}\cdot\gamma} &= xyz & & 
  \end{aligned}
\end{equation}
\fi


%%%% examples use empheq
%   \begin{empheq}[left={\mathllap{\begin{aligned}    \de\yF_{\yc}/\de\yp&=0\text{:} \\
%         \de\yF_{\yc}/\de\ym&=0\text{:}\\ \de\yF_{\yc}/\de\yl&=0\text{:}\end{aligned}}\qquad}\empheqlbrace]{align}
%     \label{eq:con_p}
% %    \de\yF_{\yc}/\de\yp &\equiv
%     -\ln\yp + \ln\yq + \yl\yM + \ym\yu &=0,\\
%     \label{eq:con_u}
% %    \de\yF_{\yc}/\de\ym &\equiv
%     \yu\yp-1 &=0,\\
%     \label{eq:con_l}
%     %\de\yF_{\yc}/\de\yl &\equiv
%     \yM\yp-\yc &=0.
%   \end{empheq}
%%%%
% \begin{empheq}[box=\widefbox]{equation}
%   \label{eq:maxent_question}
%   \p\bigl[\yE{N+1}{k} \bigcond \tsum\yo\yf{N}\in\yA, \yM\bigr] = \mathord{?}
% \end{empheq}



% \[
%   \begin{tikzcd}
%       M_{n,n}(\CC) \arrow{r}{R'_{a}(\Hat{U})} & M_{n,n}(\CC)
%     \\
%     L(\mathcal{H}) \arrow{r}{\Hat{U}} \arrow[swap]{d}{R_*}\arrow[swap]{u}{R'_*} & L(\mathcal{H}) \arrow{d}{R_*}\arrow{u}{R'_*} \\
%       M_{n,n}(\CC) \arrow{r}{R_{a}(\Hat{U})} & M_{n,n}(\CC)
%   \end{tikzcd}
% \]

% \[
%   \begin{tikzcd}
%       \CC^n \arrow{r}{R'_*(A)} & \CC^n
%     \\
%     \mathcal{H} \arrow{r}{A} \arrow[swap]{d}{R}\arrow[swap]{u}{R'} & \mathcal{H} \arrow{d}{R}\arrow{u}{R'} \\
%       \CC^n \arrow{r}{R_*(A)} & \CC^n
%   \end{tikzcd}
% \]


% \[
%   \begin{tikzcd}
%     \mathcal{H} \arrow{r}{A} \arrow[swap]{d}{R} & \mathcal{H} \arrow{d}{R} \\
%       \CC^n \arrow{r}{R_*(A)} & \CC^n
%   \end{tikzcd}
% \]

%%\setlength{\intextsep}{0ex}% with wrapfigure
%%\setlength{\columnsep}{0ex}% with wrapfigure
%\begin{figure}[p!]% with figure
%\begin{wrapfigure}{r}{0.4\linewidth} % with wrapfigure
%  \centering\includegraphics[trim={12ex 0 18ex 0},clip,width=\linewidth]{maxent_saddle.png}\\
%\caption{caption}\label{fig:comparison_a5}
%\end{figure}% exp_family_maxent.nb


%%%%%%%%%%%%%%%%%%%%%%%%%%%%%%%%%%%%%%%%%%%%%%%%%%%%%%%%%%%%%%%%%%%%%%%%%%%%
%%% Acknowledgements
%%%%%%%%%%%%%%%%%%%%%%%%%%%%%%%%%%%%%%%%%%%%%%%%%%%%%%%%%%%%%%%%%%%%%%%%%%%% 
\iffalse
\begin{acknowledgements}
  \ldots to Mari \amp\ Miri for continuous encouragement and affection, and
  to Buster Keaton and Saitama for filling life with awe and inspiration.
  To the developers and maintainers of \LaTeX, Emacs, AUC\TeX, Open Science
  Framework, R, Python, Inkscape, Sci-Hub for making a free and impartial
  scientific exchange possible.
  % Our work was supported by the Trond Mohn Research Foundation, grant number BFS2018TMT07
%\rotatebox{15}{P}\rotatebox{5}{I}\rotatebox{-10}{P}\rotatebox{10}{\reflectbox{P}}\rotatebox{-5}{O}.
%\sourceatright{\autanet}
\mbox{}\hfill\autanet
\end{acknowledgements}
\fi

%%%%%%%%%%%%%%%%%%%%%%%%%%%%%%%%%%%%%%%%%%%%%%%%%%%%%%%%%%%%%%%%%%%%%%%%%%%%
%%% Appendices
%%%%%%%%%%%%%%%%%%%%%%%%%%%%%%%%%%%%%%%%%%%%%%%%%%%%%%%%%%%%%%%%%%%%%%%%%%%% 
%\clearpage
\bigskip
% %\renewcommand*{\appendixpagename}{Appendix}
% %\renewcommand*{\appendixname}{Appendix}
% %\appendixpage
% \appendix

%%%%%%%%%%%%%%%%%%%%%%%%%%%%%%%%%%%%%%%%%%%%%%%%%%%%%%%%%%%%%%%%%%%%%%%%%%%%
%%% Bibliography
%%%%%%%%%%%%%%%%%%%%%%%%%%%%%%%%%%%%%%%%%%%%%%%%%%%%%%%%%%%%%%%%%%%%%%%%%%%% 
\renewcommand*{\finalnamedelim}{\addcomma\space}
\defbibnote{prenote}{{\footnotesize (\enquote{de $X$} is listed under D,
    \enquote{van $X$} under V, and so on, regardless of national
    conventions.)\par}}
% \defbibnote{postnote}{\par\medskip\noindent{\footnotesize% Note:
%     \arxivp \mparcp \philscip \biorxivp}}

\printbibliography[prenote=prenote%,postnote=postnote
]

\end{document}

%%%%%%%%%%%%%%%%%%%%%%%%%%%%%%%%%%%%%%%%%%%%%%%%%%%%%%%%%%%%%%%%%%%%%%%%%%%%
%%% Cut text (won't be compiled)
%%%%%%%%%%%%%%%%%%%%%%%%%%%%%%%%%%%%%%%%%%%%%%%%%%%%%%%%%%%%%%%%%%%%%%%%%%%% 


%%% Local Variables: 
%%% mode: LaTeX
%%% TeX-PDF-mode: t
%%% TeX-master: t
%%% End: 
