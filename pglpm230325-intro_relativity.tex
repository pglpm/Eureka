\pdfoutput=1
%% Author: PGL  Porta Mana
%% Created: 2015-05-01T20:53:34+0200
%% Last-Updated: 2023-03-27T17:50:54+0200
%%%%%%%%%%%%%%%%%%%%%%%%%%%%%%%%%%%%%%%%%%%%%%%%%%%%%%%%%%%%%%%%%%%%%%%%%%%%
\newif\ifarxiv
\arxivfalse
\iftrue\pdfmapfile{+classico.map}\fi
\newif\ifafour
\afourfalse% true = A4, false = A5
\newif\iftypodisclaim % typographical disclaim on the side
\typodisclaimtrue
\newcommand*{\memfontfamily}{zplx}
\newcommand*{\memfontpack}{newpxtext}
\documentclass[\ifafour a4paper,12pt,\else a5paper,10pt,\fi%extrafontsizes,%
onecolumn,oneside,article,%french,italian,german,swedish,latin,
british%
]{memoir}
\newcommand*{\firstdraft}{25 March 2023}
\newcommand*{\firstpublished}{\firstdraft}
\newcommand*{\updated}{\ifarxiv***\else\today\ [draft]\fi}
\newcommand*{\propertitle}{Introduction and teaching of relativity theory%\\{\large ***}%
}% title uses LARGE; set Large for smaller
\newcommand*{\pdftitle}{\propertitle}
\newcommand*{\headtitle}{Intro to relativity}
\newcommand*{\pdfauthor}{P.G.L.  Porta Mana}
\newcommand*{\headauthor}{Porta Mana}
\newcommand*{\reporthead}{\ifarxiv\else Open Science Framework \href{https://doi.org/10.31219/osf.io/***}{\textsc{doi}:10.31219/osf.io/***}\fi}% Report number

%%%%%%%%%%%%%%%%%%%%%%%%%%%%%%%%%%%%%%%%%%%%%%%%%%%%%%%%%%%%%%%%%%%%%%%%%%%%
%%% Calls to packages (uncomment as needed)
%%%%%%%%%%%%%%%%%%%%%%%%%%%%%%%%%%%%%%%%%%%%%%%%%%%%%%%%%%%%%%%%%%%%%%%%%%%%

%\usepackage{pifont}

%\usepackage{fontawesome}

\usepackage[T1]{fontenc} 
\input{glyphtounicode} \pdfgentounicode=1

\usepackage[utf8]{inputenx}

%\usepackage{newunicodechar}
% \newunicodechar{Ĕ}{\u{E}}
% \newunicodechar{ĕ}{\u{e}}
% \newunicodechar{Ĭ}{\u{I}}
% \newunicodechar{ĭ}{\u{\i}}
% \newunicodechar{Ŏ}{\u{O}}
% \newunicodechar{ŏ}{\u{o}}
% \newunicodechar{Ŭ}{\u{U}}
% \newunicodechar{ŭ}{\u{u}}
% \newunicodechar{Ā}{\=A}
% \newunicodechar{ā}{\=a}
% \newunicodechar{Ē}{\=E}
% \newunicodechar{ē}{\=e}
% \newunicodechar{Ī}{\=I}
% \newunicodechar{ī}{\={\i}}
% \newunicodechar{Ō}{\=O}
% \newunicodechar{ō}{\=o}
% \newunicodechar{Ū}{\=U}
% \newunicodechar{ū}{\=u}
% \newunicodechar{Ȳ}{\=Y}
% \newunicodechar{ȳ}{\=y}

\newcommand*{\bmmax}{0} % reduce number of bold fonts, before font packages
\newcommand*{\hmmax}{0} % reduce number of heavy fonts, before font packages

\usepackage{textcomp}

%\usepackage[normalem]{ulem}% package for underlining
% \makeatletter
% \def\ssout{\bgroup \ULdepth=-.35ex%\UL@setULdepth
%  \markoverwith{\lower\ULdepth\hbox
%    {\kern-.03em\vbox{\hrule width.2em\kern1.2\p@\hrule}\kern-.03em}}%
%  \ULon}
% \makeatother

\usepackage{amsmath}

\usepackage{mathtools}
%\addtolength{\jot}{\jot} % increase spacing in multiline formulae
\setlength{\multlinegap}{0pt}

%\usepackage{empheq}% automatically calls amsmath and mathtools
%\newcommand*{\widefbox}[1]{\fbox{\hspace{1em}#1\hspace{1em}}}

%%%% empheq above seems more versatile than these:
%\usepackage{fancybox}
%\usepackage{framed}

% \usepackage[misc]{ifsym} % for dice
% \newcommand*{\diceone}{{\scriptsize\Cube{1}}}

\usepackage{amssymb}

\usepackage{amsxtra}

\usepackage[main=british]{babel}\selectlanguage{british}
%\newcommand*{\langnohyph}{\foreignlanguage{nohyphenation}}
\newcommand{\langnohyph}[1]{\begin{hyphenrules}{nohyphenation}#1\end{hyphenrules}}

\usepackage[autostyle=false,autopunct=false,english=british]{csquotes}
\setquotestyle{american}
\newcommand*{\defquote}[1]{`\,#1\,'}

% \makeatletter
% \renewenvironment{quotation}%
%                {\list{}{\listparindent 1.5em%
%                         \itemindent    \listparindent
%                         \rightmargin=1em   \leftmargin=1em
%                         \parsep        \z@ \@plus\p@}%
%                 \item[]\footnotesize}%
%                 {\endlist}
% \makeatother                


\usepackage{amsthm}
%% from https://tex.stackexchange.com/a/404680/97039
\makeatletter
\def\@endtheorem{\endtrivlist}
\makeatother

\newcommand*{\QED}{\textsc{q.e.d.}}
\renewcommand*{\qedsymbol}{\QED}
\theoremstyle{remark}
\newtheorem{note}{Note}
\newtheorem*{remark}{Note}
\newtheoremstyle{innote}{\parsep}{\parsep}{\footnotesize}{}{}{}{0pt}{}
\theoremstyle{innote}
\newtheorem*{innote}{}

\usepackage[shortlabels,inline]{enumitem}
\SetEnumitemKey{para}{itemindent=\parindent,leftmargin=0pt,listparindent=\parindent,parsep=0pt,itemsep=\topsep}
% \begin{asparaenum} = \begin{enumerate}[para]
% \begin{inparaenum} = \begin{enumerate*}
\setlist{itemsep=0pt,topsep=\parsep}
\setlist[enumerate,2]{label=(\roman*)}
\setlist[enumerate]{label=(\alph*),leftmargin=1.5\parindent}
\setlist[itemize]{leftmargin=1.5\parindent}
\setlist[description]{leftmargin=1.5\parindent}
% old alternative:
% \setlist[enumerate,2]{label=\alph*.}
% \setlist[enumerate]{leftmargin=\parindent}
% \setlist[itemize]{leftmargin=\parindent}
% \setlist[description]{leftmargin=\parindent}

\usepackage[babel,theoremfont,largesc]{newpxtext}

% For Baskerville see https://ctan.org/tex-archive/fonts/baskervillef?lang=en
% and http://mirrors.ctan.org/fonts/baskervillef/doc/baskervillef-doc.pdf
% \usepackage[p]{baskervillef}
% \usepackage[varqu,varl,var0]{inconsolata}
% \usepackage[scale=.95,type1]{cabin}
% \usepackage[baskerville,vvarbb]{newtxmath}
% \usepackage[cal=boondoxo]{mathalfa}


\usepackage[bigdelims,nosymbolsc%,smallerops % probably arXiv doesn't have it
]{newpxmath}
%\useosf
%\linespread{1.083}%
%\linespread{1.05}% widely used
\linespread{1.1}% best for text with maths
%% smaller operators for old version of newpxmath
\makeatletter
\def\re@DeclareMathSymbol#1#2#3#4{%
    \let#1=\undefined
    \DeclareMathSymbol{#1}{#2}{#3}{#4}}
%\re@DeclareMathSymbol{\bigsqcupop}{\mathop}{largesymbols}{"46}
%\re@DeclareMathSymbol{\bigodotop}{\mathop}{largesymbols}{"4A}
\re@DeclareMathSymbol{\bigoplusop}{\mathop}{largesymbols}{"4C}
\re@DeclareMathSymbol{\bigotimesop}{\mathop}{largesymbols}{"4E}
\re@DeclareMathSymbol{\sumop}{\mathop}{largesymbols}{"50}
\re@DeclareMathSymbol{\prodop}{\mathop}{largesymbols}{"51}
\re@DeclareMathSymbol{\bigcupop}{\mathop}{largesymbols}{"53}
\re@DeclareMathSymbol{\bigcapop}{\mathop}{largesymbols}{"54}
%\re@DeclareMathSymbol{\biguplusop}{\mathop}{largesymbols}{"55}
\re@DeclareMathSymbol{\bigwedgeop}{\mathop}{largesymbols}{"56}
\re@DeclareMathSymbol{\bigveeop}{\mathop}{largesymbols}{"57}
%\re@DeclareMathSymbol{\bigcupdotop}{\mathop}{largesymbols}{"DF}
%\re@DeclareMathSymbol{\bigcapplusop}{\mathop}{largesymbolsPXA}{"00}
%\re@DeclareMathSymbol{\bigsqcupplusop}{\mathop}{largesymbolsPXA}{"02}
%\re@DeclareMathSymbol{\bigsqcapplusop}{\mathop}{largesymbolsPXA}{"04}
%\re@DeclareMathSymbol{\bigsqcapop}{\mathop}{largesymbolsPXA}{"06}
\re@DeclareMathSymbol{\bigtimesop}{\mathop}{largesymbolsPXA}{"10}
%\re@DeclareMathSymbol{\coprodop}{\mathop}{largesymbols}{"60}
%\re@DeclareMathSymbol{\varprod}{\mathop}{largesymbolsPXA}{16}
\makeatother
%%
%% With euler font cursive for Greek letters - the [1] means 100% scaling
\DeclareFontFamily{U}{egreek}{\skewchar\font'177}%
\DeclareFontShape{U}{egreek}{m}{n}{<-6>s*[1]eurm5 <6-8>s*[1]eurm7 <8->s*[1]eurm10}{}%
\DeclareFontShape{U}{egreek}{m}{it}{<->s*[1]eurmo10}{}%
\DeclareFontShape{U}{egreek}{b}{n}{<-6>s*[1]eurb5 <6-8>s*[1]eurb7 <8->s*[1]eurb10}{}%
\DeclareFontShape{U}{egreek}{b}{it}{<->s*[1]eurbo10}{}%
\DeclareSymbolFont{egreeki}{U}{egreek}{m}{it}%
\SetSymbolFont{egreeki}{bold}{U}{egreek}{b}{it}% from the amsfonts package
\DeclareSymbolFont{egreekr}{U}{egreek}{m}{n}%
\SetSymbolFont{egreekr}{bold}{U}{egreek}{b}{n}% from the amsfonts package
% Take also \sum, \prod, \coprod symbols from Euler fonts
\DeclareFontFamily{U}{egreekx}{\skewchar\font'177}
\DeclareFontShape{U}{egreekx}{m}{n}{%
       <-7.5>s*[0.9]euex7%
    <7.5-8.5>s*[0.9]euex8%
    <8.5-9.5>s*[0.9]euex9%
    <9.5->s*[0.9]euex10%
}{}
\DeclareSymbolFont{egreekx}{U}{egreekx}{m}{n}
\DeclareMathSymbol{\sumop}{\mathop}{egreekx}{"50}
\DeclareMathSymbol{\prodop}{\mathop}{egreekx}{"51}
\DeclareMathSymbol{\coprodop}{\mathop}{egreekx}{"60}
\makeatletter
\def\sum{\DOTSI\sumop\slimits@}
\def\prod{\DOTSI\prodop\slimits@}
\def\coprod{\DOTSI\coprodop\slimits@}
\makeatother
\input{definegreek.tex}% Greek letters not usually given in LaTeX.

%\usepackage%[scaled=0.9]%
%{classico}%  Optima as sans-serif font
\renewcommand\sfdefault{uop}
\DeclareMathAlphabet{\mathsf}  {T1}{\sfdefault}{m}{sl}
\SetMathAlphabet{\mathsf}{bold}{T1}{\sfdefault}{b}{sl}
\newcommand*{\mathte}[1]{\textbf{\textit{\textsf{#1}}}}
% Upright sans-serif math alphabet
% \DeclareMathAlphabet{\mathsu}  {T1}{\sfdefault}{m}{n}
% \SetMathAlphabet{\mathsu}{bold}{T1}{\sfdefault}{b}{n}

% DejaVu Mono as typewriter text
\usepackage[scaled=0.84]{DejaVuSansMono}

\usepackage{mathdots}

\usepackage[usenames]{xcolor}
% Tol (2012) colour-blind-, print-, screen-friendly colours, alternative scheme; Munsell terminology
\definecolor{bluepurple}{RGB}{68,119,170}
\definecolor{blue}{RGB}{102,204,238}
\definecolor{green}{RGB}{34,136,51}
\definecolor{yellow}{RGB}{204,187,68}
\definecolor{red}{RGB}{238,102,119}
\definecolor{redpurple}{RGB}{170,51,119}
\definecolor{grey}{RGB}{187,187,187}
% Tol (2012) colour-blind-, print-, screen-friendly colours; Munsell terminology
% \definecolor{lbpurple}{RGB}{51,34,136}
% \definecolor{lblue}{RGB}{136,204,238}
% \definecolor{lbgreen}{RGB}{68,170,153}
% \definecolor{lgreen}{RGB}{17,119,51}
% \definecolor{lgyellow}{RGB}{153,153,51}
% \definecolor{lyellow}{RGB}{221,204,119}
% \definecolor{lred}{RGB}{204,102,119}
% \definecolor{lpred}{RGB}{136,34,85}
% \definecolor{lrpurple}{RGB}{170,68,153}
\definecolor{lgrey}{RGB}{221,221,221}
%\newcommand*\mycolourbox[1]{%
%\colorbox{grey}{\hspace{1em}#1\hspace{1em}}}
\colorlet{shadecolor}{lgrey}

\usepackage{bm}

\usepackage{microtype}

\usepackage[backend=biber,mcite,%subentry,
citestyle=authoryear-comp,bibstyle=pglpm-authoryear,autopunct=false,sorting=ny,sortcites=false,natbib=false,maxcitenames=2,maxbibnames=8,minbibnames=8,giveninits=true,uniquename=false,uniquelist=false,maxalphanames=1,block=space,hyperref=true,defernumbers=false,useprefix=true,sortupper=false,language=british,parentracker=false,autocite=footnote]{biblatex}
\DeclareSortingTemplate{ny}{\sort{\field{sortname}\field{author}\field{editor}}\sort{\field{year}}}
\DeclareFieldFormat{postnote}{#1}
\iffalse\makeatletter%%% replace parenthesis with brackets
\newrobustcmd*{\parentexttrack}[1]{%
  \begingroup
  \blx@blxinit
  \blx@setsfcodes
  \blx@bibopenparen#1\blx@bibcloseparen
  \endgroup}
\AtEveryCite{%
  \let\parentext=\parentexttrack%
  \let\bibopenparen=\bibopenbracket%
  \let\bibcloseparen=\bibclosebracket}
\makeatother\fi
\DefineBibliographyExtras{british}{\def\finalandcomma{\addcomma}}
\renewcommand*{\finalnamedelim}{\addspace\amp\space}
% \renewcommand*{\finalnamedelim}{\addcomma\space}
\renewcommand*{\textcitedelim}{\addcomma\space}
% \setcounter{biburlnumpenalty}{1} % to allow url breaks anywhere
% \setcounter{biburlucpenalty}{0}
% \setcounter{biburllcpenalty}{1}
\DeclareDelimFormat{multicitedelim}{\addsemicolon\addspace\space}
\DeclareDelimFormat{compcitedelim}{\addsemicolon\addspace\space}
\DeclareDelimFormat{postnotedelim}{\addspace}
\ifarxiv\else\addbibresource{portamanabib.bib}\fi
\renewcommand{\bibfont}{\footnotesize}
%\appto{\citesetup}{\footnotesize}% smaller font for citations
\defbibheading{bibliography}[\bibname]{\section*{#1}\addcontentsline{toc}{section}{#1}%\markboth{#1}{#1}
}
\newcommand*{\citep}{\footcites}
\newcommand*{\citey}{\footcites}%{\parencites*}
\newcommand*{\ibid}{\unspace\addtocounter{footnote}{-1}\footnotemark{}}
%\renewcommand*{\cite}{\parencite}
%\renewcommand*{\cites}{\parencites}
\providecommand{\href}[2]{#2}
\providecommand{\eprint}[2]{\texttt{\href{#1}{#2}}}
\newcommand*{\amp}{\&}
% \newcommand*{\citein}[2][]{\textnormal{\textcite[#1]{#2}}%\addtocategory{extras}{#2}
% }
\newcommand*{\citein}[2][]{\textnormal{\textcite[#1]{#2}}%\addtocategory{extras}{#2}
}
\newcommand*{\citebi}[2][]{\textcite[#1]{#2}%\addtocategory{extras}{#2}
}
\newcommand*{\subtitleproc}[1]{}
\newcommand*{\chapb}{ch.}
%
%\def\UrlOrds{\do\*\do\-\do\~\do\'\do\"\do\-}%
\def\myUrlOrds{\do\0\do\1\do\2\do\3\do\4\do\5\do\6\do\7\do\8\do\9\do\a\do\b\do\c\do\d\do\e\do\f\do\g\do\h\do\i\do\j\do\k\do\l\do\m\do\n\do\o\do\p\do\q\do\r\do\s\do\t\do\u\do\v\do\w\do\x\do\y\do\z\do\A\do\B\do\C\do\D\do\E\do\F\do\G\do\H\do\I\do\J\do\K\do\L\do\M\do\N\do\O\do\P\do\Q\do\R\do\S\do\T\do\U\do\V\do\W\do\X\do\Y\do\Z}%
\makeatletter
%\g@addto@macro\UrlSpecials{\do={\newline}}
\g@addto@macro{\UrlBreaks}{\myUrlOrds}
\makeatother
\newcommand*{\arxiveprint}[1]{%
arXiv \doi{10.48550/arXiv.#1}%
}
\newcommand*{\mparceprint}[1]{%
\href{http://www.ma.utexas.edu/mp_arc-bin/mpa?yn=#1}{mp\_arc:\allowbreak\nolinkurl{#1}}%
}
\newcommand*{\haleprint}[1]{%
\href{https://hal.archives-ouvertes.fr/#1}{\textsc{hal}:\allowbreak\nolinkurl{#1}}%
}
\newcommand*{\philscieprint}[1]{%
\href{http://philsci-archive.pitt.edu/archive/#1}{PhilSci:\allowbreak\nolinkurl{#1}}%
}
\newcommand*{\doi}[1]{%
\href{https://doi.org/#1}{\textsc{doi}:\allowbreak\nolinkurl{#1}}%
}
\newcommand*{\biorxiveprint}[1]{%
bioRxiv \doi{10.1101/#1}%
}
\newcommand*{\osfeprint}[1]{%
Open Science Framework \doi{10.31219/osf.io/#1}%
}
\newcommand*{\osfproj}[1]{%
Open Science Framework \doi{10.17605/osf.io/#1}%
}

\usepackage{graphicx}

\usepackage{wrapfig}

%\usepackage{tikz-cd}

\PassOptionsToPackage{hyphens}{url}\usepackage[hypertexnames=false,pdfencoding=unicode,psdextra]{hyperref}

\usepackage[depth=4]{bookmark}
\hypersetup{colorlinks=true,bookmarksnumbered,pdfborder={0 0 0.25},citebordercolor={0.2667 0.4667 0.6667},citecolor=bluepurple,linkbordercolor={0.6667 0.2 0.4667},linkcolor=redpurple,urlbordercolor={0.1333 0.5333 0.2},urlcolor=green,breaklinks=true,pdftitle={\pdftitle},pdfauthor={\pdfauthor}}
% \usepackage[vertfit=local]{breakurl}% only for arXiv
\providecommand*{\urlalt}{\href}

\usepackage[british]{datetime2}
\DTMnewdatestyle{mydate}%
{% definitions
\renewcommand*{\DTMdisplaydate}[4]{%
\number##3\ \DTMenglishmonthname{##2} ##1}%
\renewcommand*{\DTMDisplaydate}{\DTMdisplaydate}%
}
\DTMsetdatestyle{mydate}

%%%%%%%%%%%%%%%%%%%%%%%%%%%%%%%%%%%%%%%%%%%%%%%%%%%%%%%%%%%%%%%%%%%%%%%%%%%%
%%% Layout. I do not know on which kind of paper the reader will print the
%%% paper on (A4? letter? one-sided? double-sided?). So I choose A5, which
%%% provides a good layout for reading on screen and save paper if printed
%%% two pages per sheet. Average length line is 66 characters and page
%%% numbers are centred.
%%%%%%%%%%%%%%%%%%%%%%%%%%%%%%%%%%%%%%%%%%%%%%%%%%%%%%%%%%%%%%%%%%%%%%%%%%%%
\ifafour\setstocksize{297mm}{210mm}%{*}% A4
\else\setstocksize{210mm}{5.5in}%{*}% 210x139.7
\fi
\settrimmedsize{\stockheight}{\stockwidth}{*}
\setlxvchars[\normalfont] %313.3632pt for a 66-characters line
\setxlvchars[\normalfont]
% \setlength{\trimtop}{0pt}
% \setlength{\trimedge}{\stockwidth}
% \addtolength{\trimedge}{-\paperwidth}
%\settrims{0pt}{0pt}
% The length of the normalsize alphabet is 133.05988pt - 10 pt = 26.1408pc
% The length of the normalsize alphabet is 159.6719pt - 12pt = 30.3586pc
% Bringhurst gives 32pc as boundary optimal with 69 ch per line
% The length of the normalsize alphabet is 191.60612pt - 14pt = 35.8634pc
\ifafour\settypeblocksize{*}{32pc}{1.618} % A4
%\setulmargins{*}{*}{1.667}%gives 5/3 margins % 2 or 1.667
\else\settypeblocksize{*}{26pc}{1.618}% nearer to a 66-line newpx and preserves GR
\fi
\setulmargins{*}{*}{1}%gives equal margins
\setlrmargins{*}{*}{*}
\setheadfoot{\onelineskip}{2.5\onelineskip}
\setheaderspaces{*}{2\onelineskip}{*}
\setmarginnotes{2ex}{10mm}{0pt}
\checkandfixthelayout[nearest]
%%% End layout
%% this fixes missing white spaces
%\pdfmapline{+dummy-space <dummy-space.pfb}
%\pdfinterwordspaceon% seems to add a white margin to Sumatrapdf

%%% Sectioning
\newcommand*{\asudedication}[1]{%
{\par\centering\textit{#1}\par}}
\newenvironment{acknowledgements}{\section*{Thanks}\addcontentsline{toc}{section}{Thanks}}{\par}
\makeatletter\renewcommand{\appendix}{\par
  \bigskip{\centering
   \interlinepenalty \@M
   \normalfont
   \printchaptertitle{\sffamily\appendixpagename}\par}
  \setcounter{section}{0}%
  \gdef\@chapapp{\appendixname}%
  \gdef\thesection{\@Alph\c@section}%
  \anappendixtrue}\makeatother
\counterwithout{section}{chapter}
\setsecnumformat{\upshape\csname the#1\endcsname\quad}
\setsecheadstyle{\large\bfseries\sffamily%
\centering}
\setsubsecheadstyle{\bfseries\sffamily%
\raggedright}
%\setbeforesecskip{-1.5ex plus 1ex minus .2ex}% plus 1ex minus .2ex}
%\setaftersecskip{1.3ex plus .2ex }% plus 1ex minus .2ex}
%\setsubsubsecheadstyle{\bfseries\sffamily\slshape\raggedright}
%\setbeforesubsecskip{1.25ex plus 1ex minus .2ex }% plus 1ex minus .2ex}
%\setaftersubsecskip{-1em}%{-0.5ex plus .2ex}% plus 1ex minus .2ex}
\setsubsecindent{0pt}%0ex plus 1ex minus .2ex}
\setparaheadstyle{\bfseries\sffamily%
\raggedright}
\setcounter{secnumdepth}{2}
\setlength{\headwidth}{\textwidth}
\newcommand{\addchap}[1]{\chapter*[#1]{#1}\addcontentsline{toc}{chapter}{#1}}
\newcommand{\addsec}[1]{\section*{#1}\addcontentsline{toc}{section}{#1}}
\newcommand{\addsubsec}[1]{\subsection*{#1}\addcontentsline{toc}{subsection}{#1}}
\newcommand{\addpara}[1]{\paragraph*{#1.}\addcontentsline{toc}{subsubsection}{#1}}
\newcommand{\addparap}[1]{\paragraph*{#1}\addcontentsline{toc}{subsubsection}{#1}}

%%% Headers, footers, pagestyle
\copypagestyle{manaart}{plain}
\makeheadrule{manaart}{\headwidth}{0.5\normalrulethickness}
\makeoddhead{manaart}{%
{\footnotesize%\sffamily%
\scshape\headauthor}}{}{{\footnotesize\sffamily%
\headtitle}}
\makeoddfoot{manaart}{}{\thepage}{}
\newcommand*\autanet{\includegraphics[height=\heightof{M}]{autanet.pdf}}
\definecolor{mygray}{gray}{0.333}
\iftypodisclaim%
\ifafour\newcommand\addprintnote{\begin{picture}(0,0)%
\put(245,149){\makebox(0,0){\rotatebox{90}{\tiny\color{mygray}\textsf{This
            document is designed for screen reading and
            two-up printing on A4 or Letter paper}}}}%
\end{picture}}% A4
\else\newcommand\addprintnote{\begin{picture}(0,0)%
\put(176,112){\makebox(0,0){\rotatebox{90}{\tiny\color{mygray}\textsf{This
            document is designed for screen reading and
            two-up printing on A4 or Letter paper}}}}%
\end{picture}}\fi%afourtrue
\makeoddfoot{plain}{}{\makebox[0pt]{\thepage}\addprintnote}{}
\else
\makeoddfoot{plain}{}{\makebox[0pt]{\thepage}}{}
\fi%typodisclaimtrue
\makeoddhead{plain}{\scriptsize\reporthead}{}{}
% \copypagestyle{manainitial}{plain}
% \makeheadrule{manainitial}{\headwidth}{0.5\normalrulethickness}
% \makeoddhead{manainitial}{%
% \footnotesize\sffamily%
% \scshape\headauthor}{}{\footnotesize\sffamily%
% \headtitle}
% \makeoddfoot{manaart}{}{\thepage}{}

\pagestyle{manaart}

\setlength{\droptitle}{-3.9\onelineskip}
\pretitle{\begin{center}\LARGE\sffamily%
\bfseries}
\posttitle{\bigskip\end{center}}

\makeatletter\newcommand*{\atf}{\includegraphics[totalheight=\heightof{@}]{atblack.png}}\makeatother
\providecommand{\affiliation}[1]{\textsl{\textsf{\footnotesize #1}}}
\providecommand{\epost}[1]{\texttt{\footnotesize\textless#1\textgreater}}
\providecommand{\email}[2]{\href{mailto:#1ZZ@#2 ((remove ZZ))}{#1\protect\atf#2}}
%\providecommand{\email}[2]{\href{mailto:#1@#2}{#1@#2}}

\preauthor{\vspace{-0.5\baselineskip}\begin{center}
\normalsize\sffamily%
\lineskip  0.5em}
\postauthor{\par\end{center}}
\predate{\DTMsetdatestyle{mydate}\begin{center}\footnotesize}
\postdate{\end{center}\vspace{-\medskipamount}}

\setfloatadjustment{figure}{\footnotesize}
\captiondelim{\quad}
\captionnamefont{\footnotesize\sffamily%
}
\captiontitlefont{\footnotesize}
%\firmlists*
\midsloppy
% handling orphan/widow lines, memman.pdf
% \clubpenalty=10000
% \widowpenalty=10000
% \raggedbottom
% Downes, memman.pdf
\clubpenalty=9996
\widowpenalty=9999
\brokenpenalty=4991
\predisplaypenalty=10000
\postdisplaypenalty=1549
\displaywidowpenalty=1602
\raggedbottom

\paragraphfootnotes
\setlength{\footmarkwidth}{2ex}
% \threecolumnfootnotes
%\setlength{\footmarksep}{0em}
\footmarkstyle{\textsuperscript{%\color{red}
\scriptsize\bfseries#1}~}
%\footmarkstyle{\textsuperscript{\color{red}\scriptsize\bfseries#1}~}
%\footmarkstyle{\textsuperscript{[#1]}~}

\selectlanguage{british}\frenchspacing

\definecolor{notecolour}{RGB}{68,170,153}
%\newcommand*{\puzzle}{\maltese}
\newcommand*{\puzzle}{{\fontencoding{U}\fontfamily{fontawesometwo}\selectfont\symbol{225}}}
\newcommand*{\wrench}{{\fontencoding{U}\fontfamily{fontawesomethree}\selectfont\symbol{114}}}
\newcommand*{\pencil}{{\fontencoding{U}\fontfamily{fontawesometwo}\selectfont\symbol{210}}}
\newcommand{\mynotew}[1]{{\footnotesize\color{notecolour}\wrench\ #1}}
\newcommand{\mynotep}[1]{{\footnotesize\color{notecolour}\pencil\ #1}}
\newcommand{\mynotez}[1]{{\footnotesize\color{notecolour}\puzzle\ #1}}

%%%%%%%%%%%%%%%%%%%%%%%%%%%%%%%%%%%%%%%%%%%%%%%%%%%%%%%%%%%%%%%%%%%%%%%%%%%%
%%% Paper's details
%%%%%%%%%%%%%%%%%%%%%%%%%%%%%%%%%%%%%%%%%%%%%%%%%%%%%%%%%%%%%%%%%%%%%%%%%%%%
\title{\propertitle}
\author{%
\hspace*{\stretch{1}}%
%% uncomment if additional authors present
% \parbox{0.5\linewidth}%\makebox[0pt][c]%
% {\protect\centering ***\\%
% \footnotesize\epost{\email{***}{***}}}%
% \hspace*{\stretch{1}}%
\parbox{1\linewidth}%\makebox[0pt][c]%
{\protect\centering P.G.L.  Porta Mana  \href{https://orcid.org/0000-0002-6070-0784}{\protect\includegraphics[scale=0.16]{orcid_32x32.png}}\\\footnotesize
Western Norway University of Applied Sciences%
\quad\epost{\email{pgl}{portamana.org}}}%
% Mohn Medical Imaging and Visualization Centre, Dept of Computer science, Electrical Engineering and Mathematical Sciences, Western Norway University of Applied Sciences, Bergen, Norway
%% uncomment if additional authors present
% \hspace*{\stretch{1}}%
% \parbox{0.5\linewidth}%\makebox[0pt][c]%
% {\protect\centering ***\\%
% \footnotesize\epost{\email{***}{***}}}%
\hspace*{\stretch{1}}%
}

%\date{Draft of \today\ (first drafted \firstdraft)}
\date{\firstpublished; updated \updated}

%%%%%%%%%%%%%%%%%%%%%%%%%%%%%%%%%%%%%%%%%%%%%%%%%%%%%%%%%%%%%%%%%%%%%%%%%%%%
%%% Macros @@@
%%%%%%%%%%%%%%%%%%%%%%%%%%%%%%%%%%%%%%%%%%%%%%%%%%%%%%%%%%%%%%%%%%%%%%%%%%%%

% Common ones - uncomment as needed
%\providecommand{\nequiv}{\not\equiv}
%\providecommand{\coloneqq}{\mathrel{\mathop:}=}
%\providecommand{\eqqcolon}{=\mathrel{\mathop:}}
%\providecommand{\varprod}{\prod}
\newcommand*{\de}{\partialup}%partial diff
\newcommand*{\pu}{\piup}%constant pi
\newcommand*{\delt}{\deltaup}%Kronecker, Dirac
%\newcommand*{\eps}{\varepsilonup}%Levi-Civita, Heaviside
%\newcommand*{\riem}{\zetaup}%Riemann zeta
%\providecommand{\degree}{\textdegree}% degree
%\newcommand*{\celsius}{\textcelsius}% degree Celsius
%\newcommand*{\micro}{\textmu}% degree Celsius
% \newcommand*{\I}{\mathrm{i}}%imaginary unit
\newcommand*{\I}{\ensuremath{\mathrm{i}}}
% \newcommand*{\e}{\mathrm{e}}%Neper
\newcommand*{\e}{\ensuremath{\mathrm{e}}}
% \newcommand*{\di}{\mathrm{d}}%differential
% \newcommand*{\dii}{\ensuremath{\mathrm{d}}}
%% From TUGboat 18 (1997) 1
\makeatletter
\providecommand*{\di}%
{\@ifnextchar^{\DIfF}{\DIfF^{}}}
\def\DIfF^#1{%
\mathop{\mathrm{\mathstrut d}}%
\nolimits^{#1}\gobblespace}
\def\gobblespace{%
\futurelet\diffarg\opspace}
\def\opspace{%
\let\DiffSpace\!%
\ifx\diffarg(%
\let\DiffSpace\relax
\else
\ifx\diffarg[%
\let\DiffSpace\relax
\else
\ifx\diffarg\{%
\let\DiffSpace\relax
\fi\fi\fi\DiffSpace}
\makeatother

% \newcommand*{\Di}{\mathrm{D}}%capital differential
%\newcommand*{\Li}{\mathrm{L}}%Lie derivative
%\newcommand*{\planckc}{\hslash}
%\newcommand*{\avogn}{N_{\textrm{A}}}
%\newcommand*{\NN}{\bm{\mathrm{N}}}
%\newcommand*{\ZZ}{\bm{\mathrm{Z}}}
%\newcommand*{\QQ}{\bm{\mathrm{Q}}}
\newcommand*{\RR}{\bm{\mathrm{R}}}
%\newcommand*{\CC}{\bm{\mathrm{C}}}
%\newcommand*{\nabl}{\bm{\nabla}}%nabla
%\DeclareMathOperator{\lb}{lb}%base 2 log
%\DeclareMathOperator{\tr}{tr}%trace
%\DeclareMathOperator{\card}{card}%cardinality
%% From TUGboat 18 (1997) 1
% \renewoperator{\Re}{\mathrm{Re}}{\nolimits}
% \renewoperator{\Im}{\mathrm{Im}}{\nolimits}
\DeclareMathOperator{\im}{Im}%im part
\DeclareMathOperator{\re}{Re}%re part
%\DeclareMathOperator{\sgn}{sgn}%signum
%\DeclareMathOperator{\ent}{ent}%integer less or equal to
\DeclareMathOperator{\Ord}{O}%same order as
%\DeclareMathOperator{\ord}{o}%lower order than
\newcommand*{\incr}{\triangle}%finite increment
\newcommand*{\defd}{\coloneqq}
\newcommand*{\defs}{\eqqcolon}
%\newcommand*{\Land}{\bigwedge}
%\newcommand*{\Lor}{\bigvee}
%\newcommand*{\lland}{\DOTSB\;\land\;}
%\newcommand*{\llor}{\DOTSB\;\lor\;}
\newcommand*{\limplies}{\mathbin{\Rightarrow}}%implies
%\newcommand*{\suchthat}{\mid}%{\mathpunct{|}}%such that (eg in sets)
%\newcommand*{\with}{\colon}%with (list of indices)
%\newcommand*{\mul}{\times}%multiplication
%\newcommand*{\inn}{\cdot}%inner product
%\newcommand*{\dotv}{\mathord{\,\cdot\,}}%variable place
%\newcommand*{\comp}{\circ}%composition of functions
%\newcommand*{\con}{\mathbin{:}}%scal prod of tensors
%\newcommand*{\equi}{\sim}%equivalent to 
\renewcommand*{\asymp}{\simeq}%equivalent to 
%\newcommand*{\corr}{\mathrel{\hat{=}}}%corresponds to
%\providecommand{\varparallel}{\ensuremath{\mathbin{/\mkern-7mu/}}}%parallel (tentative symbol)
\renewcommand*{\le}{\leqslant}%less or equal
\renewcommand*{\ge}{\geqslant}%greater or equal
%\DeclarePairedDelimiter\clcl{[}{]}
%\DeclarePairedDelimiter\clop{[}{[}
%\DeclarePairedDelimiter\opcl{]}{]}
%\DeclarePairedDelimiter\opop{]}{[}%}
\DeclarePairedDelimiter\abs{\lvert}{\rvert}
%\DeclarePairedDelimiter\norm{\lVert}{\rVert}
\DeclarePairedDelimiter\set{\{}{\}} %}
%\DeclareMathOperator{\pr}{P}%probability
\newcommand*{\p}{\mathrm{p}}%probability
\renewcommand*{\P}{\mathrm{P}}%probability
%\newcommand*{\E}{\mathrm{E}}
%% The "\:" space is chosen to correctly separate inner binary and external rels
\renewcommand*{\|}[1][]{\nonscript\:#1\vert\nonscript\:\mathopen{}}
%\DeclarePairedDelimiterX{\cp}[2]{(}{)}{#1\nonscript\:\delimsize\vert\nonscript\:\mathopen{}#2}
%\DeclarePairedDelimiterX{\ct}[2]{[}{]}{#1\nonscript\;\delimsize\vert\nonscript\:\mathopen{}#2}
%\DeclarePairedDelimiterX{\cs}[2]{\{}{\}}{#1\nonscript\:\delimsize\vert\nonscript\:\mathopen{}#2}
%\newcommand*{\+}{\lor}
%\renewcommand{\*}{\land}
%% symbol = for equality statements within probabilities
%% from https://tex.stackexchange.com/a/484142/97039
% \newcommand*{\eq}{\mathrel{\!=\!}}
% \let\texteq\=
% \renewcommand*{\=}{\TextOrMath\texteq\eq}
% \newcommand*{\eq}[1][=]{\mathrel{\!#1\!}}
\newcommand*{\mo}[1][=]{\mathord{\,#1\,}}
%%
\newcommand*{\sect}{\S}% Sect.~
\newcommand*{\sects}{\S\S}% Sect.~
\newcommand*{\chap}{ch.}%
\newcommand*{\chaps}{chs}%
\newcommand*{\bref}{ref.}%
\newcommand*{\brefs}{refs}%
%\newcommand*{\fn}{fn}%
\newcommand*{\eqn}{eq.}%
\newcommand*{\eqns}{eqs}%
\newcommand*{\fig}{fig.}%
\newcommand*{\figs}{figs}%
\newcommand*{\vs}{{vs}}
\newcommand*{\eg}{{e.g.}}
\newcommand*{\etc}{{etc.}}
\newcommand*{\ie}{{i.e.}}
%\newcommand*{\ca}{{c.}}
\newcommand*{\foll}{{ff.}}
%\newcommand*{\viz}{{viz}}
\newcommand*{\cf}{{cf.}}
%\newcommand*{\Cf}{{Cf.}}
%\newcommand*{\vd}{{v.}}
\newcommand*{\etal}{{et al.}}
%\newcommand*{\etsim}{{et sim.}}
%\newcommand*{\ibid}{{ibid.}}
%\newcommand*{\sic}{{sic}}
%\newcommand*{\id}{\mathte{I}}%id matrix
%\DeclareMathOperator{\id}{\textsf{\textbf{id}}}%probability
\newcommand*{\id}{\mathte{I}}
%\newcommand*{\nbd}{\nobreakdash}%
%\newcommand*{\bd}{\hspace{0pt}}%
%\def\hy{-\penalty0\hskip0pt\relax}
%\newcommand*{\labelbis}[1]{\tag*{(\ref{#1})$_\text{r}$}}
%\newcommand*{\mathbox}[2][.8]{\parbox[t]{#1\columnwidth}{#2}}
%\newcommand*{\zerob}[1]{\makebox[0pt][l]{#1}}
\newcommand*{\tprod}{\mathop{\textstyle\prod}\nolimits}
\newcommand*{\tsum}{\mathop{\textstyle\sum}\nolimits}
%\newcommand*{\tint}{\begingroup\textstyle\int\endgroup\nolimits}
%\newcommand*{\tland}{\mathop{\textstyle\bigwedge}\nolimits}
%\newcommand*{\tlor}{\mathop{\textstyle\bigvee}\nolimits}
%\newcommand*{\sprod}{\mathop{\textstyle\prod}}
%\newcommand*{\ssum}{\mathop{\textstyle\sum}}
%\newcommand*{\sint}{\begingroup\textstyle\int\endgroup}
%\newcommand*{\sland}{\mathop{\textstyle\bigwedge}}
%\newcommand*{\slor}{\mathop{\textstyle\bigvee}}
%\newcommand*{\T}{^\transp}%transpose
%%\newcommand*{\QEM}%{\textnormal{$\Box$}}%{\ding{167}}
%\newcommand*{\qem}{\leavevmode\unskip\penalty9999 \hbox{}\nobreak\hfill
%\quad\hbox{\QEM}}
%% from TUGboat 18 (1997) 1:
\providecommand*{\unit}[1]{\ensuremath{\mathrm{\,#1}}}

%%%%%%%%%%%%%%%%%%%%%%%%%%%%%%%%%%%%%%%%%%%%%%%%%%%%%%%%%%%%%%%%%%%%%%%%%%%%
%%% Custom macros for this file @@@
%%%%%%%%%%%%%%%%%%%%%%%%%%%%%%%%%%%%%%%%%%%%%%%%%%%%%%%%%%%%%%%%%%%%%%%%%%%%

\newcommand*{\widebar}[1]{{\mkern1.5mu\skew{2}\overline{\mkern-1.5mu#1\mkern-1.5mu}\mkern 1.5mu}}

% \newcommand{\explanation}[4][t]{%\setlength{\tabcolsep}{-1ex}
% %\smash{
% \begin{tabular}[#1]{c}#2\\[0.5\jot]\rule{1pt}{#3}\\#4\end{tabular}}%}
% \newcommand*{\ptext}[1]{\text{\small #1}}% for propositions
%\DeclareMathOperator*{\argsup}{arg\,sup}
\newcommand*{\dob}{degree of belief}
\newcommand*{\dobs}{degrees of belief}
\newcommand*{\yg}{\mathte{g}}
\newcommand*{\yv}{\bm{v}}
\newcommand*{\yu}{\bm{u}}
\newcommand*{\yvu}{\underline{\yv}}
\newcommand*{\yV}{\bm{V}}
%%% Custom macros end @@@

%%%%%%%%%%%%%%%%%%%%%%%%%%%%%%%%%%%%%%%%%%%%%%%%%%%%%%%%%%%%%%%%%%%%%%%%%%%%
%%% Beginning of document
%%%%%%%%%%%%%%%%%%%%%%%%%%%%%%%%%%%%%%%%%%%%%%%%%%%%%%%%%%%%%%%%%%%%%%%%%%%%
%\firmlists
\begin{document}
\captiondelim{\quad}\captionnamefont{\footnotesize}\captiontitlefont{\footnotesize}
\selectlanguage{british}\frenchspacing
\maketitle

%%%%%%%%%%%%%%%%%%%%%%%%%%%%%%%%%%%%%%%%%%%%%%%%%%%%%%%%%%%%%%%%%%%%%%%%%%%%
%%% Abstract
%%%%%%%%%%%%%%%%%%%%%%%%%%%%%%%%%%%%%%%%%%%%%%%%%%%%%%%%%%%%%%%%%%%%%%%%%%%%
\abstractrunin
\abslabeldelim{}
\renewcommand*{\abstractname}{}
\setlength{\absleftindent}{0pt}
\setlength{\absrightindent}{0pt}
\setlength{\abstitleskip}{-\absparindent}
\begin{abstract}\labelsep 0pt%
  \noindent Notes on a possible way to introduce and teach general, Lorentzian, and Newtonian relativity.
% \\\noindent\emph{\footnotesize Note: Dear Reader
%     \amp\ Peer, this manuscript is being peer-reviewed by you. Thank you.}
% \par%\\[\jot]
% \noindent
% {\footnotesize PACS: ***}\qquad%
% {\footnotesize MSC: ***}%
%\qquad{\footnotesize Keywords: ***}
\end{abstract}
\selectlanguage{british}\frenchspacing

%%%%%%%%%%%%%%%%%%%%%%%%%%%%%%%%%%%%%%%%%%%%%%%%%%%%%%%%%%%%%%%%%%%%%%%%%%%%
%%% Epigraph
%%%%%%%%%%%%%%%%%%%%%%%%%%%%%%%%%%%%%%%%%%%%%%%%%%%%%%%%%%%%%%%%%%%%%%%%%%%%
% \asudedication{\small ***}
% \vspace{\bigskipamount}
% \setlength{\epigraphwidth}{.7\columnwidth}
% %\epigraphposition{flushright}
% \epigraphtextposition{flushright}
% %\epigraphsourceposition{flushright}
% \epigraphfontsize{\footnotesize}
% \setlength{\epigraphrule}{0pt}
% %\setlength{\beforeepigraphskip}{0pt}
% %\setlength{\afterepigraphskip}{0pt}
% \epigraph{\emph{text}}{source}



%%%%%%%%%%%%%%%%%%%%%%%%%%%%%%%%%%%%%%%%%%%%%%%%%%%%%%%%%%%%%%%%%%%%%%%%%%%%
%%% BEGINNING OF MAIN TEXT
%%%%%%%%%%%%%%%%%%%%%%%%%%%%%%%%%%%%%%%%%%%%%%%%%%%%%%%%%%%%%%%%%%%%%%%%%%%%

\section{Motivation}
\label{sec:motivation}

The way Lorentzian and general relativity are introduced in many books leaves some uneasiness as regards some logical steps in the underlying reasoning -- although I feel no doubt about the internal logical consistency of the results. For example, if we come to the conclusion that metre sticks shortens or are seen shortened by some observer, then is our initial reasoning about some experiments, such as the Michelson-Morley one, still logically valid? How were those meters initially defined? If general relativity makes the notion of rigid body very tricky, then where does our initial reasoning involving rigid rods logically stand?

The present note tries to develop the theory from some postulates and notions that can be induced from experiments -- at least from thought experiments -- while avoiding logical loop-holes. It is only tentative, a step, probably still affected by inconsistencies.

The presentation is, for the moment, mostly in form of main points and self-reminders.

\section{Initial notions and postulates}
\label{sec:primitives}

We start with the notion of \enquote{event}, something that has very small extension in space and time. Intuitive, qualitative notions of \enquote{extension} should suffice for such a definition. Spacetime is introduced as the four-dimensional topological space made of such events. For the moment the only thing we can say about events is whether they are coincident (or very close, in a qualitative sense) or not.

Next we introduce the notion of a persistent clock. \enquote{Persistent} in the sense that it is not just an event, but rather a continuous sequence of events, which has a very small extension in space. In spacetime it can be represented by a one-dimensional curve, which is called a \enquote{worldline}. \enquote{Clock} in the sense that there is some kind of periodicity in this sequence of events. Note how tricky the notion of \enquote{periodicity} is, though: how do we establish that some phenomenon is periodic, without having a notion of clock? For the moment we must thus take \enquote{periodic} as a primitive notion. It's the refinement of our own biological feeling of \enquote{equal time intervals}. We call such a persistent clock an \enquote{observer}.

Next we make an experimentally verifiable assumption. Whenever two observers coincide -- they have coincident worldlines -- and their clocks are of the same making, then they notice and exact equality of the periods of the two clocks. The absolute times shown by the clocks -- the number of periods counted from some initial moment -- may be different, but the period intervals are the same.

Let us now imagine that two such observers are coincident in some initial and final parts of their worldlines, but not in all their internal parts. Let us also assume that in their initial, coincident part, their clocks are synchronized, in the sense that they give the same absolute time, not just same time intervals. Experimentally it is observed that if the worldlines diverge and then come to coincide again, at the point of reunion the two clocks will have generally different absolute times (though both future times with respect to the initial point of worldline separation), but they have again same time intervals in this new coincident part of the curve. Generalizing to three or more observers, in general we would find three or more different absolute times at the point of reunion.

This experimental fact is at variance with Newtonian relativity, which instead assumes that in such a situation the two clocks would still have identical absolute times upon reuniting.

We also take as primitive the possibility of sending light signals from a given observer, using some kind of device.

Another experimentally verifiable assumption is that if two coincident observers send a light signal in the same direction, using light-sending devices of the same making, then the two signals also have coincident worldlines.

This experimental assumption is extended, however. Suppose we have two observers with non-coincident worldlines, which however intersect in one event. At that event, both observers send light signals in the same direction. Then also in this case the two light signals have coincident worldlines. This is an experimentally verifiable assumption.

This last assumption is extremely important because it selects, in spacetime, a special set of worldlines and of \enquote{lightcones}. A lightcone is the set of worldlines of light signals stemming from the same event but having different directions.

The last assumption could have been restated as saying that the velocity of a light signal is independent of the velocity of its source. But we have thus far happily avoided the tricky notions of \enquote{velocity} and \enquote{distance}.


\section{Remarks on the differential-geometric representation}
\label{sec:diff_geo_intro}

To each event we can also associate scalar values representing measurement results, thus obtaining scalar-valued functions over regions of the spacetime. We can also consider functions from a one-dimensional scalar space into spacetime. Spacetime with its collection of possible functions from and into can thus be represented by a differential manifold. The functions are so chosen that, if we \enquote{zoom in} on a region around an event, the level hypersurfaces of the functions from the manifold look like hyperplanes (or a constant value over the whole region), and the images of the functions onto the manifold look like straight lines (or points).

This also implies that if we consider a tangent vector $\yv$ at the event $P$, we can approximately associate another event close to $P$ with the vector $\epsilon\,\yv$, if $\epsilon$ is small enough. The event approximately associated with $\epsilon\,\yv$ is the one having coordinates $x^{i}(P) + \epsilon\,\di x^{i}(P)\cdot\yv + \Ord(\epsilon^{2})$ in \emph{any} coordinate system $(x^{i})$ admitted by the differential manifold. This is possible because coordinate transformations only affect $\Ord(\epsilon^{2})$ terms.

\section{Metric and radar coordinates}
\label{sec:metric_radar_coords}

The metric (field) $\yg$ encodes the length of physical time intervals for all observers. Given the worldline $s \mapsto C(s)$ of an observer between events $C(s_{0})$ and $C(s_{1})$, the physical time elapsed for that observer between those two events is given by
\begin{equation}
  \label{eq:proper_time_PQ}
  \int_{s_{0}}^{s_{1}}
  \frac{1}{c}\sqrt{\abs{\dot{C}(s)\,\yg[C(s)]\,\dot{C}(s)}\,}\,\di s \ .
\end{equation}

If $\yv$ is the tangent vector at some event to a worldline, then the physical time interval between that event and one roughly corresponding to a parameter increment $\incr s$ along the worldline is $\frac{1}{c}\sqrt{\abs{\yv\,\yg\,\yv}}\,\incr s + \Ord(\incr s^{2})$, where $\yg$ is the metric at the first event. We can normalize $\yv$ in such a way that the increment $\incr s$ itself is the physical time interval; this corresponds to a local change of the worldline parameter $s$. With such normalization we must then have
\begin{equation}
  \label{eq:normalize_4-vel}
  \abs{\yv\,\yg\,\yv} = c^{2} \quad\text{if $\yv$ is normalized},
\end{equation}
and we call $\yv$ the local \emph{4-velocity}. Note that this normalization assumes that the metric has dimensions of a squared-length, and that the 4-velocity has dimensions of inverse-time. Any vector $\yu$ tangent to a worldline can be normalized by multiplying it by $c/\sqrt{\abs{\yu\,\yg\,\yu}}$. Such multiplication also gives it dimensions of inverse-time.

Given the 4-velocity $\yv$ of an observer at an event $P$, there's a special three-dimensional set of events in the spatial and temporal vicinity of $P$. Each event in this set has the property that a light signal sent by the observer a small interval of time $\incr s$ before $P$, and bouncing off the event, reaches back the observer an interval $\incr s$ after $P$, except for differences $\Ord(\incr s^{2})$. Such events are said to be locally orthogonal to the observer's 4-velocity $\yv$ at $P$. The observer can conventionally consider these local neighbouring events as \emph{simultaneous} to $P$.

To each such orthogonal or simultaneous neighbouring event, the observer can also associate a distance, by convention. The distance is defined in such a way that the \enquote{ordinary} velocity of the signal reaching the event is equal to $c$. The distance must therefore be defined as $\incr s/c$.

The constructions above can be made with respect to events on the worldline of the observer very close to $P$. This leads to a local set of coordinates in the neighbourhood of $P$ called \emph{radar coordinates}. The observer can therefore conventionally associate a \enquote{time} and a \enquote{distance}, and thus a \enquote{velocity}, to all events or worldlines in the neighbourhood of $P$.

We can associate a covector
\begin{equation}
  \label{eq:co-4-velocity}
  \yvu \defd \frac{\yv\yg}{\yv\yg\yv}
\end{equation}
to a 4-velocity $\yv$, defined by the following properties:
\begin{enumerate*}[label=(\alph*)]
\item contracted with $\yv$ it yields unity: $\yvu\yv=1$,
\item it yields zero when applied to any vector orthogonal to $\yv$.
\end{enumerate*}
We call it the \enquote*{4-covelocity}.

The 4-velocity and 4-covelocity allow us to associate a conventional radar-distance and radar-time to events in the neighbourhood of an event $P$ on a worldline with 4-velocity $\yv$ at $P$. For a neighbouring event approximately identified by the small tangent vector $\epsilon\yu$, its time with respect to $P$ is $\epsilon\yvu\yu$, and its position vector is $\epsilon\,(\id - \yv\otimes\yvu)\,\yu$.
If another worldline in the neighbourhood of $P$ has tangent vector $\yu$, then its conventional radar-coordinate velocity $\yV$ from the point of view of the observer with 4-velocity $\yu$ is
\begin{equation}
  \label{eq:3-velocity}
\yV \defd  \frac{(\id - \yv\otimes\yvu)\,\yu}{\yvu\yu}
\equiv
  \frac{\yu - \yv\,(\yvu\yu)}{\yvu\yu}
  \ .
\end{equation}
This expression is valid even if $\yu$ is not normalized. In radar coordinates with respect to $\yv$, the 4-velocity of the second worldline has components
\begin{equation}
  \label{eq:components_4-vel}
  \Bigl(1-\tfrac{\yV^{2}}{c^{2}}\Bigr)^{-\frac12}\,
  (1, \yV) \ .
\end{equation}

These formulae also reveal the \enquote{time dilation} and \enquote{length contraction} phenomena. A physical-time lapse of $\epsilon$ for the observer with 4-velocity $\yu$ has a larger \emph{radar-time lapse} of $\epsilon\yvu\yu = \epsilon/(1-\yV^{2}/c^{2})$ according to $\yv$, since $1-\yV^{2}/c^{2} \le 1$. That is,
% observer $\yv$ attributes a \emph{conventional} time difference larger than $\epsilon$ to two events to which observer $\yu$ attributes a \emph{physical} time difference of $\epsilon$. Or alternatively,
observer $\yv$ attributes a \emph{conventional} time lapse larger than $\epsilon$ to a \emph{physical} time lapse of $\epsilon$ on the clock of $\yu$, which therefore \enquote{runs slowly} according to the conventional time reckoning of $\yv$. It should again be emphasized that the radar-time lapse is just a conventional time coordinate, not a physical one. Also note that radar-time is not the time at which $\yv$ \emph{sees} other events.

\medskip

\setlength{\intextsep}{0ex}% with wrapfigure
\setlength{\columnsep}{2ex}% with wrapfigure
\begin{wrapfigure}{r}{0.3\linewidth} % with wrapfigure
\centering\includegraphics[width=\linewidth]{radar_distance1.jpg}
%  \\\caption{caption}\label{fig:radar_distance}
\end{wrapfigure}%
The extent of the neighbourhood around an event $P$ in which radar distances and velocities are well-defined depends on the problem considered; in general it cannot be extended indefinitely. It is important that the worldline of the observer measuring the radar discance can be considered straight, otherwise the result will not make much sense. In the figure on the right, for example, the observer with worldline \textcolor{red}{$D$} (\textcolor{red}{dashed red}) would attribute a shorter radar distance between events $P$ and $Q$, than the observer with worldline \textcolor{bluepurple}{$C$} (\textcolor{bluepurple}{solid blue}) would, even if both observers have the same tangent vectors at $P$ and at the events in which the light signals (\textcolor{yellow}{dot-dashed yellow}) are sent and received.

Other kinds of \enquote{distances} and \enquote{velocities} can be \emph{defined} in more extended regions. But it is important to keep in mind that all such definitions have an element of arbitrariness, and Newtonian intuition should not be applied in such situations. Speaking of the \enquote{distances} and \enquote{velocity} of faraway galaxies, for example, is a subtle matter relying on many conventions and prone to misunderstandings \autocites[see \eg][]{davisetal2003,davisetal2004,groen2004,weber2004}. It would probably be much better if less intuitive and less misleading terms were used instead.


\mynotep{Notion of \enquote{constant distance or position} with respect to an observer?}

\section{Energy, momentum, stress, and amount of matter}
\label{sec:stress_matter}

Before discussing the notions of amount of matter, force, momentum, energy, it is best to clarify and emphasize important differences, but also important similarities, between Newtonian and Lorentzian relativity.

In Newtonian relativity we somehow associate matter and mass.

\mynotew{}


% \[ \color{bluepurple}\bm{a} \color{redpurple}\mathbin{\bm{\land}}\color{bluepurple} \bm{b} \]


%%%% examples use empheq
%   \begin{empheq}[left={\mathllap{\begin{aligned}    \de\yF_{\yc}/\de\yp&=0\text{:} \\
%         \de\yF_{\yc}/\de\ym&=0\text{:}\\ \de\yF_{\yc}/\de\yl&=0\text{:}\end{aligned}}\qquad}\empheqlbrace]{align}
%     \label{eq:con_p}
% %    \de\yF_{\yc}/\de\yp &\equiv
%     -\ln\yp + \ln\yq + \yl\yM + \ym\yu &=0,\\
%     \label{eq:con_u}
% %    \de\yF_{\yc}/\de\ym &\equiv
%     \yu\yp-1 &=0,\\
%     \label{eq:con_l}
%     %\de\yF_{\yc}/\de\yl &\equiv
%     \yM\yp-\yc &=0.
%   \end{empheq}
%%%%
% \begin{empheq}[box=\widefbox]{equation}
%   \label{eq:maxent_question}
%   \p\bigl[\yE{N+1}{k} \bigcond \tsum\yo\yf{N}\in\yA, \yM\bigr] = \mathord{?}
% \end{empheq}



% \[
%   \begin{tikzcd}
%       M_{n,n}(\CC) \arrow{r}{R'_{a}(\Hat{U})} & M_{n,n}(\CC)
%     \\
%     L(\mathcal{H}) \arrow{r}{\Hat{U}} \arrow[swap]{d}{R_*}\arrow[swap]{u}{R'_*} & L(\mathcal{H}) \arrow{d}{R_*}\arrow{u}{R'_*} \\
%       M_{n,n}(\CC) \arrow{r}{R_{a}(\Hat{U})} & M_{n,n}(\CC)
%   \end{tikzcd}
% \]

% \[
%   \begin{tikzcd}
%       \CC^n \arrow{r}{R'_*(A)} & \CC^n
%     \\
%     \mathcal{H} \arrow{r}{A} \arrow[swap]{d}{R}\arrow[swap]{u}{R'} & \mathcal{H} \arrow{d}{R}\arrow{u}{R'} \\
%       \CC^n \arrow{r}{R_*(A)} & \CC^n
%   \end{tikzcd}
% \]


% \[
%   \begin{tikzcd}
%     \mathcal{H} \arrow{r}{A} \arrow[swap]{d}{R} & \mathcal{H} \arrow{d}{R} \\
%       \CC^n \arrow{r}{R_*(A)} & \CC^n
%   \end{tikzcd}
% \]

%%\setlength{\intextsep}{0ex}% with wrapfigure
%%\setlength{\columnsep}{0ex}% with wrapfigure
%\begin{figure}[p!]% with figure
%\begin{wrapfigure}{r}{0.4\linewidth} % with wrapfigure
%  \centering\includegraphics[trim={12ex 0 18ex 0},clip,width=\linewidth]{maxent_saddle.png}\\
%\caption{caption}\label{fig:comparison_a5}
%\end{figure}% exp_family_maxent.nb


%%%%%%%%%%%%%%%%%%%%%%%%%%%%%%%%%%%%%%%%%%%%%%%%%%%%%%%%%%%%%%%%%%%%%%%%%%%%
%%% Acknowledgements
%%%%%%%%%%%%%%%%%%%%%%%%%%%%%%%%%%%%%%%%%%%%%%%%%%%%%%%%%%%%%%%%%%%%%%%%%%%% 
\iffalse
\begin{acknowledgements}
  \ldots to Mari \amp\ Miri for continuous encouragement and affection, and
  to Buster Keaton and Saitama for filling life with awe and inspiration.
  To the developers and maintainers of \LaTeX, Emacs, AUC\TeX, Open Science
  Framework, R, Python, Inkscape, Sci-Hub for making a free and impartial
  scientific exchange possible.
  % Our work was supported by the Trond Mohn Research Foundation, grant number BFS2018TMT07
%\rotatebox{15}{P}\rotatebox{5}{I}\rotatebox{-10}{P}\rotatebox{10}{\reflectbox{P}}\rotatebox{-5}{O}.
%\sourceatright{\autanet}
\mbox{}\hfill\autanet
\end{acknowledgements}
\fi

%%%%%%%%%%%%%%%%%%%%%%%%%%%%%%%%%%%%%%%%%%%%%%%%%%%%%%%%%%%%%%%%%%%%%%%%%%%%
%%% Appendices
%%%%%%%%%%%%%%%%%%%%%%%%%%%%%%%%%%%%%%%%%%%%%%%%%%%%%%%%%%%%%%%%%%%%%%%%%%%% 
%\clearpage
\bigskip
% %\renewcommand*{\appendixpagename}{Appendix}
% %\renewcommand*{\appendixname}{Appendix}
% %\appendixpage
% \appendix

%%%%%%%%%%%%%%%%%%%%%%%%%%%%%%%%%%%%%%%%%%%%%%%%%%%%%%%%%%%%%%%%%%%%%%%%%%%%
%%% Bibliography
%%%%%%%%%%%%%%%%%%%%%%%%%%%%%%%%%%%%%%%%%%%%%%%%%%%%%%%%%%%%%%%%%%%%%%%%%%%% 
\renewcommand*{\finalnamedelim}{\addcomma\space}
\defbibnote{prenote}{{\footnotesize (\enquote{de $X$} is listed under D,
    \enquote{van $X$} under V, and so on, regardless of national
    conventions.)\par}}
% \defbibnote{postnote}{\par\medskip\noindent{\footnotesize% Note:
%     \arxivp \mparcp \philscip \biorxivp}}

\printbibliography[prenote=prenote%,postnote=postnote
]

\end{document}

%%%%%%%%%%%%%%%%%%%%%%%%%%%%%%%%%%%%%%%%%%%%%%%%%%%%%%%%%%%%%%%%%%%%%%%%%%%%
%%% Cut text (won't be compiled)
%%%%%%%%%%%%%%%%%%%%%%%%%%%%%%%%%%%%%%%%%%%%%%%%%%%%%%%%%%%%%%%%%%%%%%%%%%%% 


%%% Local Variables: 
%%% mode: LaTeX
%%% TeX-PDF-mode: t
%%% TeX-master: t
%%% End: 
