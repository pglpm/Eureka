\pdfoutput=1
%% Author: PGL  Porta Mana
%% Created: 2020-07-12T16:31:50+0200
%% Last-Updated: 2020-12-22T17:30:09+0100
%%%%%%%%%%%%%%%%%%%%%%%%%%%%%%%%%%%%%%%%%%%%%%%%%%%%%%%%%%%%%%%%%%%%%%%%%%%%
\newif\ifarxiv
\arxivfalse
\ifarxiv\pdfmapfile{+classico.map}\fi
\newif\ifafour
\afourfalse% true = A4, false = A5
\newif\iftypodisclaim % typographical disclaim on the side
\typodisclaimtrue
\newcommand*{\memfontfamily}{zplx}
\newcommand*{\memfontpack}{newpxtext}
\documentclass[\ifafour a4paper,12pt,\else a5paper,10pt,\fi%extrafontsizes,%
onecolumn,oneside,article,%french,italian,german,swedish,latin,
british%
]{memoir}
\newcommand*{\firstdraft}{23 July 2020}
\newcommand*{\firstpublished}{\firstdraft}
\newcommand*{\updated}{\ifarxiv***\else\today\fi}
\newcommand*{\propertitle}{\enquote{Affine} ideal gas {\large(Notes for Mila)}%
}% title uses LARGE; set Large for smaller
\newcommand*{\pdftitle}{\propertitle}
\newcommand*{\headtitle}{\enquote{Affine} ideal gas}
\newcommand*{\pdfauthor}{P.G.L.  Porta Mana}
\newcommand*{\headauthor}{Porta Mana}
\newcommand*{\reporthead}{\ifarxiv\else Open Science Framework \href{https://doi.org/10.31219/osf.io/***}{\textsc{doi}:10.31219/osf.io/***}\fi}% Report number

%%%%%%%%%%%%%%%%%%%%%%%%%%%%%%%%%%%%%%%%%%%%%%%%%%%%%%%%%%%%%%%%%%%%%%%%%%%%
%%% Calls to packages (uncomment as needed)
%%%%%%%%%%%%%%%%%%%%%%%%%%%%%%%%%%%%%%%%%%%%%%%%%%%%%%%%%%%%%%%%%%%%%%%%%%%%

%\usepackage{pifont}

%\usepackage{fontawesome}

\usepackage[T1]{fontenc} 
\input{glyphtounicode} \pdfgentounicode=1

\usepackage[utf8]{inputenx}

%\usepackage{newunicodechar}
% \newunicodechar{Ĕ}{\u{E}}
% \newunicodechar{ĕ}{\u{e}}
% \newunicodechar{Ĭ}{\u{I}}
% \newunicodechar{ĭ}{\u{\i}}
% \newunicodechar{Ŏ}{\u{O}}
% \newunicodechar{ŏ}{\u{o}}
% \newunicodechar{Ŭ}{\u{U}}
% \newunicodechar{ŭ}{\u{u}}
% \newunicodechar{Ā}{\=A}
% \newunicodechar{ā}{\=a}
% \newunicodechar{Ē}{\=E}
% \newunicodechar{ē}{\=e}
% \newunicodechar{Ī}{\=I}
% \newunicodechar{ī}{\={\i}}
% \newunicodechar{Ō}{\=O}
% \newunicodechar{ō}{\=o}
% \newunicodechar{Ū}{\=U}
% \newunicodechar{ū}{\=u}
% \newunicodechar{Ȳ}{\=Y}
% \newunicodechar{ȳ}{\=y}

\newcommand*{\bmmax}{0} % reduce number of bold fonts, before font packages
\newcommand*{\hmmax}{0} % reduce number of heavy fonts, before font packages

\usepackage{textcomp}

%\usepackage[normalem]{ulem}% package for underlining
% \makeatletter
% \def\ssout{\bgroup \ULdepth=-.35ex%\UL@setULdepth
%  \markoverwith{\lower\ULdepth\hbox
%    {\kern-.03em\vbox{\hrule width.2em\kern1.2\p@\hrule}\kern-.03em}}%
%  \ULon}
% \makeatother

\usepackage{amsmath}

\usepackage{mathtools}
%\addtolength{\jot}{\jot} % increase spacing in multiline formulae
\setlength{\multlinegap}{0pt}

%\usepackage{empheq}% automatically calls amsmath and mathtools
%\newcommand*{\widefbox}[1]{\fbox{\hspace{1em}#1\hspace{1em}}}

%%%% empheq above seems more versatile than these:
%\usepackage{fancybox}
%\usepackage{framed}

% \usepackage[misc]{ifsym} % for dice
% \newcommand*{\diceone}{{\scriptsize\Cube{1}}}

\usepackage{amssymb}

\usepackage{amsxtra}

\usepackage[main=british]{babel}\selectlanguage{british}
%\newcommand*{\langnohyph}{\foreignlanguage{nohyphenation}}
\newcommand{\langnohyph}[1]{\begin{hyphenrules}{nohyphenation}#1\end{hyphenrules}}

\usepackage[autostyle=false,autopunct=false,english=british]{csquotes}
\setquotestyle{american}
\newcommand*{\defquote}[1]{`\,#1\,'}

\usepackage{amsthm}
%% from https://tex.stackexchange.com/a/404680/97039
\makeatletter
\def\@endtheorem{\endtrivlist}
\makeatother

\newcommand*{\QED}{\textsc{q.e.d.}}
\renewcommand*{\qedsymbol}{\QED}
\theoremstyle{remark}
\newtheorem{note}{Note}
\newtheorem*{remark}{Note}
\newtheoremstyle{innote}{\parsep}{\parsep}{\footnotesize}{}{}{}{0pt}{}
\theoremstyle{innote}
\newtheorem*{innote}{}

\usepackage[shortlabels,inline]{enumitem}
\SetEnumitemKey{para}{itemindent=\parindent,leftmargin=0pt,listparindent=\parindent,parsep=0pt,itemsep=\topsep}
% \begin{asparaenum} = \begin{enumerate}[para]
% \begin{inparaenum} = \begin{enumerate*}
\setlist{itemsep=0pt,topsep=\parsep}
\setlist[enumerate,2]{label=\alph*.}
\setlist[enumerate]{label=\arabic*.,leftmargin=1.5\parindent}
\setlist[itemize]{leftmargin=1.5\parindent}
\setlist[description]{leftmargin=1.5\parindent}
% old alternative:
% \setlist[enumerate,2]{label=\alph*.}
% \setlist[enumerate]{leftmargin=\parindent}
% \setlist[itemize]{leftmargin=\parindent}
% \setlist[description]{leftmargin=\parindent}

\usepackage[babel,theoremfont,largesc]{newpxtext}

\usepackage[bigdelims,nosymbolsc%,smallerops % probably arXiv doesn't have it
]{newpxmath}
%\useosf
%\linespread{1.083}%
%\linespread{1.05}% widely used
\linespread{1.1}% best for text with maths
%% smaller operators for old version of newpxmath
\makeatletter
\def\re@DeclareMathSymbol#1#2#3#4{%
    \let#1=\undefined
    \DeclareMathSymbol{#1}{#2}{#3}{#4}}
%\re@DeclareMathSymbol{\bigsqcupop}{\mathop}{largesymbols}{"46}
%\re@DeclareMathSymbol{\bigodotop}{\mathop}{largesymbols}{"4A}
\re@DeclareMathSymbol{\bigoplusop}{\mathop}{largesymbols}{"4C}
\re@DeclareMathSymbol{\bigotimesop}{\mathop}{largesymbols}{"4E}
\re@DeclareMathSymbol{\sumop}{\mathop}{largesymbols}{"50}
\re@DeclareMathSymbol{\prodop}{\mathop}{largesymbols}{"51}
\re@DeclareMathSymbol{\bigcupop}{\mathop}{largesymbols}{"53}
\re@DeclareMathSymbol{\bigcapop}{\mathop}{largesymbols}{"54}
%\re@DeclareMathSymbol{\biguplusop}{\mathop}{largesymbols}{"55}
\re@DeclareMathSymbol{\bigwedgeop}{\mathop}{largesymbols}{"56}
\re@DeclareMathSymbol{\bigveeop}{\mathop}{largesymbols}{"57}
%\re@DeclareMathSymbol{\bigcupdotop}{\mathop}{largesymbols}{"DF}
%\re@DeclareMathSymbol{\bigcapplusop}{\mathop}{largesymbolsPXA}{"00}
%\re@DeclareMathSymbol{\bigsqcupplusop}{\mathop}{largesymbolsPXA}{"02}
%\re@DeclareMathSymbol{\bigsqcapplusop}{\mathop}{largesymbolsPXA}{"04}
%\re@DeclareMathSymbol{\bigsqcapop}{\mathop}{largesymbolsPXA}{"06}
\re@DeclareMathSymbol{\bigtimesop}{\mathop}{largesymbolsPXA}{"10}
%\re@DeclareMathSymbol{\coprodop}{\mathop}{largesymbols}{"60}
%\re@DeclareMathSymbol{\varprod}{\mathop}{largesymbolsPXA}{16}
\makeatother
%%
%% With euler font cursive for Greek letters - the [1] means 100% scaling
\DeclareFontFamily{U}{egreek}{\skewchar\font'177}%
\DeclareFontShape{U}{egreek}{m}{n}{<-6>s*[1]eurm5 <6-8>s*[1]eurm7 <8->s*[1]eurm10}{}%
\DeclareFontShape{U}{egreek}{m}{it}{<->s*[1]eurmo10}{}%
\DeclareFontShape{U}{egreek}{b}{n}{<-6>s*[1]eurb5 <6-8>s*[1]eurb7 <8->s*[1]eurb10}{}%
\DeclareFontShape{U}{egreek}{b}{it}{<->s*[1]eurbo10}{}%
\DeclareSymbolFont{egreeki}{U}{egreek}{m}{it}%
\SetSymbolFont{egreeki}{bold}{U}{egreek}{b}{it}% from the amsfonts package
\DeclareSymbolFont{egreekr}{U}{egreek}{m}{n}%
\SetSymbolFont{egreekr}{bold}{U}{egreek}{b}{n}% from the amsfonts package
% Take also \sum, \prod, \coprod symbols from Euler fonts
\DeclareFontFamily{U}{egreekx}{\skewchar\font'177}
\DeclareFontShape{U}{egreekx}{m}{n}{%
       <-7.5>s*[0.9]euex7%
    <7.5-8.5>s*[0.9]euex8%
    <8.5-9.5>s*[0.9]euex9%
    <9.5->s*[0.9]euex10%
}{}
\DeclareSymbolFont{egreekx}{U}{egreekx}{m}{n}
\DeclareMathSymbol{\sumop}{\mathop}{egreekx}{"50}
\DeclareMathSymbol{\prodop}{\mathop}{egreekx}{"51}
\DeclareMathSymbol{\coprodop}{\mathop}{egreekx}{"60}
\makeatletter
\def\sum{\DOTSI\sumop\slimits@}
\def\prod{\DOTSI\prodop\slimits@}
\def\coprod{\DOTSI\coprodop\slimits@}
\makeatother
\input{definegreek.tex}% Greek letters not usually given in LaTeX.

%\usepackage%[scaled=0.9]%
%{classico}%  Optima as sans-serif font
\renewcommand\sfdefault{uop}
\DeclareMathAlphabet{\mathsf}  {T1}{\sfdefault}{m}{sl}
\SetMathAlphabet{\mathsf}{bold}{T1}{\sfdefault}{b}{sl}
\newcommand*{\mathte}[1]{\textbf{\textit{\textsf{#1}}}}
% Upright sans-serif math alphabet
% \DeclareMathAlphabet{\mathsu}  {T1}{\sfdefault}{m}{n}
% \SetMathAlphabet{\mathsu}{bold}{T1}{\sfdefault}{b}{n}

% DejaVu Mono as typewriter text
\usepackage[scaled=0.84]{DejaVuSansMono}

\usepackage{mathdots}

\usepackage[usenames]{xcolor}
% Tol (2012) colour-blind-, print-, screen-friendly colours, alternative scheme; Munsell terminology
\definecolor{mypurpleblue}{RGB}{68,119,170}
\definecolor{myblue}{RGB}{102,204,238}
\definecolor{mygreen}{RGB}{34,136,51}
\definecolor{myyellow}{RGB}{204,187,68}
\definecolor{myred}{RGB}{238,102,119}
\definecolor{myredpurple}{RGB}{170,51,119}
\definecolor{mygrey}{RGB}{187,187,187}
% Tol (2012) colour-blind-, print-, screen-friendly colours; Munsell terminology
% \definecolor{lbpurple}{RGB}{51,34,136}
% \definecolor{lblue}{RGB}{136,204,238}
% \definecolor{lbgreen}{RGB}{68,170,153}
% \definecolor{lgreen}{RGB}{17,119,51}
% \definecolor{lgyellow}{RGB}{153,153,51}
% \definecolor{lyellow}{RGB}{221,204,119}
% \definecolor{lred}{RGB}{204,102,119}
% \definecolor{lpred}{RGB}{136,34,85}
% \definecolor{lrpurple}{RGB}{170,68,153}
\definecolor{lgrey}{RGB}{221,221,221}
%\newcommand*\mycolourbox[1]{%
%\colorbox{mygrey}{\hspace{1em}#1\hspace{1em}}}
\colorlet{shadecolor}{lgrey}

\usepackage{bm}

\usepackage{microtype}

\usepackage[backend=biber,mcite,%subentry,
citestyle=authoryear-comp,bibstyle=pglpm-authoryear,autopunct=false,sorting=ny,sortcites=false,natbib=false,maxcitenames=2,maxbibnames=8,minbibnames=8,giveninits=true,uniquename=false,uniquelist=false,maxalphanames=1,block=space,hyperref=true,defernumbers=false,useprefix=true,sortupper=false,language=british,parentracker=false]{biblatex}
\DeclareSortingTemplate{ny}{\sort{\field{sortname}\field{author}\field{editor}}\sort{\field{year}}}
\iffalse\makeatletter%%% replace parenthesis with brackets
\newrobustcmd*{\parentexttrack}[1]{%
  \begingroup
  \blx@blxinit
  \blx@setsfcodes
  \blx@bibopenparen#1\blx@bibcloseparen
  \endgroup}
\AtEveryCite{%
  \let\parentext=\parentexttrack%
  \let\bibopenparen=\bibopenbracket%
  \let\bibcloseparen=\bibclosebracket}
\makeatother\fi
\DefineBibliographyExtras{british}{\def\finalandcomma{\addcomma}}
\renewcommand*{\finalnamedelim}{\addspace\amp\space}
%\renewcommand*{\finalnamedelim}{\addcomma\space}
\setcounter{biburlnumpenalty}{1}
\setcounter{biburlucpenalty}{0}
\setcounter{biburllcpenalty}{1}
\DeclareDelimFormat{multicitedelim}{\addsemicolon\addspace\space}
\DeclareDelimFormat{compcitedelim}{\addsemicolon\addspace\space}
\DeclareDelimFormat{postnotedelim}{\addspace}
\ifarxiv\else\addbibresource{portamanabib.bib}\fi
\renewcommand{\bibfont}{\footnotesize}
%\appto{\citesetup}{\footnotesize}% smaller font for citations
\defbibheading{bibliography}[\bibname]{\section*{#1}\addcontentsline{toc}{section}{#1}%\markboth{#1}{#1}
}
\newcommand*{\citep}{\footcites}
\newcommand*{\citey}{\footcites}%{\parencites*}
\newcommand*{\ibid}{\unspace\addtocounter{footnote}{-1}\footnotemark{}}
%\renewcommand*{\cite}{\parencite}
%\renewcommand*{\cites}{\parencites}
\providecommand{\href}[2]{#2}
\providecommand{\eprint}[2]{\texttt{\href{#1}{#2}}}
\newcommand*{\amp}{\&}
% \newcommand*{\citein}[2][]{\textnormal{\textcite[#1]{#2}}%\addtocategory{extras}{#2}
% }
\newcommand*{\citein}[2][]{\textnormal{\textcite[#1]{#2}}%\addtocategory{extras}{#2}
}
\newcommand*{\citebi}[2][]{\textcite[#1]{#2}%\addtocategory{extras}{#2}
}
\newcommand*{\subtitleproc}[1]{}
\newcommand*{\chapb}{ch.}
%
\newcommand*{\arxiveprint}[1]{%
\texttt{arXiv:\urlalt{https://arxiv.org/abs/#1}{#1}}%
}
\newcommand*{\mparceprint}[1]{%
\texttt{mp\_arc:\urlalt{http://www.ma.utexas.edu/mp_arc-bin/mpa?yn=#1}{#1}}%
}
\newcommand*{\haleprint}[1]{%
\texttt{HAL:\urlalt{https://hal.archives-ouvertes.fr/#1}{#1}}%
}
\newcommand*{\philscieprint}[1]{%
\texttt{PhilSci:\urlalt{http://philsci-archive.pitt.edu/archive/#1}{#1}}%
}
\newcommand*{\doi}[1]{%
\href{https://doi.org/#1}{\textsc{doi}:\texttt{#1}}%
}
\newcommand*{\biorxiveprint}[1]{%
bioRxiv \doi{10.1101/#1}%
}
\newcommand*{\osfeprint}[1]{%
Open Science Framework \doi{10.31219/osf.io/#1}%
}

\usepackage{graphicx}

%\usepackage{wrapfig}

%\usepackage{tikz-cd}

\PassOptionsToPackage{hyphens}{url}\usepackage[hypertexnames=false,pdfencoding=unicode,psdextra]{hyperref}

\usepackage[depth=4]{bookmark}
\hypersetup{colorlinks=true,bookmarksnumbered,pdfborder={0 0 0.25},citebordercolor={0.2667 0.4667 0.6667},citecolor=mypurpleblue,linkbordercolor={0.6667 0.2 0.4667},linkcolor=myredpurple,urlbordercolor={0.1333 0.5333 0.2},urlcolor=mygreen,breaklinks=true,pdftitle={\pdftitle},pdfauthor={\pdfauthor}}
% \usepackage[vertfit=local]{breakurl}% only for arXiv
\providecommand*{\urlalt}{\href}

\usepackage[british]{datetime2}
\DTMnewdatestyle{mydate}%
{% definitions
\renewcommand*{\DTMdisplaydate}[4]{%
\number##3\ \DTMenglishmonthname{##2} ##1}%
\renewcommand*{\DTMDisplaydate}{\DTMdisplaydate}%
}
\DTMsetdatestyle{mydate}

%%%%%%%%%%%%%%%%%%%%%%%%%%%%%%%%%%%%%%%%%%%%%%%%%%%%%%%%%%%%%%%%%%%%%%%%%%%%
%%% Layout. I do not know on which kind of paper the reader will print the
%%% paper on (A4? letter? one-sided? double-sided?). So I choose A5, which
%%% provides a good layout for reading on screen and save paper if printed
%%% two pages per sheet. Average length line is 66 characters and page
%%% numbers are centred.
%%%%%%%%%%%%%%%%%%%%%%%%%%%%%%%%%%%%%%%%%%%%%%%%%%%%%%%%%%%%%%%%%%%%%%%%%%%%
\ifafour\setstocksize{297mm}{210mm}%{*}% A4
\else\setstocksize{210mm}{5.5in}%{*}% 210x139.7
\fi
\settrimmedsize{\stockheight}{\stockwidth}{*}
\setlxvchars[\normalfont] %313.3632pt for a 66-characters line
\setxlvchars[\normalfont]
\setlength{\trimtop}{0pt}
\setlength{\trimedge}{\stockwidth}
\addtolength{\trimedge}{-\paperwidth}
% The length of the normalsize alphabet is 133.05988pt - 10 pt = 26.1408pc
% The length of the normalsize alphabet is 159.6719pt - 12pt = 30.3586pc
% Bringhurst gives 32pc as boundary optimal with 69 ch per line
% The length of the normalsize alphabet is 191.60612pt - 14pt = 35.8634pc
\ifafour\settypeblocksize{*}{32pc}{1.618} % A4
%\setulmargins{*}{*}{1.667}%gives 5/3 margins % 2 or 1.667
\else\settypeblocksize{*}{26pc}{1.618}% nearer to a 66-line newpx and preserves GR
\fi
\setulmargins{*}{*}{1}%gives equal margins
\setlrmargins{*}{*}{*}
\setheadfoot{\onelineskip}{2.5\onelineskip}
\setheaderspaces{*}{2\onelineskip}{*}
\setmarginnotes{2ex}{10mm}{0pt}
\checkandfixthelayout[nearest]
%%% End layout
%% this fixes missing white spaces
%\pdfmapline{+dummy-space <dummy-space.pfb}
\pdfinterwordspaceon%

%%% Sectioning
\newcommand*{\asudedication}[1]{%
{\par\centering\textit{#1}\par}}
\newenvironment{acknowledgements}{\section*{Thanks}\addcontentsline{toc}{section}{Thanks}}{\par}
\makeatletter\renewcommand{\appendix}{\par
  \bigskip{\centering
   \interlinepenalty \@M
   \normalfont
   \printchaptertitle{\sffamily\appendixpagename}\par}
  \setcounter{section}{0}%
  \gdef\@chapapp{\appendixname}%
  \gdef\thesection{\@Alph\c@section}%
  \anappendixtrue}\makeatother
\counterwithout{section}{chapter}
\setsecnumformat{\upshape\csname the#1\endcsname\quad}
\setsecheadstyle{\large\bfseries\sffamily%
\centering}
\setsubsecheadstyle{\bfseries\sffamily%
\raggedright}
%\setbeforesecskip{-1.5ex plus 1ex minus .2ex}% plus 1ex minus .2ex}
%\setaftersecskip{1.3ex plus .2ex }% plus 1ex minus .2ex}
%\setsubsubsecheadstyle{\bfseries\sffamily\slshape\raggedright}
%\setbeforesubsecskip{1.25ex plus 1ex minus .2ex }% plus 1ex minus .2ex}
%\setaftersubsecskip{-1em}%{-0.5ex plus .2ex}% plus 1ex minus .2ex}
\setsubsecindent{0pt}%0ex plus 1ex minus .2ex}
\setparaheadstyle{\bfseries\sffamily%
\raggedright}
\setcounter{secnumdepth}{2}
\setlength{\headwidth}{\textwidth}
\newcommand{\addchap}[1]{\chapter*[#1]{#1}\addcontentsline{toc}{chapter}{#1}}
\newcommand{\addsec}[1]{\section*{#1}\addcontentsline{toc}{section}{#1}}
\newcommand{\addsubsec}[1]{\subsection*{#1}\addcontentsline{toc}{subsection}{#1}}
\newcommand{\addpara}[1]{\paragraph*{#1.}\addcontentsline{toc}{subsubsection}{#1}}
\newcommand{\addparap}[1]{\paragraph*{#1}\addcontentsline{toc}{subsubsection}{#1}}

%%% Headers, footers, pagestyle
\copypagestyle{manaart}{plain}
\makeheadrule{manaart}{\headwidth}{0.5\normalrulethickness}
\makeoddhead{manaart}{%
{\footnotesize%\sffamily%
\scshape\headauthor}}{}{{\footnotesize\sffamily%
\headtitle}}
\makeoddfoot{manaart}{}{\thepage}{}
\newcommand*\autanet{\includegraphics[height=\heightof{M}]{autanet.pdf}}
\definecolor{mygray}{gray}{0.333}
\iftypodisclaim%
\ifafour\newcommand\addprintnote{\begin{picture}(0,0)%
\put(245,149){\makebox(0,0){\rotatebox{90}{\tiny\color{mygray}\textsf{This
            document is designed for screen reading and
            two-up printing on A4 or Letter paper}}}}%
\end{picture}}% A4
\else\newcommand\addprintnote{\begin{picture}(0,0)%
\put(176,112){\makebox(0,0){\rotatebox{90}{\tiny\color{mygray}\textsf{This
            document is designed for screen reading and
            two-up printing on A4 or Letter paper}}}}%
\end{picture}}\fi%afourtrue
\makeoddfoot{plain}{}{\makebox[0pt]{\thepage}\addprintnote}{}
\else
\makeoddfoot{plain}{}{\makebox[0pt]{\thepage}}{}
\fi%typodisclaimtrue
\makeoddhead{plain}{\scriptsize\reporthead}{}{}
% \copypagestyle{manainitial}{plain}
% \makeheadrule{manainitial}{\headwidth}{0.5\normalrulethickness}
% \makeoddhead{manainitial}{%
% \footnotesize\sffamily%
% \scshape\headauthor}{}{\footnotesize\sffamily%
% \headtitle}
% \makeoddfoot{manaart}{}{\thepage}{}

\pagestyle{manaart}

\setlength{\droptitle}{-3.9\onelineskip}
\pretitle{\begin{center}\LARGE\sffamily%
\bfseries}
\posttitle{\bigskip\end{center}}

\makeatletter\newcommand*{\atf}{\includegraphics[totalheight=\heightof{@}]{atblack.png}}\makeatother
\providecommand{\affiliation}[1]{\textsl{\textsf{\footnotesize #1}}}
\providecommand{\epost}[1]{\texttt{\footnotesize\textless#1\textgreater}}
\providecommand{\email}[2]{\href{mailto:#1ZZ@#2 ((remove ZZ))}{#1\protect\atf#2}}
%\providecommand{\email}[2]{\href{mailto:#1@#2}{#1@#2}}

\preauthor{\vspace{-0.5\baselineskip}\begin{center}
\normalsize\sffamily%
\lineskip  0.5em}
\postauthor{\par\end{center}}
\predate{\DTMsetdatestyle{mydate}\begin{center}\footnotesize}
\postdate{\end{center}\vspace{-\medskipamount}}

\setfloatadjustment{figure}{\footnotesize}
\captiondelim{\quad}
\captionnamefont{\footnotesize\sffamily%
}
\captiontitlefont{\footnotesize}
%\firmlists*
\midsloppy
% handling orphan/widow lines, memman.pdf
% \clubpenalty=10000
% \widowpenalty=10000
% \raggedbottom
% Downes, memman.pdf
\clubpenalty=9996
\widowpenalty=9999
\brokenpenalty=4991
\predisplaypenalty=10000
\postdisplaypenalty=1549
\displaywidowpenalty=1602
\raggedbottom

\paragraphfootnotes
\setlength{\footmarkwidth}{2ex}
% \threecolumnfootnotes
%\setlength{\footmarksep}{0em}
\footmarkstyle{\textsuperscript{%\color{myred}
\scriptsize\bfseries#1}~}
%\footmarkstyle{\textsuperscript{\color{myred}\scriptsize\bfseries#1}~}
%\footmarkstyle{\textsuperscript{[#1]}~}

\selectlanguage{british}\frenchspacing

%%%%%%%%%%%%%%%%%%%%%%%%%%%%%%%%%%%%%%%%%%%%%%%%%%%%%%%%%%%%%%%%%%%%%%%%%%%%
%%% Paper's details
%%%%%%%%%%%%%%%%%%%%%%%%%%%%%%%%%%%%%%%%%%%%%%%%%%%%%%%%%%%%%%%%%%%%%%%%%%%%
\title{\propertitle}
\author{%
\hspace*{\stretch{1}}%
%% uncomment if additional authors present
% \parbox{0.5\linewidth}%\makebox[0pt][c]%
% {\protect\centering ***\\%
% \footnotesize\epost{\email{***}{***}}}%
% \hspace*{\stretch{1}}%
\parbox{1\linewidth}%\makebox[0pt][c]%
{\protect\centering P.G.L.  Porta Mana  \href{https://orcid.org/0000-0002-6070-0784}{\protect\includegraphics[scale=0.16]{orcid_32x32.png}}\\%
\footnotesize Western Norway University of Applied Sciences\quad\epost{\email{pgl}{portamana.org}}}%
%% uncomment if additional authors present
% \hspace*{\stretch{1}}%
% \parbox{0.5\linewidth}%\makebox[0pt][c]%
% {\protect\centering ***\\%
% \footnotesize\epost{\email{***}{***}}}%
\hspace*{\stretch{1}}%
}

%\date{Draft of \today\ (first drafted \firstdraft)}
\date{\firstpublished; updated \updated}

%%%%%%%%%%%%%%%%%%%%%%%%%%%%%%%%%%%%%%%%%%%%%%%%%%%%%%%%%%%%%%%%%%%%%%%%%%%%
%%% Macros @@@
%%%%%%%%%%%%%%%%%%%%%%%%%%%%%%%%%%%%%%%%%%%%%%%%%%%%%%%%%%%%%%%%%%%%%%%%%%%%

% Common ones - uncomment as needed
%\providecommand{\nequiv}{\not\equiv}
%\providecommand{\coloneqq}{\mathrel{\mathop:}=}
%\providecommand{\eqqcolon}{=\mathrel{\mathop:}}
%\providecommand{\varprod}{\prod}
\newcommand*{\de}{\partialup}%partial diff
\newcommand*{\pu}{\piup}%constant pi
\newcommand*{\delt}{\deltaup}%Kronecker, Dirac
%\newcommand*{\eps}{\varepsilonup}%Levi-Civita, Heaviside
%\newcommand*{\riem}{\zetaup}%Riemann zeta
%\providecommand{\degree}{\textdegree}% degree
%\newcommand*{\celsius}{\textcelsius}% degree Celsius
%\newcommand*{\micro}{\textmu}% degree Celsius
\newcommand*{\I}{\mathrm{i}}%imaginary unit
\newcommand*{\e}{\mathrm{e}}%Neper
\newcommand*{\di}{\mathrm{d}}%differential
%\newcommand*{\Di}{\mathrm{D}}%capital differential
%\newcommand*{\planckc}{\hslash}
%\newcommand*{\avogn}{N_{\textrm{A}}}
%\newcommand*{\NN}{\bm{\mathrm{N}}}
%\newcommand*{\ZZ}{\bm{\mathrm{Z}}}
%\newcommand*{\QQ}{\bm{\mathrm{Q}}}
\newcommand*{\RR}{\bm{\mathrm{R}}}
%\newcommand*{\CC}{\bm{\mathrm{C}}}
\newcommand*{\nabl}{\bm{\nabla}}%nabla
%\DeclareMathOperator{\lb}{lb}%base 2 log
\DeclareMathOperator{\tr}{tr}%trace
%\DeclareMathOperator{\card}{card}%cardinality
%\DeclareMathOperator{\im}{Im}%im part
%\DeclareMathOperator{\re}{Re}%re part
%\DeclareMathOperator{\sgn}{sgn}%signum
%\DeclareMathOperator{\ent}{ent}%integer less or equal to
%\DeclareMathOperator{\Ord}{O}%same order as
%\DeclareMathOperator{\ord}{o}%lower order than
%\newcommand*{\incr}{\triangle}%finite increment
\newcommand*{\defd}{\coloneqq}
\newcommand*{\defs}{\eqqcolon}
%\newcommand*{\Land}{\bigwedge}
%\newcommand*{\Lor}{\bigvee}
%\newcommand*{\lland}{\DOTSB\;\land\;}
%\newcommand*{\llor}{\DOTSB\;\lor\;}
%\newcommand*{\limplies}{\mathbin{\Rightarrow}}%implies
%\newcommand*{\suchthat}{\mid}%{\mathpunct{|}}%such that (eg in sets)
%\newcommand*{\with}{\colon}%with (list of indices)
%\newcommand*{\mul}{\times}%multiplication
%\newcommand*{\inn}{\cdot}%inner product
%\newcommand*{\dotv}{\mathord{\,\cdot\,}}%variable place
%\newcommand*{\comp}{\circ}%composition of functions
\newcommand*{\con}{\mathbin{:}}%scal prod of tensors
%\newcommand*{\equi}{\sim}%equivalent to 
\renewcommand*{\asymp}{\simeq}%equivalent to 
%\newcommand*{\corr}{\mathrel{\hat{=}}}%corresponds to
%\providecommand{\varparallel}{\ensuremath{\mathbin{/\mkern-7mu/}}}%parallel (tentative symbol)
\renewcommand*{\le}{\leqslant}%less or equal
\renewcommand*{\ge}{\geqslant}%greater or equal
\DeclarePairedDelimiter\clcl{[}{]}
%\DeclarePairedDelimiter\clop{[}{[}
%\DeclarePairedDelimiter\opcl{]}{]}
%\DeclarePairedDelimiter\opop{]}{[}
\DeclarePairedDelimiter\abs{\lvert}{\rvert}
%\DeclarePairedDelimiter\norm{\lVert}{\rVert}
\DeclarePairedDelimiter\set{\{}{\}}
%\DeclareMathOperator{\pr}{P}%probability
\newcommand*{\pf}{\mathrm{p}}%probability
\newcommand*{\p}{\mathrm{P}}%probability
%\newcommand*{\E}{\mathrm{E}}
%\renewcommand*{\|}{\nonscript\,\vert\nonscript\;\mathopen{}}
\renewcommand*{\|}[1][]{\nonscript\,#1\vert\nonscript\;\mathopen{}}
%\DeclarePairedDelimiterX{\cond}[2]{(}{)}{#1\nonscript\,\delimsize\vert\nonscript\;\mathopen{}#2}
%\DeclarePairedDelimiterX{\condt}[2]{[}{]}{#1\nonscript\,\delimsize\vert\nonscript\;\mathopen{}#2}
%\DeclarePairedDelimiterX{\conds}[2]{\{}{\}}{#1\nonscript\,\delimsize\vert\nonscript\;\mathopen{}#2}
%\newcommand*{\+}{\lor}
%\renewcommand{\*}{\land}
%% symbol = for equality statements within probabilities
%% from https://tex.stackexchange.com/a/484142/97039
%\newcommand*{\eq}{\mathrel{\!=\!}}
%\let\texteq\=
%\renewcommand*{\=}{\TextOrMath\texteq\eq}
%%
\newcommand*{\sect}{\S}% Sect.~
\newcommand*{\sects}{\S\S}% Sect.~
\newcommand*{\chap}{ch.}%
\newcommand*{\chaps}{chs}%
\newcommand*{\bref}{ref.}%
\newcommand*{\brefs}{refs}%
%\newcommand*{\fn}{fn}%
\newcommand*{\eqn}{eq.}%
\newcommand*{\eqns}{eqs}%
\newcommand*{\fig}{fig.}%
\newcommand*{\figs}{figs}%
\newcommand*{\vs}{{vs}}
\newcommand*{\eg}{{e.g.}}
\newcommand*{\etc}{{etc.}}
\newcommand*{\ie}{{i.e.}}
%\newcommand*{\ca}{{c.}}
\newcommand*{\foll}{{ff.}}
%\newcommand*{\viz}{{viz}}
\newcommand*{\cf}{{cf.}}
%\newcommand*{\Cf}{{Cf.}}
%\newcommand*{\vd}{{v.}}
\newcommand*{\etal}{{et al.}}
%\newcommand*{\etsim}{{et sim.}}
%\newcommand*{\ibid}{{ibid.}}
%\newcommand*{\sic}{{sic}}
\newcommand*{\id}{\mathte{I}}%id matrix
%\newcommand*{\nbd}{\nobreakdash}%
%\newcommand*{\bd}{\hspace{0pt}}%
%\def\hy{-\penalty0\hskip0pt\relax}
%\newcommand*{\labelbis}[1]{\tag*{(\ref{#1})$_\text{r}$}}
%\newcommand*{\mathbox}[2][.8]{\parbox[t]{#1\columnwidth}{#2}}
%\newcommand*{\zerob}[1]{\makebox[0pt][l]{#1}}
\newcommand*{\tprod}{\mathop{\textstyle\prod}\nolimits}
\newcommand*{\tsum}{\mathop{\textstyle\sum}\nolimits}
%\newcommand*{\tint}{\begingroup\textstyle\int\endgroup\nolimits}
%\newcommand*{\tland}{\mathop{\textstyle\bigwedge}\nolimits}
%\newcommand*{\tlor}{\mathop{\textstyle\bigvee}\nolimits}
%\newcommand*{\sprod}{\mathop{\textstyle\prod}}
%\newcommand*{\ssum}{\mathop{\textstyle\sum}}
%\newcommand*{\sint}{\begingroup\textstyle\int\endgroup}
%\newcommand*{\sland}{\mathop{\textstyle\bigwedge}}
%\newcommand*{\slor}{\mathop{\textstyle\bigvee}}
\newcommand*{\T}{^\transp}%transpose
%%\newcommand*{\QEM}%{\textnormal{$\Box$}}%{\ding{167}}
%\newcommand*{\qem}{\leavevmode\unskip\penalty9999 \hbox{}\nobreak\hfill
%\quad\hbox{\QEM}}

%%%%%%%%%%%%%%%%%%%%%%%%%%%%%%%%%%%%%%%%%%%%%%%%%%%%%%%%%%%%%%%%%%%%%%%%%%%%
%%% Custom macros for this file @@@
%%%%%%%%%%%%%%%%%%%%%%%%%%%%%%%%%%%%%%%%%%%%%%%%%%%%%%%%%%%%%%%%%%%%%%%%%%%%
\definecolor{notecolour}{RGB}{68,170,153}
% \newcommand*{\puzzle}{{\fontencoding{U}\fontfamily{fontawesometwo}\selectfont\symbol{225}}}
% \newcommand{\mynote}[1]{ {\color{notecolour}\puzzle\ #1}}
\newcommand*{\wrench}{{\fontencoding{U}\fontfamily{fontawesomethree}\selectfont\symbol{114}}}
\newcommand{\mynote}[1]{ {\color{notecolour}\wrench\ #1}}
\newcommand*{\widebar}[1]{{\mkern1.5mu\skew{2}\overline{\mkern-1.5mu#1\mkern-1.5mu}\mkern 1.5mu}}

% \newcommand{\explanation}[4][t]{%\setlength{\tabcolsep}{-1ex}
% %\smash{
% \begin{tabular}[#1]{c}#2\\[0.5\jot]\rule{1pt}{#3}\\#4\end{tabular}}%}
% \newcommand*{\ptext}[1]{\text{\small #1}}
%\DeclareMathOperator*{\argsup}{arg\,sup}
\newcommand*{\yr}{r}
\newcommand*{\yrd}{\dot{\yr}}
\newcommand*{\yrdd}{\ddot{\yr}}
\newcommand*{\yno}{N}
\newcommand*{\yt}{\theta}
\newcommand*{\yV}{\upsilon}
\newcommand*{\yVb}{\bar{\upsilon}}
\newcommand*{\yVd}{\dot{\yVb}}
\newcommand*{\yVdd}{\ddot{\yVb}}
\newcommand*{\yub}{\bar{u}}
\newcommand*{\yq}{\bm{q}}
\newcommand*{\yqb}{\bar{q}}
\newcommand*{\ywb}{\bar{w}}
\newcommand*{\yv}{\bm{v}}
\newcommand*{\yp}{\mathte{p}}
\newcommand*{\yZ}{\varZeta}
\newcommand*{\yE}{\varEta}
\newcommand*{\yL}{\varLambda}
\newcommand*{\yC}{C}
\newcommand*{\yx}{\bm{x}}
\newcommand*{\yX}{\bm{X}}
\newcommand*{\yD}{\mathte{L}}
\newcommand*{\yc}{\bm{c}}
\newcommand*{\ytb}{\bar{\yt}}
\newcommand*{\ypb}{\bar{p}}
\newcommand*{\ygb}{\widebar{\nabla\yt}}


%%% Custom macros end @@@

%%%%%%%%%%%%%%%%%%%%%%%%%%%%%%%%%%%%%%%%%%%%%%%%%%%%%%%%%%%%%%%%%%%%%%%%%%%%
%%% Beginning of document
%%%%%%%%%%%%%%%%%%%%%%%%%%%%%%%%%%%%%%%%%%%%%%%%%%%%%%%%%%%%%%%%%%%%%%%%%%%%
%\firmlists
\begin{document}
\captiondelim{\quad}\captionnamefont{\footnotesize}\captiontitlefont{\footnotesize}
\selectlanguage{british}\frenchspacing
\maketitle

%%%%%%%%%%%%%%%%%%%%%%%%%%%%%%%%%%%%%%%%%%%%%%%%%%%%%%%%%%%%%%%%%%%%%%%%%%%%
%%% Abstract
%%%%%%%%%%%%%%%%%%%%%%%%%%%%%%%%%%%%%%%%%%%%%%%%%%%%%%%%%%%%%%%%%%%%%%%%%%%%
\iffalse \abstractrunin
\abslabeldelim{}
\renewcommand*{\abstractname}{}
\setlength{\absleftindent}{0pt}
\setlength{\absrightindent}{0pt}
\setlength{\abstitleskip}{-\absparindent}
\begin{abstract}\labelsep 0pt%
  \noindent ***
% \\\noindent\emph{\footnotesize Note: Dear Reader
%     \amp\ Peer, this manuscript is being peer-reviewed by you. Thank you.}
% \par%\\[\jot]
% \noindent
% {\footnotesize PACS: ***}\qquad%
% {\footnotesize MSC: ***}%
%\qquad{\footnotesize Keywords: ***}
\end{abstract}
\fi
\selectlanguage{british}\frenchspacing

%%%%%%%%%%%%%%%%%%%%%%%%%%%%%%%%%%%%%%%%%%%%%%%%%%%%%%%%%%%%%%%%%%%%%%%%%%%%
%%% Epigraph
%%%%%%%%%%%%%%%%%%%%%%%%%%%%%%%%%%%%%%%%%%%%%%%%%%%%%%%%%%%%%%%%%%%%%%%%%%%%
% \asudedication{\small ***}
% \vspace{\bigskipamount}
% \setlength{\epigraphwidth}{.7\columnwidth}
% %\epigraphposition{flushright}
% \epigraphtextposition{flushright}
% %\epigraphsourceposition{flushright}
% \epigraphfontsize{\footnotesize}
% \setlength{\epigraphrule}{0pt}
% %\setlength{\beforeepigraphskip}{0pt}
% %\setlength{\afterepigraphskip}{0pt}
% \epigraph{\emph{text}}{source}



%%%%%%%%%%%%%%%%%%%%%%%%%%%%%%%%%%%%%%%%%%%%%%%%%%%%%%%%%%%%%%%%%%%%%%%%%%%%
%%% BEGINNING OF MAIN TEXT
%%%%%%%%%%%%%%%%%%%%%%%%%%%%%%%%%%%%%%%%%%%%%%%%%%%%%%%%%%%%%%%%%%%%%%%%%%%%

\section{Limitations of the ideal-gas model in graduate-physics teaching}
\label{sec:limitations}

The ideal gas of graduate-physics curricula, modelled with $(V,T)$ or
equivalent states and the equation of state $pV=nRt$, targets simplified
problems of (Gibbsian) thermostatics, involving comparisons of equilibria.
Many students find difficulties when questions involving motion and rates
of change are involved, for example \enquote{how long will it take for the
  volume of the gas to change from $V_{0}$ to $V_{1}$?} (given particular
conditions). Yet, experiments involving such changes can be readily thought
and taught; for example the Assmann-R\"uchardt experiment
\citep{assmann1852,ruechardt1929}, in which a massive piston closes a
cylinder containing an ideal gas, and various questions can be asked about
the motion of the piston and the changes in volume, pressure, temperature
of the gas.

Answers to such questions can for example be given \citep[See
\eg][]{portamana2010_r2011} using the thermo\emph{dynamic} model with
$(V,T)$ state as presented in \textcite[\sect~2.1 model A]{pekaretal2014}.
But this kind of analysis has some interrelated singular points and
limitations: \mynote{Make the following statements more precise; concrete
  examples}
\begin{enumerate}
\item If we assume the piston to have negligible mass, some evolution
  equations degenerate, and the gas shows infinitely fast responses
\item There is no dissipation and no differences in pressure between compression and
  expansion, as instead expected in irreversible processes.
\end{enumerate}
Owing to these peculiarities there's often an increased difficulty for the
students to move to a continuum ideal gas later on.

The limitations above are connected to the fact that, from a continuum
point of view, the gas can be seen as always having uniform pressure and
temperature, and negligible mass. \mynote{Make more precise.}

\medskip

It is possible to model the ideal gas in a slightly more realistic way,
having these useful features:
\begin{itemize}
\item the state variables are still finite, so we don't need to use
  continuum mechanics;
\item the gas shows different pressures in compression and expansion,
  viscosity and dissipation;
\item dynamical questions can be asked even with massless pistons;
\item it's possible to make clearer connections with the continuum model of
  the gas, and in particular to the question of the boundary conditions
  necessary to study an continuum model.
\end{itemize}
Such model is very similar to \emph{model~C} in \sect~2.1 of
\textcite{pekaretal2014}. The state of the gas is determined by four
quantities: volume $\yV$, rate of change of volume $\dot{\yV}$, temperature
$\yt$ at the boundary, and temperature gradient $\nabl \yt$ at the boundary
or equivalently heating flux $Q$ at the boundary \mynote{I must check
  whether it's possible to replace the latter with $\dot{\yt}$.}

This model can be introduced as a special class of solutions of the
continuum equations. I start from there.

\section{Special continuum solution}
\label{sec:cont_solution}

Some notation: constants are denoted by capital letters, spatially dependent
quantities with lowercase letters, and quantities that depend only on time
have an overbar.

Let $n$ be the volumic amount of substance, $\yv$ the velocity, $\yp$ the
(tensorial) pressure, $\yt$ the temperature, $u$ the molar internal energy,
$\yq$ the heat flux, and $N$ the total amount of substance in this body of
gas. Also,
\begin{equation}
  \label{eq:def_shear}
  \nabl\yv^{+}\defd \tfrac{1}{2}(\nabl\yv+\nabl\yv\T),
  \qquad
  \nabl\yv^{\perp}\defd \nabl\yv^{+}-\tfrac{1}{3}\nabl\cdot\yv\; \id \ .
\end{equation}

The independent fields, defining the state of the body, are taken to be
$(n, \yv,\yt)$. Pressure, internal energy, heat flux are local functions of
state. The fields satisfy the balance laws of matter, force, internal
energy (the balance of torque is automatically satisfied owing to the
symmetry of $\yp$):
\begin{gather}
  \label{eq:matter_bal}
  \de_{t}n +\nabl\cdot(n\yv) = 0
\\
  \label{eq:force_bal}
  M \; \de_{t}(n\yv) + M\; \nabl\cdot(n\yv\otimes\yv) + \nabl\cdot\yp = 0
\\
  \label{eq:energy_bal}
  \de_{t} (n u) + \nabl\cdot(n u \yv)  + \nabl\cdot\yq + \tr(\yp\nabl\yv^{+}) = 0
\end{gather}
where $M$ is the molar mass.

We define the continuum ideal gas by the following constitutive equations:
\begin{gather}
  \label{eq:const_idealgas}
  \yq = -K\,\nabl\yt %- \kappa(n,\yte)\,\nab n
  \\
  \yp = (R n \yt - \yZ\,\nabl\cdot\yv)\;\id - 2\,\yE\,\nabl\yv^{\perp}
  \\
  u = \yC\yt%f(n,\yte)+\yte\de_{\yte}f % c \yte 
  \\
 R, K, C, \yE, \yZ \ge 0
\end{gather}
$K$ is the thermal conductivity, $\yE$ the shear viscosity, $\yZ$ the bulk
or volume viscosity. The shear viscosity $\yE$ should depend on temperature
\citep{sutherland1893}, but here we consider it constant for simplicity.
% Equations for these are given \eg\ in Kannuluik
% \amp\ Carman \citey{kannuluiketal1951}, Sutherland \citey{sutherland1893},
% Cramer or Sharma \amp\ Kumar \citey{cramer2012,sharmaetal2019}.

The thermomechanic processes of the system are determined by the system of
partial differential equations above together with appropriate initial and
boundary conditions. The latter, for example, may consist in the
specification of the pressure and heat flux on the boundary of the body.

\bigskip

Let's now look for a specific class of solutions. Consider a coordinate
system $(x,y,z)$ where the centre of mass of the gas is at rest with
coordinate $x=0$, and assume that the fields have no $(y,z)$ dependence.
The problem becomes one-dimensional. The gas can be imagined to be in a
cylindrical container of given cross section $A$, within two pistons
movable in the direction $x$. Let's identify the matter elements of the gas
with the coordinate $X \in \clcl{-1, 1}$ in the reference configuration,
the centre of mass being at $X=0$.

We require the volumic amount of matter $n$ to be uniform at all times (so
it's a constant). The balance of matter~\eqref{eq:matter_bal} then requires
that the motion of a matter element of gas at the reference position $X$
can only have a motion of the form
\begin{equation}
  \label{eq:simplified_affine_motion}
  x(X,t) = \frac{1}{2A}\yVb(t) \, X \ .
\end{equation}
The boundary points of the body of gas -- that is, the positions of the two
pistons -- are $x=\pm\frac{1}{2A}\yVb(t)$.


It turns out that the class of motions~\eqref{eq:simplified_affine_motion}
and the balance laws of force and energy lead to a parabolic spatial
dependence of pressure, temperature, internal energy, and a linear spatial
dependence of the heat flux. We are also assuming this parabolic spatial
dependence to be symmetric with respect to the mass centre; this requires
the pressures on the two pistons to be equal (to keep the mass centre at
rest), and the heat fluxes at the two boundaries to be opposite and equal
in magnitude. These last symmetric restrictions can be easily lifted,
however.

The solutions for the fields of velocity, matter density,  temperature, pressure
internal energy, heat flux are found to be
\begin{align}
  \label{eq:v_motion}
  v(x,t) &= \yVd\, \frac{x}{\yVb}
\\
\label{eq:n_motion}
n(x,t) &= \frac{\yno}{\yVb}
\\
\label{eq:T_motion}
  \yt(x,t) &= \ytb - \frac{1}{2} \frac{M}{R}\frac{\yVdd}{\yVb}\
             \biggl(x^{2}-\frac{\yVb^{2}}{4A^{2}}\biggr)
\\[\jot]
\label{eq:p_motion}
  p(x,t) &= NR \frac{\ytb}{\yVb} - \frac{2}{3}\yE \frac{\yVd}{\yVb}
           - \frac{1}{2}MN\frac{\yVdd}{\yVb^{2}}\
           \biggl(x^{2}-\frac{\yVb^{2}}{4A^{2}}\biggr)
  \\[\jot]
\label{eq:u_motion}
  u(x,t) &= \yC N \frac{\ytb}{\yVb} -
                           \frac{1}{2} \frac{\yC N M}{R}\frac{\yVdd}{\yVb^{2}}\
             \biggl(x^{2}-\frac{\yVb^{2}}{4A^{2}}\biggr)
  \\[\jot]
\label{eq:q_motion}
q(x,t) &= \frac{KM}{R} \frac{\yVdd}{\yVb}\ x
\end{align}
where the time dependence of the barred quantities $\ytb(t)$, $\yVb(t)$ has
been omitted for brevity. $\ytb(t)$ is the temperature \emph{at the boundaries}.

By integration we also find the \emph{average} pressure, total internal
energy, heating, and work:
\begin{align}
  \label{eq:p_motion_avg}
  \ypb_{\text{average}}(t) &= NR \frac{\ytb}{\yVb}
                          - \frac{2}{3}\yE \frac{\yVd}{\yVb}
                          +\frac{MN}{12 A^{2}} \yVdd
  \\[\jot]
\label{eq:u_motion_tot}
  \yub_{\text{total}}(t) &= \yC N \ytb +
                           \frac{1}{3}\frac{\yC MN}{4RA^{2}} \yVb\yVdd
  \\
\label{eq:q_motion_tot}
\yqb_{\text{total}}(t) &= \frac{KM}{R} \yVdd
  \\
\label{eq:work_tot}
  \ywb_{\text{total}} &=
  \yVd \ \ypb_{\text{average}}
\end{align}

We see that in this specific class of solutions the state of the ideal gas
at time $t$ is completely determined by the variables
$(\ytb,\yVb,\yVd,\yVdd)$, similarly to model model~C of
\citeauthor{pekaretal2014}. It is possible to associate to this model the
average pressure $\ypb_{\text{average}}$ above, which is given by an
equation of state. Such a pressure shows different values at equilibrium
($\yVdd=\yVd=0$) and during compression or expansion, leading to
dissipation.

A result at variance with model~C is that the total energy does depend on
the state variable $\yVdd$, whereas it is found in
\citeauthor{pekaretal2014} that it shouldn't have such
dependence. \mynote{this can be a computation error on my part. All thee
  calculations have to be rechecked. Also, the equation of balance for the
  energy hasn't been used yet.}


\bigskip

The point of the calculations above is to see whether a connection could be
made between one of the higher-order models in \citeauthor{pekaretal2014}
and a special state of a continuum ideal gas – just like model~A can be
interpreted, in one specific case, as describing a continuum ideal gas in a
uniform state.

As already mentioned, I believe that such an association could be
pedagogically useful, allowing thermodynamics students to study a
finite-dimensional system showing dissipation and irreversibility, and at
the same time showing them a first connection with its continuum description.


\iffalse
The same symmetry holds for the parabolic
spatial dependence of the temperature if the heating flux at the boundary
is also assumed constant with respect to the surface normal.

In the following we shall assume that the body is in a cylindrical
container of given cross section, within movable pistons in the direction
$x$, and mass centre at rest at $\yx = 0$. The
motion~\eqref{eq:affine_motion} can be simplified to
\begin{equation}
  \label{eq:simplified_affine_motion}
  x(X,t) = \yr(t) \, X, \qquad X \in \clcl{-1, 1}
\end{equation}
with the $y$ and $z$ coordinates remaining constant. This motion requires
the pressures on the two pistons, at positions $\yr$ and $-\yr$, to be
equal (to keep the mass centre at rest). We also require that the heat
fluxes at the two pistons be equal in magnitude and opposite.





the motionhave an affine dependence on $x$. In the
material description we have


and mass centre at rest at $\yx = 0$. The
motion~\eqref{eq:affine_motion} can be simplified to
\begin{equation}
  \label{eq:simplified_affine_motion}
  x(X,t) = \yr(t) \, X, \qquad X \in \clcl{-1, 1}
\end{equation}



then the body can only undergo affine motions, that is, motions of the form
\begin{equation}
  \label{eq:affine_motion}
  \yx(\yX,t) = \yD(t)\,\yX + \yc(t) \;,
\end{equation}
where $\yX$ are its points in a reference configuration.



In introductory thermostatics courses the ideal gas is usually intended in
a much more restrictive way: a body for which matter density, temperature,
pressure are uniform at all times, with $p = R n \yt$ and $u = C \yt$. The
state of the system is taken to be given by the spatially-independent quantities
$(n,\yt)$ or $(V,\yt)$, where $V$ is the volume. It is possible to model
thermomechanic processes for such a body; see for example Samoh\'yl \amp\
Peka\v{r} \citep{samohyletal1987_r2014}.

This uniform model has some singular features, however. In passing from a
state of rest in an inertial frame to a compression process, for example,
parts of the body will be accelerated. This is only possible if the
divergence of the pressure is not zero at some times, owing to the balance
law for the force. And the thermal conductivity must be infinite, in order
to keep a uniform temperature when there is a heating flux at the boundary.
The introduction of additional state variables such as the rate of change
of volume $\de_{t}V$ \citep{samohyletal1987_r2014} allows us to model
interesting non-equilibrium phenomena, such as non-equilibrium pressures.
But the singular features remain.

These singular features also lead the boundary conditions to eliminate some
state quantities. For example, the specification of the pressure at the
boundary is equivalent to its specification through the whole body. This
places a constraint between $n$ and $T$ (and $\de_{t}V$) for example, from
the constitutive equation of the pressure.

\bigskip

I present here a slightly less simplified model. The idea is to allow a
non-constant spatial dependence of some fields, but of a fixed mathematical
kind. This idea can be seen as a simple generalization of the spatially
constant dependence of the uniform model.

If we require the volumic amount of matter $n$ to be uniform at all times,
then the body can only undergo affine motions, that is, motions of the form
\begin{equation}
  \label{eq:affine_motion}
  \yx(\yX,t) = \yD(t)\,\yX + \yc(t) \;,
\end{equation}
where $\yX$ are its points in a reference configuration.

It turns out that such an \enquote{affine} model is consistent with the
balance laws if we assume a parabolic spatial dependence of pressure,
temperature, and internal energy, and a linear spatial dependence of the
heat flux.

If we moreover consider an inertial frame in which the mass centre of the
body is at rest, then the pressure must be equal throughout the boundary;
its parabolic spatial dependence is therefore symmetric with respect to the
mass centre. The same symmetry holds for the parabolic spatial dependence
of the temperature if the heating flux at the boundary is also assumed
constant with respect to the surface normal.

In the following we shall assume that the body is in a cylindrical
container of given cross section, within movable pistons in the direction
$x$, and mass centre at rest at $\yx = 0$. The
motion~\eqref{eq:affine_motion} can be simplified to
\begin{equation}
  \label{eq:simplified_affine_motion}
  x(X,t) = \yr(t) \, X, \qquad X \in \clcl{-1, 1}
\end{equation}
with the $y$ and $z$ coordinates remaining constant. This motion requires
the pressures on the two pistons, at positions $\yr$ and $-\yr$, to be
equal (to keep the mass centre at rest). We also require that the heat
fluxes at the two pistons be equal in magnitude and opposite.

Under these conditions the symmetric parabolic profile along $x$ of the
temperature field $\yt(x, t)$ can be summarized by two numbers, for example
the values of the temperature $\yt[\yr(t), t] \defs \ytb(t)$ and of the
temperature gradient $(\de_{x}\yt)[\yr(t), t] \defs \ygb(t)$ at one piston
surface. These two values can be considered state quantities. One more
state quantity is the uniform $n(t)$ or equivalently the volume occupied by
the body, which is in turn equivalent to the specification of the position
$\yr(t) > 0$ of one piston, the other having position $-\yr(t)$, so that
the volume is given by $2 \yr(t) \times{}$cross-sectional area.

From the motion~\eqref{eq:simplified_affine_motion} we can calculate the
simplified space-time dependence of all the fields through the
equations~\eqref{eq:const_idealgas}--\eqref{eq:matter_bal}:
\begin{align}
  \label{eq:v_motion}
  v(x,t) &= \yrd\, \frac{x}{\yr}
\\
\label{eq:n_motion}
n(x,t) &= \frac{\yno}{\yr}
\\
\label{eq:p_motion}
  \begin{split}
p(x,t) &= R\, n\, \ytb + \tfrac{1}{2} \,R \, n \, \ygb \,
         \frac{x^{2}-\yr^{2}}{\yr} - \yr \,\frac{\yrd}{\yr}
         \\&=
         \ypb + \tfrac{1}{2} \,R \, n \, \ygb \,
         \frac{x^{2}-\yr^{2}}{\yr} 
       \end{split}
\\
\label{eq:T_motion}
\yt(x,t) &= \ytb + \tfrac{1}{2}\, \ygb\, \frac{x^{2}-\yr^{2}}{\yr}
\\
\label{eq:u_motion}
  u(x,t) &= \yC \, \ytb +
           \tfrac{1}{2}\, \yC \,\ygb \,\frac{x^{2}-\yr^{2}}{\yr}
\\
\label{eq:q_motion}
q(x,t) &= - K \, \ygb \,\frac{x}{\yr}
\end{align}
where it we have omitted the time-dependence of the quantities $\yr(t)$, $\ypb(t)$,
$\yrd(t)$, $\ytb(t)$, $\ygb(t)$


\begin{equation}
  \label{eq:data}
  \begin{gathered}
  M = 4\times 10^{-3}\ \textrm{kg/mol}
  \qquad
  C = 21\ \textrm{J/(mol\ K)}
  \qquad
  K = 0.15\ \textrm{W/(m\ K)}
  \\
  R = 8.3\ \mathrm{m^{3}\ Pa/(mol\ K)}
  \qquad
  \yE = 2\times 10^{-5}\ \textrm{Pa\ s}
  \qquad
  N = 1\ \textrm{mol}
  \qquad
  \yZ = 0\ \textrm{Pa\ s}
\end{gathered}
\end{equation}


\textcolor{white}{If you find this you can claim a postcard from me.}
\fi

%%\setlength{\intextsep}{0ex}% with wrapfigure
%%\setlength{\columnsep}{0ex}% with wrapfigure
\begin{center}%[h!]% with figure
%\begin{wrapfigure}{r}{0.4\linewidth} % with wrapfigure
\centering\includegraphics[width=\linewidth]{idealgas2_IMG_20200725_114540.jpg}\\
%\caption{caption}\label{fig:comparison_a5}
(Preliminary calculations)
\end{center}%






%%%% examples use empheq
%   \begin{empheq}[left={\mathllap{\begin{aligned}    \de\yF_{\yc}/\de\yp&=0\text{:} \\
%         \de\yF_{\yc}/\de\ym&=0\text{:}\\ \de\yF_{\yc}/\de\yr&=0\text{:}\end{aligned}}\qquad}\empheqlbrace]{align}
%     \label{eq:con_p}
% %    \de\yF_{\yc}/\de\yp &\equiv
%     -\ln\yp + \ln\yq + \yr\yM + \ym\yu &=0,\\
%     \label{eq:con_u}
% %    \de\yF_{\yc}/\de\ym &\equiv
%     \yu\yp-1 &=0,\\
%     \label{eq:con_l}
%     %\de\yF_{\yc}/\de\yr &\equiv
%     \yM\yp-\yc &=0.
%   \end{empheq}
%%%%
% \begin{empheq}[box=\widefbox]{equation}
%   \label{eq:maxent_question}
%   \p\bigl[\yE{N+1}{k} \bigcond \tsum\yo\yf{N}\in\yA, \yM\bigr] = \mathord{?}
% \end{empheq}



% \[
%   \begin{tikzcd}
%       M_{n,n}(\CC) \arrow{r}{R'_{a}(\hat{U})} & M_{n,n}(\CC)
%     \\
%     L(\mathcal{H}) \arrow{r}{\hat{U}} \arrow[swap]{d}{R_*}\arrow[swap]{u}{R'_*} & L(\mathcal{H}) \arrow{d}{R_*}\arrow{u}{R'_*} \\
%       M_{n,n}(\CC) \arrow{r}{R_{a}(\hat{U})} & M_{n,n}(\CC)
%   \end{tikzcd}
% \]

% \[
%   \begin{tikzcd}
%       \CC^n \arrow{r}{R'_*(A)} & \CC^n
%     \\
%     \mathcal{H} \arrow{r}{A} \arrow[swap]{d}{R}\arrow[swap]{u}{R'} & \mathcal{H} \arrow{d}{R}\arrow{u}{R'} \\
%       \CC^n \arrow{r}{R_*(A)} & \CC^n
%   \end{tikzcd}
% \]


% \[
%   \begin{tikzcd}
%     \mathcal{H} \arrow{r}{A} \arrow[swap]{d}{R} & \mathcal{H} \arrow{d}{R} \\
%       \CC^n \arrow{r}{R_*(A)} & \CC^n
%   \end{tikzcd}
% \]

%%\setlength{\intextsep}{0ex}% with wrapfigure
%%\setlength{\columnsep}{0ex}% with wrapfigure
%\begin{figure}[p!]% with figure
%\begin{wrapfigure}{r}{0.4\linewidth} % with wrapfigure
%  \centering\includegraphics[trim={12ex 0 18ex 0},clip,width=\linewidth]{maxent_saddle.png}\\
%\caption{caption}\label{fig:comparison_a5}
%\end{figure}% exp_family_maxent.nb


%%%%%%%%%%%%%%%%%%%%%%%%%%%%%%%%%%%%%%%%%%%%%%%%%%%%%%%%%%%%%%%%%%%%%%%%%%%%
%%% Acknowledgements
%%%%%%%%%%%%%%%%%%%%%%%%%%%%%%%%%%%%%%%%%%%%%%%%%%%%%%%%%%%%%%%%%%%%%%%%%%%% 
\iffalse
\begin{acknowledgements}
  \ldots to Mari \amp\ Miri for continuous encouragement and affection, and
  to Buster Keaton and Saitama for filling life with awe and inspiration.
  To the developers and maintainers of \LaTeX, Emacs, AUC\TeX, Open Science
  Framework, R, Python, Inkscape, Sci-Hub for making a free and impartial
  scientific exchange possible.
%\rotatebox{15}{P}\rotatebox{5}{I}\rotatebox{-10}{P}\rotatebox{10}{\reflectbox{P}}\rotatebox{-5}{O}.
%\sourceatright{\autanet}
\mbox{}\hfill\autanet
\end{acknowledgements}
\fi

%%%%%%%%%%%%%%%%%%%%%%%%%%%%%%%%%%%%%%%%%%%%%%%%%%%%%%%%%%%%%%%%%%%%%%%%%%%%
%%% Appendices
%%%%%%%%%%%%%%%%%%%%%%%%%%%%%%%%%%%%%%%%%%%%%%%%%%%%%%%%%%%%%%%%%%%%%%%%%%%% 
%\clearpage
\bigskip
% %\renewcommand*{\appendixpagename}{Appendix}
% %\renewcommand*{\appendixname}{Appendix}
% %\appendixpage
% \appendix

%%%%%%%%%%%%%%%%%%%%%%%%%%%%%%%%%%%%%%%%%%%%%%%%%%%%%%%%%%%%%%%%%%%%%%%%%%%%
%%% Bibliography
%%%%%%%%%%%%%%%%%%%%%%%%%%%%%%%%%%%%%%%%%%%%%%%%%%%%%%%%%%%%%%%%%%%%%%%%%%%% 
\renewcommand*{\finalnamedelim}{\addcomma\space}
\defbibnote{prenote}{{\footnotesize (\enquote{de $X$} is listed under D,
    \enquote{van $X$} under V, and so on, regardless of national
    conventions.)\par}}
% \defbibnote{postnote}{\par\medskip\noindent{\footnotesize% Note:
%     \arxivp \mparcp \philscip \biorxivp}}

\printbibliography[prenote=prenote%,postnote=postnote
]

\end{document}

%%%%%%%%%%%%%%%%%%%%%%%%%%%%%%%%%%%%%%%%%%%%%%%%%%%%%%%%%%%%%%%%%%%%%%%%%%%%
%%% Cut text (won't be compiled)
%%%%%%%%%%%%%%%%%%%%%%%%%%%%%%%%%%%%%%%%%%%%%%%%%%%%%%%%%%%%%%%%%%%%%%%%%%%% 


%%% Local Variables: 
%%% mode: LaTeX
%%% TeX-PDF-mode: t
%%% TeX-master: t
%%% End: 
