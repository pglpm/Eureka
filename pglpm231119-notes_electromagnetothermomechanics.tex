\pdfoutput=1
%% Author: PGL  Porta Mana
%% Created: 2022-07-20T21:21:10+0100
%% Last-Updated: 2024-12-05T07:32:35+0100
%%%%%%%%%%%%%%%%%%%%%%%%%%%%%%%%%%%%%%%%%%%%%%%%%%%%%%%%%%%%%%%%%%%%%%%%%%%%
\newif\ifarxiv
\arxivfalse
%\iftrue\pdfmapfile{+classico.map}\fi
\newif\ifafour
\afourfalse% true = A4, false = A5
\newif\iftypodisclaim % typographical disclaim on the side
\typodisclaimtrue
\newcommand*{\memfontfamily}{zplr}
\newcommand*{\memfontpack}{newpx}
\documentclass[\ifafour a4paper,12pt,\else a5paper,10pt,\fi%extrafontsizes,%
onecolumn,oneside,article,%french,italian,german,swedish,latin,
british%
]{memoir}
\newcommand*{\firstdraft}{20 July 2022}
\newcommand*{\firstpublished}{\firstdraft}
\newcommand*{\updated}{\ifarxiv***\else\today\ [draft]\fi}
\newcommand*{\propertitle}{\Large Notes on general-relativistic\\continuum electromagneto-thermo-mechanics}
\newcommand*{\pdftitle}{Notes on general-relativistic\\continuum electromagneto-thermo-mechanics}
\newcommand*{\headtitle}{General mechanics}
\newcommand*{\pdfauthor}{P.G.L.  Porta Mana}
\newcommand*{\headauthor}{Porta Mana}
\newcommand*{\reporthead}{\iftrue\else Open Science Framework \href{https://doi.org/10.31219/osf.io/***}{\textsc{doi}:10.31219/osf.io/***}\fi}% Report number

%%%%%%%%%%%%%%%%%%%%%%%%%%%%%%%%%%%%%%%%%%%%%%%%%%%%%%%%%%%%%%%%%%%%%%%%%%%%
%%% Calls to packages (uncomment as needed)
%%%%%%%%%%%%%%%%%%%%%%%%%%%%%%%%%%%%%%%%%%%%%%%%%%%%%%%%%%%%%%%%%%%%%%%%%%%%

%\usepackage{pifont}

%\usepackage{fontawesome}
\PassOptionsToPackage{obeyspaces}{url}
\usepackage[T1]{fontenc}
\input{glyphtounicode} \pdfgentounicode=1

\usepackage[utf8]{inputenx}

%\usepackage{newunicodechar}
% \newunicodechar{Ĕ}{\u{E}}
% \newunicodechar{ĕ}{\u{e}}
% \newunicodechar{Ĭ}{\u{I}}
% \newunicodechar{ĭ}{\u{\i}}
% \newunicodechar{Ŏ}{\u{O}}
% \newunicodechar{ŏ}{\u{o}}
% \newunicodechar{Ŭ}{\u{U}}
% \newunicodechar{ŭ}{\u{u}}
% \newunicodechar{Ā}{\=A}
% \newunicodechar{ā}{\=a}
% \newunicodechar{Ē}{\=E}
% \newunicodechar{ē}{\=e}
% \newunicodechar{Ī}{\=I}
% \newunicodechar{ī}{\={\i}}
% \newunicodechar{Ō}{\=O}
% \newunicodechar{ō}{\=o}
% \newunicodechar{Ū}{\=U}
% \newunicodechar{ū}{\=u}
% \newunicodechar{Ȳ}{\=Y}
% \newunicodechar{ȳ}{\=y}

\newcommand*{\bmmax}{0} % reduce number of bold fonts, before font packages
\newcommand*{\hmmax}{0} % reduce number of heavy fonts, before font packages

\usepackage{textcomp}

%\usepackage[normalem]{ulem}% package for underlining
% \makeatletter
% \def\ssout{\bgroup \ULdepth=-.35ex%\UL@setULdepth
%  \markoverwith{\lower\ULdepth\hbox
%    {\kern-.03em\vbox{\hrule width.2em\kern1.2\p@\hrule}\kern-.03em}}%
%  \ULon}
% \makeatother

\usepackage{amsmath}

\usepackage{mathtools}
%\addtolength{\jot}{\jot} % increase spacing in multiline formulae
\setlength{\multlinegap}{0pt}

% \usepackage{empheq}% automatically calls amsmath and mathtools
% \newcommand*{\widefbox}[1]{\fbox{\hspace{1em}#1\hspace{1em}}}

%%%% empheq above seems more versatile than these:
\usepackage{fancybox}
% \usepackage{framed}

% \usepackage[misc]{ifsym} % for dice
% \newcommand*{\diceone}{{\scriptsize\Cube{1}}}

\usepackage{amssymb}

\usepackage{amsxtra}

\usepackage[main=british]{babel}\selectlanguage{british}
%\newcommand*{\langnohyph}{\foreignlanguage{nohyphenation}}
\newcommand{\langnohyph}[1]{\begin{hyphenrules}{nohyphenation}#1\end{hyphenrules}}

\usepackage[autostyle=false,autopunct=false,english=british]{csquotes}
\setquotestyle{american}
\newcommand*{\defquote}[1]{`\,#1\,'}

% \makeatletter
% \renewenvironment{quotation}%
%                {\list{}{\listparindent 1.5em%
%                         \itemindent    \listparindent
%                         \rightmargin=1em   \leftmargin=1em
%                         \parsep        \z@ \@plus\p@}%
%                 \item[]\footnotesize}%
%                 {\endlist}
% \makeatother


\usepackage{amsthm}
%% from https://tex.stackexchange.com/a/404680/97039
\makeatletter
\def\@endtheorem{\endtrivlist}
\makeatother

\newcommand*{\QED}{\textsc{q.e.d.}}
\renewcommand*{\qedsymbol}{\QED}
\theoremstyle{remark}
\newtheorem{note}{Note}
\newtheorem*{remark}{Note}
\newtheoremstyle{innote}{\parsep}{\parsep}{\footnotesize}{}{}{}{0pt}{}
\theoremstyle{innote}
\newtheorem*{innote}{}

\usepackage[shortlabels,inline]{enumitem}
\SetEnumitemKey{para}{itemindent=\parindent,leftmargin=0pt,listparindent=\parindent,parsep=0pt,itemsep=\topsep}
% \begin{asparaenum} = \begin{enumerate}[para]
% \begin{inparaenum} = \begin{enumerate*}
\setlist{itemsep=0pt,topsep=\parsep}
\setlist[enumerate,2]{label=(\roman*)}
\setlist[enumerate]{label=(\alph*),leftmargin=1.5\parindent}
\setlist[itemize]{leftmargin=1.5\parindent}
\setlist[description]{leftmargin=1.5\parindent}
% old alternative:
% \setlist[enumerate,2]{label=\alph*.}
% \setlist[enumerate]{leftmargin=\parindent}
% \setlist[itemize]{leftmargin=\parindent}
% \setlist[description]{leftmargin=\parindent}

\usepackage[babel,theoremfont,largesc,smallerops,nosymbolsc]{newpx}

% For Baskerville see https://ctan.org/tex-archive/fonts/baskervillef?lang=en
% and http://mirrors.ctan.org/fonts/baskervillef/doc/baskervillef-doc.pdf
% \usepackage[p]{baskervillef}
% \usepackage[varqu,varl,var0]{inconsolata}
% \usepackage[scale=.95,type1]{cabin}
% \usepackage[baskerville,vvarbb]{newtxmath}
% \usepackage[cal=boondoxo]{mathalfa}


% \usepackage[bigdelims,nosymbolsc%,smallerops % probably arXiv doesn't have it
% ]{newpxmath}
%\useosf
%\linespread{1.083}%
%\linespread{1.05}% widely used
\linespread{1.1}% best for text with maths
%% smaller operators for old version of newpxmath
\makeatletter
\def\re@DeclareMathSymbol#1#2#3#4{%
    \let#1=\undefined
    \DeclareMathSymbol{#1}{#2}{#3}{#4}}
%\re@DeclareMathSymbol{\bigsqcupop}{\mathop}{largesymbols}{"46}
%\re@DeclareMathSymbol{\bigodotop}{\mathop}{largesymbols}{"4A}
\re@DeclareMathSymbol{\bigoplusop}{\mathop}{largesymbols}{"4C}
\re@DeclareMathSymbol{\bigotimesop}{\mathop}{largesymbols}{"4E}
\re@DeclareMathSymbol{\sumop}{\mathop}{largesymbols}{"50}
\re@DeclareMathSymbol{\prodop}{\mathop}{largesymbols}{"51}
\re@DeclareMathSymbol{\bigcupop}{\mathop}{largesymbols}{"53}
\re@DeclareMathSymbol{\bigcapop}{\mathop}{largesymbols}{"54}
%\re@DeclareMathSymbol{\biguplusop}{\mathop}{largesymbols}{"55}
\re@DeclareMathSymbol{\bigwedgeop}{\mathop}{largesymbols}{"56}
\re@DeclareMathSymbol{\bigveeop}{\mathop}{largesymbols}{"57}
%\re@DeclareMathSymbol{\bigcupdotop}{\mathop}{largesymbols}{"DF}
%\re@DeclareMathSymbol{\bigcapplusop}{\mathop}{largesymbolsPXA}{"00}
%\re@DeclareMathSymbol{\bigsqcupplusop}{\mathop}{largesymbolsPXA}{"02}
%\re@DeclareMathSymbol{\bigsqcapplusop}{\mathop}{largesymbolsPXA}{"04}
%\re@DeclareMathSymbol{\bigsqcapop}{\mathop}{largesymbolsPXA}{"06}
\re@DeclareMathSymbol{\bigtimesop}{\mathop}{largesymbolsPXA}{"10}
%\re@DeclareMathSymbol{\coprodop}{\mathop}{largesymbols}{"60}
%\re@DeclareMathSymbol{\varprod}{\mathop}{largesymbolsPXA}{16}
\makeatother
%%
%% With euler font cursive for Greek letters - the [1] means 100% scaling
\DeclareFontFamily{U}{egreek}{\skewchar\font'177}%
\DeclareFontShape{U}{egreek}{m}{n}{<-6>s*[1]eurm5 <6-8>s*[1]eurm7 <8->s*[1]eurm10}{}%
\DeclareFontShape{U}{egreek}{m}{it}{<->s*[1]eurmo10}{}%
\DeclareFontShape{U}{egreek}{b}{n}{<-6>s*[1]eurb5 <6-8>s*[1]eurb7 <8->s*[1]eurb10}{}%
\DeclareFontShape{U}{egreek}{b}{it}{<->s*[1]eurbo10}{}%
\DeclareSymbolFont{egreeki}{U}{egreek}{m}{it}%
\SetSymbolFont{egreeki}{bold}{U}{egreek}{b}{it}% from the amsfonts package
\DeclareSymbolFont{egreekr}{U}{egreek}{m}{n}%
\SetSymbolFont{egreekr}{bold}{U}{egreek}{b}{n}% from the amsfonts package
% Take also \sum, \prod, \coprod symbols from Euler fonts
\DeclareFontFamily{U}{egreekx}{\skewchar\font'177}
\DeclareFontShape{U}{egreekx}{m}{n}{%
       <-7.5>s*[0.9]euex7%
    <7.5-8.5>s*[0.9]euex8%
    <8.5-9.5>s*[0.9]euex9%
    <9.5->s*[0.9]euex10%
}{}
\DeclareSymbolFont{egreekx}{U}{egreekx}{m}{n}
\DeclareMathSymbol{\sumop}{\mathop}{egreekx}{"50}
\DeclareMathSymbol{\prodop}{\mathop}{egreekx}{"51}
\DeclareMathSymbol{\coprodop}{\mathop}{egreekx}{"60}
\makeatletter
\def\sum{\DOTSI\sumop\slimits@}
\def\prod{\DOTSI\prodop\slimits@}
\def\coprod{\DOTSI\coprodop\slimits@}
\makeatother
%%%% Greek letters not usually given in LaTeX
%%%% best to uncomment only the ones needed
%% %% \input{definegreek.tex} % originally in a separate file
\DeclareMathSymbol{\varpartial}{\mathalpha}{egreeki}{"40}
%\DeclareMathSymbol{\partialup}{\mathalpha}{egreekr}{"40}
% \DeclareMathSymbol{\alpha}{\mathalpha}{egreeki}{"0B}
% \DeclareMathSymbol{\beta}{\mathalpha}{egreeki}{"0C}
% \DeclareMathSymbol{\gamma}{\mathalpha}{egreeki}{"0D}
% \DeclareMathSymbol{\delta}{\mathalpha}{egreeki}{"0E}
% \DeclareMathSymbol{\epsilon}{\mathalpha}{egreeki}{"0F}
% \DeclareMathSymbol{\zeta}{\mathalpha}{egreeki}{"10}
% \DeclareMathSymbol{\eta}{\mathalpha}{egreeki}{"11}
% \DeclareMathSymbol{\theta}{\mathalpha}{egreeki}{"12}
% \DeclareMathSymbol{\iota}{\mathalpha}{egreeki}{"13}
% \DeclareMathSymbol{\kappa}{\mathalpha}{egreeki}{"14}
% \DeclareMathSymbol{\lambda}{\mathalpha}{egreeki}{"15}
% \DeclareMathSymbol{\mu}{\mathalpha}{egreeki}{"16}
% \DeclareMathSymbol{\nu}{\mathalpha}{egreeki}{"17}
% \DeclareMathSymbol{\xi}{\mathalpha}{egreeki}{"18}
% \DeclareMathSymbol{\omicron}{\mathalpha}{egreeki}{"6F}
% \DeclareMathSymbol{\pi}{\mathalpha}{egreeki}{"19}
% \DeclareMathSymbol{\rho}{\mathalpha}{egreeki}{"1A}
% \DeclareMathSymbol{\sigma}{\mathalpha}{egreeki}{"1B}
% \DeclareMathSymbol{\tau}{\mathalpha}{egreeki}{"1C}
% \DeclareMathSymbol{\upsilon}{\mathalpha}{egreeki}{"1D}
% \DeclareMathSymbol{\phi}{\mathalpha}{egreeki}{"1E}
% \DeclareMathSymbol{\chi}{\mathalpha}{egreeki}{"1F}
% \DeclareMathSymbol{\psi}{\mathalpha}{egreeki}{"20}
% \DeclareMathSymbol{\omega}{\mathalpha}{egreeki}{"21}
% \DeclareMathSymbol{\varepsilon}{\mathalpha}{egreeki}{"22}
% \DeclareMathSymbol{\vartheta}{\mathalpha}{egreeki}{"23}
% \DeclareMathSymbol{\varpi}{\mathalpha}{egreeki}{"24}
% \let\varrho\rho 
% \let\varsigma\sigma
% \let\varkappa\kappa
% \DeclareMathSymbol{\varphi}{\mathalpha}{egreeki}{"27}
% %
% \DeclareMathSymbol{\varAlpha}{\mathalpha}{egreeki}{"41}
% \DeclareMathSymbol{\varBeta}{\mathalpha}{egreeki}{"42}
% \DeclareMathSymbol{\varGamma}{\mathalpha}{egreeki}{"00}
% \DeclareMathSymbol{\varDelta}{\mathalpha}{egreeki}{"01}
% \DeclareMathSymbol{\varEpsilon}{\mathalpha}{egreeki}{"45}
% \DeclareMathSymbol{\varZeta}{\mathalpha}{egreeki}{"5A}
% \DeclareMathSymbol{\varEta}{\mathalpha}{egreeki}{"48}
% \DeclareMathSymbol{\varTheta}{\mathalpha}{egreeki}{"02}
% \DeclareMathSymbol{\varIota}{\mathalpha}{egreeki}{"49}
% \DeclareMathSymbol{\varKappa}{\mathalpha}{egreeki}{"4B}
% \DeclareMathSymbol{\varLambda}{\mathalpha}{egreeki}{"03}
% \DeclareMathSymbol{\varMu}{\mathalpha}{egreeki}{"4D}
% \DeclareMathSymbol{\varNu}{\mathalpha}{egreeki}{"4E}
% \DeclareMathSymbol{\varXi}{\mathalpha}{egreeki}{"04}
% \DeclareMathSymbol{\varOmicron}{\mathalpha}{egreeki}{"4F}
% \DeclareMathSymbol{\varPi}{\mathalpha}{egreeki}{"05}
% \DeclareMathSymbol{\varRho}{\mathalpha}{egreeki}{"50}
% \DeclareMathSymbol{\varSigma}{\mathalpha}{egreeki}{"06}
% \DeclareMathSymbol{\varTau}{\mathalpha}{egreeki}{"54}
% \DeclareMathSymbol{\varUpsilon}{\mathalpha}{egreeki}{"07}
% \DeclareMathSymbol{\varPhi}{\mathalpha}{egreeki}{"08}
% \DeclareMathSymbol{\varChi}{\mathalpha}{egreeki}{"58}
% \DeclareMathSymbol{\varPsi}{\mathalpha}{egreeki}{"09}
% \DeclareMathSymbol{\varOmega}{\mathalpha}{egreeki}{"0A} 
% %
% \DeclareMathSymbol{\Alpha}{\mathalpha}{egreekr}{"41}
% \DeclareMathSymbol{\Beta}{\mathalpha}{egreekr}{"42}
% \DeclareMathSymbol{\Gamma}{\mathalpha}{egreekr}{"00}
% \DeclareMathSymbol{\Delta}{\mathalpha}{egreekr}{"01}
% \DeclareMathSymbol{\Epsilon}{\mathalpha}{egreekr}{"45}
% \DeclareMathSymbol{\Zeta}{\mathalpha}{egreekr}{"5A}
% \DeclareMathSymbol{\Eta}{\mathalpha}{egreekr}{"48}
% \DeclareMathSymbol{\Theta}{\mathalpha}{egreekr}{"02}
% \DeclareMathSymbol{\Iota}{\mathalpha}{egreekr}{"49}
% \DeclareMathSymbol{\Kappa}{\mathalpha}{egreekr}{"4B}
% \DeclareMathSymbol{\Lambda}{\mathalpha}{egreekr}{"03}
% \DeclareMathSymbol{\Mu}{\mathalpha}{egreekr}{"4D}
% \DeclareMathSymbol{\Nu}{\mathalpha}{egreekr}{"4E}
% \DeclareMathSymbol{\Xi}{\mathalpha}{egreekr}{"04}
% \DeclareMathSymbol{\Omicron}{\mathalpha}{egreekr}{"4F}
% \DeclareMathSymbol{\Pi}{\mathalpha}{egreekr}{"05}
% \DeclareMathSymbol{\Rho}{\mathalpha}{egreekr}{"50}
% \DeclareMathSymbol{\Sigma}{\mathalpha}{egreekr}{"06}
% \DeclareMathSymbol{\Tau}{\mathalpha}{egreekr}{"54}
% \DeclareMathSymbol{\Upsilon}{\mathalpha}{egreekr}{"07}
% \DeclareMathSymbol{\Phi}{\mathalpha}{egreekr}{"08}
% \DeclareMathSymbol{\Chi}{\mathalpha}{egreekr}{"58}
% \DeclareMathSymbol{\Psi}{\mathalpha}{egreekr}{"09}
% \DeclareMathSymbol{\Omega}{\mathalpha}{egreekr}{"0A}
% %
% \DeclareMathSymbol{\alphaup}{\mathalpha}{egreekr}{"0B}
% \DeclareMathSymbol{\betaup}{\mathalpha}{egreekr}{"0C}
% \DeclareMathSymbol{\gammaup}{\mathalpha}{egreekr}{"0D}
\DeclareMathSymbol{\deltaup}{\mathalpha}{egreekr}{"0E}
% \DeclareMathSymbol{\epsilonup}{\mathalpha}{egreekr}{"0F}
% \DeclareMathSymbol{\zetaup}{\mathalpha}{egreekr}{"10}
% \DeclareMathSymbol{\etaup}{\mathalpha}{egreekr}{"11}
% \DeclareMathSymbol{\thetaup}{\mathalpha}{egreekr}{"12}
% \DeclareMathSymbol{\iotaup}{\mathalpha}{egreekr}{"13}
% \DeclareMathSymbol{\kappaup}{\mathalpha}{egreekr}{"14}
% \DeclareMathSymbol{\lambdaup}{\mathalpha}{egreekr}{"15}
% \DeclareMathSymbol{\muup}{\mathalpha}{egreekr}{"16}
% \DeclareMathSymbol{\nuup}{\mathalpha}{egreekr}{"17}
% \DeclareMathSymbol{\xiup}{\mathalpha}{egreekr}{"18}
% \DeclareMathSymbol{\omicronup}{\mathalpha}{egreekr}{"6F}
\DeclareMathSymbol{\piup}{\mathalpha}{egreekr}{"19}
% \DeclareMathSymbol{\rhoup}{\mathalpha}{egreekr}{"1A}
% \DeclareMathSymbol{\sigmaup}{\mathalpha}{egreekr}{"1B}
% \DeclareMathSymbol{\tauup}{\mathalpha}{egreekr}{"1C}
% \DeclareMathSymbol{\upsilonup}{\mathalpha}{egreekr}{"1D}
% \DeclareMathSymbol{\phiup}{\mathalpha}{egreekr}{"1E}
% \DeclareMathSymbol{\chiup}{\mathalpha}{egreekr}{"1F}
% \DeclareMathSymbol{\psiup}{\mathalpha}{egreekr}{"20}
% \DeclareMathSymbol{\omegaup}{\mathalpha}{egreekr}{"21}
% \DeclareMathSymbol{\varepsilonup}{\mathalpha}{egreekr}{"22}
% \DeclareMathSymbol{\varthetaup}{\mathalpha}{egreekr}{"23}
% \DeclareMathSymbol{\varpiup}{\mathalpha}{egreekr}{"24}
% \let\varrhoup\rhoup 
% \let\varsigmaup\sigmaup
% \let\varkappaup\kappaup
% \DeclareMathSymbol{\varphiup}{\mathalpha}{egreekr}{"27}


% \usepackage%[scaled=0.9]%
% {classico}%  Optima as sans-serif font
\renewcommand\sfdefault{uop}
\DeclareMathAlphabet{\mathsf}  {T1}{\sfdefault}{m}{sl}
\SetMathAlphabet{\mathsf}{bold}{T1}{\sfdefault}{b}{sl}
\newcommand*{\mathte}[1]{\textbf{\textit{\textsf{#1}}}}
% Upright sans-serif math alphabet
% \DeclareMathAlphabet{\mathsu}  {T1}{\sfdefault}{m}{n}
% \SetMathAlphabet{\mathsu}{bold}{T1}{\sfdefault}{b}{n}

% DejaVu Mono as typewriter text
\usepackage[scaled=0.84]{DejaVuSansMono}

\usepackage{mathdots}

\usepackage[usenames]{xcolor}
% Tol (2012) colour-blind-, print-, screen-friendly colours, alternative scheme; Munsell terminology
\definecolor{bluepurple}{RGB}{68,119,170}
\definecolor{blue}{RGB}{102,204,238}
\definecolor{green}{RGB}{34,136,51}
\definecolor{yellow}{RGB}{204,187,68}
\definecolor{red}{RGB}{238,102,119}
\definecolor{redpurple}{RGB}{170,51,119}
\definecolor{grey}{RGB}{187,187,187}
% Tol (2012) colour-blind-, print-, screen-friendly colours; Munsell terminology
% \definecolor{lbpurple}{RGB}{51,34,136}
% \definecolor{lblue}{RGB}{136,204,238}
% \definecolor{lbgreen}{RGB}{68,170,153}
% \definecolor{lgreen}{RGB}{17,119,51}
% \definecolor{lgyellow}{RGB}{153,153,51}
% \definecolor{lyellow}{RGB}{221,204,119}
% \definecolor{lred}{RGB}{204,102,119}
% \definecolor{lpred}{RGB}{136,34,85}
% \definecolor{lrpurple}{RGB}{170,68,153}
\definecolor{lgrey}{RGB}{221,221,221}
%\newcommand*\mycolourbox[1]{%
%\colorbox{grey}{\hspace{1em}#1\hspace{1em}}}
\colorlet{shadecolor}{lgrey}

\usepackage{bm}

\usepackage{microtype}

\usepackage[backend=biber,mcite,%subentry,
citestyle=authoryear-comp,bibstyle=pglpm_latex/pglpm-authoryear,autopunct=false,sorting=ny,sortcites=false,natbib=false,maxcitenames=2,maxbibnames=8,minbibnames=8,giveninits=true,uniquename=false,uniquelist=false,maxalphanames=1,block=space,hyperref=true,defernumbers=false,useprefix=true,sortupper=false,language=british,parentracker=false,autocite=footnote]{biblatex}
\DeclareSortingTemplate{ny}{\sort{\field{sortname}\field{author}\field{editor}}\sort{\field{year}}}
\DeclareFieldFormat{postnote}{#1}
\iffalse\makeatletter%%% replace parenthesis with brackets
\newrobustcmd*{\parentexttrack}[1]{%
  \begingroup
  \blx@blxinit
  \blx@setsfcodes
  \blx@bibopenparen#1\blx@bibcloseparen
  \endgroup}
\AtEveryCite{%
  \let\parentext=\parentexttrack%
  \let\bibopenparen=\bibopenbracket%
  \let\bibcloseparen=\bibclosebracket}
\makeatother\fi
\DefineBibliographyExtras{british}{\def\finalandcomma{\addcomma}}
\renewcommand*{\finalnamedelim}{\addspace\amp\space}
% \renewcommand*{\finalnamedelim}{\addcomma\space}
\renewcommand*{\textcitedelim}{\addcomma\space}
% \setcounter{biburlnumpenalty}{1} % to allow url breaks anywhere
% \setcounter{biburlucpenalty}{0}
% \setcounter{biburllcpenalty}{1}
\setcounter{biburlucpenalty}{1}  %break URL after uppercase character
\setcounter{biburlnumpenalty}{1} %break URL after number
\setcounter{biburllcpenalty}{1}  %break URL after lowercase character
\DeclareDelimFormat{multicitedelim}{\addsemicolon\addspace\space}
\DeclareDelimFormat{compcitedelim}{\addsemicolon\addspace\space}
\DeclareDelimFormat{postnotedelim}{\addspace}
\ifarxiv\else\addbibresource{portamanabib.bib}\fi
\renewcommand{\bibfont}{\footnotesize}
%\appto{\citesetup}{\footnotesize}% smaller font for citations
\defbibheading{bibliography}[\bibname]{\section*{#1}\addcontentsline{toc}{section}{#1}%\markboth{#1}{#1}
}
\newcommand*{\citep}{\footcites}
\newcommand*{\citey}{\footcites}%{\parencites*}
\newcommand*{\ibid}{\unspace\addtocounter{footnote}{-1}\footnotemark{}}
%\renewcommand*{\cite}{\parencite}
%\renewcommand*{\cites}{\parencites}
\providecommand{\href}[2]{#2}
\providecommand{\eprint}[2]{\texttt{\href{#1}{#2}}}
\newcommand*{\amp}{\&}
% \newcommand*{\citein}[2][]{\textnormal{\textcite[#1]{#2}}%\addtocategory{extras}{#2}
% }
\newcommand*{\citein}[2][]{\textnormal{\textcite[#1]{#2}}%\addtocategory{extras}{#2}
}
\newcommand*{\citebi}[2][]{\textcite[#1]{#2}%\addtocategory{extras}{#2}
}
\newcommand*{\subtitleproc}[1]{}
\newcommand*{\chapb}{ch.}
%
%\def\UrlOrds{\do\*\do\-\do\~\do\'\do\"\do\-}%
% \def\myUrlOrds{\do\0\do\1\do\2\do\3\do\4\do\5\do\6\do\7\do\8\do\9\do\a\do\b\do\c\do\d\do\e\do\f\do\g\do\h\do\i\do\j\do\k\do\l\do\m\do\n\do\o\do\p\do\q\do\r\do\s\do\t\do\u\do\v\do\w\do\x\do\y\do\z\do\A\do\B\do\C\do\D\do\E\do\F\do\G\do\H\do\I\do\J\do\K\do\L\do\M\do\N\do\O\do\P\do\Q\do\R\do\S\do\T\do\U\do\V\do\W\do\X\do\Y\do\Z}%
\makeatletter
%\g@addto@macro\UrlSpecials{\do={\newline}}
\g@addto@macro{\UrlBreaks}{%
\do\0\do\1\do\2\do\3\do\4\do\5\do\6\do\7\do\8\do\9\do\a\do\b\do\c\do\d\do\e\do\f\do\g\do\h\do\i\do\j\do\k\do\l\do\m\do\n\do\o\do\p\do\q\do\r\do\s\do\t\do\u\do\v\do\w\do\x\do\y\do\z\do\A\do\B\do\C\do\D\do\E\do\F\do\G\do\H\do\I\do\J\do\K\do\L\do\M\do\N\do\O\do\P\do\Q\do\R\do\S\do\T\do\U\do\V\do\W\do\X\do\Y\do\Z%
}
\g@addto@macro\UrlSpecials{%
\do\/{\mbox{\UrlFont/}\hskip 0pt plus 10pt}%
}
\makeatother
\newcommand*{\arxiveprint}[1]{%
arXiv \doi{10.48550/arXiv.#1}%
}
\newcommand*{\mparceprint}[1]{%
\href{http://www.ma.utexas.edu/mp_arc-bin/mpa?yn=#1}{mp_arc:\allowbreak\nolinkurl{#1}}%
}
\newcommand*{\haleprint}[1]{%
\href{https://hal.archives-ouvertes.fr/#1}{\textsc{hal}:\allowbreak\nolinkurl{#1}}%
}
\newcommand*{\philscieprint}[1]{%
\href{http://philsci-archive.pitt.edu/archive/#1}{PhilSci:\allowbreak\nolinkurl{#1}}%
}
\newcommand*{\doi}[1]{%
\href{https://doi.org/#1}{\textsc{doi}:\allowbreak\nolinkurl{#1}}%
}
\newcommand*{\biorxiveprint}[1]{%
bioRxiv \doi{10.1101/#1}%
}
\newcommand*{\osfeprint}[1]{%
Open Science Framework \doi{10.31219/osf.io/#1}%
}
\newcommand*{\osfproj}[1]{%
Open Science Framework \doi{10.17605/osf.io/#1}%
}

\usepackage{graphicx}

%\usepackage{wrapfig}

%\usepackage{tikz-cd}

\PassOptionsToPackage{hyphens}{url}\usepackage[hypertexnames=false,pdfencoding=unicode,psdextra]{hyperref}

\usepackage{pdfrender}
% \newcommand*{\textxbf}[1]{\textpdfrender{TextRenderingMode=2,LineWidth=0.3pt}{\textbf{#1}}}
\renewcommand*{\bm}[1]{\textpdfrender{TextRenderingMode=2,LineWidth=0.1pt}{\boldsymbol{#1}}}

\usepackage[depth=4]{bookmark}
\hypersetup{colorlinks=true,bookmarksnumbered,pdfborder={0 0 0.25},citebordercolor={0.2667 0.4667 0.6667},citecolor=bluepurple,linkbordercolor={0.6667 0.2 0.4667},linkcolor=redpurple,urlbordercolor={0.1333 0.5333 0.2},urlcolor=green,breaklinks=true,pdftitle={\pdftitle},pdfauthor={\pdfauthor}}
% \usepackage[vertfit=local]{breakurl}% only for arXiv
\providecommand*{\urlalt}{\href}

\usepackage{tensor}


\usepackage[british]{datetime2}
\DTMnewdatestyle{mydate}%
{% definitions
\renewcommand*{\DTMdisplaydate}[4]{%
\number##3\ \DTMenglishmonthname{##2} ##1}%
\renewcommand*{\DTMDisplaydate}{\DTMdisplaydate}%
}
\DTMsetdatestyle{mydate}

%%%%%%%%%%%%%%%%%%%%%%%%%%%%%%%%%%%%%%%%%%%%%%%%%%%%%%%%%%%%%%%%%%%%%%%%%%%%
%%% Layout. I do not know on which kind of paper the reader will print the
%%% paper on (A4? letter? one-sided? double-sided?). So I choose A5, which
%%% provides a good layout for reading on screen and save paper if printed
%%% two pages per sheet. Average length line is 66 characters and page
%%% numbers are centred.
%%%%%%%%%%%%%%%%%%%%%%%%%%%%%%%%%%%%%%%%%%%%%%%%%%%%%%%%%%%%%%%%%%%%%%%%%%%%
\ifafour\setstocksize{297mm}{210mm}%{*}% A4
\else\setstocksize{210mm}{5.5in}%{*}% 210x139.7
\fi
\settrimmedsize{\stockheight}{\stockwidth}{*}
\setlxvchars[\normalfont] %313.3632pt for a 66-characters line
\setxlvchars[\normalfont]
% \setlength{\trimtop}{0pt}
% \setlength{\trimedge}{\stockwidth}
% \addtolength{\trimedge}{-\paperwidth}
%\settrims{0pt}{0pt}
% The length of the normalsize alphabet is 133.05988pt - 10 pt = 26.1408pc
% The length of the normalsize alphabet is 159.6719pt - 12pt = 30.3586pc
% Bringhurst gives 32pc as boundary optimal with 69 ch per line
% The length of the normalsize alphabet is 191.60612pt - 14pt = 35.8634pc
\ifafour\settypeblocksize{*}{32pc}{1.618} % A4
%\setulmargins{*}{*}{1.667}%gives 5/3 margins % 2 or 1.667
\else\settypeblocksize{*}{26pc}{1.618}% nearer to a 66-line newpx and preserves GR
\fi
\setulmargins{*}{*}{1}%gives equal margins
\setlrmargins{*}{*}{*}
\setheadfoot{\onelineskip}{2.5\onelineskip}
\setheaderspaces{*}{2\onelineskip}{*}
\setmarginnotes{2ex}{10mm}{0pt}
\checkandfixthelayout[nearest]
%%% End layout
%% this fixes missing white spaces
%\pdfmapline{+dummy-space <dummy-space.pfb}
%\pdfinterwordspaceon% seems to add a white margin to Sumatrapdf

%%% Sectioning
\newcommand*{\asudedication}[1]{%
{\par\centering\textit{#1}\par}}
\newenvironment{acknowledgements}{\section*{Thanks}\addcontentsline{toc}{section}{Thanks}}{\par}
\makeatletter\renewcommand{\appendix}{\par
  \bigskip{\centering
   \interlinepenalty \@M
   \normalfont
   \printchaptertitle{\sffamily\appendixpagename}\par}
  \setcounter{section}{0}%
  \gdef\@chapapp{\appendixname}%
  \gdef\thesection{\@Alph\c@section}%
  \anappendixtrue}\makeatother
\counterwithout{section}{chapter}
\setsecnumformat{\upshape\csname the#1\endcsname\quad}
\setsecheadstyle{\large\bfseries\sffamily%
\centering}
\setsubsecheadstyle{\bfseries\sffamily%
\raggedright}
%\setbeforesecskip{-1.5ex plus 1ex minus .2ex}% plus 1ex minus .2ex}
%\setaftersecskip{1.3ex plus .2ex }% plus 1ex minus .2ex}
%\setsubsubsecheadstyle{\bfseries\sffamily\slshape\raggedright}
%\setbeforesubsecskip{1.25ex plus 1ex minus .2ex }% plus 1ex minus .2ex}
%\setaftersubsecskip{-1em}%{-0.5ex plus .2ex}% plus 1ex minus .2ex}
\setsubsecindent{0pt}%0ex plus 1ex minus .2ex}
\setparaheadstyle{\bfseries\sffamily%
\raggedright}
\setcounter{secnumdepth}{2}
\setlength{\headwidth}{\textwidth}
\newcommand{\addchap}[1]{\chapter*[#1]{#1}\addcontentsline{toc}{chapter}{#1}}
\newcommand{\addsec}[1]{\section*{#1}\addcontentsline{toc}{section}{#1}}
\newcommand{\addsubsec}[1]{\subsection*{#1}\addcontentsline{toc}{subsection}{#1}}
\newcommand{\addpara}[1]{\paragraph*{#1.}\addcontentsline{toc}{subsubsection}{#1}}
\newcommand{\addparap}[1]{\paragraph*{#1}\addcontentsline{toc}{subsubsection}{#1}}

%%% Headers, footers, pagestyle
\copypagestyle{manaart}{plain}
\makeheadrule{manaart}{\headwidth}{0.5\normalrulethickness}
\makeoddhead{manaart}{%
{\footnotesize%\sffamily%
\scshape\headauthor}}{}{{\footnotesize\sffamily%
\headtitle}}
\makeoddfoot{manaart}{}{\thepage}{}
\newcommand*\autanet{\includegraphics[height=\heightof{M}]{autanet.pdf}}
\definecolor{mygray}{gray}{0.333}
\iftypodisclaim%
\ifafour\newcommand\addprintnote{\begin{picture}(0,0)%
\put(245,149){\makebox(0,0){\rotatebox{90}{\tiny\color{mygray}\textsf{This
            document is designed for screen reading and
            two-up printing on A4 or Letter paper}}}}%
\end{picture}}% A4
\else\newcommand\addprintnote{\begin{picture}(0,0)%
\put(176,112){\makebox(0,0){\rotatebox{90}{\tiny\color{mygray}\textsf{This
            document is designed for screen reading and
            two-up printing on A4 or Letter paper}}}}%
\end{picture}}\fi%afourtrue
\makeoddfoot{plain}{}{\makebox[0pt]{\thepage}\addprintnote}{}
\else
\makeoddfoot{plain}{}{\makebox[0pt]{\thepage}}{}
\fi%typodisclaimtrue
\makeoddhead{plain}{\scriptsize\reporthead}{}{}
% \copypagestyle{manainitial}{plain}
% \makeheadrule{manainitial}{\headwidth}{0.5\normalrulethickness}
% \makeoddhead{manainitial}{%
% \footnotesize\sffamily%
% \scshape\headauthor}{}{\footnotesize\sffamily%
% \headtitle}
% \makeoddfoot{manaart}{}{\thepage}{}

\pagestyle{manaart}

\setlength{\droptitle}{-3.9\onelineskip}
\pretitle{\begin{center}\LARGE\sffamily%
\bfseries}
\posttitle{\bigskip\end{center}}

\makeatletter\newcommand*{\atf}{\includegraphics[totalheight=\heightof{@}]{pglpm_latex/atblack.png}}\makeatother
\providecommand{\affiliation}[1]{\textsl{\textsf{\footnotesize #1}}}
\providecommand{\epost}[1]{\texttt{\footnotesize\textless#1\textgreater}}
\providecommand{\email}[2]{\href{mailto:#1ZZ@#2 ((remove ZZ))}{#1\protect\atf#2}}
%\providecommand{\email}[2]{\href{mailto:#1@#2}{#1@#2}}

\preauthor{\vspace{-0.5\baselineskip}\begin{center}
\normalsize\sffamily%
\lineskip  0.5em}
\postauthor{\par\end{center}}
\predate{\DTMsetdatestyle{mydate}\begin{center}\footnotesize}
\postdate{\end{center}\vspace{-\medskipamount}}

\setfloatadjustment{figure}{\footnotesize}
\captiondelim{\quad}
\captionnamefont{\footnotesize\sffamily%
}
\captiontitlefont{\footnotesize}
%\firmlists*
\midsloppy
% handling orphan/widow lines, memman.pdf
% \clubpenalty=10000
% \widowpenalty=10000
% \raggedbottom
% Downes, memman.pdf
\clubpenalty=9996
\widowpenalty=9999
\brokenpenalty=4991
\predisplaypenalty=10000
\postdisplaypenalty=1549
\displaywidowpenalty=1602
\raggedbottom

\paragraphfootnotes
\setlength{\footmarkwidth}{2ex}
% \threecolumnfootnotes
%\setlength{\footmarksep}{0em}
\footmarkstyle{\textsuperscript{%\color{red}
\scriptsize\bfseries#1}~}
%\footmarkstyle{\textsuperscript{\color{red}\scriptsize\bfseries#1}~}
%\footmarkstyle{\textsuperscript{[#1]}~}

\selectlanguage{british}\frenchspacing

\definecolor{notecolour}{RGB}{68,170,153}
%\newcommand*{\puzzle}{\maltese}
\newcommand*{\puzzle}{{\fontencoding{U}\fontfamily{fontawesometwo}\selectfont\symbol{225}}}
\newcommand*{\wrench}{{\fontencoding{U}\fontfamily{fontawesomethree}\selectfont\symbol{114}}}
\newcommand*{\pencil}{{\fontencoding{U}\fontfamily{fontawesometwo}\selectfont\symbol{210}}}
\newcommand{\mynotew}[1]{{\footnotesize\color{notecolour}\wrench\ #1}}
\newcommand{\mynotep}[1]{{\footnotesize\color{notecolour}\pencil\ #1}}
\newcommand{\mynotez}[1]{{\footnotesize\color{notecolour}\puzzle\ #1}}

%%%%%%%%%%%%%%%%%%%%%%%%%%%%%%%%%%%%%%%%%%%%%%%%%%%%%%%%%%%%%%%%%%%%%%%%%%%%
%%% Paper's details
%%%%%%%%%%%%%%%%%%%%%%%%%%%%%%%%%%%%%%%%%%%%%%%%%%%%%%%%%%%%%%%%%%%%%%%%%%%%
\title{\propertitle}
\author{%
\hspace*{\stretch{1}}%
%% uncomment if additional authors present
% \parbox{0.5\linewidth}%\makebox[0pt][c]%
% {\protect\centering ***\\%
% \footnotesize\epost{\email{***}{***}}}%
% \hspace*{\stretch{1}}%
\parbox{1\linewidth}%\makebox[0pt][c]%
{\protect\centering P.G.L.  Porta Mana  \href{https://orcid.org/0000-0002-6070-0784}{\protect\includegraphics[scale=0.16]{pglpm_latex/orcid_32x32.png}}\\\footnotesize
Western Norway University of Applied Sciences%
\quad\epost{\email{pgl}{portamana.org}}}%
% Mohn Medical Imaging and Visualization Centre, Dept of Computer science, Electrical Engineering and Mathematical Sciences, Western Norway University of Applied Sciences, Bergen, Norway
%% uncomment if additional authors present
% \hspace*{\stretch{1}}%
% \parbox{0.5\linewidth}%\makebox[0pt][c]%
% {\protect\centering ***\\%
% \footnotesize\epost{\email{***}{***}}}%
\hspace*{\stretch{1}}%
}

%\date{Draft of \today\ (first drafted \firstdraft)}
\date{\firstpublished; updated \updated}

%%%%%%%%%%%%%%%%%%%%%%%%%%%%%%%%%%%%%%%%%%%%%%%%%%%%%%%%%%%%%%%%%%%%%%%%%%%%
%%% Macros @@@
%%%%%%%%%%%%%%%%%%%%%%%%%%%%%%%%%%%%%%%%%%%%%%%%%%%%%%%%%%%%%%%%%%%%%%%%%%%%

% Common ones - uncomment as needed
%\providecommand{\nequiv}{\not\equiv}
%\providecommand{\coloneqq}{\mathrel{\mathop:}=}
%\providecommand{\eqqcolon}{=\mathrel{\mathop:}}
%\providecommand{\varprod}{\prod}
\newcommand*{\de}{\uppartial}%partial diff
\newcommand*{\pu}{\piup}%constant pi
\newcommand*{\delt}{\deltaup}%Kronecker, Dirac
%\newcommand*{\eps}{\varepsilonup}%Levi-Civita, Heaviside
%\newcommand*{\riem}{\zetaup}%Riemann zeta
%\providecommand{\degree}{\textdegree}% degree
%\newcommand*{\celsius}{\textcelsius}% degree Celsius
%\newcommand*{\micro}{\textmu}% degree Celsius
% \newcommand*{\I}{\mathrm{i}}%imaginary unit
\newcommand*{\I}{\ensuremath{\mathrm{i}}}
% \newcommand*{\e}{\mathrm{e}}%Neper
\newcommand*{\e}{\ensuremath{\mathrm{e}}}
\newcommand*{\di}{\mathrm{d}}%differential
% \newcommand*{\dii}{\ensuremath{\mathrm{d}}}
% %% From TUGboat 18 (1997) 1 - leads to very strange spacing
% \makeatletter
% \providecommand*{\di}%
% {\@ifnextchar^{\DIfF}{\DIfF^{}}}
% \def\DIfF^#1{%
% \mathop{\mathrm{\mathstrut d}}%
% \nolimits^{#1}\gobblespace}
% \def\gobblespace{%
% \futurelet\diffarg\opspace}
% \def\opspace{%
% \let\DiffSpace\!%
% \ifx\diffarg(%
% \let\DiffSpace\relax
% \else
% \ifx\diffarg[%
% \let\DiffSpace\relax
% \else
% \ifx\diffarg\{%
% \let\DiffSpace\relax
% \fi\fi\fi\DiffSpace}
% \makeatother

\newcommand*{\Di}{\mathrm{D}}%capital differential
\newcommand*{\Li}{\mathrm{L}}%Lie derivative
%\newcommand*{\planckc}{\hslash}
%\newcommand*{\avogn}{N_{\textrm{A}}}
%\newcommand*{\NN}{\bm{\mathrm{N}}}
%\newcommand*{\ZZ}{\bm{\mathrm{Z}}}
%\newcommand*{\QQ}{\bm{\mathrm{Q}}}
\newcommand*{\RR}{\bm{\mathrm{R}}}
%\newcommand*{\CC}{\bm{\mathrm{C}}}
\newcommand*{\nab}{\bm{\nabla}}%nabla
%\DeclareMathOperator{\lb}{lb}%base 2 log
\DeclareMathOperator{\tr}{tr}%trace
%\DeclareMathOperator{\card}{card}%cardinality
%% From TUGboat 18 (1997) 1
% \renewoperator{\Re}{\mathrm{Re}}{\nolimits}
% \renewoperator{\Im}{\mathrm{Im}}{\nolimits}
\DeclareMathOperator{\im}{Im}%im part
\DeclareMathOperator{\re}{Re}%re part
%\DeclareMathOperator{\sgn}{sgn}%signum
%\DeclareMathOperator{\ent}{ent}%integer less or equal to
\DeclareMathOperator{\Ord}{O}%same order as
%\DeclareMathOperator{\ord}{o}%lower order than
%\newcommand*{\incr}{\triangle}%finite increment
\newcommand*{\defd}{\coloneqq}
\newcommand*{\defs}{\eqqcolon}
%\newcommand*{\Land}{\bigwedge}
%\newcommand*{\Lor}{\bigvee}
%\newcommand*{\lland}{\DOTSB\;\land\;}
%\newcommand*{\llor}{\DOTSB\;\lor\;}
\newcommand*{\limplies}{\mathbin{\Rightarrow}}%implies
%\newcommand*{\suchthat}{\mid}%{\mathpunct{|}}%such that (eg in sets)
%\newcommand*{\with}{\colon}%with (list of indices)
%\newcommand*{\mul}{\times}%multiplication
%\newcommand*{\inn}{\cdot}%inner product
\newcommand*{\dotv}{\blacksquare}%variable place
%\newcommand*{\comp}{\circ}%composition of functions
%\newcommand*{\con}{\mathbin{:}}%scal prod of tensors
%\newcommand*{\equi}{\sim}%equivalent to 
\renewcommand*{\asymp}{\simeq}%equivalent to 
\newcommand*{\corr}{\mathrel{\hat{=}}}%corresponds to
%\providecommand{\varparallel}{\ensuremath{\mathbin{/\mkern-7mu/}}}%parallel (tentative symbol)
\renewcommand*{\le}{\leqslant}%less or equal
\renewcommand*{\ge}{\geqslant}%greater or equal
%\DeclarePairedDelimiter\clcl{[}{]}
%\DeclarePairedDelimiter\clop{[}{[}
%\DeclarePairedDelimiter\opcl{]}{]}
%\DeclarePairedDelimiter\opop{]}{[}%}
\DeclarePairedDelimiter\abs{\lvert}{\rvert}
%\DeclarePairedDelimiter\norm{\lVert}{\rVert}
\DeclarePairedDelimiter\set{\{}{\}} %}
%\DeclareMathOperator{\pr}{P}%probability
\newcommand*{\p}{\mathrm{p}}%probability
\renewcommand*{\P}{\mathrm{P}}%probability
%\newcommand*{\E}{\mathrm{E}}
%% The "\:" space is chosen to correctly separate inner binary and external rels
\renewcommand*{\|}[1][]{\nonscript\:#1\vert\nonscript\:\mathopen{}}
%\DeclarePairedDelimiterX{\cp}[2]{(}{)}{#1\nonscript\:\delimsize\vert\nonscript\:\mathopen{}#2}
%\DeclarePairedDelimiterX{\ct}[2]{[}{]}{#1\nonscript\;\delimsize\vert\nonscript\:\mathopen{}#2}
%\DeclarePairedDelimiterX{\cs}[2]{\{}{\}}{#1\nonscript\:\delimsize\vert\nonscript\:\mathopen{}#2}
%\newcommand*{\+}{\lor}
%\renewcommand{\*}{\land}
%% symbol = for equality statements within probabilities
%% from https://tex.stackexchange.com/a/484142/97039
% \newcommand*{\eq}{\mathrel{\!=\!}}
% \let\texteq\=
% \renewcommand*{\=}{\TextOrMath\texteq\eq}
% \newcommand*{\eq}[1][=]{\mathrel{\!#1\!}}
\newcommand*{\mo}[1][=]{\mathclose{}\mathord{\nonscript\mkern0.5mu#1\nonscript\mkern0.5mu}\mathopen{}}
%%
\newcommand*{\sect}{\S}% Sect.~
\newcommand*{\sects}{\S\S}% Sect.~
\newcommand*{\chap}{ch.}%
\newcommand*{\chaps}{chs}%
\newcommand*{\bref}{ref.}%
\newcommand*{\brefs}{refs}%
%\newcommand*{\fn}{fn}%
\newcommand*{\eqn}{eq.}%
\newcommand*{\eqns}{eqs}%
\newcommand*{\fig}{fig.}%
\newcommand*{\figs}{figs}%
\newcommand*{\vs}{{vs}}
\newcommand*{\eg}{{e.g.}}
\newcommand*{\etc}{{etc.}}
\newcommand*{\ie}{{i.e.}}
%\newcommand*{\ca}{{c.}}
\newcommand*{\foll}{{ff.}}
%\newcommand*{\viz}{{viz}}
\newcommand*{\cf}{{cf.}}
%\newcommand*{\Cf}{{Cf.}}
%\newcommand*{\vd}{{v.}}
\newcommand*{\etal}{{et al.}}
%\newcommand*{\etsim}{{et sim.}}
%\newcommand*{\ibid}{{ibid.}}
%\newcommand*{\sic}{{sic}}
%\newcommand*{\id}{\mathte{I}}%id matrix
%\newcommand*{\nbd}{\nobreakdash}%
%\newcommand*{\bd}{\hspace{0pt}}%
%\def\hy{-\penalty0\hskip0pt\relax}
\newcommand*{\labelbis}[1]{\tag*{(\ref{#1})$_\text{r}$}}
%\newcommand*{\mathbox}[2][.8]{\parbox[t]{#1\columnwidth}{#2}}
\newcommand*{\zerob}[1]{\makebox[0pt][l]{#1}}
\newcommand*{\tprod}{\mathop{\textstyle\prod}\nolimits}
\newcommand*{\tsum}{\mathop{\textstyle\sum}\nolimits}
%\newcommand*{\tint}{\begingroup\textstyle\int\endgroup\nolimits}
%\newcommand*{\tland}{\mathop{\textstyle\bigwedge}\nolimits}
%\newcommand*{\tlor}{\mathop{\textstyle\bigvee}\nolimits}
%\newcommand*{\sprod}{\mathop{\textstyle\prod}}
%\newcommand*{\ssum}{\mathop{\textstyle\sum}}
%\newcommand*{\sint}{\begingroup\textstyle\int\endgroup}
%\newcommand*{\sland}{\mathop{\textstyle\bigwedge}}
%\newcommand*{\slor}{\mathop{\textstyle\bigvee}}
\newcommand*{\T}{^\transp}%transpose
%%\newcommand*{\QEM}%{\textnormal{$\Box$}}%{\ding{167}}
%\newcommand*{\qem}{\leavevmode\unskip\penalty9999 \hbox{}\nobreak\hfill
%\quad\hbox{\QEM}}
%% from TUGboat 18 (1997) 1:
\providecommand*{\unit}[1]{\ensuremath{\mathrm{\,#1}}}

%%%%%%%%%%%%%%%%%%%%%%%%%%%%%%%%%%%%%%%%%%%%%%%%%%%%%%%%%%%%%%%%%%%%%%%%%%%%
%%% Custom macros for this file @@@
%%%%%%%%%%%%%%%%%%%%%%%%%%%%%%%%%%%%%%%%%%%%%%%%%%%%%%%%%%%%%%%%%%%%%%%%%%%%

\newcommand*{\widebar}[1]{{\mkern1.5mu\skew{2}\overline{\mkern-1.5mu#1\mkern-1.5mu}\mkern 1.5mu}}

\newcommand*{\iT}{^{-\transp}}%transpose
\newcommand*{\id}{\mathbf{id}}%id matrix

\renewcommand*{\i}{{}\indices}
%%
% from https://tex.stackexchange.com/a/424252/97039
\makeatletter
\newcommand*{\q}{}% Check if undefined
\DeclareRobustCommand*{\q}{%
  \mathord{\mathpalette\bigcdot@{}}% changed mathbin to mathord
}
\newcommand*{\bigcdot@scalefactor}{0.7}
\newcommand*{\bigcdot@widthfactor}{1.5}
\newcommand*{\bigcdot@}[2]{%
  % #1: math style
  % #2: unused
  \sbox0{$#1\vcenter{}$}% math axis
  \sbox2{$#1\cdot\m@th$}%
  \hbox to \bigcdot@widthfactor\wd2{%
    \hfil
    \raise\ht0\hbox{%
      \scalebox{\bigcdot@scalefactor}{%
        \lower\ht0\hbox{$#1\bullet\m@th$}%
      }%
    }%
    \hfil
  }%
}
\makeatother
%%
\newcommand*{\rul}{{\mkern2mu\rule[-0.1ex]{0.75pt}{1.1ex}\mkern2mu}}
\DeclarePairedDelimiter\mul{\rul}{\rul}%{{\bm{\shortmid}}}

% wedge with superimposed dot
\newcommand*{\dand}{\mathbin{\mathclap{\mkern12.5mu\bm{\cdot}}\wedge}}

% \newcommand{\explanation}[4][t]{%\setlength{\tabcolsep}{-1ex}
% %\smash{
% \begin{tabular}[#1]{c}#2\\[0.5\jot]\rule{1pt}{#3}\\#4\end{tabular}}%}
% \newcommand*{\ptext}[1]{\text{\small #1}}% for propositions
%\DeclareMathOperator*{\argsup}{arg\,sup}
% % \newcommand*{\tw}[2]{\underbracket[0pt][0pt]{#1}_{\scriptscriptstyle #2}}
% % \newcommand*{\ti}{\tw{\di}}
% % \newcommand*{\te}{\tw{\de}}
% % \newcommand*{\tw}[2][\scriptstyle\sim]{\smash[b]{\underset{\mathclap{{}^{#1}}}{#2}}}
% % \newcommand*{\ti}[1][\scriptstyle\sim]{\tw[#1]{\di}}
% % \newcommand*{\te}[1][\scriptstyle\sim]{\tw[#1]{\de}}
% \newcommand*{\da}{^{\dagger}}
\newcommand*{\ddi}{\bm{\di}}

\newcommand*{\se}[1]{\de_{#1}}
\newcommand*{\sse}[1]{\de^{2}_{#1}}
\newcommand*{\ssse}[1]{\de^{3}_{#1}}
\newcommand*{\sssse}[1]{\de^{4}_{#1}}
%
\newcommand*{\si}[1]{\di{#1}}
\newcommand*{\ssi}[1]{\di^{2}{#1}}
\newcommand*{\sssi}[1]{\di^{3}{#1}}
\newcommand*{\ssssi}[1]{\di^{4}{#1}}
% %
% \newcommand*{\te}[1]{\underset{\raisebox{1ex}{$\sim$}}{\de}{#1}}
% \newcommand*{\tte}[1]{\underset{\raisebox{1ex}{$\sim$}}{\de}^{2}{#1}}
% \newcommand*{\ttte}[1]{\underset{\raisebox{1ex}{$\sim$}}{\de}^{3}{#1}}
% \newcommand*{\tttte}[1]{\underset{\raisebox{1ex}{$\sim$}}{\de}^{4}{#1}}
% %
% \newcommand*{\ti}[1]{\smash[b]{\underset{\raisebox{1ex}{$\sim$}}{\di}}_{#1}}
% \newcommand*{\tti}[1]{\smash[b]{\underset{\raisebox{1ex}{$\sim$}}{\di}}^{2}_{#1}}
% \newcommand*{\ttti}[1]{\smash[b]{\underset{\raisebox{1ex}{$\sim$}}{\di}}^{3}_{#1}}
% \newcommand*{\tttti}[1]{\smash[b]{\underset{\raisebox{1ex}{$\sim$}}{\di}}^{4}_{#1}}
%
\newcommand*{\tw}[1]{\tilde{#1}}
\newcommand*{\te}[1]{\de{#1}}
\newcommand*{\tte}[1]{\de^{2}{#1}}
\newcommand*{\ttte}[1]{\de^{3}{#1}}
\newcommand*{\tttte}[1]{\de^{4}{#1}}
%
\newcommand*{\ti}[1]{\di_{#1}}
\newcommand*{\tti}[1]{\di^{2}_{#1}}
\newcommand*{\ttti}[1]{\di^{3}_{#1}}
\newcommand*{\tttti}[1]{\di^{4}_{#1}}
%
\newcommand*{\yg}{\mathte{g}}
\newcommand*{\ygg}[1][]{\overset{#1}{\bm{g}}{}}
\newcommand*{\dg}{\sqrt{g}}
\DeclarePairedDelimiter\nor{\lVert}{\rVert}
\newcommand*{\ve}{\tfrac{\bm{\sqrt{g}}}{c}}
\newcommand*{\vi}{\tfrac{c}{\bm{\sqrt{g}}}}
% \newcommand*{\vi}{\bar{\mathcal{G}}}
%
\newcommand*{\yN}{\bm{\mathcal{N}}}
\newcommand*{\yP}{\bm{P}}
\newcommand*{\yQ}{\bm{Q}}
\newcommand*{\yS}{\mathte{S}}
\newcommand*{\yTT}{\bm{\mathcal{T}}}
\newcommand*{\yT}{\mathcal{T}}
\newcommand*{\ye}{e}
\newcommand*{\yt}{\bm{\tau}}
\newcommand*{\yU}{\bm{U}}
\newcommand*{\yUd}{\bar{\bm{U}}}
\newcommand*{\yV}{\bm{V}}
\newcommand*{\yu}{\bm{u}}
\newcommand*{\yv}{\bm{v}}
\newcommand*{\yo}{\bm{\omega}}
\newcommand*{\yh}{\bm{\eta}}
\newcommand*{\yphi}{\bm{\phi}}
\newcommand*{\ypsi}{\bm{\psi}}
\newcommand*{\yJ}{\bm{J}}

\newcommand*{\yW}{W}

\newcommand*{\yF}{\mathte{F}}
\newcommand*{\yG}{\mathte{G}}
\newcommand*{\yE}{\bm{E}}
\newcommand*{\yB}{\mathte{B}}

\newcommand*{\Oc}[1]{\Ord\bigl(\tfrac{1}{c^{#1}}\bigr)}


%%% Custom macros end @@@

%%%%%%%%%%%%%%%%%%%%%%%%%%%%%%%%%%%%%%%%%%%%%%%%%%%%%%%%%%%%%%%%%%%%%%%%%%%%
%%% Beginning of document
%%%%%%%%%%%%%%%%%%%%%%%%%%%%%%%%%%%%%%%%%%%%%%%%%%%%%%%%%%%%%%%%%%%%%%%%%%%%
%\firmlists
\begin{document}
\captiondelim{\quad}\captionnamefont{\footnotesize}\captiontitlefont{\footnotesize}
\selectlanguage{british}\frenchspacing
\maketitle

%%%%%%%%%%%%%%%%%%%%%%%%%%%%%%%%%%%%%%%%%%%%%%%%%%%%%%%%%%%%%%%%%%%%%%%%%%%%
%%% Abstract
%%%%%%%%%%%%%%%%%%%%%%%%%%%%%%%%%%%%%%%%%%%%%%%%%%%%%%%%%%%%%%%%%%%%%%%%%%%%
\abstractrunin
\abslabeldelim{}
\renewcommand*{\abstractname}{}
\setlength{\absleftindent}{0pt}
\setlength{\absrightindent}{0pt}
\setlength{\abstitleskip}{-\absparindent}
\begin{abstract}\labelsep 0pt%
  \noindent Personal notes on topics in general-relativistic continuum electromagneto-thermo-mechanics.
% \\\noindent\emph{\footnotesize Note: Dear Reader
%     \amp\ Peer, this manuscript is being peer-reviewed by you. Thank you.}
% \par%\\[\jot]
% \noindent
% {\footnotesize PACS: ***}\qquad%
% {\footnotesize MSC: ***}%
%\qquad{\footnotesize Keywords: ***}
\end{abstract}
\selectlanguage{british}\frenchspacing

%%%%%%%%%%%%%%%%%%%%%%%%%%%%%%%%%%%%%%%%%%%%%%%%%%%%%%%%%%%%%%%%%%%%%%%%%%%%
%%% Epigraph
%%%%%%%%%%%%%%%%%%%%%%%%%%%%%%%%%%%%%%%%%%%%%%%%%%%%%%%%%%%%%%%%%%%%%%%%%%%%
% \asudedication{\small ***}
% \vspace{\bigskipamount}
% \setlength{\epigraphwidth}{.7\columnwidth}
% %\epigraphposition{flushright}
% \epigraphtextposition{flushright}
% %\epigraphsourceposition{flushright}
% \epigraphfontsize{\footnotesize}
% \setlength{\epigraphrule}{0pt}
% %\setlength{\beforeepigraphskip}{0pt}
% %\setlength{\afterepigraphskip}{0pt}
% \epigraph{\emph{text}}{source}



%%%%%%%%%%%%%%%%%%%%%%%%%%%%%%%%%%%%%%%%%%%%%%%%%%%%%%%%%%%%%%%%%%%%%%%%%%%%
%%% BEGINNING OF MAIN TEXT
%%%%%%%%%%%%%%%%%%%%%%%%%%%%%%%%%%%%%%%%%%%%%%%%%%%%%%%%%%%%%%%%%%%%%%%%%%%%

\section{Bases of multivector spaces}
\label{sec:bases_multivectors}

Take an ordered coordinate system $(t,x,y,z)$, which also defines an orientation. We shall usually assume that $t$ has dimensions of time and $x,y,z$ of length, and usually that they are orthogonal if a metric is defined.

There are 18 basic types of geometric objects, which represent physical quantities of different types. We have inner-oriented scalars, inner-oriented 1-vectors up to 4-vectors, inner-oriented 1-covectors up to 4-covectors, and all their outer-oriented counterparts.

The exterior product \enquote*{$\land$} of vectors and covectors is usually represented just by juxtaposition, and a shortened notation is used for exterior products of several exterior derivatives \enquote*{$\di$}. For instance
\begin{equation}
  \label{eq:notation_dwedge}
  \di^{2}xy \defd \di x \land \di y
  \qquad
  \de^{2}_{xy} \defd \de_{x} \land \de_{y} \ .
\end{equation}

The associated bases for inner-oriented covector fields are
\begin{gather}
  \label{eq:basis_1vec}
  \si{t}\quad
  \si{x}\quad
  \si{y}\quad
  \si{z}
  \\[\jot]
  \label{eq:basis_2vec}
  \ssi{tx}\quad
  -\ssi{ty}\quad
  \ssi{tz}\quad
  \ssi{yz}\quad
  \ssi{zx}\quad
  \ssi{xy}
  \\
  \label{eq:basis_3vec}
  \sssi{xyz}\quad
  -\sssi{tyz}\quad
  -\sssi{tzx}\quad
  -\sssi{txy}
  \\
  \label{eq:basis_4vec}
  \ssssi{txyz}
\end{gather}
and analogously for inner-oriented vector fields. The particular choice of ordering and sign is such that these bases are related by volume duality, see below.

The outer-oriented unit scalar is $\tw{1}$, with outer orientation $txyz$; note that it's only defined on a coordinate patch. It is idempotent: $\tw{1}\tw{1} =1$.

A twisted or outer-oriented 3-covector such as $\sssi{\tw{x}yz}$ has an associated outer direction, in this case positive $t$. We adopt this shorter notation for the outer-oriented versions of the bases above (analogous to the notation in  \cites{gotayetal1992} \sect\,2 p.~371):
\begin{subequations}\label{eq:basis_twshort}
  \begin{gather}
    \label{eq:basis_1vec_twshort}
    -\ti{xyz}\quad
    \ti{tyz}\quad
    \ti{tzx}\quad
    \ti{txy}
    \\[\jot]
    \label{eq:basis_2vec_twshort}
    \tti{yz}\quad
    -\tti{zx}\quad
    \tti{xy}\quad
    \tti{tx}\quad
    \tti{ty}\quad
    \tti{tz}
    \\
    \label{eq:basis_3vec_twshort}
    \ttti{t}\quad
    \ttti{x}\quad
    \ttti{y}\quad
    \ttti{z}
    \\
    \label{eq:basis_4vec_twshort}
    \tttti{}
  \end{gather}
\end{subequations}
so that $-\ti{xyz} \defd \si{\tw{t}}$ and so on. Similar notation is used for outer-oriented multivector fields; for instance $-\te{xyz} \defd \se{\tw{t}}$. Here is an example of an outer-oriented 3-covector written in terms of the bases and notation above. Note the position of the super- and sub-scripts:
\begin{equation}
  \label{eq:examples_expansion}
  \begin{split}
   &\quad n_{xyz}\,\sssi{\tw{x}yz}
    -n_{tyz}\,\sssi{\tw{t}yz}
    -n_{tzx}\,\sssi{\tw{t}zx}
    -n_{txy}\,\sssi{\tw{t}xy}
    \\[\jot]
    &\equiv
    n^{t}\,\ttti{t} +
    n^{x}\,\ttti{x} +
    n^{y}\,\ttti{y} +
    n^{z}\,\ttti{z}
    \\[2\jot]
    &\qquad\text{with }\ 
    n^{t} \equiv n_{xyz}\ ,\
    n^{x} \equiv n_{tyz}\ ,\
    \text{ and so on}.
  \end{split}
\end{equation}

\medskip

Contraction or dot-product of vectors and covectors is denoted by \enquote*{$\cdot$}, and always contracts the maximal possible amount of contiguous elements. For instance
\begin{equation}
  \label{eq:contraction_ex}
  \sssi{xyz} \cdot \sse{yz} = \si{x}
  \qquad
  -\ssse{txy} \cdot \si{x} = \sse{ty} \ .
\end{equation}

\medskip

Contractions with the 4-vector $\tttte{}$ and 4-covector $\tttti{}$ establish a duality between outer $n$-covectors and inner $(4-n)$-vectors:
\begin{equation}
  \label{eq:coord_volume_duality}
  \left.
    \begin{gathered}
      \sssse{}
      \\
  \ssse{xyz}\enspace
  \ssse{tyz}\enspace
  \ssse{tzx}\enspace
  \ssse{txy}
  \\[\jot]
  \sse{tx}\enspace
  \sse{ty}\enspace
  \sse{tz}\enspace
  \sse{yz}\enspace
  \sse{zx}\enspace
  \sse{xy}
  \\
  \se{t}\enspace
  \se{x}\enspace
  \se{y}\enspace
  \se{z}
  \\
  1
\end{gathered}
\right)\quad
\begin{gathered}
  \xleftarrow{\qquad\mathclap{\textstyle\tttte{}\cdot{}}\qquad}
  \\
  \xrightarrow{\qquad\mathclap{\textstyle{}\cdot\tttti{}}\qquad}
\end{gathered}
\quad%\tfrac{c}{\dg}
\left(
  \begin{gathered}
    \tw{1}
    \\
  \ti{xyz}\enspace
  \ti{tyz}\enspace
  \ti{tzx}\enspace
  \ti{txy}
  \\[\jot]
  \tti{yz}\enspace
  \tti{zx}\enspace
  \tti{xy}\enspace
  \tti{tx}\enspace
  \tti{ty}\enspace
  \tti{tz}
  \\
  \ttti{t}\enspace
  \ttti{x}\enspace
  \ttti{y}\enspace
  \ttti{z}
  \\
  \tttti{}
\end{gathered}
\right.
\end{equation}

These duals have special properties. For instance, for any 3-covector $\yN$, we have
\begin{equation}
  \label{eq:dual_kernel}
  \yN\cdot (\tttte{}\cdot\yN) \cdot \yN = 0
\end{equation}
that is, the dual of $\yN$ is a vector belonging in the kernel of $\yN$.

If $\ve$ is a non-zero 4-covector and $\vi$ the inverse 4-vector, that is, $\ve\cdot\vi=\vi\cdot\ve=1$, and if $\yN$ is a 3-covector and $\yphi$ a 1-covector, we have the useful identity
\begin{equation}
  \label{eq:wedge_to_dot}
  \yN \land \yphi = (\yN\cdot\vi\cdot\yphi)\, \ve \ ,
\end{equation}
which also holds as long as the degrees of $\yN$ and $\yphi$ sum up to 4.

\medskip

We often consider vector- or covector-valued covectors, written as sums of tensor products. They can be outer- or inner-oriented. For instance
\begin{equation}
  \label{eq:cov-valued_cov}
  \ttti{t} \otimes \si{x} \equiv \sssi{\tw{x}yz} \otimes \si{x} \ ,
  \qquad
  \ttti{x} \otimes \se{y} \ .
\end{equation}

The operation $\dand$ between a vector-valued covector and a covector-valued covector is the contraction of their vector- and covector-valued parts and the exterior product of their covector parts. For instance, if $\yphi$ and $\ypsi$ are covectors, $\yo$ is a covector, and $\yu$ is a vector, then
\begin{equation}
  \label{eq:dot_cov_valued}
  (\yphi \otimes \yo) \dand (\ypsi \otimes \yu) \defd
  (\yphi \land \ypsi) \otimes (\yo \cdot \yu)
\end{equation}
% It is usually clear from the context which parts go into the exterior product and which parts need to be contracted. 
As another example,
\begin{equation}
  \label{eq:ex_dand}
  (\ttti{t} \otimes \si{x}) \dand (\ti{t} \otimes \se{x}) = (\ttti{t} \land \ti{t})\, (\si{x} \cdot \se{x}) = -\tttti{} \ .
\end{equation}

\subsection{Moving area elements in spacetime}
\label{sec:moving_area_elements}

In 3D space we can consider a small area with a crossing orientation; for instance a small area parallel to the $y, z$ coordinates, with a positive-$x$ crossing orientation. From a spacetime viewpoint we must also consider also a temporal direction for this \enquote{crossing} orientation. In a manner of speaking we have two choices: cross the surface in the positive-$x$ direction, then \enquote{move forward} in time, that is in the positive-$t$ direction; or move forward in time, then cross the surface. The first represents an $xt$ orientation; the second a $tx$ orientation.

If the area element in the example above has area $a_{x}$, expressed in units of the coordinates $y$ and $z$, then the area element with the $xt$ orientation is
\begin{equation*}
  a_{x}\, \te{xt} \equiv a_{x}\, \se{-\tw{y}z} \equiv - a_{x}\, \se{\tw{y}z} \ ,
\end{equation*}
whereas the one with the $tx$ orientation is
\begin{equation*}
  a_{x}\, \te{tx} \equiv a_{x}\, \se{\tw{y}z} \ .
\end{equation*}

An area element which persists for a short time lapse, and which therefore has a motion with respect to the coordinate system (including rest as a special case), is a timelike 3D-volume from a spacetime viewpoint. For it we can for example choose a positive-$x$ crossing direction; the temporal direction is irrelevant in this case. We can consider this timelike volume as produced by the area element, displaced by a timelike vector.

Take the generic area element
\begin{equation}
  \label{eq:area_element}
  \bm{A} \defd a_{i}\, \tte{x^{i}t}
  \equiv a_{x}\,\tte{xt} + a_{y}\,\tte{yt} + a_{z}\,\tte{zt}
\end{equation}
an displace it by the timelike vector
\begin{equation}
  \label{eq:timelike_displacement}
  \bm{V} \defd \delta_{t}\, (\se{t} + v^{i}\, \se{x^{i}})
  \equiv \delta_{t}\, \se{t}
  + \delta_{t}\,v^{x}\,\se{x} + \delta_{t}\,v^{y}\,\se{y} + \delta_{t}\,v^{z}\,\se{z}
\end{equation}
which represent a motion with coordinate velocity $v^{i}$ for a time lapse $\delta_{t}$.
The resulting timelike volume is
\begin{equation}
  \label{eq:timelike_volume}
  \bm{A}\land\bm{V} =
  \delta_{t}\,(-a_{i}\,v^{i}\,\ttte{t} + a_{i}\,\ttte{x^{i}}) \ .
\end{equation}

\section{Linear transformations of multivector bases}
\label{sec:lintransf_multiv}

Often we must consider multiplication of the basis of vectors or covectors by a square matrix, say
\begin{equation*}
  \label{eq:matrix_mult_bases}
  M_{\alpha\beta}\,\di u^{\beta} \ ,
\end{equation*}
leading to an object in a vector space of the same dimension. The two most important examples for us are:
\begin{itemize}
\item coordinate transformations, for example
  \begin{equation*}
   \di \alpha'= \frac{\de \alpha'}{\de \alpha}\,\di \alpha
  \end{equation*}
\item raising or lowering of indices with the metric, for example
  \begin{equation*}
  \di \alpha \mapsto  g^{\alpha\beta}\, \de_{\beta} \ .
  \end{equation*}
\end{itemize}

The corresponding operations for multivectors involve {compound matrices} of the original transformation matrix \autocites[\sect~IV.A.1 p.~199]{choquetbruhatetal1977_r1996}. For the spaces of 3-vectors and 3-covectors we have simplified formulae \autocites[\sect~I.4 \eqn~(33)]{gantmacher1959_r2000}:
\begin{equation}
  \label{eq:transf3_simple}
  \begin{gathered}
    \ttti{\alpha'} =
    \det\bigl(
    \tfrac{\de \alpha'}{\de \alpha}
    \bigr)
    \ 
    \frac{\de \alpha}{\de \alpha'}
    \ 
    \ttti{\alpha}
    \\[\jot]
    \ttte{\alpha'} =
    \det\bigl(
    \tfrac{\de \alpha}{\de \alpha'}
    \bigr)
    \ 
    \frac{\de \alpha'}{\de \alpha}
    \ 
    \ttte{\alpha}
  \end{gathered}
\end{equation}
note how the 3-covectors almost transforms as 1-vectors and 3-vectors and 1-covectors, apart from a determinant factor.

In an analogous way
\begin{equation}
  \label{eq:raise3_simple}
  \begin{gathered}
    \ttti{\alpha}
    \mapsto
\frac{1}{\abs{g}}\ g_{\alpha\beta}\
    \ttte{\beta}
    \\[\jot]
    \ttte{\alpha}
    \mapsto
\abs{g}\ g^{\alpha\beta}\
    \ttti{\beta}
  \end{gathered}
\end{equation}


\section{Metric}
\label{sec:metric}

We take the metric $\yg$ to have signature $(-,+,+,+)$ and dimensions of area. The determinant and the negative of its square root are denoted shortly
\begin{equation}
  \label{eq:det_g}
  g\defd \det\yg
  \qquad
\dg \defd \sqrt{-\det\yg} \ .
\end{equation}

The volume element induced by the metric $\yg$ has dimensions of volume-time and is denoted (note the boldface)
\begin{equation}
  \label{eq:vol_element}
  \ve \defd \frac{\dg}{c} \ssssi{\tw{t}xyz} \equiv \frac{\dg}{c}\ \tttti{}
\end{equation}
and its corresponding twisted 4-vector:
\begin{equation}
  \label{eq:inv_vol_element}
  \vi \defd \frac{c}{\dg}\ \tttte{} \ .
\end{equation}

Contraction with the volume element or its inverse establishes a \enquote{volume duality} between outer $n$-covectors and inner $(4-n)$-vectors:
\begin{equation}
  \label{eq:volume_duality}
  \left.
    \begin{gathered}
      \sssse{}
      \\
  \ssse{xyz}\enspace
  \ssse{tyz}\enspace
  \ssse{tzx}\enspace
  \ssse{txy}
  \\[\jot]
  \sse{tx}\enspace
  \sse{ty}\enspace
  \sse{tz}\enspace
  \sse{yz}\enspace
  \sse{zx}\enspace
  \sse{xy}
  \\
  \se{t}\enspace
  \se{x}\enspace
  \se{y}\enspace
  \se{z}
  \\
  1
\end{gathered}
\right)\quad
\begin{gathered}
  \xleftarrow{\qquad\mathclap{\textstyle\frac{\dg}{c}\,\vi\cdot{}}\qquad}
  \\
  \xrightarrow{\qquad\mathclap{\textstyle{}\cdot\ve\,\frac{c}{\dg}}\qquad}
\end{gathered}
\quad%\tfrac{c}{\dg}
\left(
  \begin{gathered}
    \tw{1}
    \\
  \ti{xyz}\enspace
  \ti{tyz}\enspace
  \ti{tzx}\enspace
  \ti{txy}
  \\[\jot]
  \tti{yz}\enspace
  \tti{zx}\enspace
  \tti{xy}\enspace
  \tti{tx}\enspace
  \tti{ty}\enspace
  \tti{tz}
  \\
  \ttti{t}\enspace
  \ttti{x}\enspace
  \ttti{y}\enspace
  \ttti{z}
  \\
  \tttti{}
\end{gathered}
\right.
\end{equation}
This is the reason why in older literature an outer-oriented $n$-covector is treated as a $(4-n)$-\enquote{vector density}, that is, a vector divided by the square root of the volume element.

The transpose of the map $\vi\cdot{}$ depends on what this map is acting upon:
\begin{equation}
  \label{eq:transpose_inv_vol_element}
  \dotv\cdot\vi =
  \begin{dcases*}
    -\vi\cdot\dotv& on 1- or 3-covectors,
    \\
    \vi\cdot\dotv& on 0-, 2-, 4-covectors.
  \end{dcases*}
\end{equation}
Therefore we have the following inverse relationships:
\begin{equation}
  \label{eq:inverse_vol_element}
  \ve\cdot\bigl(\vi\cdot \dotv\bigr) =
  \begin{dcases*}
    -\dotv& on 1- or 3-covectors,
    \\
    \dotv& on 0-, 2-, 4-covectors.
  \end{dcases*}
\end{equation}


\medskip

The metric on the space of 1-vectors also induces metrics on the spaces of 1-covectors, 2-vectors and 2-covectors, and so on. In particular, the metrics $\ygg[3]^{-1}$ on the space of 3-covectors and $\ygg[3]$ on the space of 3-vectors can be written in coordinates as
\begin{gather}
  \label{eq:3-covvect_metrics}
  \ygg[3]^{-1} = \frac{g_{\mu\nu}}{g}\,\ttte{\mu}\otimes\ttte{\nu}
  \qquad\text{with dimensions length${}^{-6}$}
  \\
  \label{eq:3-vect_metrics}
  \ygg[3] = g\,g^{\mu\nu}\,\ttti{\mu}\otimes\ttti{\nu}
  \qquad\text{with dimensions length${}^{6}$} \ .
\end{gather}

With these we can define squared norms $\nor{\mathord{\centerdot}}^{2}$ on all those spaces. Note in particular the following identity:
\begin{equation}
  \label{eq:norm_3cov_identity}
  \nor{\vi\cdot\yN}^{2} = -c^{2}\,\nor{\yN}
  \qquad\text{for every 3-covector $\yN$}.
\end{equation}

If the coordinates are orthonormal at some point, then the metric at that point has components
\begin{equation}
  \label{eq:metric_orthonormal}
  \begin{bmatrix}
    -c^{2}&0&0&0\\0&1&0&0\\0&0&1&0\\0&0&0&1
  \end{bmatrix} \ ,
\end{equation}
and the volume element is simply $\tttti{}$.

\section{Matter current}
\label{sec:matter_current}

The amount-of-matter current $\yN$ is an outer-oriented 3-covector
\begin{equation}
  \label{eq:matter_current}
  \yN = N\,\ttti{t} + J^{i}\,\ttti{x^{i}}
\end{equation}
of dimensions \enquote{amount of matter}, typically measured in moles, where
\begin{itemize}
\item $N$ is the volumic amount of matter, measured per unit coordinate volume.
\item $J^{i}$ is the aeric flux of amount of matter (equivalent to a molar flux), measured per unit coordinate area and unit coordinate time.
\end{itemize}

The current for every particular kind of matter satisfies the conservation law
\begin{equation}
  \label{eq:cons_matter}
  \di\yN = 0
  \quad\text{or}\quad
  \de_{t}N + \de_{i} J^{i} = 0
\end{equation}
independent of any metric.

\medskip

The common contravariant form of the matter current, \enquote{$N^{\mu}$}, is obtained by contracting the matter current with the inverse volume element:
\begin{equation}
  \label{eq:contrav_matter_current}
  \text{`\,}N^{\mu}\text{\,'} \corr  \vi \cdot \yN =
  \frac{c}{\dg}\, N\, \se{t} + \frac{c}{\dg}\, J^{i}\, \se{t} \ .
\end{equation}

\medskip

If a metric is present, a four-velocity $\yU$ can be associated with the matter current $\yN$, defined by the following properties and identity:
\begin{gather}
  \label{eq:prop_UN}
  \yU \cdot \yN = 0 \qquad \nor{\yU}^{2} = -c^{2}
  \\
  \label{eq:def_UN}
  \yU = \frac{1}{\abs{\nor{\yN}}}\, \vi\cdot\yN
  \\
  \shortintertext{which also implies (for normal matter)}
  \yN = \nor{\yN}\, \yU\cdot\ve
\end{gather}
For normal matter (as opposed to antimatter) $\nor{\yN}^{2} \ge 0$.



\section{Four-stress and its equations}
\label{sec:tensor_energy}

\subsection{Expressions}
\label{sec:expressions_4stress}


The energy-momentum tensor, or simply four-stress, is a covector-valued 3-covector field, the 3-covector being outer-oriented. It has the dimensions of an action, and can be decomposed as
\begin{equation}
  \label{eq:decomp_T}
  \begin{split}
    \yTT &= \yT\i{^{\mu}_{\nu}}\,\ttti{\mu}\otimes\si{x^{\nu}}
    \\&=
    -\ye\, \ttti{t}\otimes\si{t} 
    -q^{i}\, \ttti{i}\otimes\si{t} 
    +p_{j}\, \ttti{t}\otimes\si{x^{j}} 
    +\pi\i{^{i}_{j}}\, \ttti{i}\otimes\si{x^{j}}
  \end{split}
\end{equation}
the indices $i,j$ running over $x,y,z$, and where:
\begin{itemize}
\item The energy $\ye$ is a density per unit \emph{coordinate} volume $xyz$, and possibly includes a conversion factor for the time unit. If the point at which the four-stress is considered has a matter-current, then this energy comprises \enquote{rest} energy, kinetic energy, potential gravitational energy, and centrifugal potential energy.
\item The energy flux $q^{j}$ is an energy per coordinate area and coordinate time, and possibly includes a conversion factor for the time unit. If the point at which the four-stress is considered has a matter-current, then this flux term includes transport of the energy term above and also heat flux.
\item The momentum $p_{i}$ is a momentum density per unit coordinate volume, and includes a conversion factor for the length $x^{i}$.
\item The compressive three-stress $\pi\i{^{i}_{j}}$ are forces per unit coordinate area, possibly including conversion factors for the time unit and the length $x^{j}$.  If the point at which the four-stress is considered has a matter-current, then this stress comprises momentum transport and internal stresses.
\end{itemize}

Suppose the coordinates $txyz$ are orthogonal and the metric at a point has diagonal components
\begin{equation}
  \label{eq:metric_orthonormal}
  \begin{bmatrix}
    -\abs{g_{tt}}&0&0&0\\0&g_{xx}&0&0\\0&0&g_{yy}&0\\0&0&0&g_{zz}
  \end{bmatrix} \ .
\end{equation}
Note that if, for instance, $x$ has dimensions of angle, then $g_{xx}$ has dimensions of area per squared angle, and similarly for the other metric components.

The common contra-contra-variant form of the stress, \enquote{$T^{\mu\nu}$}, is obtained by contracting the four-stress with the inverse volume element and the inverse metric:
\begin{equation}
  \label{eq:contr_metric_T}
  \begin{split}
\text{`\,}T^{\mu\nu}\text{\,'} \equiv \vi \cdot \yTT \cdot \yg^{-1} &=
    \frac{c\,\abs{g^{tt}}}{\dg}\,\ye\, \se{t}\otimes\se{t}
    +\frac{c\,\abs{g^{tt}}}{\dg}\,q^{i}\, \se{i}\otimes\se{t} \\
    &\qquad{}+\frac{c\,g^{ij}}{\dg}\,p_{i}\, \se{t}\otimes\se{j}
    +\frac{c\,g^{kj}}{\dg}\,\pi\i{^{i}_{k}}\, \se{i}\otimes\se{j}
    \ .
  \end{split}
\end{equation}


% % mynotew{move this}
% One important detail in finding the Newtonian approximation of \enquote{energy density} is that \emph{one takes different zeros of energy density in different coordinate systems}: the zero is taken as the molar mass times the molar density \emph{in the current coordinate system}. By \enquote*{zero} I mean the arbitrary separation between \enquote{mass} and \enquote{energy}.

\subsection{Equations for the four-stress}
\label{sec:equations4stress_eqns}

If $\yphi$ is a covector, $\yo$ a vector or covector, the \emph{exterior covariant derivative} $\Di$ is defined as
\begin{equation}
  \label{eq:ext_cov_der_def}
  \Di(\yphi \otimes \yo) \defd
  (\di\yphi)\otimes\yo + (-1)^{\deg\yphi}\, \yphi\land\nabla\yo
\end{equation}
and is extended by linearity (with respect to constant coefficients).

\medskip

As a consequence of the Einstein equations, the \emph{total} four-stress satisfies two equations:

The differential equation
\begin{gather}
  \label{eq:4stress_balance}
  \ovalbox{
    $\Di\yTT
    % \equiv
    % \di( \yT\i{^{\mu}_{\nu}}\,\ttti{\mu}) \otimes \si{x^{\nu}}
    = 0$
}
    \\
    \shortintertext{which is equivalent to the component equations}
    \label{eq:4stress_balance_comps}
    \de_{\mu}\yT\i{^{\mu}_{\alpha}} =
    \yT\i{^{\mu}_{\nu}}\,\varGamma^{\nu}_{\mu\alpha}
    \\
    \shortintertext{or}
    \label{eq:4stress_balance_explicitcomps}
    \begin{aligned}
  \de_{t}\ye + \de_{i}q^{i} &=
  \ye\,\varGamma^{t}_{tt}
  +q^{i}\, \varGamma^{t}_{it}
  -p_{j}\, \varGamma^{j}_{tt}
  -\pi\i{^{i}_{k}}\, \varGamma^{k}_{it}
  \\
  \de_{t}p_{j} +  \de_{i}\pi\i{^{i}_{j}} &=
  -\ye\,\varGamma^{t}_{tj}
  -q^{i}\,\varGamma^{t}_{ij}
  +p_{k}\,\varGamma^{k}_{tj}
  +\pi\i{^{i}_{k}}\,\varGamma^{k}_{ij} \ .
\end{aligned}
\end{gather}

And the tensor equation
\begin{gather}
  \label{eq:4stress_symm}
  \ovalbox{$
    \bigl(\vi\cdot\yTT\cdot\yg^{-1}\bigr)\T = \vi\cdot\yTT\cdot\yg^{-1}
    $}
    \\
    \shortintertext{which is equivalent to the component equations}
    \label{eq:4stress_symm_comps}
    \yT\i{^{\nu}_{\alpha}}\, g\i{^{\alpha\mu}} =
    \yT\i{^{\mu}_{\alpha}}\, g\i{^{\alpha\nu}} \ .
\end{gather}

These two equations can be re-expressed in several ways, which we now discuss.

\section{Balance laws from the four-stress}
\label{sec:balancelaws}

The four-stress determines an association between any 1-vector field $\yV$ and an outer-oriented 3-covector field, which can then be interpreted as a current:\autocites[\sect~II.7.III p.~87]{choquetbruhatetal1989_r2000}[\sect~3.2 p.~62]{hawkingetal1973_r1994}{gotayetal1992}[see also the discussion in][part~4 \sect~1]{vandantzig1934b}
\begin{equation}
  \label{eq:1vec_to_current_T}
  \yV \mapsto
  \yTT \cdot \yV
  \equiv
  (\yT\i{^{\mu}_{\nu}}\, V^{\nu})\,\ttti{\mu}
\end{equation}
Taking the differential of the current $\yTT\cdot\yV$ we have the following \emph{identity}:
\begin{equation}
  \label{eq:identity_Tcurrent}
  \di(\yTT\cdot\yV) =
(\Di\yTT) \cdot \yV
-\yTT \dand \nabla\yV
\end{equation}

The identity above, combined with the four-stress equations~\eqref{eq:4stress_balance} and~\eqref{eq:4stress_symm}, leads to a non-trivial balance law for the current $\yTT\cdot\yV$ associated with any $\yV$: \autocites[\sect~3.2 p.~62]{hawkingetal1973_r1994}[\sect~3.9.5.2 esp. \eqn~(3.424) and \sect~3.9.6.1 esp. \eqn~(3.458)]{kopeikinetal2011}
\begin{equation}
  \label{eq:balance_VT_current}
  \begin{aligned}
    \di(\yTT\cdot\yV) &= -\yTT \dand
    \tfrac12 \bigl(\nabla\yV + \yg\cdot\nabla\yV\T \cdot \yg^{-1}\bigr)
\\
&= \tr\bigl[\yTT\T\cdot\vi\cdot
\tfrac12 \bigl(\nabla\yV + \yg\cdot\nabla\yV\T \cdot \yg^{-1}\bigr)
\bigr]\,\ve
\\
&\equiv \tr\bigl[\yTT\T\cdot\vi\cdot
\tfrac12 (\Li_{\yV}\yg)\cdot\yg^{-1}
\bigr]\,\ve \ .
\end{aligned}
\end{equation}
This is a \emph{conservation} law if $\yV$ is a Killing vector, that is, if it satisfies
\begin{equation}
  \label{eq:Killing}
  \nabla\yV + \yg\cdot\nabla\yV\T \cdot \yg^{-1}
  \equiv (\Li_{\yV}\yg)\cdot\yg^{-1} = 0 \ .
\end{equation}

\medskip

Some examples. For $\yV=\se{\alpha}$ the formula above becomes, thanks to equations~\eqref{eq:d_3cov} and~\eqref{eq:d_3ext1cov},
\begin{equation}
  \label{eq:balance_VT_current_coord}
  \begin{gathered}
    \di(\yTT\cdot\se{\alpha}) = -\yTT \dand \nabla\se{\alpha}
    \quad\Longleftrightarrow\quad
    \di(\yT\i{^{\mu}_{\alpha}}\,\ttti{\mu}) = \yT\i{^{\mu}_{\nu}}\,\varGamma^{\nu}_{\mu\alpha}\,\tttti{}
    \quad\Longleftrightarrow
    \\
    \de_{\mu}\yT\i{^{\mu}_{\alpha}} = \yT\i{^{\mu}_{\nu}}\,\varGamma^{\nu}_{\mu\alpha}
  \end{gathered}
\end{equation}
which is the common expression of $\Di\yTT=0$ in books where $\yTT$ is interpreted as a tensor density.

For $\yV=\yg^{-1}\cdot\si{x^{\alpha}} = g^{\mu\alpha}\se{\mu}$ we have 
\begin{equation}
  \label{eq:balance_VT_current_co-coord}
  \begin{gathered}
    \di(\yTT\cdot\yg^{-1}\cdot\si{x^{\alpha}}) =
    -\yTT \dand \nabla(\yg^{-1}\cdot\si{x^{\alpha}})
    \quad\Longleftrightarrow
    \\
    \di(\yTT\cdot\yg^{-1}\cdot\si{x^{\alpha}}) =
    -\yTT \dand (\yg^{-1}\cdot\nabla\si{x^{\alpha}})
    \quad\Longleftrightarrow
    \\
    \di(\yT\i{^{\mu\alpha}}\,\ttti{\mu}) =
    -\yT\i{^{\mu\nu}}\,\varGamma^{\alpha}_{\mu\nu}\,\tttti{}
    \quad\Longleftrightarrow
    \\
    \de_{\mu}\yT\i{^{\mu\alpha}} = -\yT\i{^{\mu\nu}}\,\varGamma^{\alpha}_{\mu\nu} \ ,
  \end{gathered}
\end{equation}
another common expression.

\medskip

Equation~\eqref{eq:balance_VT_current} gives rise to an infinity of currents and related balance laws. Such balance laws are not independent; only ten of them are. The four-stress equations~\eqref{eq:4stress_balance} and~\eqref{eq:4stress_symm} can be re-expressed as a set of balance laws associated with ten different vector fields \autocites[\sect~3.2]{hawkingetal1973_r1994}[see also][]{gotayetal1992}. These balance laws are usually divided into a set of four for four-momentum, and a set of six for angular momentum, including \enquote{boost momentum}.

In the literature we find several inequivalent choices for these ten vector fields, leading to definitions of four-momentum and angular momentum that are different, even if they usually agree in a Newtonian approximation.

For the four-momentum, that is, energy and three-momentum, sometimes the vector fields
\begin{equation}
  \label{eq:four-momentum_up}
  \yg^{-1}\cdot\si{t} \ ,\quad \yg^{-1}\cdot\si{x^{i}}
\end{equation}
are used \autocites[for instance in][\sect~5.3, \chap~20]{misneretal1970_r2017}[\sect~6.1.3]{poissonetal2014}. Authors who make this choice can be recognized from their writing \enquote{$T\i{^{\mu\nu}_{;\mu}}$}, or similar, for the covariant divergence of the energy-momentum tensor, with the non-contracted index up.

Sometimes the vector fields
\begin{equation}
  \label{eq:four-momentum_down}
  \se{t} \ ,\quad \se{x^{i}}
\end{equation}
are used instead \autocites[for instance in][\sect~3.2 after \eqn~(3.2)]{hawkingetal1973_r1994}. This choice is recognized by expressions such as \enquote{$T\i{^{\mu}_{\nu;\mu}}$}, with the non-contracted index down.

In typical coordinate systems (for instance, locally free-falling), the vector $\yg^{-1}\cdot\si{x^{i}}$ is the one that gives the traditional momentum in the Newtonian limit, which is zero in a coordinate system where a matter element is at rest. The vector $\se{x^{i}}$ instead yields additional terms related to the velocity of the frame with respect to another reference frame.

On the other hand, it is the vector $-c\se{t}/\abs{\nor{\se{t}}}$ (whose squared-norm is negative) which corresponds to the traditional total, \enquote{internal plus kinetic} energy in the Newtonian limit, where the \enquote{kinetic} part only involves the velocity of the matter element in the particular coordinates used.

For the angular-boost-momentum, one choice of vector fields, corresponding to~\eqref{eq:four-momentum_up} is
\begin{equation}
  \label{eq:ang-momentum_up}
  \yg^{-1}\cdot(t\,\si{x^{i}} - x^{i}\,\si{t})
\ ,\quad
  \yg^{-1}\cdot(x^{i}\,\si{x^{j}} - x^{j}\,\si{x^{i}}) \ .
\end{equation}
Another choice \autocites[\sect~3.9.6.3 before \eqn~(3.475)]{kopeikinetal2011} is
\begin{equation}
  \label{eq:ang-momentum_down}
  g_{t\mu}\,x^{\mu}\,\se{x^{i}} - g_{i\mu}\,x^{\mu}\,\se{t}
\ ,\quad
  g_{i\mu}\,x^{\mu}\,\se{x^{j}} - g_{j\mu}\,x^{\mu}\,\se{i} \ .
\end{equation}
Some texts \autocites[\sects~III.21, IX.62 p.~385]{burke1985_r1987}[\sect~3.2 after \eqn~(3.2)]{hawkingetal1973_r1994} also make the choice \enquote{$x^{\mu}\,\se{x^{\nu}} - x^{\nu}\,\se{x^{\mu}}$}, but note that this in general may be dimensionally inconsistent.

Typically it is the first choice of vector that leads to the traditional angular momentum, is a similar way as discussed above for mmomentum.


With these vector fields it is possible to replace the four-stress equations~\eqref{eq:4stress_balance} and~\eqref{eq:4stress_symm} with a set of four plus six. For instance:
\begin{gather}
  \label{eq:fourstress_balance_1}
  \di(\yTT\cdot\se{x^{\mu}}) =
  -\yTT \dand \nabla\se{x^{\nu}}
  \\[2\jot]
  \label{eq:fourstress_balance_2}
  \begin{multlined}[0.9\linewidth]
    \di[\yTT\cdot
    (x^{\mu}\,\yg^{-1}\cdot\si{x^{\nu}}
    - x^{\nu}\,\yg^{-1}\cdot\si{x^{\mu}} )] ={}\\
    - x^{\mu}\,\yTT \dand \yg^{-1}\cdot\nabla\si{x^{\nu}}
    + x^{\nu}\,\yTT \dand \yg^{-1}\cdot\nabla\si{x^{\mu}}
  \end{multlined}
\end{gather}
If Killing vectors are used, the corresponding supply terms vanish.



\section{Four-stress for matter}
\label{sec:fourstress_matter}

Consider a region where there is a non-vanishing matter current $\yN$ with associated four-velocity $\yU$, and define
\begin{gather}
  \label{eq:U_dual}
  \yUd = -\frac{1}{c}\,\yg \cdot \yU
  \\
  \shortintertext{which statisfies}
  \yUd \cdot \yU = 1 \ ,
  \qquad
  \nabla\yU \cdot \yUd = 0 \ .
\end{gather}
The last equality can be proved from $\nabla\yg=0$ and 
\begin{equation}
  \label{eq:proof_UU_nabla}
  0=-\nabla(c^{2}) = \nabla(\yU \cdot \yg \cdot \yU) = 2\,(\nabla\yU) \cdot \yg \cdot \yU \ .
\end{equation}

\medskip


We can associate with the matter a four-stress $\yTT$ which can be decomposed as follows:
\begin{equation}
  \label{eq:4stress_matter}
  \begin{gathered}
    \yTT =
    -\ye\,\yN \otimes \yUd
    + \yN \otimes \yP
    -(\yUd\land \yQ)\otimes\yUd
    +\yUd \land \yS
    \\
    \text{with}\quad
    \yP \cdot\yU = 0
    \quad
    \yQ \cdot \yU = 0
    \quad
    \yU\cdot\yS = 0
    \quad
    \yS \cdot \yU= 0
  \end{gathered}
\end{equation}
where
\begin{itemize}
\item $\ye$ is a scalar, the molar energy density.
\item $\yP$ is a 1-covector, the molar momentum density.
\item $\yQ$ is a 2-covector, the areic energy-flux density.
\item $\yS$ is a 1-covector-valued 2-covector, the three-stress related to the Cauchy stress.
\end{itemize}

Using the four-velocity $\yU$ associated with the matter current we obtain what could be called the \enquote{internal-energy current}:\autocites[\sect~II.7.III p.~87]{choquetbruhatetal1989_r2000}[see also][]{gotayetal1992}
\begin{equation}
  \label{eq:inten_current}
  \yTT\cdot\yU \equiv -\ye\,\yN - \yUd\, \yQ
\end{equation}
which, from \eqns~\eqref{eq:balance_VT_current}, \eqref{eq:U_dual}, \eqref{eq:cons_matter}, satisfies the balance law
\begin{gather}
  \di\bigl(\yTT\cdot\yU\bigr) = - \yTT \dand \nabla\yU
  \\
  \shortintertext{or}
  \label{eq:balance_TU_matter}
  \begin{multlined}[0.9\linewidth]
    \di\bigl(-\ye\,\yN - \yUd\, \yQ\bigr) =
\bigl(\ye\,\yN \otimes \yUd\bigr) \cdot\nabla\yU
- \bigl(\yN \otimes \yP\bigr) \cdot\nabla\yU \\
{}
    +\bigl(\yUd \,\yQ\otimes\yUd\bigr) \cdot\nabla\yU
    -\yUd \,\yS \cdot\nabla\yU
  \end{multlined}
  \\[\jot]
  \shortintertext{or simply}
  \yN\di\ye + \yUd\di\yQ - \yQ\,\di\yUd =
  -\yN\,\nabla\yU \cdot \yP
  -(\yUd\,\yS)\cdot\nabla\yU \ .
\end{gather}
The last balance equation corresponds to \eqn~(28) in \textcites{eckart1940c}, except for the fact that here we are keeping the momentum $\yP$ and heat flux $\yQ$ distinct.

When $\yP$, $\yS$, $\yQ$ are zero, one speaks of the four-stress of \enquote{dust}. It is noteworthy that in this case the four-stress is essentially proportional to the matter 3-covector, multiplied by a vector collinear with the matter four-velocity:
\begin{equation}
  \label{eq:4stress_dust}
  \yTT = -\yN \otimes (\ye\,\yUd) \ .
\end{equation}
If we take the four-velocity itself as reference, given that $\yUd\yU = -c^{2}$, the energy 3-covector we obtain is proportional to the matter 3-covector:
\begin{equation}
  \label{eq:energy_dust}
  -c^{2}\ye\,\yN
\end{equation}
For this reason $\ye$ is called the \enquote{proper internal energy}. If we take other vector fields as reference, then this energy will pick up further multiplicative terms consisting in the projection of the matter four-velocity $\yU$ onto the reference vector field. In particular taking $\de_{t}$ or $\de_{i}$ as reference we'll have the projections of the four-velocity onto the coordinate covectors as multiplicative factors.



\section{Electromagnetic field}
\label{sec:EM_field}

The electromagnetic field, represented by the Faraday 2-covector, is typically decomposed as \autocites[\chap~9]{frankel1979}
\begin{equation}
  \label{eq:F_decomp_EB}
  \begin{split}
    \yF &= \yE\,\si{t} + \yB
    \\&\equiv
    E_{x}\, \ssi{xt}
    +  E_{y}\, \ssi{yt}
    + E_{z}\, \ssi{zt}
    + B^{x}\,\ssi{yz}
    + B^{y}\,\ssi{zx}
    + B^{z}\,\ssi{xy}
    \ .
  \end{split}
\end{equation}
Given a system of coordinates, this decomposition is unique: the 2-covector $\yB$ does not have $\si{t}$ components, that is, it has $\se{t}$ in its kernel; and so does the 1-covector $\yE$.


The conservation of magnetic flux is expressed by
\begin{gather}
  \label{eq:dFzero}
  \di\yF = 0
  \\
  \shortintertext{or equivalently}
  \begin{aligned}
    &\de_{i} B^{i} = 0
    &&\qquad\text{($\sssi{xyz}$ component)}
    \\[\jot]
    &\de_{t}B^{x} + \de_{y}E_{z} - \de_{z} E_{y} = 0
    &&\qquad\text{($\sssi{tyz}$)}
    \\
    &\de_{t}B^{y} + \de_{z}E_{x} - \de_{x} E_{z} = 0
    &&\qquad\text{($\sssi{tzx}$)}
    \\
    &\de_{t}B^{z} + \de_{x}E_{y} - \de_{y} E_{x} = 0
    &&\qquad\text{($\sssi{txy}$)}
  \end{aligned}
\end{gather}

It should be noted that all components of $\di\yE$ containing $\si{t}$ also disappear, because of the exterior product $\yE\,\si{t}$. The differential $\di$ therefore operates on $\yE$ as if this 1-covector belonged to a three-dimensional manifolds with coordinates $(x,y,z)$. Let's denote this operation by $\ddi$; it is connected with the \enquote{curl} operator. Then the equations above can be written
\begin{equation}
\ddi\yB = 0 \qquad
\de_{t}\yB + \ddi\yE = 0
\ .
\end{equation}


\section{Explicit formulae for a GCRS-like coordinate system}
\label{sec:equations_on_earth}

\subsection{Metric}
\label{sec:GCRS_metric}

Consider a coordinate system $(x^{\mu}) \equiv (t,x,y,z)$ like the Geocentric Celestial Reference System or the Barycentric Celestial Reference System \autocites{soffeletal2003,petitetal2005}[\sect\,8.1.1]{poissonetal2014}[see also][\sect\,2.3.1.1]{moyer2000}.

Neglecting the timelike off-diagonal terms, the metric $\yg$ has components
\begin{multline}
  \label{eq:metric_GCRS}
  (g_{\mu\nu}) ={}\\
  \begin{bmatrix}
    -c^{2} + 2 \yW + \Ord\bigl(\frac{1}{c^{2}}\bigr) &
    \Ord\bigl(\frac{1}{c^{3}}\bigr) &
    \Ord\bigl(\frac{1}{c^{3}}\bigr) &
    \Ord\bigl(\frac{1}{c^{3}}\bigr)
    \\
    \Ord\bigl(\frac{1}{c^{3}}\bigr) &
    1 + \frac{2}{c^{2}} \yW + \Ord\bigl(\frac{1}{c^{4}}\bigr) &
    0 &
    0
    \\
    \Ord\bigl(\frac{1}{c^{3}}\bigr) &
    0 &
    1 + \frac{2}{c^{2}} \yW + \Ord\bigl(\frac{1}{c^{4}}\bigr) &
    0
    \\
    \Ord\bigl(\frac{1}{c^{3}}\bigr) &
    0 &
    0 &
    1 + \frac{2}{c^{2}} \yW + \Ord\bigl(\frac{1}{c^{4}}\bigr)
  \end{bmatrix}
\end{multline}
and its inverse $\yg^{-1}$ has components
\begin{multline}
  \label{eq:inv_metric_GCRS}
  (g^{\mu\nu}) ={}\\
  \begin{bmatrix}
    -\frac{1}{c^{2}} - \frac{2}{c^{4}} \yW + \Oc{6} &
    \Oc{3} &
    \Oc{3} &
    \Oc{3}
    \\
    \Oc{3} &
    1 - \frac{2}{c^{2}} \yW + \Oc{4} &
    0 &
    0
    \\
    \Oc{3} &
    0 &
    1 - \frac{2}{c^{2}} \yW + \Oc{4} &
    0
    \\
    \Oc{3} &
    0 &
    0 &
    1 - \frac{2}{c^{2}} \yW + \Oc{4}
  \end{bmatrix}
\end{multline}
where the gravitational potential $\yW$ generally depends on all coordinates.

Near Earth we have
\begin{equation}
  \label{eq:grav_potential_GCRS}
  \yW = \frac{G M}{r}
\end{equation}
where $M$ is Earth's mass and $r=\sqrt{x^{2}+y^{2}+z^{2}}$ the distance from its centre. Close to the surface, choosing a coordinate system in which $z$ is vertical and pointing upward, we have $\yW\approx g\,(z_{0}-z)$.

The volume element $\ve$ and its inverse are
\begin{equation}
  \label{eq:vol_element_GCRS}
  \ve = \biggl[1 + \frac{2}{c^{2}} \yW + \Oc{4}\biggr]\,\tttti{} \ ,
  \qquad
  \vi = \biggl[1 - \frac{2}{c^{2}} \yW + \Oc{4}\biggr]\,\tttte{} \ .
\end{equation}

\subsection{Matter current}
\label{sec:GCRS_matter}

Suppose the matter current in the geocentric coordinate system is given by
\begin{equation}
  \labelbis{eq:matter_current}
  \yN = N\,\ttti{t} + J^{i}\,\ttti{x^{i}} \ .
\end{equation}

The net flow of matter through a moving area element, that is, through a timelike volume, as the one discussed in \sect~\ref{sec:moving_area_elements}, is
\begin{equation}
  \label{eq:matter_flow_GCRS}
  \yN \cdot (\bm{A}\land\bm{V}) =
  \delta_{t}\, (  - N\,a_{i}\,v^{i} + J^{i} \,a_{i} ) \ .
\end{equation}
This flow is zero if the area element is moving with velocity $v^{i} = U^{i}\defd J^{i}/N$; this is what defines a \enquote{velocity of matter}. When represented as a normalized timelike 4-vector, this is the four-velocity $\yU$ discussed in \sect~\ref{sec:matter_current}.

Note that the equations above are exact in full general relativity as well as in the Newtonian approximation. This happens because we're working with tensor densities.

The representation of matter current as a 1-vector is
\begin{equation}
  \label{eq:matter_1vector}
  \begin{bmatrix}
    N - \frac{2}{c^{2}}N\,\yW + \Oc{4}\ , &
    J^{i} - \frac{2}{c^{2}}J^{i}\,\yW + \Oc{4}
  \end{bmatrix} \ .
\end{equation}
The four-velocity of matter normalized to $-c^{2}$ is given by
\begin{equation}
  \label{eq:4velocity_matter_normalized}
  \yU =
  \begin{bmatrix}
    1 + \frac{1}{c^{2}}\bigl(\tfrac12 U^{2} + \yW\bigr) + \Oc{4} \ , &
    U^{i} + \frac{1}{c^{2}}U^{i}\,\bigl(\tfrac12 U^{2} + \yW\bigr) + \Oc{4}
  \end{bmatrix} \ .
\end{equation}

\subsection{Four-stress for matter}
\label{sec:GCRS_fourstress}

Using the decompositions~\eqref{eq:decomp_T} and~\eqref{eq:4stress_matter}, and defining $U_{i}\defd U^{i}$, the energy-momentum tensor for matter is \autocites[\cf][\sect\,8.1.2, where it is assumed that there's no heat flux and that the 3-stress is isotropic]{poissonetal2014}
\begin{equation}
  \label{eq:fourstress_matter_GCRS}
  \begin{aligned}
    \yT\i{^{t}_{t}} &=
    - N\rho c^{2} -N\,(\epsilon + \tfrac12 \rho U^{2} - \rho\yW) + \Oc{2}
    \\
    \yT\i{^{i}_{t}} &=
    -  N\rho U^{i}\, c^{2} -
    [N\,(\epsilon + \tfrac12\rho U^{2} -\rho\yW)\,U^{i} +
    q^{i} + \pi\i{^{i}_{k}}\,U^{k}] + \Oc{2}
    \\[\jot]
    \yT\i{^{t}_{j}} &=
    N\rho U^{j} + \frac{1}{c^{2}} [
    N\,(\epsilon + \tfrac12 \rho U^{2} + 3\rho\yW)\,U_{j} +
    p_{j} + U_{k}\,\pi\i{^{k}_{j}} ] + \Oc{4}
    \\
    \yT\i{^{i}_{j}} &=
    \!\begin{multlined}[t][0.8\linewidth]
      \pi\i{^{i}_{j}} + N\rho \,U^{i}\,U_{j} + {}\\
      \frac{1}{c^{2}} [
      N\,(\epsilon + \tfrac12 \rho U^{2} + 3\rho\yW)\,U^{i}\,U_{j} +
      U^{i}\,p_{j} + q^{i}\,U_{j}] + \Oc{4}
    \end{multlined}
  \end{aligned}
\end{equation}
with $p_{i}=q^{i}$ and $\pi\i{^{i}_{j}}=\pi\i{^{j}_{i}}$ when the symmetry equation~\eqref{eq:4stress_symm} is enforced.


\subsection{Energy, momentum, balance laws}
\label{sec:GCRS_energy_momentum}

Define the energy current with respect to the vector $-\se{t}$, as discussed in \sect~\ref{sec:balancelaws}. Then the volumic energy content, the areic energy flux across the moving area element $\bm{A}\land\bm{V}$, and the volumic energy supply are
\begin{equation}
  \label{eq:energy_matter_GCRS}
  \begin{aligned}
    \ye &=
    N\rho c^{2} + N\,(\epsilon + \tfrac12 \rho U^{2} - \rho\yW) + \Oc{2}
    \\
    \phi &=
    \!\begin{multlined}[t][0.8\linewidth]
      a_{k}\,(U^{k}-v^{k})\,N\rho c^{2} +{}\\
      a_{k}\,(U^{k}-v^{k})\,N\,(\epsilon + \tfrac12 \rho U^{2} - \rho\yW) +
      a_{k}\,q^{k} + a_{k}\,\pi\i{^{k}_{j}}\,U^{j} +
      \Oc{2}
    \end{multlined}
    \\
    r &= - N\rho\,\de_{t}\yW + \Oc{2} \ .
  \end{aligned}
\end{equation}

Define the $x^{a}$-momentum current with respect to the vector $\se{x^{a}}$. Then the volumic $x^{a}$-momentum content, the areic flux across the moving area element $\bm{A}\land\bm{V}$, and the volumic supply are
\begin{equation}
  \label{eq:momentum_matter_GCRS}
  \begin{aligned}
    p_{a} &=
    N\rho U^{a} + \frac{1}{c^{2}} [
    q^{a} + \pi\i{^{a}_{j}}\,U^{j} +
    N\,(\epsilon + \tfrac12 \rho U^{2} + 3\rho\yW)\, U^{a}
    ] + \Oc{4}
    \\
    f_{a} &=
    \!\begin{multlined}[t][0.8\linewidth]
      a_{k}\,(U^{k}-v^{k})\,U_{a}\,N\rho + a_{j}\,\pi\i{^{j}_{a}} +
      \Oc{2}
    \end{multlined}
    \\
    g_{a} &= N\rho\,\de_{a}\yW + \Oc{2} \ .
  \end{aligned}
\end{equation}

Define the $x^{a}x^{b}$-angular momentum current as in \sect~\ref{sec:balancelaws}. Then the volumic $x^{a}x^{b}$-angular momentum content, the areic flux across the moving area element $\bm{A}\land\bm{V}$, and the volumic supply are
\begin{equation}
  \label{eq:ang_momentum_matter_GCRS}
  \begin{aligned}
    l^{ab} &=
    N\rho\, (x^{a}\,U^{b} - x^{b}\,U^{a}) + \Oc{2}
    \\
    &\equiv x^{a}\,p_{b} - x^{b}\,p_{a} + \Oc{2}
    \\[\jot]
    \tau^{ab} &=
    \!\begin{multlined}[t][0.8\linewidth]
      a_{k}\,(U^{k}-v^{k}) \, (x^{a}\,U^{b} - x^{b}\,U^{a})\,N\rho +
      x^{a}\,a_{k}\,\pi\i{^{k}_{b}} - x^{b}\,a_{k}\,\pi\i{^{k}_{a}} +
      \Oc{2}
    \end{multlined}
    \\
    &\equiv x^{a}\,f_{b} - x^{b}\,f_{a} + \Oc{2}
    \\[\jot]
    m^{ab} &= N\rho\,(x^{a}\,\de_{b}\yW - x^{b}\,\de_{a}\yW)+ \Oc{2} 
    \\
    &\equiv x^{a}\,g_{b} - x^{b}\,g_{a} + \Oc{2} \ .
  \end{aligned}
\end{equation}

Finally define the $x^{a}$-boost momentum current as in \sect~\ref{sec:balancelaws}. Then the volumic $x^{a}$-boost momentum content, the areic flux across the moving area element $\bm{A}\land\bm{V}$, and the volumic supply are
\begin{equation}
  \label{eq:boost_momentum_matter_GCRS}
  \begin{aligned}
    \text{\small content}&= N\rho\, (t\,U^{a} - x^{a}) + \Oc{2}
    \\
    &\equiv t\,p_{a} - x^{a}\,e/c^{2} + \Oc{2}
    \\[\jot]
    \text{\small flux}&=
    \!\begin{multlined}[t][0.8\linewidth]
      a_{k}\,(U^{k}-v^{k}) \, (t\,U^{a} - x^{a}) \,N\rho +
      t\,a_{k}\,\pi\i{^{k}_{a}}  + \Oc{2}
    \end{multlined}
    \\
    &\equiv t\,f_{a} - x^{a}\,\phi/c^{2} + \Oc{2}
    \\[\jot]
    \text{\small supply}&= N\rho\,t\,\de_{a}\yW+ \Oc{2} 
    \\
    &\equiv t\,g_{a} + \Oc{2} \ .
  \end{aligned}
\end{equation}

Combining \eqns~\eqref{eq:momentum_matter_GCRS} and \eqref{eq:ang_momentum_matter_GCRS} we obtain
\begin{equation}
  \label{eq:stress_sym_GCRS}
  \pi\i{^{a}_{b}} = \pi\i{^{b}_{a}} \ .
\end{equation}
Combining \eqns~\eqref{eq:energy_matter_GCRS}, \eqref{eq:momentum_matter_GCRS}, \eqref{eq:boost_momentum_matter_GCRS} we obtain
\begin{equation}
  \label{eq:equiv_mom_enflux_GCRS}
  p_{a} = \phi^{a}/c^{2} \ .
\end{equation}

%% Refs to check
%% brumberg2004

%%%% examples use empheq
%   \begin{empheq}[left={\mathllap{\begin{aligned}    \de\yF_{\yc}/\de\yp&=0\text{:} \\
%         \de\yF_{\yc}/\de\ym&=0\text{:}\\ \de\yF_{\yc}/\de\yl&=0\text{:}\end{aligned}}\qquad}\empheqlbrace]{align}
%     \label{eq:con_p}
% %    \de\yF_{\yc}/\de\yp &\equiv
%     -\ln\yp + \ln\yq + \yl\yM + \ym\yu &=0,\\
%     \label{eq:con_u}
% %    \de\yF_{\yc}/\de\ym &\equiv
%     \yu\yp-1 &=0,\\
%     \label{eq:con_l}
%     %\de\yF_{\yc}/\de\yl &\equiv
%     \yM\yp-\yc &=0.
%   \end{empheq}
%%%%
% \begin{empheq}[box=\widefbox]{equation}
%   \label{eq:maxent_question}
%   \p\bigl[\yE{N+1}{k} \bigcond \tsum\yo\yf{N}\in\yA, \yM\bigr] = \mathord{?}
% \end{empheq}



% \[
%   \begin{tikzcd}
%       M_{n,n}(\CC) \arrow{r}{R'_{a}(\Hat{U})} & M_{n,n}(\CC)
%     \\
%     L(\mathcal{H}) \arrow{r}{\Hat{U}} \arrow[swap]{d}{R_*}\arrow[swap]{u}{R'_*} & L(\mathcal{H}) \arrow{d}{R_*}\arrow{u}{R'_*} \\
%       M_{n,n}(\CC) \arrow{r}{R_{a}(\Hat{U})} & M_{n,n}(\CC)
%   \end{tikzcd}
% \]

% \[
%   \begin{tikzcd}
%       \CC^n \arrow{r}{R'_*(A)} & \CC^n
%     \\
%     \mathcal{H} \arrow{r}{A} \arrow[swap]{d}{R}\arrow[swap]{u}{R'} & \mathcal{H} \arrow{d}{R}\arrow{u}{R'} \\
%       \CC^n \arrow{r}{R_*(A)} & \CC^n
%   \end{tikzcd}
% \]


% \[
%   \begin{tikzcd}
%     \mathcal{H} \arrow{r}{A} \arrow[swap]{d}{R} & \mathcal{H} \arrow{d}{R} \\
%       \CC^n \arrow{r}{R_*(A)} & \CC^n
%   \end{tikzcd}
% \]

%%\setlength{\intextsep}{0ex}% with wrapfigure
%%\setlength{\columnsep}{0ex}% with wrapfigure
%\begin{figure}[p!]% with figure
%\begin{wrapfigure}{r}{0.4\linewidth} % with wrapfigure
%  \centering\includegraphics[trim={12ex 0 18ex 0},clip,width=\linewidth]{maxent_saddle.png}\\
%\caption{caption}\label{fig:comparison_a5}
%\end{figure}% exp_family_maxent.nb


%%%%%%%%%%%%%%%%%%%%%%%%%%%%%%%%%%%%%%%%%%%%%%%%%%%%%%%%%%%%%%%%%%%%%%%%%%%%
%%% Acknowledgements
%%%%%%%%%%%%%%%%%%%%%%%%%%%%%%%%%%%%%%%%%%%%%%%%%%%%%%%%%%%%%%%%%%%%%%%%%%%% 
\iffalse
\begin{acknowledgements}
  \ldots to Mari \amp\ Miri for continuous encouragement and affection, and
  to Buster Keaton and Saitama for filling life with awe and inspiration.
  To the developers and maintainers of \LaTeX, Emacs, AUC\TeX, Open Science
  Framework, R, Python, Inkscape, Sci-Hub for making a free and impartial
  scientific exchange possible.
  % Our work was supported by the Trond Mohn Research Foundation, grant number BFS2018TMT07
%\rotatebox{15}{P}\rotatebox{5}{I}\rotatebox{-10}{P}\rotatebox{10}{\reflectbox{P}}\rotatebox{-5}{O}.
%\sourceatright{\autanet}
\mbox{}\hfill\autanet
\end{acknowledgements}
\fi

%%%%%%%%%%%%%%%%%%%%%%%%%%%%%%%%%%%%%%%%%%%%%%%%%%%%%%%%%%%%%%%%%%%%%%%%%%%%
%%% Appendices
%%%%%%%%%%%%%%%%%%%%%%%%%%%%%%%%%%%%%%%%%%%%%%%%%%%%%%%%%%%%%%%%%%%%%%%%%%%%
%\clearpage
\bigskip
% %\renewcommand*{\appendixpagename}{Appendix}
% %\renewcommand*{\appendixname}{Appendix}
% %\appendixpage
\appendix

\section{Matrix representation}
\label{sec:matrix_repr}

Column-vectors, row-vectors, and matrices are useful to encode the components of geometric objects, but they are too few to faithfully encode the difference between co- and contra-variant objects and multivectors of different orders. Here are some useful conventions and procedures.

\begin{description}
\item[1-vectors] represented by column-matrices
\item[1-covectors] represented by row-matrices
\item[3-vectors] represented by row-matrices
\item[3-covectors] represented by column-matrices
\end{description}
objects such as metrics, covector-valued 3-covectors, tensor products of 1-vector and 1-covector, and similar must be represented by matrices.

Here are the matrix representations of several operations. Consider these objects:
\begin{itemize}
\item $\yu\i{^{\q}}$ is a 1-vector, represented by the column-matrix $\yu$.
\item Similarly for $\yv\i{^{\q}}$.
\item $\yo\i{_{\q}}$ is a 1-covector, represented by the row-matrix $\yo$.
\item Similarly for $\yh\i{_{\q}}$.
\item $\yF\i{_{\q\q}}$ is a 2-covector, represented by the matrix $\yF$.
\item $\yG\i{_{\q\q}}$ is a twisted 2-covector, represented by the matrix $\yG$.
\item $\yg\i{_{\q\q}}$ is a covector-valued covector, represented by the matrix $\yg$. The rows of the matrix represent the left side of the tensor product; and the columns, the right side.
\item ${\yg^{-1}}\i{^{\q\q}}$ is a vector-valued vector, inverse of $\yg$, that is: $\yg\cdot\yg^{1} = \id\i{_{\q}^{\q}}$. The rows of the matrix represent the left side of the tensor product; and the columns, the right side.

\item $\yN\i{_{\mul{\q\q\q}}}$ is an twisted 3-covector, represented by the column-matrix $\yN$.

\item $\yTT\i{_{\mul{\q\q\q}\q}}$ is a covector-valued 3-covector, represented by the matrix $\yTT$. The rows represent the 3-covector components; the columns, the covector components.
\item $\ve\i{_{\mul{\q\q\q\q}}}$ is a 4-covector, represented by the number $\dg/c$.
\item ${\vi}\i{^{\mul{\q\q\q\q}}}$ is a 4-vector, represented by the number $c/\dg$
\item The Jacobian matrix $\frac{\de x'}{\de x}$ from \enquote{old} coordinates $x$ to \enquote{new} coordinates $x'$ is represented by the matrix $\yJ$. The rows correspond to the new coordinates $x'$; the columns, to the old $x$.

\item The inverse-Jacobian matrix $\frac{\de x}{\de x'}$ from new coordinates $x'$ to old coordinates $x'$ is represented by the matrix $\yJ^{-1}$. The rows correspond to the old coordinates $x$; the columns, to the new $x'$.
\end{itemize}

Then -- note that the order on the right side is important:
\begin{itemize}
\item Contractions, wedge product, index raising and lowering
  \begin{align}
    &\text{\small(object)}
    &&\text{\small(matrix repr)}
    \\[\jot]
    &\yo \cdot \yu
    &&\yo\yu \equiv \yu\T\yo\T
    \quad\text{(number)}
    \\
    &\yv\cdot\yg\cdot\yu
    &&\yv\T\yg\yu
    \quad\text{(number)}
    \\
    &\yg\cdot\yu
    &&\yu\T\yg\T
    \quad\text{(row-matrix)}
    \\
    &\yo\land\yh
    &&\yo\T\yh - \yh\T\yo
    \quad\text{(matrix)}
    \\
    &\yu\cdot\yF
    &&\yu\T\yF
    \quad\text{(row-matrix)}
    \\
    &\yF\cdot\yv
    &&(\yF\yv)\T
    \quad\text{(row-matrix)}
    \\
    &\yo\cdot\yg^{-1}
    &&\yg\iT\yo\T
    \quad\text{(column-matrix)}
    \\
    &\vi\cdot\yN
    &&\yN c/\dg
    \quad\text{(column-matrix)}
    \\
    &\yu\cdot\yN
    &&\yu\yN\T - \yN\yu\T
    \quad\text{(matrix)}
    \\
    &\yo\land\yG
    &&(\yo\yG)\T
    \quad\text{(column-matrix)}
    \\
    &\vi\cdot\yTT
    &&\yTT\ c/\dg
    \quad\text{(matrix)}
    \\
    &\yTT\cdot\yu
    &&\yTT\yu
    \quad\text{(column-matrix)}
    \\
    &\yg\cdot(\vi\cdot\yTT)\cdot\yu
    &&\yg\yTT\yu c/\dg
    \quad\text{(matrix)}
  \end{align}

\item Transformations
  \begin{align}
    \text{\small(old coords)} &\quad\mapsto\quad\text{\small(new coords)}
    \\[\jot]
    \yu &\quad\mapsto\quad \yJ\yu
    \\
    \yo &\quad\mapsto\quad \yo\yJ^{-1}
    \\
    \yg &\quad\mapsto\quad \yJ\iT\yg\yJ^{-1}
    \\
    \yN &\quad\mapsto\quad \frac{1}{\det\yJ}\ \yJ\yN
    \\
    \dg/c &\quad\mapsto\quad \frac{1}{\det\yJ}\ \dg/c
    \\
    c/dg &\quad\mapsto\quad \det\yJ\ c/\dg
    \\
    \yTT &\quad\mapsto\quad \frac{1}{\det\yJ}\ \yJ\yTT\yJ^{-1}
  \end{align}

\end{itemize}
\section{Checks about optimal representation of four-stress}
\label{sec:checks_4stress}

A lemma about the exterior derivative of some outer-oriented 3-covectors:
\begin{equation}
  \label{eq:d_3cov}
  \begin{gathered}
    \di\bigl(f\,\ttti{t}\bigr) =
    \di\Bigl( f\,\tw{\sssi{xyz}}\Bigr) =
    \se{t}f \di t \land \tw{\sssi{xyz}} + 0
    = \se{t}f\ \tw{\ssssi{txyz}}
    \\
    \di\bigl(f\,\ttti{x}\bigr) =
    -\di\Bigl( f\,\tw{\sssi{tyz}}\Bigr) =
    -\se{x}f \di x \land \tw{\sssi{tyz}} + 0
    = \se{x}f\ \tw{\ssssi{txyz}}
    \\
    \di\bigl(f\,\ttti{\alpha}\bigr) =
    \se{\alpha}f\ \tw{\ssssi{txyz}}
  \end{gathered}
\end{equation}

Another lemma about the exterior product of 3- and 1-covectors:
\begin{equation}
  \label{eq:d_3ext1cov}
  \begin{gathered}
    \ttti{t} \land \si{t} = - \tttti{}
    \qquad
    \ttti{x} \land \si{x} = - \tttti{}
    \qquad
    \dotso
    \\
    \ttti{\alpha} \land \si{x^{\beta}} = - \delta_{\alpha}^{\beta}\tttti{}
  \end{gathered}
\end{equation}

And a lemma about the covariant derivative of some inner-oriented 1-covectors:
\begin{equation}
  \label{eq:nab_1cov}
  \begin{gathered}
    \nab(\si{t}) =
    -\varGamma^{t}_{tt}\, \si{t} \otimes \si{t}
    -\varGamma^{t}_{tj}\, \si{t} \otimes \si{x^{j}}
    -\varGamma^{t}_{it}\, \si{x^{i}} \otimes \si{t}
    -\varGamma^{t}_{ij}\, \si{x^{i}} \otimes \si{x^{j}}
    \\
    \nab\bigl(\si{x^{k}}\bigr) =
    -\varGamma^{k}_{tt}\, \si{t} \otimes \si{t}
    -\varGamma^{k}_{tj}\, \si{t} \otimes \si{x^{j}}
    -\varGamma^{k}_{it}\, \si{x^{i}} \otimes \si{t}
    -\varGamma^{k}_{ij}\, \si{x^{i}} \otimes \si{x^{j}}
  \end{gathered}
\end{equation}

If $\yphi$ is a 3-covector, $\yo$ a 1-covector, and $\Di$ the exterior covariant derivative, then
\begin{gather}
  \label{eq:ext_cov_der_31}
  \Di(\yphi \otimes \yo) = (\di\yphi)\otimes\yo - \yphi\land\nabla\yo
  \\
  \shortintertext{and in particular}
  \label{eq:ext_cov_der_3basis}
  \Di\bigl(\yphi \otimes \di x^{\alpha}\bigr) =
  (\di\yphi)\otimes\di x^{\alpha} +
  \varGamma^{\alpha}_{\mu\nu}\,\bigl(\yphi\land\di x^{\mu}\bigr) \otimes \di x^{\nu}
  \ .
\end{gather}

Let's also consider the contraction with a 1-vector $\yu$:
\begin{gather}
  \label{eq:ext_cov_contracted_vector}
  \begin{split}
    \Di(\yphi \otimes \yo \cdot \yu) &\equiv
    \di[\yphi \, (\yo \cdot \yu)]
    \\
    &= (\di\yphi)\, (\yo \cdot \yu) - \yphi \land \di(\yo \cdot \yu)
    \\
    &= \di\yphi \otimes \yo \cdot \yu
    -\yphi \land \nabla\yo \cdot \yu
    -\yphi \land \nabla\yu \cdot \yo
    \\
&\equiv (\di\yphi \otimes \yo 
    -\yphi \land \nabla\yo) \cdot \yu
    -\yphi \land \nabla\yu \cdot \yo
    \\
    &\equiv  \Di(\yphi \otimes \yo) \cdot \yu
    -(\yphi \otimes \yo) \dand \nabla\yu
    \ .
  \end{split}
  \\
  \shortintertext{and in particular}
    \label{eq:ext_cov_basis_contracted_vector}
    \Di(\yphi \otimes \di x^{\alpha} \cdot \yu) =
    \di\yphi\, u^{\alpha}- \yphi \land \di x^{\beta}\, \de_{\beta} u^{\alpha}
\end{gather}

A balance equation with the exterior covariant derivative then becomes
\begin{equation}
  \label{eq:balance_contracted}
\Di\yTT = 0 \qquad\implies\qquad
  \di(\yTT \cdot \yu) = -\yTT \dand \nabla\yu
\end{equation}

Then
\begin{equation}
  \label{eq:div_T31}
  \begin{aligned}
    &0=\Di\yTT
    \\
    &\quad{}= \Di\bigl(
    -e\, \ttti{t}\otimes\si{t}
    -q^{i}\, \ttti{i}\otimes\si{t}
    +p_{j}\, \ttti{t}\otimes\si{x^{j}}
    +\pi\i{^{i}_{j}}\, \ttti{i}\otimes\si{x^{j}}
    \bigr)
    \\
    &\quad{}=
    -\de_{t}e\ \tttti{}\otimes\si{t}
    -\de_{i}q^{i}\, \tttti{}\otimes\si{t}
    +\de_{t}p_{j}\, \tttti{}\otimes\si{x^{j}}
    +\de_{i}\pi\i{^{i}_{j}}\, \tttti{}\otimes\si{x^{j}}
    -{}
    \\
    &\qquad
    \Bigl[
    e\,\varGamma^{t}_{tt}\, \ttti{t}\land\si{t} \otimes \si{t}
    +e\,\varGamma^{t}_{tj}\, \ttti{t}\land\si{t} \otimes \si{x^{j}}
    + 0
    \\
    &\qquad
    q^{i}\, \varGamma^{t}_{it}\, \ttti{i}\land\si{x^{i}} \otimes \si{t}
    +q^{i}\,\varGamma^{t}_{ij}\, \ttti{i}\land\si{x^{i}} \otimes \si{x^{j}}
    + 0
    \\
    &\qquad
    {}-p_{j}\, \varGamma^{j}_{tt}\, \ttti{t}\land\si{t} \otimes \si{t}
    -p_{k}\,\varGamma^{k}_{tj}\, \ttti{t}\land\si{t} \otimes \si{x^{j}}
    + 0
    \\
    &\qquad
    {}-\pi\i{^{i}_{k}}\, \varGamma^{k}_{it}\, \ttti{i}\land\si{x^{i}} \otimes \si{t}
    -\pi\i{^{i}_{k}}\,\varGamma^{k}_{ij}\, \ttti{i}\land\si{x^{i}} \otimes \si{x^{j}}
    \Bigr]
    \\
    &\quad{}=
    \tttti{}\otimes \bigl[{}
    \\
    &\qquad
    -\de_{t}e\ \si{t} 
    + e\,\varGamma^{t}_{tt}\, \si{t}
    +e\,\varGamma^{t}_{tj}\,  \si{x^{j}}
    \\
    &\qquad
    {}-\de_{i}q^{i}\, \si{t} 
    +q^{i}\, \varGamma^{t}_{it}\,  \si{t}
    +q^{i}\,\varGamma^{t}_{ij}\,  \si{x^{j}}
    \\
    &\qquad
    {}+\de_{t}p_{j}\, \si{x^{j}} 
    -p_{j}\, \varGamma^{j}_{tt}\, \si{t}
    -p_{k}\,\varGamma^{k}_{tj}\,  \si{x^{j}}
    \\
    &\qquad
    {}+\de_{i}\pi\i{^{i}_{j}}\, \si{x^{j}} 
    -\pi\i{^{i}_{k}}\, \varGamma^{k}_{it}\, \si{t}
    -\pi\i{^{i}_{k}}\,\varGamma^{k}_{ij}\, \si{x^{j}}
    \\
    &\qquad{}\bigr]
  \end{aligned}
\end{equation}

The equation above corresponds to the four balance equations
\begin{align}
  \label{eq:div_e}
  \de_{t}e + \de_{i}q^{i} &=
  e\,\varGamma^{t}_{tt}
  +q^{i}\, \varGamma^{t}_{it}
  -p_{j}\, \varGamma^{j}_{tt}
  -\pi\i{^{i}_{k}}\, \varGamma^{k}_{it}
  \\
  \label{eq:div_p}
  \de_{t}p_{j} +  \de_{i}\pi\i{^{i}_{j}} &=
  -e\,\varGamma^{t}_{tj}
  -q^{i}\,\varGamma^{t}_{ij}
  +p_{k}\,\varGamma^{k}_{tj}
  +\pi\i{^{i}_{k}}\,\varGamma^{k}_{ij}
\end{align}

\medskip

For $\yTT\cdot\yu$ we find first
\begin{align}
    \yTT\cdot\yu &=
    (-e\,u^{t} + p_{j}\,u^{j})\,\ttti{t} +
    (-q^{i}\,u^{t} + \pi\i{^{i}_{j}}\,u^{j})\,\ttti{i}
    \\
    \begin{split}
      \yTT \dand \nabla\yu &=
      (-e\,\de_{t} u^{t} + p_{j}\,\de_{t}u^{j})\,\ttti{t}\land\di t
      +(-q^{i}\,\de_{i}u^{t} + \pi\i{^{i}_{j}}\,\de_{i}u^{j})\,\ttti{i}\land\di x^{i}
      \\
      &\qquad{}+\text{$\varGamma$ terms}
    \end{split}
  \end{align}
and therefore
\begin{multline}
  \label{eq:DT_contracted_u}
  \de_{t}(-e\,u^{t} + p_{j}\,u^{j})
  +\de_{i}(-q^{i}\,u^{t} + \pi\i{^{i}_{j}}\,u^{j})
  ={}\\
  \shoveright{
    -e\,\de_{t} u^{t} + p_{j}\,\de_{t}u^{j}
  -q^{i}\,\de_{i}u^{t} + \pi\i{^{i}_{j}}\,\de_{i}u^{j}
}
  \\
  \shoveright{
  {}-(e\,\varGamma^{t}_{tt}
  +q^{i}\, \varGamma^{t}_{it}
  -p_{j}\, \varGamma^{j}_{tt}
  -\pi\i{^{i}_{k}}\, \varGamma^{k}_{it})\,u^{t}
}
  \\
  {}-(e\,\varGamma^{t}_{tj}
  +q^{i}\,\varGamma^{t}_{ij}
  -p_{k}\,\varGamma^{k}_{tj}
  -\pi\i{^{i}_{k}}\,\varGamma^{k}_{ij})\,u^{j}
\end{multline}


\bigskip

\paragraph{Radial case}

\begin{align}
  \label{eq:div_e_r}
  \de_{t}e + \de_{r}q^{r} &=
  e\,\varGamma^{t}_{tt}
  +q^{r}\, \varGamma^{t}_{rt}
  +p_{r}\, \varGamma^{r}_{tt}
  +\pi\i{^{r}_{r}}\, \varGamma^{r}_{rt}
  \\
  \label{eq:div_p_r}
  \de_{t}p_{r} +  \de_{r}\pi\i{^{r}_{r}} &=
  e\,\varGamma^{t}_{tr}
  +q^{r}\,\varGamma^{t}_{rr}
  +p_{r}\,\varGamma^{r}_{tr}
  +\pi\i{^{r}_{r}}\,\varGamma^{r}_{rr}
\end{align}

\begin{align}
  \label{eq:div_e_r_S}
  \de_{t}e + \de_{r}q^{r} &=
 % e\,\varGamma^{t}_{tt}
  q^{r}\, \frac{g}{c^{2}}
  +p_{r}\, g
  %+\pi\i{^{r}_{r}}\, \varGamma^{r}_{rt}
  \\
  \label{eq:div_p_r_S}
  \de_{t}p_{r} +  \de_{r}\pi\i{^{r}_{r}} &=
  e\, \frac{g}{c^{2}}
  %+q^{r}\,\varGamma^{t}_{rr}
  %+p_{r}\,\varGamma^{r}_{tr}
  %+\pi\i{^{r}_{r}}\,\varGamma^{r}_{rr}
\end{align}





Let's consider a Cartesian coordinate system over a small neighbourhood on the Earth's surface, with $z$ pointing upwards. In the Newtonian approximation we have \autocites[\sect\,5.2.3]{poissonetal2014}
\begin{equation}
  \label{eq:christof_newt}
  \varGamma^{t}_{jt}=\varGamma^{t}_{tj} =
    \frac{GM}{c^{2}}\frac{x^{j}}{r^{3}} \approx \frac{g}{c^{2}}
  \qquad\varGamma^{j}_{tt} = GM\frac{x^{j}}{r^{3}} \approx g
\end{equation}
where $g$ is the standard acceleration, considered positive. Take also $p_{j} \approx m v_{j}$ and $e\approx m c^{2} + \frac{1}{2}mv^{2}$.



The balances above become
\begin{align}
  \label{eq:div_e_appr}
  \de_{t}e + \de_{i}q^{i} &=
  q^{z}\, \frac{g}{c^{2}}
  +p_{z}\, g
  \\
  \label{eq:div_p_appr}
  \de_{t}p_{z} +  \de_{i}\pi\i{^{i}_{z}} &=
  e\,\frac{g}{c^{2}}
\end{align}

Also,

\begin{equation}
  \label{eq:christof_body}
  \begin{gathered}
    \varGamma^{t}_{tt} \approx
    -2\frac{g}{c^{2}} v(t)
    \qquad
    \varGamma^{t}_{jt}=\varGamma^{t}_{tj}
    \approx \frac{g}{c^{2}}
    \\
    \varGamma^{j}_{tt} \approx
    g - 2\frac{g}{c^{2}}\,v(t)^{2} - \dot{v}(t)
    \qquad
    \varGamma^{j}_{jt} = \varGamma^{j}_{tj} \approx
    \frac{g}{c^{2}} v(t)
  \end{gathered}
\end{equation}

\begin{align}
  \label{eq:div_e}
  \de_{t}e + \de_{i}q^{i} &=
  -e\,2\frac{g}{c^{2}} v(t)
  +q^{z}\, \frac{g}{c^{2}}
  +p_{j}\, \Bigl(g - 2\frac{g}{c^{2}}\,v(t)^{2} - \dot{v}(t)\Bigr)
  +\pi\i{^{z}_{z}}\, \frac{g}{c^{2}} v(t)
  \\
  \label{eq:div_p}
  \de_{t}p_{z} +  \de_{i}\pi\i{^{i}_{z}} &=
  e\, \frac{g}{c^{2}}
  % +q^{i}\,\varGamma^{t}_{ij}
  +p_{z}\,\frac{g}{c^{2}} v(t)
%  +\pi\i{^{i}_{k}}\,\varGamma^{k}_{ij}
\end{align}


\section{Works with useful content}
\label{sec:works}

\begin{itemize}
\item Eq.~(21) in \cites{maugin1974b}: fluid with heat conduction.
\item Factor $1/2$ in kinetic energy: \sect~6.3.3 in \cites{gourgoulhon2007_r2012}
% \[ \color{bluepurple}\bm{a} \color{redpurple}\mathbin{\bm{\land}}\color{bluepurple} \bm{b} \]

\item Tensor-density expression of divergence law:  \cites{castrillonlopezetal2008} also  \cites{kopeikinetal2011,soffeletal2019}.

\item For transformation or raising:  \cites[\sect~I.4 \eqn~(33)]{gantmacher1959_r2000}.

\begin{equation}
  \gamma\cdot (B\land A) = (-1)^{\deg{A}\deg{B}} \gamma\cdot(A \land B)
\end{equation}
with $\deg(B) = n-\deg(A)$
\cites[prop.~4.1]{barnabeietal1985}


\item Compound matrices:  \cites[\sect~IV.A.1 p.~199, Problem~1 p.~270]{choquetbruhatetal1977_r1996}
\end{itemize}

%%%%%%%%%%%%%%%%%%%%%%%%%%%%%%%%%%%%%%%%%%%%%%%%%%%%%%%%%%%%%%%%%%%%%%%%%%%% 
%%% Bibliography
%%%%%%%%%%%%%%%%%%%%%%%%%%%%%%%%%%%%%%%%%%%%%%%%%%%%%%%%%%%%%%%%%%%%%%%%%%%%
\renewcommand*{\finalnamedelim}{\addcomma\space}
\defbibnote{prenote}{{\footnotesize (\enquote{de $X$} is listed under D,
    \enquote{van $X$} under V, and so on, regardless of national
    conventions.)\par}}
% \defbibnote{postnote}{\par\medskip\noindent{\footnotesize% Note:
%     \arxivp \mparcp \philscip \biorxivp}}

\printbibliography[prenote=prenote%,postnote=postnote
]

\end{document}

%%%%%%%%%%%%%%%%%%%%%%%%%%%%%%%%%%%%%%%%%%%%%%%%%%%%%%%%%%%%%%%%%%%%%%%%%%%%
%%% Cut text (won't be compiled)
%%%%%%%%%%%%%%%%%%%%%%%%%%%%%%%%%%%%%%%%%%%%%%%%%%%%%%%%%%%%%%%%%%%%%%%%%%%%


%%% Local Variables: 
%%% mode: LaTeX
%%% TeX-PDF-mode: t
%%% TeX-master: t
%%% End: 
