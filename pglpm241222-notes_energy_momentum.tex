\pdfoutput=1
%% Author: PGL  Porta Mana
%% Created: 2015-05-01T20:53:34+0200
%% Last-Updated: 2025-01-05T12:07:07+0100
%%%%%%%%%%%%%%%%%%%%%%%%%%%%%%%%%%%%%%%%%%%%%%%%%%%%%%%%%%%%%%%%%%%%%%%%%%%%
\newif\ifarxiv
\arxivfalse
%\iftrue\pdfmapfile{+classico.map}\fi
\newif\ifafour
\afourfalse% true = A4, false = A5
\newif\iftypodisclaim % typographical disclaim on the side
\typodisclaimfalse
\newcommand*{\memfontfamily}{zpl}
%\newcommand*{\memfontenc}{T1}
\newcommand*{\memfontpack}{newpx}
\documentclass[\ifafour a4paper,12pt,\else a5paper,10pt,\fi%extrafontsizes,%
onecolumn,oneside,article,%french,italian,german,swedish,latin,
british%
]{memoir}
\newcommand*{\firstdraft}{22 December 2024}
\newcommand*{\firstpublished}{\firstdraft}
\newcommand*{\updated}{\ifarxiv***\else\today\ [draft]\fi}
\newcommand*{\propertitle}{Notes on energy and momentum%\\{\large ***}%
}% title uses LARGE; set Large for smaller
\newcommand*{\pdftitle}{\propertitle}
\newcommand*{\headtitle}{Notes on energy and momentum}
\newcommand*{\pdfauthor}{P.G.L.  Porta Mana}
\newcommand*{\headauthor}{Porta Mana}
\newcommand*{\reporthead}{\ifarxiv\else Open Science Framework \href{https://doi.org/10.31219/osf.io/***}{\textsc{doi}:10.31219/osf.io/***}\fi}% Report number

%%%%%%%%%%%%%%%%%%%%%%%%%%%%%%%%%%%%%%%%%%%%%%%%%%%%%%%%%%%%%%%%%%%%%%%%%%%%
%%% Calls to packages (uncomment as needed)
%%%%%%%%%%%%%%%%%%%%%%%%%%%%%%%%%%%%%%%%%%%%%%%%%%%%%%%%%%%%%%%%%%%%%%%%%%%%

%\usepackage{pifont}

%\usepackage{fontawesome}
\PassOptionsToPackage{obeyspaces}{url}
\usepackage[T1]{fontenc}
\input{glyphtounicode} \pdfgentounicode=1

\usepackage[utf8]{inputenx}

%\usepackage{newunicodechar}
% \newunicodechar{Ĕ}{\u{E}}
% \newunicodechar{ĕ}{\u{e}}
% \newunicodechar{Ĭ}{\u{I}}
% \newunicodechar{ĭ}{\u{\i}}
% \newunicodechar{Ŏ}{\u{O}}
% \newunicodechar{ŏ}{\u{o}}
% \newunicodechar{Ŭ}{\u{U}}
% \newunicodechar{ŭ}{\u{u}}
% \newunicodechar{Ā}{\=A}
% \newunicodechar{ā}{\=a}
% \newunicodechar{Ē}{\=E}
% \newunicodechar{ē}{\=e}
% \newunicodechar{Ī}{\=I}
% \newunicodechar{ī}{\={\i}}
% \newunicodechar{Ō}{\=O}
% \newunicodechar{ō}{\=o}
% \newunicodechar{Ū}{\=U}
% \newunicodechar{ū}{\=u}
% \newunicodechar{Ȳ}{\=Y}
% \newunicodechar{ȳ}{\=y}

% note: with \newcommand*{\bmmax}{0} commands like \bm{\left(} won't work
\newcommand*{\bmmax}{3} % reduce number of bold fonts, before font packages
\newcommand*{\hmmax}{0} % reduce number of heavy fonts, before font packages

\usepackage{textcomp}

%\usepackage[normalem]{ulem}% package for underlining
% \makeatletter
% \def\ssout{\bgroup \ULdepth=-.35ex%\UL@setULdepth
%  \markoverwith{\lower\ULdepth\hbox
%    {\kern-.03em\vbox{\hrule width.2em\kern1.2\p@\hrule}\kern-.03em}}%
%  \ULon}
% \makeatother

\usepackage{amsmath}

\usepackage{mathtools}
%\addtolength{\jot}{\jot} % increase spacing in multiline formulae
\setlength{\multlinegap}{0pt}

%\usepackage{empheq}% automatically calls amsmath and mathtools
%\newcommand*{\widefbox}[1]{\fbox{\hspace{1em}#1\hspace{1em}}}

%%%% empheq above seems more versatile than these:
%\usepackage{fancybox}
%\usepackage{framed}

% \usepackage[misc]{ifsym} % for dice
% \newcommand*{\diceone}{{\scriptsize\Cube{1}}}

\usepackage{amssymb}

\usepackage{amsxtra}

\usepackage[main=british]{babel}\selectlanguage{british}
%\newcommand*{\langnohyph}{\foreignlanguage{nohyphenation}}
\newcommand{\langnohyph}[1]{\begin{hyphenrules}{nohyphenation}#1\end{hyphenrules}}

\usepackage[autostyle=false,autopunct=false,english=british]{csquotes}
\setquotestyle{american}
\newcommand*{\defquote}[1]{`\,#1\,'}

% \makeatletter
% \renewenvironment{quotation}%
%                {\list{}{\listparindent 1.5em%
%                         \itemindent    \listparindent
%                         \rightmargin=1em   \leftmargin=1em
%                         \parsep        \z@ \@plus\p@}%
%                 \item[]\footnotesize}%
%                 {\endlist}
% \makeatother


\usepackage{amsthm}
%% from https://tex.stackexchange.com/a/404680/97039
\makeatletter
\def\@endtheorem{\endtrivlist}
\makeatother

% \newcommand*{\QED}{\textsc{q.e.d.}}
% \renewcommand*{\qedsymbol}{\QED}
% \theoremstyle{remark}
% \newtheorem{note}{Note}
% \newtheorem*{remark}{Note}
% \newtheoremstyle{innote}{\parsep}{\parsep}{\footnotesize}{}{}{}{0pt}{}
% \theoremstyle{innote}
% \newtheorem*{innote}{}

\usepackage[shortlabels,inline]{enumitem}
\SetEnumitemKey{para}{itemindent=\parindent,leftmargin=0pt,listparindent=\parindent,parsep=0pt,itemsep=\topsep}
% \begin{asparaenum} = \begin{enumerate}[para]
% \begin{inparaenum} = \begin{enumerate*}
\setlist{itemsep=0pt,topsep=\parsep}
\setlist[enumerate,2]{label=(\roman*)}
\setlist[enumerate]{label=(\alph*),leftmargin=1.5\parindent}
\setlist[itemize]{leftmargin=1.5\parindent}
\setlist[description]{leftmargin=1.5\parindent}
% old alternative:
% \setlist[enumerate,2]{label=\alph*.}
% \setlist[enumerate]{leftmargin=\parindent}
% \setlist[itemize]{leftmargin=\parindent}
% \setlist[description]{leftmargin=\parindent}

\usepackage[babel,theoremfont,largesc,smallerops,nosymbolsc]{newpx}

% For Baskerville see https://ctan.org/tex-archive/fonts/baskervillef?lang=en
% and http://mirrors.ctan.org/fonts/baskervillef/doc/baskervillef-doc.pdf
% \usepackage[p]{baskervillef}
% \usepackage[varqu,varl,var0]{inconsolata}
% \usepackage[scale=.95,type1]{cabin}
% \usepackage[baskerville,vvarbb]{newtxmath}
% \usepackage[cal=boondoxo]{mathalfa}


% \usepackage[bigdelims,nosymbolsc%,smallerops % probably arXiv doesn't have it
% ]{newpxmath}
%\useosf
%\linespread{1.083}%
%\linespread{1.05}% widely used
\linespread{1.1}% best for text with maths
%% smaller operators for old version of newpxmath
\makeatletter
\def\re@DeclareMathSymbol#1#2#3#4{%
    \let#1=\undefined
    \DeclareMathSymbol{#1}{#2}{#3}{#4}}
%\re@DeclareMathSymbol{\bigsqcupop}{\mathop}{largesymbols}{"46}
%\re@DeclareMathSymbol{\bigodotop}{\mathop}{largesymbols}{"4A}
\re@DeclareMathSymbol{\bigoplusop}{\mathop}{largesymbols}{"4C}
\re@DeclareMathSymbol{\bigotimesop}{\mathop}{largesymbols}{"4E}
\re@DeclareMathSymbol{\sumop}{\mathop}{largesymbols}{"50}
\re@DeclareMathSymbol{\prodop}{\mathop}{largesymbols}{"51}
\re@DeclareMathSymbol{\bigcupop}{\mathop}{largesymbols}{"53}
\re@DeclareMathSymbol{\bigcapop}{\mathop}{largesymbols}{"54}
%\re@DeclareMathSymbol{\biguplusop}{\mathop}{largesymbols}{"55}
\re@DeclareMathSymbol{\bigwedgeop}{\mathop}{largesymbols}{"56}
\re@DeclareMathSymbol{\bigveeop}{\mathop}{largesymbols}{"57}
%\re@DeclareMathSymbol{\bigcupdotop}{\mathop}{largesymbols}{"DF}
%\re@DeclareMathSymbol{\bigcapplusop}{\mathop}{largesymbolsPXA}{"00}
%\re@DeclareMathSymbol{\bigsqcupplusop}{\mathop}{largesymbolsPXA}{"02}
%\re@DeclareMathSymbol{\bigsqcapplusop}{\mathop}{largesymbolsPXA}{"04}
%\re@DeclareMathSymbol{\bigsqcapop}{\mathop}{largesymbolsPXA}{"06}
\re@DeclareMathSymbol{\bigtimesop}{\mathop}{largesymbolsPXA}{"10}
%\re@DeclareMathSymbol{\coprodop}{\mathop}{largesymbols}{"60}
%\re@DeclareMathSymbol{\varprod}{\mathop}{largesymbolsPXA}{16}
\makeatother
%%
%% With euler font cursive for Greek letters - the [1] means 100% scaling
\DeclareFontFamily{U}{egreek}{\skewchar\font'177}%
\DeclareFontShape{U}{egreek}{m}{n}{<-6>s*[1]eurm5 <6-8>s*[1]eurm7 <8->s*[1]eurm10}{}%
\DeclareFontShape{U}{egreek}{m}{it}{<->s*[1]eurmo10}{}%
\DeclareFontShape{U}{egreek}{b}{n}{<-6>s*[1]eurb5 <6-8>s*[1]eurb7 <8->s*[1]eurb10}{}%
\DeclareFontShape{U}{egreek}{b}{it}{<->s*[1]eurbo10}{}%
\DeclareSymbolFont{egreeki}{U}{egreek}{m}{it}%
\SetSymbolFont{egreeki}{bold}{U}{egreek}{b}{it}% from the amsfonts package
\DeclareSymbolFont{egreekr}{U}{egreek}{m}{n}%
\SetSymbolFont{egreekr}{bold}{U}{egreek}{b}{n}% from the amsfonts package
% Take also \sum, \prod, \coprod symbols from Euler fonts
\DeclareFontFamily{U}{egreekx}{\skewchar\font'177}
\DeclareFontShape{U}{egreekx}{m}{n}{%
       <-7.5>s*[0.9]euex7%
    <7.5-8.5>s*[0.9]euex8%
    <8.5-9.5>s*[0.9]euex9%
    <9.5->s*[0.9]euex10%
}{}
\DeclareSymbolFont{egreekx}{U}{egreekx}{m}{n}
\DeclareMathSymbol{\sumop}{\mathop}{egreekx}{"50}
\DeclareMathSymbol{\prodop}{\mathop}{egreekx}{"51}
\DeclareMathSymbol{\coprodop}{\mathop}{egreekx}{"60}
\makeatletter
\def\sum{\DOTSI\sumop\slimits@}
\def\prod{\DOTSI\prodop\slimits@}
\def\coprod{\DOTSI\coprodop\slimits@}
\makeatother
%%%% Greek letters not usually given in LaTeX
%%%% best to uncomment only the ones needed
%% %% \input{definegreek.tex} % originally in a separate file
\DeclareMathSymbol{\varpartial}{\mathalpha}{egreeki}{"40}
%\DeclareMathSymbol{\partialup}{\mathalpha}{egreekr}{"40}
% \DeclareMathSymbol{\alpha}{\mathalpha}{egreeki}{"0B}
% \DeclareMathSymbol{\beta}{\mathalpha}{egreeki}{"0C}
% \DeclareMathSymbol{\gamma}{\mathalpha}{egreeki}{"0D}
% \DeclareMathSymbol{\delta}{\mathalpha}{egreeki}{"0E}
% \DeclareMathSymbol{\epsilon}{\mathalpha}{egreeki}{"0F}
% \DeclareMathSymbol{\zeta}{\mathalpha}{egreeki}{"10}
% \DeclareMathSymbol{\eta}{\mathalpha}{egreeki}{"11}
% \DeclareMathSymbol{\theta}{\mathalpha}{egreeki}{"12}
% \DeclareMathSymbol{\iota}{\mathalpha}{egreeki}{"13}
% \DeclareMathSymbol{\kappa}{\mathalpha}{egreeki}{"14}
% \DeclareMathSymbol{\lambda}{\mathalpha}{egreeki}{"15}
% \DeclareMathSymbol{\mu}{\mathalpha}{egreeki}{"16}
% \DeclareMathSymbol{\nu}{\mathalpha}{egreeki}{"17}
% \DeclareMathSymbol{\xi}{\mathalpha}{egreeki}{"18}
% \DeclareMathSymbol{\omicron}{\mathalpha}{egreeki}{"6F}
% \DeclareMathSymbol{\pi}{\mathalpha}{egreeki}{"19}
% \DeclareMathSymbol{\rho}{\mathalpha}{egreeki}{"1A}
% \DeclareMathSymbol{\sigma}{\mathalpha}{egreeki}{"1B}
% \DeclareMathSymbol{\tau}{\mathalpha}{egreeki}{"1C}
% \DeclareMathSymbol{\upsilon}{\mathalpha}{egreeki}{"1D}
% \DeclareMathSymbol{\phi}{\mathalpha}{egreeki}{"1E}
% \DeclareMathSymbol{\chi}{\mathalpha}{egreeki}{"1F}
% \DeclareMathSymbol{\psi}{\mathalpha}{egreeki}{"20}
% \DeclareMathSymbol{\omega}{\mathalpha}{egreeki}{"21}
% \DeclareMathSymbol{\varepsilon}{\mathalpha}{egreeki}{"22}
% \DeclareMathSymbol{\vartheta}{\mathalpha}{egreeki}{"23}
% \DeclareMathSymbol{\varpi}{\mathalpha}{egreeki}{"24}
% \let\varrho\rho 
% \let\varsigma\sigma
% \let\varkappa\kappa
% \DeclareMathSymbol{\varphi}{\mathalpha}{egreeki}{"27}
% %
% \DeclareMathSymbol{\varAlpha}{\mathalpha}{egreeki}{"41}
% \DeclareMathSymbol{\varBeta}{\mathalpha}{egreeki}{"42}
% \DeclareMathSymbol{\varGamma}{\mathalpha}{egreeki}{"00}
% \DeclareMathSymbol{\varDelta}{\mathalpha}{egreeki}{"01}
% \DeclareMathSymbol{\varEpsilon}{\mathalpha}{egreeki}{"45}
% \DeclareMathSymbol{\varZeta}{\mathalpha}{egreeki}{"5A}
% \DeclareMathSymbol{\varEta}{\mathalpha}{egreeki}{"48}
% \DeclareMathSymbol{\varTheta}{\mathalpha}{egreeki}{"02}
% \DeclareMathSymbol{\varIota}{\mathalpha}{egreeki}{"49}
% \DeclareMathSymbol{\varKappa}{\mathalpha}{egreeki}{"4B}
% \DeclareMathSymbol{\varLambda}{\mathalpha}{egreeki}{"03}
% \DeclareMathSymbol{\varMu}{\mathalpha}{egreeki}{"4D}
% \DeclareMathSymbol{\varNu}{\mathalpha}{egreeki}{"4E}
% \DeclareMathSymbol{\varXi}{\mathalpha}{egreeki}{"04}
% \DeclareMathSymbol{\varOmicron}{\mathalpha}{egreeki}{"4F}
% \DeclareMathSymbol{\varPi}{\mathalpha}{egreeki}{"05}
% \DeclareMathSymbol{\varRho}{\mathalpha}{egreeki}{"50}
% \DeclareMathSymbol{\varSigma}{\mathalpha}{egreeki}{"06}
% \DeclareMathSymbol{\varTau}{\mathalpha}{egreeki}{"54}
% \DeclareMathSymbol{\varUpsilon}{\mathalpha}{egreeki}{"07}
% \DeclareMathSymbol{\varPhi}{\mathalpha}{egreeki}{"08}
% \DeclareMathSymbol{\varChi}{\mathalpha}{egreeki}{"58}
% \DeclareMathSymbol{\varPsi}{\mathalpha}{egreeki}{"09}
% \DeclareMathSymbol{\varOmega}{\mathalpha}{egreeki}{"0A} 
% %
% \DeclareMathSymbol{\Alpha}{\mathalpha}{egreekr}{"41}
% \DeclareMathSymbol{\Beta}{\mathalpha}{egreekr}{"42}
% \DeclareMathSymbol{\Gamma}{\mathalpha}{egreekr}{"00}
% \DeclareMathSymbol{\Delta}{\mathalpha}{egreekr}{"01}
% \DeclareMathSymbol{\Epsilon}{\mathalpha}{egreekr}{"45}
% \DeclareMathSymbol{\Zeta}{\mathalpha}{egreekr}{"5A}
% \DeclareMathSymbol{\Eta}{\mathalpha}{egreekr}{"48}
% \DeclareMathSymbol{\Theta}{\mathalpha}{egreekr}{"02}
% \DeclareMathSymbol{\Iota}{\mathalpha}{egreekr}{"49}
% \DeclareMathSymbol{\Kappa}{\mathalpha}{egreekr}{"4B}
% \DeclareMathSymbol{\Lambda}{\mathalpha}{egreekr}{"03}
% \DeclareMathSymbol{\Mu}{\mathalpha}{egreekr}{"4D}
% \DeclareMathSymbol{\Nu}{\mathalpha}{egreekr}{"4E}
% \DeclareMathSymbol{\Xi}{\mathalpha}{egreekr}{"04}
% \DeclareMathSymbol{\Omicron}{\mathalpha}{egreekr}{"4F}
% \DeclareMathSymbol{\Pi}{\mathalpha}{egreekr}{"05}
% \DeclareMathSymbol{\Rho}{\mathalpha}{egreekr}{"50}
% \DeclareMathSymbol{\Sigma}{\mathalpha}{egreekr}{"06}
% \DeclareMathSymbol{\Tau}{\mathalpha}{egreekr}{"54}
% \DeclareMathSymbol{\Upsilon}{\mathalpha}{egreekr}{"07}
% \DeclareMathSymbol{\Phi}{\mathalpha}{egreekr}{"08}
% \DeclareMathSymbol{\Chi}{\mathalpha}{egreekr}{"58}
% \DeclareMathSymbol{\Psi}{\mathalpha}{egreekr}{"09}
% \DeclareMathSymbol{\Omega}{\mathalpha}{egreekr}{"0A}
% %
% \DeclareMathSymbol{\alphaup}{\mathalpha}{egreekr}{"0B}
% \DeclareMathSymbol{\betaup}{\mathalpha}{egreekr}{"0C}
% \DeclareMathSymbol{\gammaup}{\mathalpha}{egreekr}{"0D}
\DeclareMathSymbol{\deltaup}{\mathalpha}{egreekr}{"0E}
% \DeclareMathSymbol{\epsilonup}{\mathalpha}{egreekr}{"0F}
% \DeclareMathSymbol{\zetaup}{\mathalpha}{egreekr}{"10}
% \DeclareMathSymbol{\etaup}{\mathalpha}{egreekr}{"11}
% \DeclareMathSymbol{\thetaup}{\mathalpha}{egreekr}{"12}
% \DeclareMathSymbol{\iotaup}{\mathalpha}{egreekr}{"13}
% \DeclareMathSymbol{\kappaup}{\mathalpha}{egreekr}{"14}
% \DeclareMathSymbol{\lambdaup}{\mathalpha}{egreekr}{"15}
% \DeclareMathSymbol{\muup}{\mathalpha}{egreekr}{"16}
% \DeclareMathSymbol{\nuup}{\mathalpha}{egreekr}{"17}
% \DeclareMathSymbol{\xiup}{\mathalpha}{egreekr}{"18}
% \DeclareMathSymbol{\omicronup}{\mathalpha}{egreekr}{"6F}
\DeclareMathSymbol{\piup}{\mathalpha}{egreekr}{"19}
% \DeclareMathSymbol{\rhoup}{\mathalpha}{egreekr}{"1A}
% \DeclareMathSymbol{\sigmaup}{\mathalpha}{egreekr}{"1B}
% \DeclareMathSymbol{\tauup}{\mathalpha}{egreekr}{"1C}
% \DeclareMathSymbol{\upsilonup}{\mathalpha}{egreekr}{"1D}
% \DeclareMathSymbol{\phiup}{\mathalpha}{egreekr}{"1E}
% \DeclareMathSymbol{\chiup}{\mathalpha}{egreekr}{"1F}
% \DeclareMathSymbol{\psiup}{\mathalpha}{egreekr}{"20}
% \DeclareMathSymbol{\omegaup}{\mathalpha}{egreekr}{"21}
% \DeclareMathSymbol{\varepsilonup}{\mathalpha}{egreekr}{"22}
% \DeclareMathSymbol{\varthetaup}{\mathalpha}{egreekr}{"23}
% \DeclareMathSymbol{\varpiup}{\mathalpha}{egreekr}{"24}
% \let\varrhoup\rhoup 
% \let\varsigmaup\sigmaup
% \let\varkappaup\kappaup
% \DeclareMathSymbol{\varphiup}{\mathalpha}{egreekr}{"27}


% \usepackage%[scaled=0.9]%
% {classico}%  Optima as sans-serif font
\renewcommand\sfdefault{uop}
\DeclareMathAlphabet{\mathsf}  {T1}{\sfdefault}{m}{sl}
\SetMathAlphabet{\mathsf}{bold}{T1}{\sfdefault}{b}{sl}
\newcommand*{\mathte}[1]{\textbf{\textit{\textsf{#1}}}}
% Upright sans-serif math alphabet
% \DeclareMathAlphabet{\mathsu}  {T1}{\sfdefault}{m}{n}
% \SetMathAlphabet{\mathsu}{bold}{T1}{\sfdefault}{b}{n}

% DejaVu Mono as typewriter text
\usepackage[scaled=0.83]{DejaVuSansMono}% was 0.84

\usepackage{mathdots}

\usepackage[usenames]{xcolor}
% Tol (2012) colour-blind-, print-, screen-friendly colours, alternative scheme; Munsell terminology
\definecolor{blue}{HTML}{4477AA}
\definecolor{cyan}{HTML}{66CCEE}
\definecolor{green}{HTML}{228833}
\definecolor{yellow}{HTML}{CCBB44}
\definecolor{red}{HTML}{EE6677}
\definecolor{purple}{HTML}{AA3377}
\definecolor{grey}{HTML}{BBBBBB}
\definecolor{midgrey}{HTML}{888888}
\definecolor{darkgrey}{HTML}{555555}
\definecolor{lgrey}{HTML}{DDDDDD}
%\newcommand*\mycolourbox[1]{%
%\colorbox{grey}{\hspace{1em}#1\hspace{1em}}}
\colorlet{shadecolor}{lgrey}

\usepackage{bm}

\usepackage{pdfrender}
\newcommand*{\textxbf}[1]{\textpdfrender{TextRenderingMode=2,LineWidth=0.3pt}{\textbf{#1}}}
\renewcommand*{\bm}[1]{\textpdfrender{TextRenderingMode=2,LineWidth=0.1pt}{\boldsymbol{#1}}}
% \newcommand*{\bmm}[1]{\textpdfrender{TextRenderingMode=2,LineWidth=0.1pt}{\boldsymbol{#1}}}
% \newcommand*{\bmmm}[1]{\textpdfrender{TextRenderingMode=2,LineWidth=0.2pt}{\boldsymbol{#1}}}
% \newcommand*{\bmmmm}[1]{\textpdfrender{TextRenderingMode=2,LineWidth=0.3pt}{\boldsymbol{#1}}}

\usepackage{microtype}

\usepackage[backend=biber,mcite,%subentry,
citestyle=authoryear-comp,bibstyle=pglpm_latex/pglpm-authoryear,autopunct=false,sorting=ny,sortcites=false,natbib=false,maxcitenames=2,maxbibnames=8,minbibnames=8,giveninits=true,uniquename=false,uniquelist=false,maxalphanames=1,block=space,hyperref=true,defernumbers=false,useprefix=true,sortupper=false,language=british,parentracker=false,autocite=footnote,dashed=false]{biblatex}
\DeclareSortingTemplate{ny}{\sort{\field{sortname}\field{author}\field{editor}}\sort{\field{year}}}
\DeclareFieldFormat{postnote}{#1}
\iffalse\makeatletter%%% replace parenthesis with brackets
\newrobustcmd*{\parentexttrack}[1]{%
  \begingroup
  \blx@blxinit
  \blx@setsfcodes
  \blx@bibopenparen#1\blx@bibcloseparen
  \endgroup}
\AtEveryCite{%
  \let\parentext=\parentexttrack%
  \let\bibopenparen=\bibopenbracket%
  \let\bibcloseparen=\bibclosebracket}
\makeatother\fi
\DefineBibliographyExtras{british}{\def\finalandcomma{\addcomma}}
\renewcommand*{\finalnamedelim}{\addspace\amp\space}
% \renewcommand*{\finalnamedelim}{\addcomma\space}
\renewcommand*{\textcitedelim}{\addcomma\space}
% \setcounter{biburlnumpenalty}{1} % to allow url breaks anywhere
% \setcounter{biburlucpenalty}{0}
% \setcounter{biburllcpenalty}{1}
\setcounter{biburlucpenalty}{1}  %break URL after uppercase character
\setcounter{biburlnumpenalty}{1} %break URL after number
\setcounter{biburllcpenalty}{1}  %break URL after lowercase character
\DeclareDelimFormat{multicitedelim}{\addsemicolon\addspace\space}
\DeclareDelimFormat{compcitedelim}{\addsemicolon\addspace\space}
\DeclareDelimFormat{postnotedelim}{\addspace}
\ifarxiv\else\addbibresource{portamanabib.bib}\fi
\renewcommand{\nameyeardelim}{~}% avoid linebreak between name and year
\renewcommand{\bibfont}{\footnotesize}
%\appto{\citesetup}{\footnotesize}% smaller font for citations
\defbibheading{bibliography}[\bibname]{\section*{#1}\addcontentsline{toc}{section}{#1}%\markboth{#1}{#1}
}
\newcommand*{\citep}{\footcites}
\newcommand*{\citey}{\footcites}%{\parencites*}
\newcommand*{\ibid}{\unspace\addtocounter{footnote}{-1}\footnotemark{}}
%\renewcommand*{\cite}{\parencite}
%\renewcommand*{\cites}{\parencites}
\providecommand{\href}[2]{#2}
\providecommand{\eprint}[2]{\texttt{\href{#1}{#2}}}
\newcommand*{\amp}{\&}
% \newcommand*{\citein}[2][]{\textnormal{\textcite[#1]{#2}}%\addtocategory{extras}{#2}
% }
\newcommand*{\citein}[2][]{\textnormal{\textcite[#1]{#2}}%\addtocategory{extras}{#2}
}
\newcommand*{\citebi}[2][]{\textcite[#1]{#2}%\addtocategory{extras}{#2}
}
\newcommand*{\subtitleproc}[1]{}
\newcommand*{\chapb}{ch.}
%
%\def\UrlOrds{\do\*\do\-\do\~\do\'\do\"\do\-}%
% \def\myUrlOrds{\do\0\do\1\do\2\do\3\do\4\do\5\do\6\do\7\do\8\do\9\do\a\do\b\do\c\do\d\do\e\do\f\do\g\do\h\do\i\do\j\do\k\do\l\do\m\do\n\do\o\do\p\do\q\do\r\do\s\do\t\do\u\do\v\do\w\do\x\do\y\do\z\do\A\do\B\do\C\do\D\do\E\do\F\do\G\do\H\do\I\do\J\do\K\do\L\do\M\do\N\do\O\do\P\do\Q\do\R\do\S\do\T\do\U\do\V\do\W\do\X\do\Y\do\Z}%
\makeatletter
%\g@addto@macro\UrlSpecials{\do={\newline}}
\g@addto@macro{\UrlBreaks}{%
\do\0\do\1\do\2\do\3\do\4\do\5\do\6\do\7\do\8\do\9\do\a\do\b\do\c\do\d\do\e\do\f\do\g\do\h\do\i\do\j\do\k\do\l\do\m\do\n\do\o\do\p\do\q\do\r\do\s\do\t\do\u\do\v\do\w\do\x\do\y\do\z\do\A\do\B\do\C\do\D\do\E\do\F\do\G\do\H\do\I\do\J\do\K\do\L\do\M\do\N\do\O\do\P\do\Q\do\R\do\S\do\T\do\U\do\V\do\W\do\X\do\Y\do\Z%
}
% % The command below add spurious space to urls
% \g@addto@macro\UrlSpecials{%
% \do\/{\mbox{\UrlFont/}\hskip 0pt plus 10pt}%
% }
\makeatother
\newcommand*{\arxiveprint}[1]{%
arXiv \doi{10.48550/arXiv.#1}%
}
\newcommand*{\mparceprint}[1]{%
\href{http://www.ma.utexas.edu/mp_arc-bin/mpa?yn=#1}{mp_arc:\allowbreak\nolinkurl{#1}}%
}
\newcommand*{\haleprint}[1]{%
\href{https://hal.archives-ouvertes.fr/#1}{\textsc{hal}:\allowbreak\nolinkurl{#1}}%
}
\newcommand*{\philscieprint}[1]{%
\href{http://philsci-archive.pitt.edu/archive/#1}{PhilSci:\allowbreak\nolinkurl{#1}}%
}
\newcommand*{\doi}[1]{%
\href{https://doi.org/#1}{\textsc{doi}:\allowbreak\nolinkurl{#1}}%
}
\newcommand*{\biorxiveprint}[1]{%
bioRxiv \doi{10.1101/#1}%
}
\newcommand*{\osfeprint}[1]{%
Open Science Framework \doi{10.31219/osf.io/#1}%
}
\newcommand*{\osfproj}[1]{%
Open Science Framework \doi{10.17605/osf.io/#1}%
}

\usepackage{graphicx}
%\usepackage{graphbox}% to align includegraphics vertically
%\usepackage{wrapfig}

%\usepackage{tikz-cd}

\PassOptionsToPackage{hyphens}{url}\usepackage[hypertexnames=false,pdfencoding=unicode,psdextra]{hyperref}

\usepackage[depth=4]{bookmark}
\hypersetup{%
colorlinks=true,
%pdfborderstyle={/S/U/W 0.5},
bookmarksnumbered,pdfborder={0 0 0.25},
citebordercolor=blue,citecolor=blue,linkbordercolor=blue,linkcolor=blue,urlbordercolor=blue,urlcolor=blue,breaklinks=true,pdftitle={\pdftitle},pdfauthor={\pdfauthor}}
% \usepackage[vertfit=local]{breakurl}% only for arXiv
\providecommand*{\urlalt}{\href}

\usepackage{tensor}

\usepackage[british]{datetime2}
\DTMnewdatestyle{mydate}%
{% definitions
\renewcommand*{\DTMdisplaydate}[4]{%
\number##3\ \DTMenglishmonthname{##2} ##1}%
\renewcommand*{\DTMDisplaydate}{\DTMdisplaydate}%
}
\DTMsetdatestyle{mydate}

%%%%%%%%%%%%%%%%%%%%%%%%%%%%%%%%%%%%%%%%%%%%%%%%%%%%%%%%%%%%%%%%%%%%%%%%%%%%
%%% Layout. I do not know on which kind of paper the reader will print the
%%% paper on (A4? letter? one-sided? double-sided?). So I choose A5, which
%%% provides a good layout for reading on screen and save paper if printed
%%% two pages per sheet. Average length line is 66 characters and page
%%% numbers are centred.
%%%%%%%%%%%%%%%%%%%%%%%%%%%%%%%%%%%%%%%%%%%%%%%%%%%%%%%%%%%%%%%%%%%%%%%%%%%%
\ifafour\setstocksize{297mm}{210mm}%{*}% A4
\else\setstocksize{210mm}{5.5in}%{*}% 210x139.7
\fi
\settrimmedsize{\stockheight}{\stockwidth}{*}
\setlxvchars[\normalfont] %313.3632pt for a 66-characters line
\setxlvchars[\normalfont]
% \setlength{\trimtop}{0pt}
% \setlength{\trimedge}{\stockwidth}
% \addtolength{\trimedge}{-\paperwidth}
%\settrims{0pt}{0pt}
% The length of the normalsize alphabet is 133.05988pt - 10 pt = 26.1408pc
% The length of the normalsize alphabet is 159.6719pt - 12pt = 30.3586pc
% Bringhurst gives 32pc as boundary optimal with 69 ch per line
% The length of the normalsize alphabet is 191.60612pt - 14pt = 35.8634pc
\ifafour\settypeblocksize{*}{32pc}{1.618} % A4
%\setulmargins{*}{*}{1.667}%gives 5/3 margins % 2 or 1.667
\else\settypeblocksize{*}{26pc}{1.618}% nearer to a 66-line newpx and preserves GR
\fi
\setulmargins{*}{*}{1}%gives equal margins
\setlrmargins{*}{*}{*}
\setheadfoot{\onelineskip}{2.5\onelineskip}
\setheaderspaces{*}{2\onelineskip}{*}
\setmarginnotes{2ex}{10mm}{0pt}
\checkandfixthelayout[nearest]
%%% End layout
%% this fixes missing white spaces
%\pdfmapline{+dummy-space <dummy-space.pfb}
%\pdfinterwordspaceon% seems to add a white margin to Sumatrapdf

%%% Sectioning
\newcommand*{\asudedication}[1]{%
{\par\centering\textit{#1}\par}}
\newenvironment{acknowledgements}{\section*{Thanks}\addcontentsline{toc}{section}{Thanks}}{\par}
\makeatletter\renewcommand{\appendix}{\par
  \bigskip{\centering
   \interlinepenalty \@M
   \normalfont
   \printchaptertitle{\sffamily\appendixpagename}\par}
  \setcounter{section}{0}%
  \gdef\@chapapp{\appendixname}%
  \gdef\thesection{\@Alph\c@section}%
  \anappendixtrue}\makeatother
\counterwithout{section}{chapter}
\setsecnumformat{\upshape\csname the#1\endcsname\quad}
\setsecheadstyle{\large\bfseries\sffamily%
\centering}
\setsubsecheadstyle{\bfseries\sffamily%
\raggedright}
%\setbeforesecskip{-1.5ex plus 1ex minus .2ex}% plus 1ex minus .2ex}
%\setaftersecskip{1.3ex plus .2ex }% plus 1ex minus .2ex}
%\setsubsubsecheadstyle{\bfseries\sffamily\slshape\raggedright}
%\setbeforesubsecskip{1.25ex plus 1ex minus .2ex }% plus 1ex minus .2ex}
%\setaftersubsecskip{-1em}%{-0.5ex plus .2ex}% plus 1ex minus .2ex}
\setsubsecindent{0pt}%0ex plus 1ex minus .2ex}
\setparaheadstyle{\bfseries\sffamily%
\raggedright}
\setcounter{secnumdepth}{2}
\setlength{\headwidth}{\textwidth}
\newcommand{\addchap}[1]{\chapter*[#1]{#1}\addcontentsline{toc}{chapter}{#1}}
\newcommand{\addsec}[1]{\section*{#1}\addcontentsline{toc}{section}{#1}}
\newcommand{\addsubsec}[1]{\subsection*{#1}\addcontentsline{toc}{subsection}{#1}}
\newcommand{\addpara}[1]{\paragraph*{#1.}\addcontentsline{toc}{subsubsection}{#1}}
\newcommand{\addparap}[1]{\paragraph*{#1}\addcontentsline{toc}{subsubsection}{#1}}

%%% Headers, footers, pagestyle
\copypagestyle{manaart}{plain}
\makeheadrule{manaart}{\headwidth}{0.5\normalrulethickness}
\makeoddhead{manaart}{%
{\footnotesize%\sffamily%
\scshape\headauthor}}{}{{\footnotesize\sffamily%
\headtitle}}
\makeoddfoot{manaart}{}{\thepage}{}
\newcommand*\autanet{\includegraphics[height=\heightof{M}]{autanet.pdf}}
\definecolor{mygray}{gray}{0.333}
\iftypodisclaim%
\ifafour\newcommand\addprintnote{\begin{picture}(0,0)%
\put(245,149){\makebox(0,0){\rotatebox{90}{\tiny\color{mygray}\textsf{This
            document is designed for screen reading and
            two-up printing on A4 or Letter paper}}}}%
\end{picture}}% A4
\else\newcommand\addprintnote{\begin{picture}(0,0)%
\put(176,112){\makebox(0,0){\rotatebox{90}{\tiny\color{mygray}\textsf{This
            document is designed for screen reading and
            two-up printing on A4 or Letter paper}}}}%
\end{picture}}\fi%afourtrue
\makeoddfoot{plain}{}{\makebox[0pt]{\thepage}\addprintnote}{}
\else
\makeoddfoot{plain}{}{\makebox[0pt]{\thepage}}{}
\fi%typodisclaimtrue
\makeoddhead{plain}{\scriptsize\reporthead}{}{}
% \copypagestyle{manainitial}{plain}
% \makeheadrule{manainitial}{\headwidth}{0.5\normalrulethickness}
% \makeoddhead{manainitial}{%
% \footnotesize\sffamily%
% \scshape\headauthor}{}{\footnotesize\sffamily%
% \headtitle}
% \makeoddfoot{manaart}{}{\thepage}{}

\pagestyle{manaart}

\setlength{\droptitle}{-3.9\onelineskip}
\pretitle{\begin{center}\LARGE\sffamily%
\bfseries}
\posttitle{\bigskip\end{center}}

\makeatletter\newcommand*{\atf}{\includegraphics[totalheight=\heightof{@}]{pglpm_latex/atblack.png}}\makeatother
\providecommand{\affiliation}[1]{\textsl{\textsf{\footnotesize #1}}}
\providecommand{\epost}[1]{\texttt{\footnotesize\textless#1\textgreater}}
\providecommand{\email}[2]{\href{mailto:#1ZZ@#2 ((remove ZZ))}{#1\protect\atf#2}}
%\providecommand{\email}[2]{\href{mailto:#1@#2}{#1@#2}}

\preauthor{\vspace{-0.5\baselineskip}\begin{center}
\normalsize\sffamily%
\lineskip  0.5em}
\postauthor{\par\end{center}}
\predate{\DTMsetdatestyle{mydate}\begin{center}\footnotesize}
\postdate{\end{center}\vspace{-\medskipamount}}

\setfloatadjustment{figure}{\footnotesize}
\captiondelim{\quad}
\captionnamefont{\footnotesize\sffamily%
}
\captiontitlefont{\footnotesize}
%\firmlists*
\midsloppy
% handling orphan/widow lines, memman.pdf
% \clubpenalty=10000
% \widowpenalty=10000
% \raggedbottom
% Downes, memman.pdf
\clubpenalty=9996
\widowpenalty=9999
\brokenpenalty=4991
\predisplaypenalty=10000
\postdisplaypenalty=1549
\displaywidowpenalty=1602
\raggedbottom

\paragraphfootnotes
\setlength{\footmarkwidth}{2ex}
% \threecolumnfootnotes
%\setlength{\footmarksep}{0em}
\footmarkstyle{\textsuperscript{%\color{red}
\scriptsize\bfseries#1}~}
%\footmarkstyle{\textsuperscript{\color{red}\scriptsize\bfseries#1}~}
%\footmarkstyle{\textsuperscript{[#1]}~}

\selectlanguage{british}\frenchspacing

\colorlet{notecolour}{green}
%\newcommand*{\puzzle}{\maltese}
\newcommand*{\puzzle}{{\fontencoding{U}\fontfamily{fontawesometwo}\selectfont\symbol{225}}}
\newcommand*{\wrench}{{\fontencoding{U}\fontfamily{fontawesomethree}\selectfont\symbol{114}}}
\newcommand*{\pencil}{{\fontencoding{U}\fontfamily{fontawesometwo}\selectfont\symbol{210}}}
\newcommand{\mynotew}[1]{{\footnotesize\color{notecolour}\wrench\ #1}}
\newcommand{\mynotep}[1]{{\footnotesize\color{notecolour}\pencil\ #1}}
\newcommand{\mynotez}[1]{{\footnotesize\color{notecolour}\puzzle\ #1}}

%%%%%%%%%%%%%%%%%%%%%%%%%%%%%%%%%%%%%%%%%%%%%%%%%%%%%%%%%%%%%%%%%%%%%%%%%%%%
%%% Paper's details
%%%%%%%%%%%%%%%%%%%%%%%%%%%%%%%%%%%%%%%%%%%%%%%%%%%%%%%%%%%%%%%%%%%%%%%%%%%%
\title{\propertitle}
\author{%
\hspace*{\stretch{1}}%
%% uncomment if additional authors present
% \parbox{0.5\linewidth}%\makebox[0pt][c]%
% {\protect\centering ***\\%
% \footnotesize\epost{\email{***}{***}}}%
% \hspace*{\stretch{1}}%
\parbox{1\linewidth}%\makebox[0pt][c]%
{\protect\centering P.G.L.  Porta Mana  \href{https://orcid.org/0000-0002-6070-0784}{\protect\includegraphics[scale=0.16]{pglpm_latex/orcid_32x32.png}}%
\\\footnotesize
Western Norway University of Applied Sciences%
\quad\epost{\email{pgl}{portamana.org}}%
}%
%% uncomment if additional authors present
% \hspace*{\stretch{1}}%
% \parbox{0.5\linewidth}%\makebox[0pt][c]%
% {\protect\centering ***\\%
% \footnotesize\epost{\email{***}{***}}}%
\hspace*{\stretch{1}}%
}

%\date{Draft of \today\ (first drafted \firstdraft)}
\date{\firstpublished; updated \updated}

%%%%%%%%%%%%%%%%%%%%%%%%%%%%%%%%%%%%%%%%%%%%%%%%%%%%%%%%%%%%%%%%%%%%%%%%%%%%
%%% Macros @@@
%%%%%%%%%%%%%%%%%%%%%%%%%%%%%%%%%%%%%%%%%%%%%%%%%%%%%%%%%%%%%%%%%%%%%%%%%%%%

% Common ones - uncomment as needed
%\providecommand{\nequiv}{\not\equiv}
%\providecommand{\coloneqq}{\mathrel{\mathop:}=}
%\providecommand{\eqqcolon}{=\mathrel{\mathop:}}
%\providecommand{\varprod}{\prod}
\newcommand*{\de}{\uppartial}%partial diff
\newcommand*{\pu}{\piup}%constant pi
\newcommand*{\delt}{\deltaup}%Kronecker, Dirac
%\newcommand*{\eps}{\varepsilonup}%Levi-Civita, Heaviside
%\newcommand*{\riem}{\zetaup}%Riemann zeta
%\providecommand{\degree}{\textdegree}% degree
%\newcommand*{\celsius}{\textcelsius}% degree Celsius
%\newcommand*{\micro}{\textmu}% degree Celsius
% \newcommand*{\I}{\mathrm{i}}%imaginary unit
\newcommand*{\I}{\ensuremath{\mathrm{i}}}
% \newcommand*{\e}{\mathrm{e}}%Neper
\newcommand*{\e}{\ensuremath{\mathrm{e}}}
\newcommand*{\di}{\mathrm{d}}%differential
% \newcommand*{\dii}{\ensuremath{\mathrm{d}}}
% %% From TUGboat 18 (1997) 1 - leads to very strange spacing
% \makeatletter
% \providecommand*{\di}%
% {\@ifnextchar^{\DIfF}{\DIfF^{}}}
% \def\DIfF^#1{%
% \mathop{\mathrm{\mathstrut d}}%
% \nolimits^{#1}\gobblespace}
% \def\gobblespace{%
% \futurelet\diffarg\opspace}
% \def\opspace{%
% \let\DiffSpace\!%
% \ifx\diffarg(%
% \let\DiffSpace\relax
% \else
% \ifx\diffarg[%
% \let\DiffSpace\relax
% \else
% \ifx\diffarg\{%
% \let\DiffSpace\relax
% \fi\fi\fi\DiffSpace}
% \makeatother

\newcommand*{\Di}{\mathrm{D}}%capital differential
\newcommand*{\Li}{\mathrm{L}}%Lie derivative
%\newcommand*{\planckc}{\hslash}
%\newcommand*{\avogn}{N_{\textrm{A}}}
%\newcommand*{\NN}{\bm{\mathrm{N}}}
%\newcommand*{\ZZ}{\bm{\mathrm{Z}}}
%\newcommand*{\QQ}{\bm{\mathrm{Q}}}
\newcommand*{\RR}{\bm{\mathrm{R}}}
%\newcommand*{\CC}{\bm{\mathrm{C}}}
%\newcommand*{\nabl}{\bm{\nabla}}%nabla
%\DeclareMathOperator{\lb}{lb}%base 2 log
\DeclareMathOperator{\tr}{tr}%trace
%\DeclareMathOperator{\card}{card}%cardinality
%% From TUGboat 18 (1997) 1
% \renewoperator{\Re}{\mathrm{Re}}{\nolimits}
% \renewoperator{\Im}{\mathrm{Im}}{\nolimits}
\DeclareMathOperator{\im}{Im}%im part
\DeclareMathOperator{\re}{Re}%re part
%\DeclareMathOperator{\sgn}{sgn}%signum
%\DeclareMathOperator{\ent}{ent}%integer less or equal to
\DeclareMathOperator{\Ord}{O}%same order as
%\DeclareMathOperator{\ord}{o}%lower order than
% \newcommand*{\incr}{\triangle}%finite increment
\newcommand*{\incr}{\Delta}%finite increment
\newcommand*{\defd}{\coloneqq}
\newcommand*{\defs}{\eqqcolon}
%\newcommand*{\Land}{\bigwedge}
%\newcommand*{\Lor}{\bigvee}
%\newcommand*{\lland}{\DOTSB\;\land\;}
%\newcommand*{\llor}{\DOTSB\;\lor\;}
\newcommand*{\limplies}{\mathbin{\Rightarrow}}%implies
%\newcommand*{\suchthat}{\mid}%{\mathpunct{|}}%such that (eg in sets)
%\newcommand*{\with}{\colon}%with (list of indices)
%\newcommand*{\mul}{\times}%multiplication
%\newcommand*{\inn}{\cdot}%inner product
%\newcommand*{\dotv}{\mathord{\,\cdot\,}}%variable place
%\newcommand*{\comp}{\circ}%composition of functions
%\newcommand*{\con}{\mathbin{:}}%scal prod of tensors
%\newcommand*{\equi}{\sim}%equivalent to 
\renewcommand*{\asymp}{\simeq}%equivalent to 
%\newcommand*{\corr}{\mathrel{\hat{=}}}%corresponds to
%\providecommand{\varparallel}{\ensuremath{\mathbin{/\mkern-7mu/}}}%parallel (tentative symbol)
% \renewcommand*{\le}{\leqslant}%less or equal
% \renewcommand*{\ge}{\geqslant}%greater or equal
%\DeclarePairedDelimiter\clcl{[}{]}
%\DeclarePairedDelimiter\clop{[}{[}
%\DeclarePairedDelimiter\opcl{]}{]}
%\DeclarePairedDelimiter\opop{]}{[}%}
\DeclarePairedDelimiter\abs{\lvert}{\rvert}
%\DeclarePairedDelimiter\norm{\lVert}{\rVert}
\DeclarePairedDelimiter\set{\{}{\}} %}
%\DeclareMathOperator{\pr}{P}%probability
\newcommand*{\p}{\mathrm{p}}%probability
\renewcommand*{\P}{\mathrm{P}}%probability
%\newcommand*{\E}{\mathrm{E}}
%% The "\:" space is chosen to correctly separate inner binary and external rels
\renewcommand*{\|}[1][]{\nonscript\:#1\vert\nonscript\:\mathopen{}}
%\DeclarePairedDelimiterX{\cp}[2]{(}{)}{#1\nonscript\:\delimsize\vert\nonscript\:\mathopen{}#2}
%\DeclarePairedDelimiterX{\ct}[2]{[}{]}{#1\nonscript\;\delimsize\vert\nonscript\:\mathopen{}#2}
%\DeclarePairedDelimiterX{\cs}[2]{\{}{\}}{#1\nonscript\:\delimsize\vert\nonscript\:\mathopen{}#2}
%\newcommand*{\+}{\lor}
%\renewcommand{\*}{\land}
%% symbol = for equality statements within probabilities
%% from https://tex.stackexchange.com/a/484142/97039
% \newcommand*{\eq}{\mathrel{\!=\!}}
% \let\texteq\=
% \renewcommand*{\=}{\TextOrMath\texteq\eq}
% \newcommand*{\eq}[1][=]{\mathrel{\!#1\!}}
\newcommand*{\mo}[1][=]{\mathclose{}\mathord{\nonscript\mkern0.5mu#1\nonscript\mkern0.5mu}\mathopen{}}
%%
\newcommand*{\sect}{\S}% Sect.~
\newcommand*{\sects}{\S\S}% Sect.~
\newcommand*{\chap}{ch.}%
\newcommand*{\chaps}{chs}%
\newcommand*{\bref}{ref.}%
\newcommand*{\brefs}{refs}%
%\newcommand*{\fn}{fn}%
\newcommand*{\eqn}{eq.}%
\newcommand*{\eqns}{eqs}%
\newcommand*{\fig}{fig.}%
\newcommand*{\figs}{figs}%
\newcommand*{\vs}{{vs}}
\newcommand*{\eg}{{e.g.}}
\newcommand*{\etc}{{etc.}}
\newcommand*{\ie}{{i.e.}}
%\newcommand*{\ca}{{c.}}
\newcommand*{\foll}{{ff.}}
%\newcommand*{\viz}{{viz}}
\newcommand*{\cf}{{cf.}}
%\newcommand*{\Cf}{{Cf.}}
%\newcommand*{\vd}{{v.}}
\newcommand*{\etal}{{et al.}}
%\newcommand*{\etsim}{{et sim.}}
%\newcommand*{\ibid}{{ibid.}}
%\newcommand*{\sic}{{sic}}
%\newcommand*{\id}{\mathte{I}}%id matrix
%\newcommand*{\nbd}{\nobreakdash}%
%\newcommand*{\bd}{\hspace{0pt}}%
%\def\hy{-\penalty0\hskip0pt\relax}
%\newcommand*{\labelbis}[1]{\tag*{(\ref{#1})$_\text{r}$}}
%\newcommand*{\mathbox}[2][.8]{\parbox[t]{#1\columnwidth}{#2}}
\newcommand*{\zerob}[1]{\makebox[0pt][c]{#1}}
\newcommand*{\tprod}{\mathop{\textstyle\prod}\nolimits}
\newcommand*{\tsum}{\mathop{\textstyle\sum}\nolimits}
%\newcommand*{\tint}{\begingroup\textstyle\int\endgroup\nolimits}
%\newcommand*{\tland}{\mathop{\textstyle\bigwedge}\nolimits}
%\newcommand*{\tlor}{\mathop{\textstyle\bigvee}\nolimits}
%\newcommand*{\sprod}{\mathop{\textstyle\prod}}
%\newcommand*{\ssum}{\mathop{\textstyle\sum}}
%\newcommand*{\sint}{\begingroup\textstyle\int\endgroup}
%\newcommand*{\sland}{\mathop{\textstyle\bigwedge}}
%\newcommand*{\slor}{\mathop{\textstyle\bigvee}}
\newcommand*{\T}{^\transp}%transpose
%%\newcommand*{\QEM}%{\textnormal{$\Box$}}%{\ding{167}}
%\newcommand*{\qem}{\leavevmode\unskip\penalty9999 \hbox{}\nobreak\hfill
%\quad\hbox{\QEM}}
%% from TUGboat 18 (1997) 1:
%\providecommand*{\unit}[1]{\ensuremath{\mathrm{\,#1}}}

%%%%%%%%%%%%%%%%%%%%%%%%%%%%%%%%%%%%%%%%%%%%%%%%%%%%%%%%%%%%%%%%%%%%%%%%%%%%
%%% Custom macros for this file @@@
%%%%%%%%%%%%%%%%%%%%%%%%%%%%%%%%%%%%%%%%%%%%%%%%%%%%%%%%%%%%%%%%%%%%%%%%%%%%

\newcommand*{\widebar}[1]{{\mkern1.5mu\skew{2}\overline{\mkern-1.5mu#1\mkern-1.5mu}\mkern 1.5mu}}

\newcommand*{\iT}{^{-\transp}}%transpose
\newcommand*{\id}{\mathbf{id}}%id matrix

\renewcommand*{\i}{{}\indices}
%%
% from https://tex.stackexchange.com/a/424252/97039
\makeatletter
\newcommand*{\q}{}% Check if undefined
\DeclareRobustCommand*{\q}{%
  \mathord{\mathpalette\bigcdot@{}}% changed mathbin to mathord
}
\newcommand*{\bigcdot@scalefactor}{0.7}
\newcommand*{\bigcdot@widthfactor}{1.5}
\newcommand*{\bigcdot@}[2]{%
  % #1: math style
  % #2: unused
  \sbox0{$#1\vcenter{}$}% math axis
  \sbox2{$#1\cdot\m@th$}%
  \hbox to \bigcdot@widthfactor\wd2{%
    \hfil
    \raise\ht0\hbox{%
      \scalebox{\bigcdot@scalefactor}{%
        \lower\ht0\hbox{$#1\bullet\m@th$}%
      }%
    }%
    \hfil
  }%
}
\makeatother
%%
\newcommand*{\rul}{{\mkern2mu\rule[-0.1ex]{0.75pt}{1.1ex}\mkern2mu}}
\DeclarePairedDelimiter\mul{\rul}{\rul}%{{\bm{\shortmid}}}

% wedge with superimposed dot
\newcommand*{\dand}{\mathbin{\mathclap{\mkern12.5mu\bm{\cdot}}\wedge}}

\newcommand*{\ddi}{\bm{\di}}

\newcommand*{\ve}[1]{\bm{e}_{#1}}
\newcommand*{\vve}[1]{\bm{e}^{2}_{#1}}
\newcommand*{\vvve}[1]{\bm{e}^{3}_{#1}}
\newcommand*{\vvvve}[1]{\bm{e}^{4}_{#1}}
%
\newcommand*{\vi}[1]{\di{#1}}
\newcommand*{\vvi}[1]{\di^{2}{#1}}
\newcommand*{\vvvi}[1]{\di^{3}{#1}}
\newcommand*{\vvvvi}[1]{\di^{4}{#1}}
% %
% \newcommand*{\te}[1]{\underset{\raisebox{1ex}{$\sim$}}{e}{#1}}
% \newcommand*{\tte}[1]{\underset{\raisebox{1ex}{$\sim$}}{e}^{2}{#1}}
% \newcommand*{\ttte}[1]{\underset{\raisebox{1ex}{$\sim$}}{e}^{3}{#1}}
% \newcommand*{\tttte}[1]{\underset{\raisebox{1ex}{$\sim$}}{e}^{4}{#1}}
% %
% \newcommand*{\ti}[1]{\smash[b]{\underset{\raisebox{1ex}{$\sim$}}{\di}}_{#1}}
% \newcommand*{\tti}[1]{\smash[b]{\underset{\raisebox{1ex}{$\sim$}}{\di}}^{2}_{#1}}
% \newcommand*{\ttti}[1]{\smash[b]{\underset{\raisebox{1ex}{$\sim$}}{\di}}^{3}_{#1}}
% \newcommand*{\tttti}[1]{\smash[b]{\underset{\raisebox{1ex}{$\sim$}}{\di}}^{4}_{#1}}
%
\newcommand*{\tw}[1]{\tilde{#1}}
\newcommand*{\te}[1]{\bm{e}{#1}}
\newcommand*{\tte}[1]{\bm{e}^{2}{#1}}
\newcommand*{\ttte}[1]{\bm{e}^{3}{#1}}
\newcommand*{\tttte}[1]{\bm{e}^{4}{#1}}
%
\newcommand*{\ti}[1]{\di X_{#1}}
\newcommand*{\tti}[1]{\di^{2}X_{#1}}
\newcommand*{\ttti}[1]{\di^{3}X_{#1}}
\newcommand*{\tttti}[1]{\di^{4}X_{#1}}
%
\newcommand*{\yg}{\mathte{g}}
\newcommand*{\ygg}[1][]{\overset{#1}{\bm{g}}{}}
\newcommand*{\dg}{\sqrt{\abs{g}}}
\DeclarePairedDelimiter\nor{\lVert}{\rVert}
\newcommand*{\vol}{\tfrac{\bm{\sqrt{g}}}{c}}
\newcommand*{\ivol}{\tfrac{c}{\bm{\sqrt{g}}}}
% \newcommand*{\vi}{\bar{\mathcal{G}}}
%
\newcommand*{\yN}{\bm{\mathcal{N}}}
\newcommand*{\yQ}{\bm{Q}}
\newcommand*{\yS}{\mathte{S}}
\newcommand*{\yTTe}{\mathte{T}}
\newcommand*{\yTe}{T}
\newcommand*{\yTT}{\bm{\mathcal{T}}}
\newcommand*{\yT}{\mathcal{T}}
\newcommand*{\yJJ}{\bm{\mathcal{J}}}
\newcommand*{\yJ}{\mathcal{J}}
\newcommand*{\yPP}{\bm{\mathcal{P}}}
\newcommand*{\yP}{\mathcal{P}}
\newcommand*{\yEE}{\bm{\mathcal{E}}}
\newcommand*{\yEc}{\mathcal{E}}
\newcommand*{\yEi}{\bm{\mathcal{E}}_{\textrm{i}}}
\newcommand*{\yEk}{\bm{\mathcal{E}}_{\textrm{k}}}
\newcommand*{\yEp}{\bm{\mathcal{E}}_{\textrm{p}}}
\newcommand*{\ye}{e}
\newcommand*{\yt}{\bm{\tau}}
\newcommand*{\yU}{\bm{U}}
\newcommand*{\yUm}{\yU_{\textrm{m}}}
\newcommand*{\yUd}{\bar{\bm{U}}}
\newcommand*{\yV}{\bm{V}}
\newcommand*{\yu}{\bm{u}}
\newcommand*{\yv}{\bm{v}}
\newcommand*{\yo}{\bm{\omega}}
\newcommand*{\yh}{\bm{\eta}}
\newcommand*{\yphi}{\bm{\phi}}
\newcommand*{\ypsi}{\bm{\psi}}

\newcommand*{\yW}{W}

\newcommand*{\yF}{\mathte{F}}
\newcommand*{\yG}{\mathte{G}}
\newcommand*{\yE}{\bm{E}}
\newcommand*{\yB}{\mathte{B}}
%%% Custom macros end @@@

\usepackage[retain-explicit-plus,quantity-product=\:]{siunitx}
%\sisetup{input-digits = 0123456789\piup}


%%%%%%%%%%%%%%%%%%%%%%%%%%%%%%%%%%%%%%%%%%%%%%%%%%%%%%%%%%%%%%%%%%%%%%%%%%%%
%%% Beginning of document
%%%%%%%%%%%%%%%%%%%%%%%%%%%%%%%%%%%%%%%%%%%%%%%%%%%%%%%%%%%%%%%%%%%%%%%%%%%%
%\firmlists
\begin{document}
\captiondelim{\quad}\captionnamefont{\footnotesize}\captiontitlefont{\footnotesize}
\selectlanguage{british}\frenchspacing
\maketitle

%%%%%%%%%%%%%%%%%%%%%%%%%%%%%%%%%%%%%%%%%%%%%%%%%%%%%%%%%%%%%%%%%%%%%%%%%%%%
%%% Abstract
%%%%%%%%%%%%%%%%%%%%%%%%%%%%%%%%%%%%%%%%%%%%%%%%%%%%%%%%%%%%%%%%%%%%%%%%%%%%
\abstractrunin
\abslabeldelim{}
\renewcommand*{\abstractname}{}
\setlength{\absleftindent}{0pt}
\setlength{\absrightindent}{0pt}
\setlength{\abstitleskip}{-\absparindent}
\begin{abstract}\labelsep 0pt%
  \noindent Some notes on energy, momentum, angular momentum, and on how their general-relativistic definitions affect their appearance in Newtonian approximations.
% \\\noindent\emph{\footnotesize Note: Dear Reader
%     \amp\ Peer, this manuscript is being peer-reviewed by you. Thank you.}
% \par%\\[\jot]
% \noindent
% {\footnotesize PACS: ***}\qquad%
% {\footnotesize MSC: ***}%
%\qquad{\footnotesize Keywords: ***}
\end{abstract}
\selectlanguage{british}\frenchspacing

%%%%%%%%%%%%%%%%%%%%%%%%%%%%%%%%%%%%%%%%%%%%%%%%%%%%%%%%%%%%%%%%%%%%%%%%%%%%
%%% Epigraph
%%%%%%%%%%%%%%%%%%%%%%%%%%%%%%%%%%%%%%%%%%%%%%%%%%%%%%%%%%%%%%%%%%%%%%%%%%%%
% \asudedication{\small ***}
% \vspace{\bigskipamount}
% \setlength{\epigraphwidth}{.7\columnwidth}
% %\epigraphposition{flushright}
% \epigraphtextposition{flushright}
% %\epigraphsourceposition{flushright}
% \epigraphfontsize{\footnotesize}
% \setlength{\epigraphrule}{0pt}
% %\setlength{\beforeepigraphskip}{0pt}
% %\setlength{\afterepigraphskip}{0pt}
% \epigraph{\emph{text}}{source}



%%%%%%%%%%%%%%%%%%%%%%%%%%%%%%%%%%%%%%%%%%%%%%%%%%%%%%%%%%%%%%%%%%%%%%%%%%%%
%%% BEGINNING OF MAIN TEXT
%%%%%%%%%%%%%%%%%%%%%%%%%%%%%%%%%%%%%%%%%%%%%%%%%%%%%%%%%%%%%%%%%%%%%%%%%%%%

% \setcounter{section}{-1} % to start numbering at 0
\section{Questions about momentum and energy}
\label{sec:questions}

The notions of momentum and energy in Newtonian physics present many peculiarities, at least to the eye of a student who's learning about them. % Many of these peculiarities are then gradually hammered away or swept under the carpet under teaching, by means of definitions, verbal mnemonics, and mathematical gesticulation. Yet they resurface again from time to time.
Consider for instance the following questions, some of which have indeed been asked by students, including me:
\begin{enumerate}
\item\label{item:tot_energy} How should \emph{total energy} be defined? Is it just the sum of internal and kinetic energies? or should it also include gravitational potential energy?
\item\label{item:internalenergy_invariant} Why is \emph{internal energy} the same in two reference frames, whereas \emph{kinetic energy} differs?
\item\label{item:force_energyflux} Why does force, which is related to change of momentum, appear in the laws for the change of energy (in the formula for \enquote{work})? Why doesn't change of energy appear, vice versa, in the law for change of momentum?
\item\label{item:mom_transf} What is the \emph{law of transformation for momentum} between reference frames? For instance, if we know that the components of momentum in an inertial frame are $[0,1,2]\:\unit{N\,s}$, then what are the components in another inertial frame with constant velocity $\bm{v}$ with respect to the first?
\end{enumerate}

Tentative answers to some of these questions may lead to embarrassing further questions.

Take question~\ref{item:tot_energy} for example. A possible answer is that the definition of \enquote{total energy} is arbitrary. For an object near Earth's surface, we may avoid speaking of gravitational potential energy if we include the work done by gravitational forces in the balance of energy; or vice versa we may include a gravitational potential energy in the total energy, avoiding the inclusion of work by gravitational forces in the energy balance. To this explanation the student may ask why we have such definition freedom for energy, but not an analogous one for momentum.

Or take question~\ref{item:mom_transf}. A possible answer is that in order to know the momentum in another frame we need to know both the mass and the velocity of an object. At this answer the student may have the following questions: \emph{How about the case of an electromagnetic field, where there's no mass or velocity?} \emph{And why do we need such extra information for the transformation of momentum, when we don't need extra information for the transformation of velocity or of mass?}

There are different perspectives from which one can try to answer questions like these. One can take a historical, rather than physical, point of view. Or one may say that it's just a matter of definitions. The literature also offers more physical answers for some of these questions, based for instance on symmetry, or on variational principles, or on the Newton-Cartan theory. As an example, some work of \v{S}ilhav\'y\autocites{silhavy1989,silhavy1992}[see also][]{serrin1995b_r1998}, derives the notions of mass, momentum, and kinetic energy from Galileian invariance.

Here we show how these questions receive answers from the point of view of general relativity.

\section{Energy-momentum tensor}
\label{sec:energy_momentum_tensor}

\subsection{Representation}
\label{sec:EMt_representation}

Energy, momentum, flux of energy, and force, or more generally flux of momentum, are all components of a single object, the \emph{energy-momentum tensor}, also called \enquote*{energy-momentum-stress tensor}, with permutations, or \enquote*{stress-energy tensor}, or \enquote*{four-stress}, or also \enquote*{mass tensor}. It can be represented by a 4-by-4 matrix. Roughly speaking, the entries in the matrix can be related as follows with energy and momentum:
\begin{equation}
  \label{eq:energy_momentum_entries}
  \begin{bmatrix}
    \text{\footnotesize energy} &
    \text{\footnotesize momentum} \\[2\jot]
    \rotatebox[origin=c]{90}{\text{\footnotesize energy flux}} &
    \rotatebox[origin=c]{45}{\parbox{8em}{\centering\footnotesize force\\ (momentum flux)}}
  \end{bmatrix} \ .
\end{equation}
The \enquote{roughly speaking} will be made precise soon.

The fact that energy, momentum, and their fluxes are components of a single object, already gives a partial answer to question~\ref{item:force_energyflux}. We know in the simpler examples of vectors that their components get mixed in a change of frame. This makes it plausible that forces, for instance, should appear in the energy flux upon a change of frame. Also question~\ref{item:mom_transf} starts to receive an answer. Generally all components of an object are needed to find the new ones in another frame. The three components of momentum are not really a vector by themselves, but are partial components of something larger; this is why they are not enough, by themselves, for finding the momentum components in a new frame. This also hints at the fact that \emph{mass} or something related to it should be one of the additional components.

\medskip

The energy-momentum tensor can be represented in several equivalent ways, differing in their contravariant or covariant character. Here we discuss two: one that is common in textbook, and one that has a more immediate physical meaning and is more often used for numerical computations. The first is by a contra-contra-variant tensor
\begin{equation}
  \label{eq:energy-momentum-contraco}
  \yTTe = T\i{^{\mu\nu}}\,\ve{\mu}\otimes\ve{\nu}
\end{equation}
and the second by a so-called tensor density, or with more modern terminology a covector-valued 3-covector field:
\begin{equation}
  \label{eq:energy-momentum-tdens}
  \yTT = \yT\i{^{\mu}_{\nu}}\,\ttti{\mu}\otimes\vi{x^{\nu}}
\end{equation}
The components of the two representations are related by
\begin{equation}
  \label{eq:relation_components_EMt}
  \yT\i{^{\mu}_{\nu}} = \tfrac{\dg}{c} \, T\i{^{\mu\alpha}} \, g_{\alpha\nu}
\end{equation}
where $g$ is the determinant of the metric $\yg$.

The component $\yT\i{^{\mu}_{\nu}}$ expresses, roughly speaking, the flow of $x^{\nu}$-momentum \emph{per unit coordinates}, across a volume (hypersurface in spacetime) of constant $x^{\mu}$. Again the \enquote{roughly speaking} will be made precise soon.


In the expressions above, $\vi{x^{\mu}}$ are the differentials of the coordinates $x^{mu}$ and form a basis for covector fields, and $\ve{\mu}$ are their dual vectors -- generally not orthonormal -- and form a basis for vector fields. The $\ttti{\mu}$ are volume forms defined as follows:\autocites[notation similar to][\sect~2 p.~371]{gotayetal1992}
\begin{equation}
  \label{eq:vol_elements}
  \begin{gathered}
    \ttti{0} \defd \vi{\tw{x}^{1}} \vi{x^{2}} \vi{x^{3}}
    \defd \vi{\tw{x}^{1}} \land \vi{x^{2}} \land \vi{x^{3}}
    \\[2\jot]
    \ttti{1} \defd -\vi{\tw{x}^{0}} \vi{x^{2}} \vi{x^{3}}
    \qquad
    \ttti{2} \defd -\vi{\tw{x}^{0}} \vi{x^{3}} \vi{x^{1}}
    \qquad
    \ttti{3} \defd -\vi{\tw{x}^{0}} \vi{x^{1}} \vi{x^{2}}
  \end{gathered}
\end{equation}
that is, they are a basis for 3-covector fields. They can be integrated over three-dimensional regions of spacetime and therefore are the main building block for the definition of the total amount of any quantity in a spatial region, and of the flow of any quantity through a surface during a time lapse. The tilde $\tw{\phantom{-}}$, called an outer-oriented scalar, indicates that an outer orientation is chosen for these 3-covectors, instead of an inner orientation. For instance, the 3-covector $\vi{\tw{x}^{1}} \vi{x^{2}} \vi{x^{3}}$ does not have a screw-sense $x^{1}x^{2}x^{3}$, but an orientation along the complementary direction $x^{0}$, towards positive $x^{0}$. This complementary orientation is indicated by the subscript on \enquote{$\ttti{}$}. Compare with the similar notation in   \cites[\chap~2 Box~5.4]{misneretal1970_r2017}, who however define the volume forms slightly differently; or in \cites[\sect~4.11]{weinberg1972}.

\subsection{Equations for the energy-momentum tensor}
\label{sec:EMt_equations}

The energy-momentum tensor satisfies two fundamental equations or constraints, as a consequence of the Einstein equations. The first is
\begin{equation}
  \label{eq:div_EMt}
  \Di\yTT = 0
  \qquad\text{\footnotesize or in components}\quad
  \de_{\mu}\yT\i{^{\mu}_{\alpha}}
  - \yT\i{^{\mu}_{\nu}} \, \varGamma^{\nu}_{\mu\alpha}
  = 0
\end{equation}
where $\Di$ is the covariant exterior derivative. The second is, in components,
\begin{equation}
  \label{eq:sym_EMt}
  \tfrac{c}{\dg}\,
  \yT\i{^{\mu}_{\alpha}} \, g^{\alpha\nu} \,
\ve{\mu} \otimes \ve{\nu}
  =
  \tfrac{c}{\dg}\,
  \yT\i{^{\nu}_{\alpha}} \, g^{\alpha\mu} \,
\ve{\mu} \otimes \ve{\nu}
  \ .
\end{equation}

The first equation is often presented as saying that \enquote{the divergence of the energy-momentum tensor is zero}. But we must keep in mind that this \enquote{divergence} does \emph{not} satisfy a Stokes theorem. The fundamental problem underlying this is that \emph{the energy-momentum tensor cannot be integrated over a spacetime region}, be it four- or three-dimensional. So it doesn't make sense to speak of the \enquote{total} energy-momentum in a region. The same is true for the divergence of the energy-momentum tensor. This is a basic point in differential geometry: tensors at different points cannot be compared, added, and so on, in any standard way; there are different inequivalent ways to realize such comparisons and sums.

Then how come that we usually speak of the total energy or momentum in a region, and of their fluxes through surfaces?

We shall now see how the energy-momentum tensor can be used to define energy and momenta that satisfy particular balance laws, thanks to equations~\eqref{eq:div_EMt}--\eqref{eq:sym_EMt}. The definition, however, depends on extra fields; so it's important to keep in mind is that \emph{in general there is no unique, canonical definition} of such energy and momentum. The energy and momentum used in Newtonian mechanics are just particular choices, related to the approximations involved in Newtonian mechanics.

\subsection{Definitions of energy and momentum}
\label{sec:def_energy_momentum}

Given an energy-momentum tensor $\yTT$ we can construct an associated four-current as follows.\autocites{gotayetal1992}[\sect~3.2 p.~62]{hawkingetal1973_r1994}[\sect~II.7.III p.~87]{choquetbruhatetal1989_r2000}[cf also][\sect~2.5]{malament2012}[and the discussion in][part~4 \sect~1]{vandantzig1934b}

Choose an arbitrary non-zero vector field $\yV$ over a four-dimensional spacetime region. We call it a \emph{reference vector field}. Keep in mind that this reference field doesn't need to be related in any way with the system of coordinates under use.

From the energy momentum tensor $\yTT$ and the reference field $\yV$ we can construct a four-current
\begin{equation}
  \label{eq:current_V}
\yJJ \defd    \yTT \cdot \yV
  \qquad\text{\footnotesize in components}\quad
  \yJ\i{^{\mu}} \, \ttti{\mu} =
  (\yT\i{^{\mu}_{\alpha}} V^{\alpha}) \, \ttti{\mu} \ .
\end{equation}
This current is a 3-covector, so it can be integrated over any three-dimensional region of spacetime. Integration over a three-dimensional volume at some constant time coordinate is usually interpreted as the total, net amount in that volume. Integration over the three-dimensional span of a two-dimensional surface is usually interpreted as the total, net flow through the surface during a time lapse.

What's more, thanks to \eqns~\eqref{eq:div_EMt} and~\eqref{eq:sym_EMt} this current satisfies the following balance law: \autocites[\sect~III.7.III \eqns~(5), (6)]{choquetbruhatetal1989_r2000}
\begin{equation}
  \label{eq:balance_V}
  \di\yJJ =
  \tfrac12 \tr\bigl(\yTTe \cdot \Li_{\yV}\yg \bigr)
  \,\tfrac{\dg}{c} \tttti{} \ .
\end{equation}
Stokes's theorem holds for this current: for any four-dimensional region $R$ where $\yV$ is defined,\ibid
\begin{equation}
  \label{eq:Stokes_V}
    \int_{\de R} \yJJ = \int_{R} \di\yJJ
    =  \int_{R}\tfrac12 \tr\bigl(\yTTe \cdot \Li_{\yV}\yg \bigr)
  \,\tfrac{\dg}{c} \tttti{} \ .
\end{equation}
If the reference field $\yV$ is a Killing vector in the spacetime region, so that $\Li_{\yV}\yg = 0$ there, then the current $\yJJ$ is conserved there.

The ten equations~\eqref{eq:div_EMt}--\eqref{eq:sym_EMt} satisfied by the energy-momentum tensor can be alternatively formulated as balance laws satisfied by ten currents associated to ten independent reference fields. This is how the balances of energy, of the three components of  momentum, of three components of angular momentum, and of three components of boost-momentum, come about.

When the reference field is timelike and has intrinsic dimension of inverse time, then the associated current is called an \emph{energy} current, whose components typically are an energy density and  energy fluxes in three spatial directions.

When the reference field is spacelike and has intrinsic dimension of inverse length, then the associated current is called a \emph{momentum} current, whose components typically are a momentum density and momentum fluxes -- stresses -- in three spatial directions.

This view of energy and momentum as defined with respect to reference vector fields encompasses all other main points of view for presenting energy and momentum. For example in special relativity, when inertial coordinates are used, the reference fields are the coordinate vectors themselves; see below. Another example is the virtual-work point of view: the vector $\yV$ in this case represents a field of virtual displacements. The connection with the point of view of symmetries and Noether's theorem is also evident.


We have infinite choices of vector fields to associate energy and momentum with. We could also define currents that are somehow hybrids of energy and momentum, or \enquote{momentum} whose components don't really form a vector. How to choose such reference fields?

\medskip

Suppose we have a coordinate system $(t,x,y,z)$, with timelike $t$ and spacelike $x,y,z$. Then there are several natural choices of reference fields with which we can associate 4-currents in the manner above.

One choice are the basis vector fields
\begin{equation}
  \label{eq:choice_bvectors}
  -\ve{t} \qquad \ve{x} \qquad \ve{y} \qquad \ve{z} \ ,
\end{equation}
energy being associated with the first, and momentum with the last three. The minus sign is related to the hyperbolic character of the metric. For spacelike coordinates that have angular dimension, the associated momentum has the character of an angular momentum. Note that these vector fields may not be orthonormal.

Another natural choice are the vectors above, but appropriately normalized:
\begin{equation}
  \label{eq:choice_norm_bvectors}
- \frac{c}{\sqrt{\abs{g_{tt}}}}\ve{t} \qquad
  \frac{1}{\sqrt{\abs{g_{xx}}}}\ve{x} \qquad
  \frac{1}{\sqrt{\abs{g_{yy}}}}\ve{y} \qquad
  \frac{1}{\sqrt{\abs{g_{zz}}}}\ve{z} \ ,
\end{equation}
because $\ve{t} \cdot \yg \cdot \ve{t} = g_{tt}$ and so on. Normalizing them makes sense in that the displacements represented by these vectors are in terms of unit physical time and unit physical length.

Two other natural choices are the vectors obtained from the basis covectors:
\begin{equation}
  \label{eq:choice_bcovectors}
 - c\, \yg^{-1} \cdot \vi{t} \qquad
  \yg^{-1} \cdot \vi{x} \qquad
  \yg^{-1} \cdot \vi{y} \qquad
  \yg^{-1} \cdot \vi{z} \ ,
\end{equation}
as well as their normalized counterparts analogous to~\eqref{eq:choice_norm_bvectors}.

Further natural choices of reference fields are possible, unrelated to the coordinate system. One may choose, for instance, Killing vector fields. Or one may choose vectors fields which are orthonormal to one another, a so-called tetrad of fields. In the presence of matter there is a four-velocity field $-\yUm$, and we can choose this timelike field as the reference one to associate an energy with.

\subsubsection{Momentum}
\label{sec:momentum}

In low-curvature regions typical of the Newtonian approximation, and with approximately inertial coordinates, it turns out that all choices above for the \emph{spacelike} reference fields lead to approximately the same three currents. This is why there seems to be a unique definition of momentum, and why we can consider the $x$-, $y$-, $z$-momenta as components of a vector:
\begin{equation}
  \label{eq:xmomentum_current}
 \yPP_{x} \defd
 \yT\i{^{\mu}_{x}} \, \ttti{\mu} \equiv
  \yTT \cdot \ve{x} \approx
 \tfrac{1}{\sqrt{\abs{g_{xx}}}} \, \yTT \cdot \ve{x} \approx
 \yTT \cdot \yg^{-1} \cdot\vi{x}
\end{equation}
and similarly for $y$ and $z$.

The $x$-momentum thus defined, being a 3-covector, can be integrated over a spacelike three-dimensional region $R$:
\begin{equation}
  \label{eq:integral_Px}
  \int_{R} \yPP_{x} \equiv \int_{R} \yP^{\mu}\, \ttti{\mu}
\end{equation}
which is the total, net amount of $x$-momentum in that region. Note that this is a coordinate-independent result -- although of course it depends on the region $R$ and on the reference field used to define the $x$-momentum. Analogously we can integrate over a timelike three-dimensional region, which can be seen as the time evolution of a spacelike two-dimensional one. The result is the total, net amount of $x$-momentum flowing through that evolving surface -- that is, the $x$-force from one side of the surface to the other. Analogous discussion holds for $y$ and $z$.


\subsubsection{Energies}
\label{sec:energies}

The different choices of \emph{timelike} reference field, however, lead to slightly different associated currents. This is why there seems to be some freedom in the definition of energy. In particular:
\begin{itemize}
\item The energy four-current
  \begin{equation}
    \label{eq:internal_energy_current}
    \yEi \defd -\yTT \cdot \yUm
  \end{equation}
  associated with the matter four-velocity $\yUm$ is what we call \emph{internal energy-mass}. Its flux includes heat flux and transport terms. Note that the reference field $\yUm$ is unrelated to the coordinates; this is why internal energy and heat flux are invariants in the Newtonian approximation and in the exact case. This energy cannot be defined where there is no matter, as the four-velocity field is undefined there.

\item The energy four-current
  \begin{equation}
    \label{eq:internalkinetic_energy_current}
    \yEi + \yEk \defd -\tfrac{c}{\sqrt{\abs{g_{tt}}}} \, \yTT \cdot \ve{t}
  \end{equation}
  associated with the normalized basis field $\frac{c}{\sqrt{\abs{g_{tt}}}}\ve{t}$ is represented as the sum of internal energy above and of \emph{kinetic energy}. Kinetic energy is therefore associated with the difference between the two reference fields above:
  \begin{equation}
    \label{eq:kinetic_energy_current}
    \yEk \defd -\yTT \cdot
    \Bigl(\tfrac{c}{\sqrt{\abs{g_{tt}}}} \ve{t} - \yUm\Bigr) \ .
  \end{equation}
  Its flux includes the \emph{mechanical power} done by the stresses.
%
  % In Newtonian-approximation conditions we can write the matter four-velocity as
  % \begin{equation}
  %   \label{eq:Umatter_approx}
  %   \yUm \approx
  %   \frac{c}{\sqrt{\abs{g_{tt}}}}
  %   \biggl(1-
  %   \frac12 \frac{ {g_{t j}u^{j} + u^{i}g_{ij} u^{j}} }{ \abs{g_{tt} }}
  %   \biggr)
  %   \, (\ve{t} + u^{i}\ve{i})
  % \end{equation}
  % from which we see, by \eqn~\eqref{eq:kinetic_energy_current}, that
  % \begin{equation}
  %   \label{eq:kinetic_energy_newton}
  %   \yEk \approx -\yTT \cdot (u^{i}\ve{i}) =
  %   u^{i}\, \yPP_{i}
  %   - \frac12 \frac{ {g_{t j}u^{j} + u^{i}g_{ij} u^{j}} }{ \abs{g_{tt} }}
  % \end{equation}
%
  The difference above is a spacelike vector field, therefore kinetic energy has something akin to momentum, so to speak.

\item The energy four-current
  \begin{equation}
    \label{eq:internalkineticpotential_energy_current}
    \yEi + \yEk + \yEp \defd  -\yTT \cdot \ve{t}
    \equiv -\yT\i{^{\mu}_{t}} \, \ttti{\mu} \ .
  \end{equation}
  associated with the  basis field $\ve{t}$ is represented as the sum of internal, kinetic, and \emph{gravitational potential} energy. Potential gravitational energy is therefore associated with the difference between two reference fields:
  \begin{equation}
    \label{eq:gravit_energy_current}
    \yEp \defd -\yTT \cdot
    \Bigl(1-\tfrac{c}{\sqrt{\abs{g_{tt}}}}\Bigr) \, \ve{t} \ .
  \end{equation}
  % \begin{equation}
  %   \bmm{\mathcal{E}}_{\textrm{p}} \defd \bmm{\mathcal{T}} \cdot
  %   \Bigl(1-\tfrac{c}{\sqrt{\abs{g_{tt}}}}\Bigr) \, \bmm{e}_{t} \ .
  % \end{equation}
  % \begin{equation}
  %   \bmmm{\mathcal{E}}_{\textrm{p}} \defd \bmmm{\mathcal{T}} \cdot
  %   \Bigl(1-\tfrac{c}{\sqrt{\abs{g_{tt}}}}\Bigr) \, \bmmm{e}_{t} \ .
  % \end{equation}
  % \begin{equation}
  %   \bmmmm{\mathcal{E}}_{\textrm{p}} \defd \bmmmm{\mathcal{T}} \cdot
  %   \Bigl(1-\tfrac{c}{\sqrt{\abs{g_{tt}}}}\Bigr) \, \bmmmm{e}_{t} \ .
  % \end{equation}
  % $$v \bm{v} \bmm{v} \bmmm{v} \bmmmm{v}
  % $$
\end{itemize}

In Schwarzschild isotropic coordinates typical of the barycentric or geocentric celestial reference systems (BCRS, GCRS), introduced by the International Astronomical Union and used in astronomy \autocites{kaplan2005,soffeletal2003,petitetal2005,soffeletal2013}, the vector field $\ve{t}$ is a Killing vector field; therefore the associated energy $\yEi + \yEk + \yEp$ is conserved, not just balanced; that is, the right side of \eqn~\eqref{eq:balance_V} is zero.

It must be emphasized that any one of the three energies $\yEi$ (wherever $\yUm$ is defined), $\yEk$, $\yEp$, and any sum of the three, could be used as \enquote{the} energy, and its balance~\eqref{eq:balance_V}--\eqref{eq:Stokes_V} used as \enquote{the} energy balance. These balances are all equivalent: any terms missing on one side of the equality appears on the other side in different mathematical guise.


Each of the energies defined above, being a 3-covector, can be integrated over a spacelike three-dimensional region $R$:
\begin{equation}
  \label{eq:integral_E}
  \int_{R} \yEE_{x} \equiv \int_{R} \yEc^{\mu}\, \ttti{\mu}
\end{equation}
which is the total, net amount of energy in that region. Also this is a coordinate-independent result, depending only on the region $R$ and on the reference field used to define the energy. Analogously we can integrate over a timelike three-dimensional region; the result is the total, net amount of energy flowing through that evolving surface.

\medskip

We now see where the arbitrariness in the definition of total energy in Newtonian mechanics comes from: it corresponds to the arbitrariness in choosing a timelike reference field for defining energy.

\medskip

We can also give a more precise definition of the component $\yT\i{^{\mu}_{\nu}}$ of the energy-momentum tensor, at a given spacetime point:
\begin{equation}
  \label{eq:def_Tmunu}
  \yT\i{^{\mu}_{\nu}} =
  \left\{
    \parbox[c]{0.67\linewidth}{\footnotesize
      flux of generalized energy associated with reference field $\ve{\nu}$
      \\
      across 3D volume element of constant $x^{\mu}$,
      per unit coordinates
    }
  \right.
\end{equation}

\subsection{Definition of angular momentum and boost momentum}
\label{sec:def_angular_momentum}

The balances for any one of the energies above and for the three components of momentum are together mathematically equivalent to the divergence equation~\eqref{eq:div_EMt} of the energy-momentum tensor,
\begin{equation*}
    \Di\yTT = 0
  \qquad\text{\footnotesize or }\qquad
  \de_{\mu}\yT\i{^{\mu}_{\alpha}}
  = \yT\i{^{\mu}_{\nu}} \, \varGamma^{\nu}_{\mu\alpha} \ .
\end{equation*}
In fact the four balances can be viewed as sorts of projections of that equation along four different spacetime directions.

The symmetry equation~\eqref{eq:sym_EMt} for the energy-momentum tensor,
\begin{equation*}
  \tfrac{c}{\dg}\,
  \yT\i{^{\mu}_{\alpha}} \, g^{\alpha\nu} \,
\ve{\mu} \otimes \ve{\nu}
  =
  \tfrac{c}{\dg}\,
  \yT\i{^{\nu}_{\alpha}} \, g^{\alpha\mu} \,
\ve{\mu} \otimes \ve{\nu}
  \ ,
\end{equation*}
has a point-wise nature: it does not involve any kind of derivative. It could be re-expressed as a collection of six point-wise relationship between components of energy- and momentum-currents defined above.

Alternatively it can be re-expressed as a set of six balance laws for the currents associated with six additional reference fields. These additional reference fields must be independent, in a field sense, from the ones considered in the previous section. They cannot therefore be simple linear combinations, with constant coefficients, of the reference fields above. But they can be combinations with functional coefficients.

We can consider for instance the six reference fields\autocites[cf.][\sect~3.2 p.~62]{hawkingetal1973_r1994}
\begin{equation}
  \label{eq:ref_angmomentum}
  \begin{gathered}
    \yg^{-1} \cdot (y\,\vi{z} - z\,\vi{y}) \qquad
    \yg^{-1} \cdot (z\,\vi{x} - x\,\vi{z}) \qquad
    \yg^{-1} \cdot (x\,\vi{y} - y\,\vi{x}) \ ,
    \\
    \yg^{-1} \cdot (t\,\vi{x} - x\,\vi{t}) \qquad
    \yg^{-1} \cdot (t\,\vi{y} - y\,\vi{t}) \qquad
    \yg^{-1} \cdot (t\,\vi{z} - z\,\vi{t}) \ ,
  \end{gathered}
\end{equation}
possibly normalized. The currents associated with the first three are the $x$-, $y$-, $z$-components of angular momentum. In fact from \eqns~\eqref{eq:xmomentum_current} we see that in the Newtonian approximation these currents are
\begin{equation}
  \label{eq:angmomentum_current}
  \begin{gathered}
    y\,\yPP_{z} - z\,\yPP_{y} \qquad
    z\,\yPP_{x} - x\,\yPP_{z} \qquad
    x\,\yPP_{y} - y\,\yPP_{x} \ ,
    \\
t\,\yPP_{x}  - x\,\yEE \qquad
t\,\yPP_{y}  - y\,\yEE \qquad
t\,\yPP_{z}  - z\,\yEE \ ,
  \end{gathered}
\end{equation}
where $\yEE$ is the energy associated with the reference field $\ygg^{-1} \cdot \vi{t}$. The currents associated with the last three reference fields are the analogous components of boost momentum;\autocites[cf.][\chap~2 \sect~14]{landauetal1939_t1996} they depend on the specific choice of reference field for energy. In some important situations the reference fields~\eqref{eq:ref_angmomentum} are Killing fields, and therefore the associated currents~\eqref{eq:angmomentum_current} satisfy conservation laws; in the case of boost momentum the law is a form of the theorem of the centre of energy, also called centre of inertia\ibid.

\medskip

Thus we see the reason why there is a special connection between angular momentum and momentum, even though their balances are independent. Angular momentum must be defined with respect to reference fields different from the ones for momentum and energy. Yet, since spacetime is four dimensional, \emph{at each spacetime point} the reference fields for angular momentum must be equivalent to linear combinations of the ones for momentum and energy; the coefficients of these linear combinations will differ from point to point, though. This is exactly what happens in \eqns~\eqref{eq:angmomentum_current}.

\medskip

But it must be remarked that nothing forces us to use the reference fields~\eqref{eq:ref_angmomentum}. We could use different independent fields, for instance replacing the first three with
\begin{equation}
  \label{eq:alt_ref_fields}
  x\,\ve{x} \qquad
  y\,\ve{y} \qquad
  z\,\ve{z} \ ,
\end{equation}
associated with the three currents
\begin{equation}
  \label{eq:alt_angmomentum}
  x\,\yPP_{x} \qquad
  y\,\yPP_{y} \qquad
  z\,\yPP_{z}
\end{equation}
which would satisfy appropriate balance laws. These three alternative \enquote{angular momenta} and their balances could be used instead of the traditional angular momentum to frame and solve physics problems.

\subsection{Consequences and caveats for coordinate transformations}
\label{sec:caveats_transf}

Energy and momentum depend on the specification of a reference field; and the reference field may or may not be chosen to be related to the coordinate system. These facts make the rules for transformation of energy and momentum a little tricky.

For instance, suppose that we have defined energy with respect to the reference field $\ve{t}$ of the coordinate system under use. If we change coordinate system and have a new time coordinate $t'$, then the energy defined analogously in the new coordinates is associated with the reference field $\ve{t'}$ -- so \emph{it is actually a differently defined energy}, not the same energy as before but in a different coordinate system. This peculiarity does not need to be a problem, as long as we're aware of it.

\section{Newtonian approximation}
\label{sec:newton_approx}

As mentioned above, the Newtonian approximation consists in considering a low-curvature situation, with a spacetime flat at infinity, and therefore with ten approximate Killing vector fields, at least at inifinity. A set of coordinates corresponding to these Killing fields and for which the metric has almost Minkowskian components is chosen, with dimensions of time and length. Its coordinate time is treated as \enquote{absolute}. Approximate expressions for the energy-momentum tensor and for transformations to other inertial coordinates are obtained.

In this approximation, the sets of Killing vector fields~\eqref{eq:choice_bvectors} and
\begin{equation}
  \label{eq:ref_angmom_contrav}
    \begin{gathered}
    y\,\ve{z} - z\,\ve{y} \qquad
    z\,\ve{x} - x\,\ve{z} \qquad
    x\,\ve{y} - y\,\ve{x} \ ,
    \\
    c t\,\ve{x} + \frac{1}{c} x\,\ve{t} \qquad
    c t\,\ve{y} + \frac{1}{c} y\,\ve{t} \qquad
    c t\,\ve{z} + \frac{1}{c} z\,\ve{t} \ ,
  \end{gathered}
\end{equation}
are chosen as reference fields, and energy and momentum are defined accordingly.

The components $\yT\i{^{\mu}_{\nu}}$ of the energy-momentum tensor then have the meaning sketched in \eqn~\eqref{eq:energy_momentum_entries}:
\begin{equation}
  \label{eq:energy_momentum_entries_correct}
  \begin{bmatrix}
    \text{\footnotesize $-$energy density} &
    \text{\footnotesize momentum density} \\[2\jot]
    \rotatebox[origin=c]{90}{\text{\footnotesize $-$energy flux density}} &
    \rotatebox[origin=c]{45}{\parbox{9em}{\centering\footnotesize stress\\ (momentum flux density)}}
  \end{bmatrix}
  \equiv
    \begin{bmatrix}
    -m c^{2} & p_{x} & p_{y} & p_{z}
    \\
    -c^{2} q_{x} & \sigma_{xx} & \sigma_{xy} & \sigma_{xz}
    \\
    -c^{2} q_{y} & \sigma_{yx} & \sigma_{yy} & \sigma_{yz}
    \\
    -c^{2} q_{z} & \sigma_{zx} & \sigma_{zy} & \sigma_{zz}
  \end{bmatrix}
\ ,
\end{equation}
with the following clarifications:
\begin{itemize}[noitemsep]
\item Energy and momentum are those associated with the reference fields $\ve{\nu}$.
\item The densities or fluxes are across three-dimensional spacetime regions of constant $x^{\mu}$.
\item The densities are measured per unit coordinates.
\end{itemize}


As already mentioned in \sect~\ref{sec:energies}, there still remains the possibility of defining energy in slightly different ways. The use of the reference field $\yUm$ is typical in continuum mechanics. In continuum and point mechanics one sees the choices $\ve{t}$ and $\tfrac{c}{\sqrt{\abs{g_{tt}}}}\ve{t}$ used equally often.

\medskip

In a new coordinate system obtained by a Lorentz transformation
\begin{equation}
  \label{eq:lorentz_transf}
  \begin{gathered}
    t \mapsto \tfrac{1}{\sqrt{1-v^{2}/c^{2}}}\, (t - v x/c^{2})
    \\
    x \mapsto \tfrac{1}{\sqrt{1-v^{2}/c^{2}}}\, (x - v t)
    \qquad y \mapsto y \qquad z \mapsto z \ ,
  \end{gathered}
\end{equation}
the energy-momentum tensor~\eqref{eq:energy_momentum_entries_correct} has the following components up to $\Ord(c^{-2})$ terms:
\begin{multline}
  \label{eq:energy_momentum_boost}
    \left[\begin{smallmatrix}
        -m c^{2} - m v^{2} + v p_{x} + v q_{x}
        & p_{x} - m v & p_{y} & p_{z}
    \\
    -c^{2} q_{x} + m v c^{2} + m v^{3}- v^{2} p_{x} - v^{2} q_{x} + v \sigma_{xx}
    &\quad
    \sigma_{xx} + m v^{2} - v p_{x} - v q_{x}
    &\quad \sigma_{xy} - v p_{y}
    &\quad \sigma_{xz} - v p_{z}
    \\[\jot]
    -c q_{y} - \frac12 v^{2} q_{y} + v \sigma_{yx}
    & \sigma_{yx} - v q_{y}
    & \sigma_{yy} & \sigma_{yz}
    \\[\jot]
    -c q_{z} - \frac12 v^{2} q_{z} + v \sigma_{zx}
    & \sigma_{zx} - v q_{z}
    & \sigma_{zy} & \sigma_{zz}
  \end{smallmatrix}\right]
+{}\\[2\jot] \Ord(c^{-2})
\end{multline}
which shows several peculiarities:
\begin{itemize}[noitemsep]
\item To order $\Ord(c^{2})$, the mass-energy is the same.
\item To order $\Ord(c^{0})$, the mass-energy acquires terms proportional to the squared velocity.
\item The momentum acquires terms proportional to the mass-energy and the velocity.
\item To order $\Ord(c^{0})$, the mass-energy flux acquires terms proportional to the mechanical power of the stresses.
\item To order $\Ord(c^{0})$, the momentum flux acquires terms proportional to the energy flux.
\item Energy and momentum \enquote*{transport} terms appear in the energy and momentum fluxes, as expected from Newtonian continuum mechanics.
\end{itemize}


\section{Answers to the initial questions}
\label{sec:answers}

\begin{enumerate}[itemsep=1ex]
\item \emph{How should \emph{total energy} be defined? Is it just the sum of internal and kinetic energies? or should it also include gravitational potential energy?}

  Answer: The definition depends on the choice of a reference vector field; all such choices are acceptable. If we choose the vector field related to the coordinate $t$, then the associated total energy includes a gravitational potential energy, and is moreover conserved.


\item \emph{Why is \emph{internal energy} the same in two reference frames, whereas \emph{kinetic energy} differs?}

Answer: Because internal energy is the energy associated to the reference field $\yUm$, which is  the same in all coordinate systems. In reality also internal energy will change between reference frames in a higher approximation. Kinetic energy is instead associated with a reference field that is connected with the coordinates, and therefore changes under coordinate transformations already in a first approximation.


\item \emph{Why does force, which is related to change of momentum, appear in the laws for the change of energy (in the formula for \enquote{work})? Why doesn't change of energy appear, vice versa, in the law for change of momentum?}

  Answer: Because energy, momentum, flux of energy, and force (flux of momentum) are just components of one geometric object, and they mix together under coordinate transformations. The peculiarity of Lorentz transformations make it so that, in a first approximation, the flux of momentum in the previous coordinates enters the flux of energy in the new ones. But in higher-order approximations also the flux of energy enters the flux of momentum.


\item \emph{What is the \emph{law of transformation for momentum} between reference frames? For instance, if we know that the components of momentum in an inertial frame are $[0,1,2]\:\unit{N\,s}$, then what are the components in another inertial frame with constant velocity $\bm{v}$ with respect to the first?}

Answer: Momentum is not a vector in itself, but a set of components of a more complex geometric object. Their transformation therefore depends on the values of the other components of that object.
\end{enumerate}




% ***ref to MTW p.222


%%%%%%%%%%%%%%%%%%%%%%%%%%%%%%%%%%%%%%%%%%%%%%%%%%%%%%%%%%%%%%%%%%%%%%%%%%%%
%%% Acknowledgements
%%%%%%%%%%%%%%%%%%%%%%%%%%%%%%%%%%%%%%%%%%%%%%%%%%%%%%%%%%%%%%%%%%%%%%%%%%%% 
\iffalse
\begin{acknowledgements}
  \ldots to Mari \amp\ Miri for continuous encouragement and affection, and
  to Buster Keaton and Saitama for filling life with awe and inspiration.
  To the developers and maintainers of \LaTeX, Emacs, AUC\TeX, Open Science
  Framework, R, Python, Inkscape, Sci-Hub for making a free and impartial
  scientific exchange possible.
  % Our work was supported by the Trond Mohn Research Foundation, grant number BFS2018TMT07
%\rotatebox{15}{P}\rotatebox{5}{I}\rotatebox{-10}{P}\rotatebox{10}{\reflectbox{P}}\rotatebox{-5}{O}.
%\sourceatright{\autanet}
\mbox{}\hfill\autanet
\end{acknowledgements}
\fi

%%%%%%%%%%%%%%%%%%%%%%%%%%%%%%%%%%%%%%%%%%%%%%%%%%%%%%%%%%%%%%%%%%%%%%%%%%%%
%%% Appendices
%%%%%%%%%%%%%%%%%%%%%%%%%%%%%%%%%%%%%%%%%%%%%%%%%%%%%%%%%%%%%%%%%%%%%%%%%%%% 
\clearpage
%\bigskip
% %\renewcommand*{\appendixpagename}{Appendix}
% %\renewcommand*{\appendixname}{Appendix}
% %\appendixpage
% \appendix

%%%%%%%%%%%%%%%%%%%%%%%%%%%%%%%%%%%%%%%%%%%%%%%%%%%%%%%%%%%%%%%%%%%%%%%%%%%%
%%% Bibliography
%%%%%%%%%%%%%%%%%%%%%%%%%%%%%%%%%%%%%%%%%%%%%%%%%%%%%%%%%%%%%%%%%%%%%%%%%%%% 
\renewcommand*{\finalnamedelim}{\addcomma\space}
\defbibnote{prenote}{{\footnotesize (\enquote{de $X$} is listed under D,
    \enquote{van $X$} under V, and so on, regardless of national
    conventions.)\par}}
% \defbibnote{postnote}{\par\medskip\noindent{\footnotesize% Note:
%     \arxivp \mparcp \philscip \biorxivp}}

\printbibliography[prenote=prenote%,postnote=postnote
]

\end{document}

%%%%%%%%%%%%%%%%%%%%%%%%%%%%%%%%%%%%%%%%%%%%%%%%%%%%%%%%%%%%%%%%%%%%%%%%%%%%
%%% Cut text (won't be compiled)
%%%%%%%%%%%%%%%%%%%%%%%%%%%%%%%%%%%%%%%%%%%%%%%%%%%%%%%%%%%%%%%%%%%%%%%%%%%% 


% \begin{figure}
% \centering
% \hspace*{\fill}{ILLUSTRATION 1} \hfill {ILLUSTRATION 2}\hspace*{\fill}
% \end{figure}

%%% Local Variables: 
%%% mode: LaTeX
%%% TeX-PDF-mode: t
%%% TeX-master: t
%%% End: 
