%\pdfoutput=1
%% portamana161009-einstein_from_newton.tex --- 
%% 
%% Author: PGL  Porta Mana
%% Created: 2016-10-07T23:51:28+0200
%% Last-Updated: 2016-10-12T13:06:25+0200
%%%%%%%%%%%%%%%%%%%%%%%%%%%%%%%%%%%%%%%%%%%%%%%%%%%%%%%%%%%%%%%%%%%%%%
%Report-no: ***
\newcommand*{\memfontfamily}{zplx}
\newcommand*{\memfontpack}{newpxtext}
\documentclass[10pt,%extrafontsizes,%
%dvips,% comment this for arXiv
onecolumn,oneside,a5paper,article,frenchb,italian,german,swedish,latin,british%
%,draft%***
%,final%
]{memoir}

\newcommand*{\pdftitle}{Einstein's equations from a Newtonian perspective}
\newcommand*{\headtitle}{\pdftitle}
\newcommand*{\pdfauthor}{P.G.L.  Porta Mana}
\newcommand*{\headauthor}{\autanet\ Porta Mana}
\newcommand*{\reporthead}{}

%%% Call to the various packages, most of them in bringhurst3.sty %%%
%\usepackage[25-31,files]{pagesel}
\usepackage[%nomicrotype,%
%nosum,%
%nomathpazo% loads mathpazo - Palatino text & math
%,noosf% no old-style figures
%,fakesc,%,
%noeugreek,% no greek letters in Euler font
%,nooptima% no Zapf's Optima font as sans serif
%,noitsans% no math sans serif in italic
%,nolmodern% use Latin Modern instead of CM
%,noeucal%
%,commadec% print a comma as dec separator when . is used
%,ttassf% sans serif instead of typewriter font
%,bb% blackboard letters for naturals, integers, reals, etc.
datetime]{bringhurst3}

%%%% Paper's details %%%%
\title{\pdftitle%***
}
\author{%
%\hspace*{\stretch{1}}\protect\makebox[0pt][c]%
%{\firstname{***}\ \surname{***}}%
%\hspace*{\stretch{1}}\protect\makebox[0pt][c]%
{\firstname{P.G.L.}\ \surname{Porta\,Mana}}%
%\hspace*{\stretch{1}}%
\\[2\jot]%
\affiliation{INM-6, Forschungszentrum Jülich, Germany}
%\\[2\jot]%
\quad
\epost{\email{piergianluca}{portamana.org}}%
}

\date{Draft of \mydate\today\ (first drafted 9 October 2016)}

%@@@@@@@@@@@@@@@@@@@@@@@ new commands @@@@@@@@@@@@@@@@@@@@@
\definecolor{notecolour}{RGB}{68,170,153}
\newcommand*{\mynote}[1]{ {\color{notecolour}\maltese\ #1}}
%\DeclareMathOperator{\artanh}{artanh}
%\DeclareMathOperator{\nabnab}{\bm{\nabla\nabla}}
%\newcommand*{\nabnab}{\mathord{\bm{\nabla}\otimes\bm{\nabla}}}
\newcommand*{\yK}{\mathte{K}}
\newcommand*{\yh}{\mathte{h}}
\newcommand*{\yR}{\mathte{R}}
\newcommand*{\yT}{\mathte{T}}
\newcommand*{\yp}{\bm{p}}
\newcommand*{\yr}{\rho} %volumic mass
\newcommand*{\yv}{\bm{v}} %veloc
\newcommand*{\yb}{\bm{b}} %body force
\newcommand*{\yu}{u} %mass energy
\newcommand*{\yq}{\bm{q}} % areic heat flow rate (q)
\newcommand*{\yQ}{Q} %massic heat flow rate (Q) (phi)
\newcommand*{\yte}{\varTheta} %temp T Theta
\newcommand*{\ys}{s} % mass entropy
\providecommand{\diad}{}\renewcommand*{\diad}{\otimes}

%@@@@@@@@@@@@@@@@@@@@@@@ new commands end @@@@@@@@@@@@@@@@@

\firmlists
\begin{document}
\captiondelim{\quad}\captionnamefont{\footnotesize}\captiontitlefont{\footnotesize}
\selectlanguage{british}\frenchspacing

%%% Title and abstract %%%
\maketitle
\myabstract
\begin{abstract}\labelsep 0pt%
\noindent ***
\par%\\[\jot]
\noindent\pacs{***}\qquad\msc{***}
\end{abstract}

\selectlanguage{british}\frenchspacing
%\asudedication{to ***}
%\setlength{\epigraphwidth}{.7\columnwidth}
%\setlength{\epigraphrule}{0pt}
%\epigraph{}

\section{Introduction}
\label{sec:introduction}

\iffalse
The 3+1 decomposition of G=T yields an evolution equation and two constraint equations. The latter should, in principle, be enforced on all 3-folia of the evolution. The question is whether they can be enforced on the initial folium only, and be automatically propagated in the evolution.

One can prove that the constraints are automatically propagated if (and only if?) the Bianchi equations hold for G-T. Now, the Bianchi equations can be made to hold in two ways: either enforcing D.T=0 (since D.G=0 by construction), or enforcing G=T. But the latter requires the constraints to be enforced on all folia. Hence we must choose D.T=0.

So, if I've understood correctly, we don't need to include the evolution equations D.T=0 if we're willing to enforce the constraints on each folium (which is computationally more expensive in a simulation).

This seems to make sense from a discrete point of view: given the 3-metric, extrinsic curvature, and matter fields at time t, the evolution equation and the definition of extrinsic curvature yield the 3-metric and extrinsic curvature at t+dt. These can be used in the constraint equations at t+dt to obtain the matter fields at t+dt. And the iteration repeats. If we don't want to use the constraints at t+dt we must find another way of determining the matter fields at t+dt, i.e. by using D.T=0 (which can itself be obtained by combining the "time derivative" of the constraints with the evolution equation).
\fi


* If space metric and chronometry are influenced by matter, then all “rates
of change” are determined by matter.

* Rate of change of momentum reinterpreted as inertial force (Mach)

* “Motion” is not clear any longer. Topologically we can consider a bulk of
matter as static and the metric among its elements as changing, or vice
versa.

* The problem is that only the relations of the geometrical objects with
respect to one another is observable; but such relations are not all
arbitrary, there are constraints among them. In Newtonian mechanics we use
a reference frame -- a \emph{rigid} body -- to find a compact description
that gets rid of the constraints
\cite{zanstra1922,zanstra1923,zanstra1924,zanstra1946,barbour2010}. In
general relativity there is no “rigidity” and we have more choice.

* Do the equations for the metric have constitutive parts?

* If mass is twisted 3-form, then (analogously to displacement current)
gravitational potential must be related to a twisted 2-form
\cite[ch.~V, pp.~192--195]{kottler1922,whittaker1953}. Then is there an
equivalent “magnetic field strength” that relates to momentum?
\begin{gather}
  \di m =0, \qquad \de_t\rho+\di j =0,
  \\
  \di F = \rho, \qquad \de_t F + \di H = j
\end{gather}

* Extrinsic curvature can be understood (kinematically) as the
time-derivative of 3-metric (no Einstein eqns needed).

Setting lapse $\alpha=1$ and shift $\beta=0$:
\begin{gather}
  \de_t \yh=-2 \yK \cdot \yh
  \\
  \de_t \yK = 
  \yR + \yK\tr \yK  + 4\pu (\tr \yT - E -2 \yT)
  \\
  \tr\yR + (\tr \yK)^2 - \yK \con \yK = 16\pu E
  \\
  \nabl\cdot \yK - \nabl\tr \yK = 8\pu \yp
\end{gather}
the second should come from
\begin{equation}
  \de_t (\yK\cdot \yh) =
  \yR\cdot \yh + \yh\cdot \yK\tr \yK -2 \yh\cdot \yK\cdot \yK + 4\pu [\yh(\tr \yT - E) -2
  \yT\cdot \yh)
\end{equation}

\begin{gather}
  \de_t \yp =\tfrac{1}{8\pu}
  (\nabl\cdot\yR  -\de_t\nabl\tr\yK + \tr\yK\,\nabl\cdot\yK)
  +\tfrac{1}{2}\nabl\tr\yT -\tfrac{1}{2}\nabl E -\nabl\cdot\yT
  \\
  \de_t \yp -\yp\tr\yK=\tfrac{1}{8\pu}
  [\nabl\cdot\yR  -\de_t\nabl\tr\yK + \tfrac{1}{2}\nabl(\tr\yK)^2]
  +\tfrac{1}{2}\nabl\tr\yT -\tfrac{1}{2}\nabl E -\nabl\cdot\yT  
\end{gather}


\begin{subequations}\label{eq:balances}
  \begin{align}
    \label{eq:mass_bal}
    &\de_t \yr + \nabla\cdot(\yr\yv) =0 , \\\label{eq:mom_bal}
    % \nabla\cdot\yT + \yb - \yr\de_t\yv -\yr \yv\nabla\cdot\yv =0,
    & \de_t(\yr\yv) + \nabla\cdot(\yr\yv\diad\yv)
- \nabla\cdot\yT - \yr\yb  =0,
    \\\label{eq:spin_bal} 
    &\yT\T - \yT =0, \\\label{eq:en_bal}
     &\de_t(\yr\yu) +
\nabla\cdot(\yr\yv\yu) -
\yT:\nabla\yv 
       +\nabla\cdot\yq +\yr\yQ =0,
       \\
&
\begin{multlined}[b][.9\columnwidth]
 \de_t\bigl(\yr\yu +\tfrac{1}{2}\yr\yv^2\bigr) +
\nabla\cdot\bigl(\yr\yv\yu
  +\tfrac{1}{2}\yr\yv\yv^2\bigr) -{}\\
\nabla\cdot(\yT\cdot \yv) -\yr\yb\cdot\yv
  -\nabla\cdot\yq -\yr\yQ =0,
\end{multlined}
\\\label{eq:ent_bal} 
&\de_t(\yr \ys) + 
    \nabla\cdot(\yr\yv\ys) - \nabla\cdot(\yq/\yte) - \yr\yQ/\yte \ge 0.
  \end{align}
\end{subequations}

\iffalse
\begin{figure}[!b]
\centering
\includegraphics[width=0.4\columnwidth]{***}%
\caption{***}
\label{***}
\end{figure}
\fi

\clearpage

\begin{acknowledgements}
  Many thanks to Mari \amp\ Miri for continuous encouragement and affection,
  to Buster for filling life with awe and inspiration, and to the
  developers and maintainers of \LaTeX, Emacs, AUC\TeX, MiK\TeX, arXiv,
  biorXiv, PhilSci, Hal archives, Python, Inkscape, Sci-Hub for making a
  free and unfiltered scientific exchange possible.
%\rotatebox{15}{P}\rotatebox{5}{I}\rotatebox{-10}{P}\rotatebox{10}{\reflectbox{P}}\rotatebox{-5}{O}.
\sourceatright{\autanet}
\end{acknowledgements}

%\appendixpage
%\appendix

%%%%%%%%%%%%%%% BIB %%%%%%%%%%%%%%%

\defbibnote{postnote}{\small\par\medskip\noindent{\footnotesize% Note:
\arxivp \mparcp \philscip \biorxivp}%
}

\newcommand{\citein}[2][]{\textnormal{\textcite[#1]{#2}}%\addtocategory{extras}{#2}
}
\newcommand{\citebi}[2][]{ref.\ \citep[#1]{#2}%\addtocategory{extras}{#2}
}

\printbibliography[postnote=postnote]


\end{document}
---------- cut text ----------------


%%% Local Variables: 
%%% mode: LaTeX
%%% TeX-PDF-mode: t
%%% TeX-master: t
%%% End: 
